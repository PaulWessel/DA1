%  $Id: DA1_Chap5.tex 688 2019-06-09 09:24:42Z pwessel $
%
\chapter{LINEAR (MATRIX) ALGEBRA}
\epigraph{``Never send a human to do a machine's job.''}{\textit{Agent Smith} in \textit{The Matrix}}
\label{ch:matrix}
A large subset of data analysis techniques is simply the practical application of
linear algebra.  Undergraduate students in the natural sciences often lack any formal introduction
to this material, or they suffered through an overly theoretical presentation in a course
offered by a mathematics department.  In this book we will not dwell too much on the finer
theoretical points of linear algebra but instead present a simple overview of the aspects that are particularly
pertinent in data analysis.  There are of course an infinite number of books on matrix and
linear algebra that the eager reader could consult beyond this brief introduction.

\section{Matrix Algebra Terminology}

\index{Matrix!definition}
\index{Matrix!order}
	A \emph{matrix} is simply a rectangular array of elements arranged in a series of $m$  rows and $n$  
columns. The \emph{order} of a matrix is the specification of the number of rows by the number of 
columns. \emph{Elements} of a matrix are denoted $a_{ij}$, where index $i$ specifies the \emph{row} position 
and index $j$ specifies the \emph{column} position; thus $a_{ij}$ identifies the element at position $i,j$.

\index{Matrix!element}
	An element can be a number (real or complex), an algebraic expression, or (with some 
restrictions) another matrix or matrix expression. As an example of a real matrix,  consider
\begin{equation}
\mathbf{A} = \left[ \begin{array}{rcc}
12 & 4 & 10 \\
8 & 1 & 11\\
15 & 3 & 11\\
14 & 1 & 11
\end{array}   \right]	 
 \ = \
\left[  \begin{array}{ccc}
a_{11} & a_{12} & a_{13} \\
a_{21} & a_{22} & a_{23}\\
a_{31} & a_{32} & a_{33} \\
a_{41} & a_{42} & a_{43}
\end{array} \right].
\end{equation}
This matrix, $\mathbf A$, is of order $4 \times 3$, with elements $a_{11}$ = 12, $a_{21}$ = 8, and so on. The notation for matrices is 
not always consistent, but it is usually one of the following schemes:

\begin{description}

\item [Matrix:] Designated by a bold, uppercase letter (the most common scheme), brackets, or hat $(\hat{ \ })$,
sometimes with one or (more commonly) two underscores.
The order is also sometimes explicitly given.  E.g., $\mathbf{A}_{(4,3)}$ means 
$\mathbf A$ is of order $4 \times 3$.

\item [Order:] Always given as rows x columns but uses letters $n,m,p$ differently: $n$ (rows) x $m$ (columns) or 
$m$ (rows) $\times n$ (columns).

\item [Element:] Most commonly $a_{ij}$, with $i =$ row; $j =$ column (sometimes other dummy indices like $k,l,p$ are used).
\end{description}

The advantages of matrix algebra mainly lies in the fact that it provides a concise and simple 
method for manipulating large sets of numbers or computations, making it ideal for computers. 
Furthermore,

\begin{enumerate}

\item The compact form of matrices allows a convenient notation for describing large tables of data.

\item Matrix operations allow complex relationships to be seen, which otherwise would be obscured by the 
shear size of the data (i.e., it aids in clarification).

\item Most matrix manipulations involve just a 
few standard operations for which standard subroutines are readily available.
\end{enumerate}

MATLAB, which stands for ``Matrix Laboratory'', is ideally suited to perform such manipulations,
as is its Open Source clone, Octave. Python with numPy is also a good choice.
\index{MATLAB}
\index{Octave}

	As a convention with data matrices (i.e., when the elements are data values), the columns 
usually represent the different \emph{variables} (e.g., one column contains temperatures, another salinity, 
etc.) while rows contain the \emph{observations} (e.g., the values of the variables at different depths, times, or positions). Since 
there are usually more observations than variables, such data matrices are typically rectangular, having 
more rows $(m)$ than columns $(n)$, i.e., the matrix has an order $m \times n$ where $m > n$.

\section{Matrix Definitions}

Matrices whose smallest dimension equals one are called \emph{vectors} and are typically designated
by a bold, lowercase letter, but sometimes they may be typeset normally with either an
arrow above them or a single underscore beneath. By having only one dimension, one of the two indices (row or column) is dropped.
A \emph{column vector} is a matrix containing only a single column of elements, such as
\index{Column vector}
\index{Vector!column}
\begin{equation}
\mathbf{a} = \left[ \begin{array}{c}
a_1\\
a_2\\
\vdots\\
a_n
\end{array}  \right] . 
\end{equation}
\noindent
A \emph{row vector} is a matrix containing only a single row of elements, e.g.,
\index{Row vector}
\index{Vector!row}
\begin{equation}
\mathbf{a} = \left[ \begin{array}{cccc}
a_1 & a_2 & \cdots & a_n
\end{array}  \right] .
\end{equation}
The size of a vector is simply the number of elements it contains ($n$, in both examples above).
\index{Null matrix}
\index{Matrix!null}
The \emph{null matrix}, written as $\mathbf{0}$ or $\mathbf{0}_{(m,n)}$ has all its elements set equal to 0 --- it plays the
role of ``zero'' in matrix algebra.
A \emph{square matrix} has the same numbers of rows as columns, so its order is $n \times n$.
\index{Square matrix}
\index{Matrix!square}
A \emph{diagonal matrix} is a square matrix with zeros in all positions except along the principal (or 
leading) diagonal:
\begin{equation}
	\index{Diagonal matrix}
	\index{Matrix!diagonal}
\mathbf{D} = \left[ 
\begin{array}{ccc}
3 & 0 & 0 \\
0 & 1 & 0\\
0 & 0 & 6
\end{array} \right]	 
\end{equation}
or
\index{Matrix!identity}
\index{Identity matrix}
\begin{equation}
d_{ij} = \left \{ \begin{array}{cl}
0 & \mbox{for } i \neq j\\
\mbox{nonzero} & \mbox{for } i = j 
\end{array} \right.
\end{equation}	 
This type of matrix is important for scaling the rows or columns of other matrices.  The \emph{identity 
matrix} is a diagonal matrix with all of its nonzero elements equal to one. Written as $\mathbf{I}$ or $\mathbf{I}_n$,
it plays the role of ``one'' in matrix algebra.
A \emph{lower triangular matrix} ($\mathbf{L}$) is a square 
matrix with all elements equal to zero \emph{above} the principal diagonal:
\begin{equation}
	\index{Matrix!lower triangular}
	\index{Lower triangular matrix}
\mathbf{L} = \left [ \begin{array}{ccc}
1 & 0 & 0 \\
3 & 7 & 0 \\
8 & 2 & 6
\end{array}
\right ] =
\left[ \begin{array}{ccc}
1 \\
3 & 7 \\
8 & 2 & 6 \end{array} \right]
\end{equation}
or
\begin{equation}
\ell_{ij} = \left \{ \begin{array}
{cl}
0 & \mbox{for } i < j\\
\mbox{nonzero} & \mbox{for } i \geq j 
\end{array} \right .
\end{equation}	 
An \emph{upper triangular matrix} is thus a square matrix with all elements equal to zero \emph{below} the principal 
diagonal
\begin{equation}
	\index{Matrix!upper triangular}
	\index{Upper triangular matrix}
u_{ij} = \left \{ \begin{array}{cl}
0 & \mbox{for } i > j\\
\mbox{nonzero} & \mbox{for } i \leq j 
\end{array} \right .
\end{equation}
If one multiplies two triangular matrices of the same form, the result is a third matrix of the same 
form.

\index{Fully populated matrix}
\index{Matrix!fully populated}
\index{Matrix!sparse}
\index{Sparse matrix}
	We also have the \emph{fully populated matrix} which is a matrix with all of its elements nonzero, 
the \emph{sparse matrix} which is a matrix with only a small proportion of its elements nonzero, and the \emph{scalar} 
which simply is a number (i.e., a matrix of order 1x1, representing a single element).

   A \emph{matrix transpose} (or the transpose of a matrix) is obtained by \emph{interchanging} the rows and columns 
of the matrix. Thus, row $i$ becomes column $i$ and column $j$ becomes row $j$. As a consequence, the order of the matrix 
is reversed:
\begin{equation}
	\index{Matrix!transpose}
	\index{Transpose of matrix}
\mathbf{A} = \left[     \begin{array}{cc}
1 & 14 \\
6 & 7 \\
8 & 2 
\end{array} \right]
 \Rightarrow
\mathbf{A}^T = \left[ \begin{array}{ccr}
1 & 6 & 8 \\
14 & 7 & 2 \end{array} \right]
\end{equation}
As shown, taking the transpose is indicated by the superscript ``T''.
Repeated transposing yields the original matrix, i.e.,
\begin{equation}
(\mathbf{A}^T)^T = \mathbf{A}.
\end{equation} 
A diagonal matrix is its own transpose: $\mathbf{D}^T = \mathbf{D}$. In general, we find the transpose rule
\begin{equation}
a_{ij} \Leftrightarrow a_{ji}.
\end{equation}	 
   A \emph{symmetric matrix} is a square matrix that is symmetric about its principal diagonal, so 
$a_{ij} = a_{ji}$.  Therefore, a symmetric matrix is equal to its own transpose:
\begin{equation}
	\index{Matrix!symmetric}
	\index{Symmetric matrix}
\mathbf{A} = \left[ \begin{array}{ccc} 
1 & 2 & 5 \\
2 & 6 & 3 \\
5 & 3 & 4
\end{array}
\right]
= \mathbf{A}^T .
\end{equation} 	 
A \emph{skew symmetric matrix} is a matrix in which
\begin{equation}
a_{ij} = -a_{ji}.
\end{equation}
Therefore, $\mathbf{A}^T = - \mathbf{A}$.  Thus, $a_{ii}$, the principal diagonal elements, must all be zero.
The following matrix is skew symmetric:
\begin{equation}
	\index{Matrix!skew symmetric}
	\index{Skew symmetric matrix}
\mathbf{A} = \left[ \begin{array}{ccc} 
0 & 4 & -5 \\
-4 & 0 & 3 \\
5 & -3 & 0
\end{array}
\right].
\end{equation}
Any square matrix can be decomposed into the \emph{sum} of a symmetric and a skew-symmetric matrix:
\begin{equation}
\mathbf{A} = \frac{1}{2} (\mathbf{A} + \mathbf{A}^T) + \frac{1}{2} (\mathbf{A} - \mathbf{A}^T).
\end{equation}
The \emph{trace} of a square matrix is simply the sum of the elements along the principal diagonal. It 
is symbolized as tr ($\mathbf{A}$).\index{Matrix!trace}\index{Trace of matrix}
This property is useful in calculating various quantities from matrices. 
\index{Matrix!submatrix}
\index{Submatrix}
\index{Matrix!supermatrix}
\index{Supermatrix}
\emph{Submatrices} are smaller matrix partitions of the larger \emph{supermatrix}, i.e.,
\begin{equation}
\left [ \frac{\mbox{Supermatrix}}{\mathbf{F}} \right] = \left[	\frac{\mathbf{A} | \mathbf{B}}{\mathbf{C} | \mathbf{D}} \right].
\end{equation}
Such partitioning will frequently be useful.
	 
\section{Matrix Addition}
\index{Matrix!addition}

	Matrix addition and subtraction require matrices of the \emph{same order} since each operation 
simply involves the addition or subtraction of corresponding elements. So, if $\mathbf{C} = \mathbf{A} + \mathbf{B}$ then
\begin{equation}
\mathbf{A} = \left[ \begin{array}{cc}
a_{11} & a_{12}\\
a_{21} & a_{22}\\
a_{31} & a_{32} \end{array} \right]	 , \mathbf{B} = 
\left[ \begin{array}{cc}
b_{11} & b_{12}\\
b_{21} & b_{22}\\
b_{31} & b_{32} \end{array} \right]	 ,
\mathbf{C} = \left[ \begin{array}{cc}
a_{11}+  b_{11} & a_{12} +  b_{12}\\
a_{21}+  b_{21} & a_{22} +  b_{22}\\
a_{31} + b_{31} & a_{32} + b_{32} \end{array} \right]	 ,
\end{equation}
and (with apologies to ABBA fans)
\begin{equation}
\mathbf{A} + \mathbf{B} = \mathbf{B} + \mathbf{A},
\end{equation}
\begin{equation}
(\mathbf{A} + \mathbf{B}) + \mathbf{C} = \mathbf{A} + (\mathbf{B} + \mathbf{C}),
\end{equation}	 
where all matrices must be of the same order. \emph{Scalar multiplication} of a matrix is achieved by multiplying all elements of a matrix 
by a constant (the scalar):
\begin{equation}
\beta \mathbf{A} = \beta \left[ \begin{array}{cc}
a_{11} & a_{12} \\
a_{21} & a_{32}\\
a_{31} & a_{32}
\end{array}
\right]
=
\left[ \begin{array}{cc}
\beta a_{11} & \beta a_{12} \\
\beta a_{21} & \beta a_{22}\\
\beta a_{31} & \beta a_{32}
\end{array}
\right],
\end{equation}	 
where $\beta$ is a scalar. 
Thus, every element is multiplied by the scalar. 

\section{Dot Product}
\index{Vector!product|(}
\index{Dot product|(}

The \emph{scalar product} (or \emph{dot product} or 
\emph{inner product}) is the product of two vectors of the same size, e.g.,
\begin{equation}
\mathbf{a}\cdot \mathbf{b} = \beta,
\end{equation}	 
where $\mathbf a$ is a row vector (or the transpose of a column vector) of length $n, \mathbf{b}$ is a column vector (or 
the transpose of a row vector), also of length $n$, and $\beta$ is the scalar product of $\mathbf a \cdot \mathbf b$.
Given the two vectors
\begin{equation}
\mathbf{a} = [a_1 \ a_2 \ a_3 ], \quad \mathbf{b} = \left[ \begin{array}{c}
b_1\\
b_2\\
b_3
\end{array}
\right],
\end{equation}	 
we sum the products of corresponding elements in the two vectors, obtaining
\begin{equation}
\beta = a_1 b_1 + a_2b_2 + a_3b_3.
\end{equation}
We may visualize this multiplication, as illustrated in Figure~\ref{fig:Fig1_dotproduct} for two 4-D vectors.
\PSfig[H]{Fig1_dotproduct}{The dot product of the two 4-D vectors $\mathbf{a} = [2 \ 1 \ 4 \ 5]$
and $\mathbf{b} = [1 \ 3 \ 4 \ 2]$
is obtained by multiplying the component pairs and calculating the sum of these products.}

Geometrically, this product can be thought of as multiplying the length of one vector by the 
component of the other vector that is parallel to the first, as shown in Figure~\ref{fig:Fig1_dotvector}:
\PSfig[H]{Fig1_dotvector}{Geometrical meaning of the dot product of two vectors.  Regardless of dimension, the
dot product is proportional to the cosine of the angle between the two vectors.}


As an example, think of $\mathbf b$ as a force and $|\mathbf a|$ as the magnitude of displacement, with their product equal to the work in the 
direction of $\mathbf a$. Thus:
\begin{equation}
\mathbf{a \cdot b = |a||b|} \cos (\theta),
\end{equation}	 
where the \emph{magnitude} of a vector $\mathbf x$ is given by
\begin{equation}
|\mathbf{x}| = \sqrt{x^2_1 + x ^2_2 + \ldots + x^2_n}. 
\end{equation}
The maximum principle says that the unit vector $\mathbf{(\hat{n})}$ making $\mathbf{ a \cdot \hat{n}}$ a maximum is that unit vector 
pointing in the same direction as $\mathbf{a}:$ If $\mathbf{\hat{n} \parallel a}$ then $\cos(\theta) = \cos(0^{\circ}) = 1$ and $\mathbf{a\cdot n = |a| |n|}\cos(\theta) =  
\mathbf{|a||\hat{n}| = |a|}$. This is equally true where $\mathbf{d}$ is any vector of a given magnitude --- that vector $\mathbf{\hat{n}}$ which 
parallels $\mathbf{d}$ will give the largest scalar product.
Parallel vectors thus have $\cos(\theta) = 1$, then $\mathbf{a \cdot b = |a||b|}$ and 
$\mathbf{a = \beta b}$ (i.e., two vectors are parallel if 
one is simply a scalar multiple of the other --- this property comes from equating direction cosines), 
where
\begin{equation}
\beta =  \mathbf{|a|/|b|}.
\end{equation}
In contrast, perpendicular vectors have $\cos \theta = \cos 90^\circ = 0$, so that $\mathbf{ a \cdot b} = 0$, where $\mathbf{a \bot   b}$.

Squaring vectors is simply the dot product of a vector with its own transpose, i.e.,
\begin{equation}
\mathbf{a}^2 = \mathbf{ a \cdot a}^T \mbox{ for row vectors}
\end{equation}
and
\begin{equation}
\mathbf{a}^2 = \mathbf{ a}^T \cdot \mathbf{a} \mbox{ for column vectors}.
\end{equation}
\index{Vector!product|)}
\index{Dot product|)}

\section{Matrix Multiplication}
\index{Matrix!multiplication|(}

Matrix multiplication requires ``conformable'' matrices. Matrices are conformable when 
there are as many columns in the first matrix as there are rows in the second matrix.  Consider
\index{Matrix!conformable}
\begin{equation}
\mathbf{C}_{(m,n)} = \mathbf{A}_{(m,p)} \cdot \mathbf{B}_{(p,n)}.
\end{equation}	 
The matrix product $\mathbf{C}$ is of order $m \times n$ and has elements $c_{ij}$, given by
\begin{equation}
c_{ij} = \sum^p _{k=1} a_{ik}b_{kj}.
\end{equation}	 
This is an extension of the scalar product --- in this case, each element of $\mathbf{C}$ is the scalar product of a row 
vector in $\mathbf{A}$ and a column vector in $\mathbf{B}$.  For instance, if
\begin{equation}
\left[
\begin{array}{cc}
c_{11} & c_{12} \\
c_{21} & c_{22} 
\end{array}
\right ]
=
\left[
\begin{array}{ccc}
a_{11}&  a_{12} &  a_{13}\\
a_{21}&  a_{22} &  a_{23}
\end{array}
\right ]
\left [
\begin{array}{cc}
b_{11} & b_{12}\\
b_{21} & b_{22}\\
b_{31} & b_{32}
\end{array}
\right],
\end{equation}
then
\begin{equation}
c_{12} = a_{11}b_{12} + a_{12}b_{22} + a_{13}b_{32}.
\end{equation}
We illustrate the situation in Figure~\ref{fig:Fig1_matprod1}.
\PSfig[h]{Fig1_matprod1}{The matrix product of two matrices $\mathbf{A}$ and $\mathbf{B}$.  The light blue row in $\mathbf{A}$ is ``dotted'' with
the light blue column in $\mathbf{B}$, resulting in the single, light blue element in $\mathbf{C}$.  Similarly, the dot
product of the light green vectors result in the single, light green element.  This process is repeated by letting
all rows in $\mathbf{A}$ be ``dotted'' with all the columns in $\mathbf{B}$.}
The order of multiplication is critical. Usually
\begin{equation}
\mathbf{A \cdot B \neq B \cdot A},
\end{equation}	 
and unless $\mathbf{A}$ and $\mathbf{B}$ are square matrices or the order of $\mathbf{A}^T$ is the same as the order of $\mathbf{B}$ (or vice 
versa), one of the two products cannot even be formed. The multiplication order is specified by stating

\begin{description}
\item [A] is \emph{pre}-multiplied by $\mathbf{B}$ (yielding $\mathbf{B \cdot  A}$)
\index{Matrix!premultiply}
\item [A] is \emph{post}-multiplied by $\mathbf{B}$ (yielding $\mathbf{ A \cdot B}$)
\index{Matrix!postmultiply}
\end{description}

The order in which the pairs are multiplied is not important \emph{mathematically}, i.e.,
\begin{equation}
\mathbf{ D = (A \cdot B) \cdot C = A \cdot (B \cdot C)},
\end{equation}
but we will see later that the order matter \emph{computationally}.
The \emph{transpose of a matrix product} is simply the multiplication of the transpose of each individual matrix in 
reverse order, i.e.,
\begin{equation}
\mathbf{D = A \cdot B \cdot C},
\end{equation}
\begin{equation}
\mathbf{D}^T = \mathbf{C}^T \cdot \mathbf{B}^T \cdot \mathbf{A}^T.
\label{eq:transposerule}
\end{equation}
\emph{Multiplication by $\mathbf{I}$} leaves the matrix unchanged, i.e.,
\begin{equation}
\mathbf{A \cdot I = I \cdot A = A}.
\end{equation}
For example, 
\begin{equation}
\left[ \begin{array}{ccc}
3 & 6 & 9 \\
2 & 8 & 7 
\end{array} \right]
\left[ \begin{array}{ccc}
1 & 0 & 0\\
0 & 1 & 0\\
0 & 0 & 1 \end{array}
\right]
 = 
\left[
\begin{array}{ccc}
3 & 6 & 9 \\
2 & 8 & 7
\end{array}
\right] .
\end{equation}
\emph{Premultiplication by a diagonal matrix} is written
$\mathbf{C = D \cdot A}$,
where $\mathbf{D}$  is  a diagonal matrix. Here, $\mathbf{C}$  is the $\mathbf{A}$ matrix with each \emph{row} scaled by the corresponding diagonal 
element of $\mathbf{D}$:
\begin{equation}
\mathbf{D}   =  \left[ \begin{array}{ccc}
d_{11}\\
& d_{22} \\
& & d_{33}
\end{array} \right] ,    \quad  \mathbf{A} = \left[ \begin{array}{ccc}
a_{11} & a_{12} & a_{13}\\
a_{21} & a_{22} & a_{23}\\
a_{31} & a_{32} & a_{33}
\end{array} \right ]
\end{equation}
\begin{equation}
\mathbf{C = D \cdot A}  =  \left[ \begin{array}{ccc}
a_{11}d_{11} & a_{12}d_{11} & a_{13}d_{11}\\
a_{21}d_{22} & a_{22}d_{22} & a_{23}d_{22}\\
a_{31}d_{33} & a_{32}d_{33} & a_{33}d_{33}
\end{array} \right]
\quad 
\begin{array}{l}
\leftarrow \mbox{ each element } \times d_{11}\\
\leftarrow \mbox{ each element } \times d_{22}\\
\leftarrow \mbox{ each element } \times d_{33}
\end{array} 
\end{equation}
\emph{Postmultiplication by a diagonal matrix} produces a matrix in which each \emph{column} has been 
scaled by the corresponding diagonal element $\mathbf{D}$.  Hence,
\begin{equation}
\mathbf{C = A \cdot D} = \left[\begin{array}{ccc} a_{11} d_{11} & a_{12} d_{22} & a_{13} d_{33}\\
a_{21} d_{11} & a_{22} d_{22} & a_{23} d_{33}\\
a_{31} d_{11} & a_{32}d_{22} & a_{33} d_{33}
\end{array} \right],
\end{equation}	 
where each column in $\mathbf{A}$ has been scaled by the corresponding diagonal matrix elements, $d_{ii}$.

\subsection{Computational considerations}
The matrix product
\begin{equation}
\mathbf{C}_{(m,n)} = \mathbf{A}_{(m,p)} \cdot \mathbf{B}_{(p,n)}
\end{equation}
involves $m \times n \times p$ multiplications and $m \times n  \times (p -1)$ additions.  hence,
\begin{equation}
\mathbf{E}_{(m,n)} = [ \mathbf{A}_{(m,p)} \cdot \mathbf{B}_{(p,q)}] \cdot \mathbf{C}_{(q,n)}
\end{equation}	 
gives $m \times p \times q$ multiplications, so
\begin{equation}
\mathbf{E} = [\mathbf{D}_{(m,q)}] \cdot \mathbf{C}_{(q,n)}
\end{equation}
gives $m\times q \times n$ multiplications, and
\begin{equation}
\mathbf{E} _{(m,n)} = \mathbf{A}_{(m,p)} \cdot [\mathbf{B}_{(p,q)} \cdot \mathbf{C}_{(q,n)}]
\end{equation}
gives $p\times q \times n$ multiplications, while
\begin{equation}
\mathbf{E}_{(m,n)} = \mathbf{A}_{(m,p)}\cdot [\mathbf{D}_{(p,n)} ]
\end{equation}	
gives $m \times p \times n$ multiplications. Therefore, the total number of operations depend on the order of multiplications:
\begin{enumerate}
\item $\mathbf{(A \cdot B) \cdot C} \Rightarrow mq(p+n)$ total multiplications
\item $\mathbf{A \cdot (B \cdot C)} \Rightarrow pn(m+q)$ total multiplications
\end{enumerate}
If both $\mathbf{A}$ and $\mathbf{B}$ are $100 \times 100$ matrices and $\mathbf{C}$ is $100 \times 1$, then $m = p = q = 100$, and $n 
= 1$. Multiplying using form (1) involves $\sim 1 \times 10^6$ multiplications, whereas form (2) involves $2 \times 
10^4$; so computing $\mathbf{B \cdot C}$ first, then premultiplying by $\mathbf{A}$ saves almost a million multiplications 
and almost an equal number of additions. Therefore, the order of operations is extremely important 
computationally for both speed and accuracy, as more operations lead to a greater accumulation of 
\emph{round-off errors}.
\index{Matrix!multiplication|)}

\section{Matrix Determinant}
\index{Matrix!determinant|(}
The \emph{determinant of a matrix} is a single number representing a property of a square matrix 
(and is dependent upon what the matrix represents). The main use here is for finding the inverse of a 
matrix or for solving simultaneous linear equations. Symbolically, the determinant is usually written as
det $(\mathbf{A})$, $\mathbf{|A|}$ or $\mathbf{||A||}$ (to differentiate from magnitude).
The calculation of a $2 \times 2$ determinant is carried out using the definition
\begin{equation}
|\mathbf{A}| = \left| \begin{array}{cc} a_{11} & a_{12} \\
a_{21} & a_{22}
\end{array} \right| = a_{11}a_{22} - a_{12}a_{21},
\end{equation}	 
which is the difference of the cross products. The calculation of an $n \times n$ determinant is given by
\begin{equation}
|\mathbf{A}| = a_{11} m_{11} - a_{12} m_{12} + a_{13} m_{13} - \cdots - (-1)^na_{1n} m_{1n},
\end{equation}	 
where $m_{11}$ is the determinant of $\mathbf{A}$ with the first row and column missing; $m_{12}$ is the determinant with 
the first row and second column missing, etc. For larger matrices the procedure is used recursively.
The determinant of a $1 \times 1$ matrix is just the 
particular element. An example of a $3 \times 3$ determinant follows:
\begin{equation}
|\mathbf{A} | = \left | \begin{array}{ccc}
a_{11} & a_{12} & a_{13}\\
a_{21} & a_{22} & a_{23}\\
a_{31} & a_{32} & a_{33}
\end{array}\right |
\end{equation}
\begin{equation}
m_{11} = \left| \begin{array}{ccc}
a_{11} & a_{12} & a_{13}\\
a_{21} & a_{22} & a_{23}\\
a_{31} & a_{32} & a_{33}
\end{array} \right |
= a_{22}a_{33} - a_{23}a_{32}
\end{equation}
\begin{equation}
m_{12} = \left| \begin{array}{ccc}
a_{11} & a_{12} & a_{13}\\
a_{21} & a_{22} & a_{23}\\
a_{31} & a_{32} & a_{33}
\end{array} \right |
= a_{21}a_{33} - a_{23}a_{31}
\end{equation}

\begin{equation}
m_{13} = \left| \begin{array}{ccc}
a_{11} & a_{12} & a_{13}\\
a_{21} & a_{22} & a_{23}\\
a_{31} & a_{32} & a_{33}
\end{array} \right |
= a_{21}a_{32} - a_{22}a_{31}
\end{equation}
So
\begin{equation}
\begin{array}{c}
|\mathbf{A}| = a_{11}m_{11} - a_{12}m_{12} + a_{13}m_{13}	 \\
= a_{11}(a_{22}a_{33} - a_{23}a_{32}) - a_{12}(a_{21}a_{33} - a_{23}a_{31}) + a_{13}(a_{21}a_{32} - a_{22}a_{31}).
\end{array}
\end{equation}
For a $4 \times 4$ determinant, each $m_{1i}$ would be an entire expansion given above for the 
$3 \times 3$ 
determinant --- one quickly needs a computer.

\index{Matrix!singular}
\index{Singular matrix}
A \emph{singular matrix} is a square matrix whose determinant is zero. A determinant is zero if:
\begin{enumerate}
\item Any row or column is zero.
\item Any row or column is equal to a linear combination of other rows or columns.
\end{enumerate}
As an example of a singular matrix, consider
\begin{equation}
|{\mathbf A}| = \left | \begin{array}{ccc} 1 & 6 & 4 \\
2 & 1 & 0 \\
5 & -3 & -4 
\end{array} \right |,
\end{equation}
where row $1 = 3$(row 2) $-$ row 3.  Then, the determinant becomes
\begin{equation}
\begin{array}{ccl}
|\mathbf{A}| & = & a_{11}(a_{22}a_{33}-a_{23}a_{32})-a_{12}(a_{21}a_{33}-a_{23}a_{31})+a_{13}(a_{21}a_{32}-a_{22}a_{31})     \\
& = & 1[1(-4)-0(-3)]-6[2(-4)-0(5)]+4[2(-3)-1(5)]=-4+48-44=0.  \\
\end{array}
\end{equation}

\index{Matrix!degree of clustering}
\index{Matrix!rank}
The \emph{degree of clustering} symmetrically about the principal diagonal is another (of many) 
properties of a determinant. The more the clustering, the higher the value of the determinant.
The \emph{rank} of a matrix is the number of linearly 
independent vectors that it contains (either row or column vectors).  Consider
\begin{equation}
\mathbf{A} = \left [ \begin{array}{cccc}
1 & 4 & 0 & 2 \\1 & 0 & 1 & -1\\
-3 & -4 & -2 & 0
\end{array} \right] .
\end{equation}
Since row $3 = -$(row 1)$ - 2$ (row 2), or col $3 = $ col $1 - 1/4$ (col 2) and col 4 $= -$(col 1)$ + 3/4$ (col 2), 
the matrix $\mathbf{A}$ has rank 2 (i.e., it has only two linearly independent vectors, independent of whether 
viewed by rows or columns).

\index{Matrix!rank}
The \emph{rank} of a \emph{matrix product} must be less than or equal to the 
smallest rank of the matrices being multiplied:
\begin{equation}
\mathbf{A}_{\mbox{(rank 2)}}\cdot \mathbf{B}_{\mbox{(rank 1)}} = \mathbf{C}_{\mbox{(rank 1)}}.
\end{equation}	 
Therefore (and seen from another angle), if a matrix has rank $r$ then any matrix factor of it must have 
rank of at least $r$. Since the rank cannot be greater than the smallest of $m$ or $n$ in a $m \times n$ matrix, 
this definition also limits the size (order) of factor matrices. (That is, one cannot factor a matrix 
of rank 2, into two matrices of which either is of less than rank 2, so $m$ and $n$ of each factor must 
also be $\geq 2$).

\index{Matrix!determinant|)}

\section{Matrix Division (Matrix Inverse)}
\index{Matrix!division}
\index{Matrix!inverse}
%Was {Fig1_matrixalien}{Abducted by an alien circus company, Professor Wessel
%is forced to write Linear Algebra equations in center ring.}
\emph{Matrix division} can be thought of as multiplying by the \emph{inverse}.  Consider the scalar division
\begin{equation}
\frac{x}{b}=x\frac{1}{b}=xb^{-1},
\end{equation}	 
where we can write
\begin{equation}
bb^{-1}=1.
\end{equation}
Likewise, matrices can be effectively divided by multiplying by an inverse matrix. \emph{Nonsingular square 
matrices} may have an inverse symbolized as $\mathbf{A}^{-1}$ and satisfying $\mathbf{AA}^{-1} = \mathbf{A}^{-1}\mathbf{A} = \mathbf{I}$.
The calculation of a matrix inverse is usually done using elimination methods on the computer.
For a simple 2 x 2 matrix, its inverse is given by
\begin{equation}
\mathbf{A}^{-1} = \frac{1}{|\mathbf{A}|}   
\left [ \begin{array}{cc} a_{22} & -a_{12} \\
-a_{21} & a_{11} \\
\end{array}
\right ] .
\end{equation}
	 As an example, let
\begin{equation}
\mathbf{A}=
\left [\begin{array}{cc} 7 & 2\\
10 & 3\\
\end{array}
\right].
\end{equation}
We solve for the inverse as
\begin{equation}
\mathbf{A}^{-1}=\frac{1}{21-20}
\left [\begin{array}{cc}3 & -2\\
 -10 & 7\\
\end{array}
\right]=
\left [\begin{array}{cc}3 & -2\\
-10 & 7\\
\end{array}
\right],
\end{equation}
and as a check we note that
\begin{equation}
\mathbf{AA}^{-1}=
\left [\begin{array}{cc} 7 & 2\\ 10 & 3\\ 
\end{array} \right] \left [\begin{array}{cc}3 & -2\\ -10 & 7\\ \end{array} \right]=
\left [\begin{array}{ccr} 1 & 0\\ 0 & 1\\ \end{array} \right]=\mathbf{I}.
\end{equation}
Given the concept of a matrix inverse we may summarize a few useful matrix properties:
\begin{equation}
(\mathbf{A}^{-1})^{-1} = \mathbf{A},
\end{equation}
\begin{equation}
(\mathbf{A}^{-1})^T = (\mathbf{A}^T)^{-1} = \mathbf{A}^{-T},
\end{equation}
\begin{equation}
\mathbf{D} = \mathbf{ABC} \mbox{ then } \mathbf{D}^{-1} = \mathbf{C}^{-1} \mathbf{B}^{-1} \mathbf{A}^{-1}.
\end{equation}
	This ``reversal rule'' for inverse products may be useful for eliminating or minimizing the number 
of matrix inverses requiring calculation.

\section{Matrix Manipulation and Normal Scores}
\index{Normal scores|(}
We will look at a few examples of matrix manipulations. Consider the data matrix
\begin{equation}
\mathbf{A} = \left[ \begin{array}{c}
1 \ 2 \ 3 \\
4 \ 5 \ 6 \\7 \ 8 \ 9
\end{array} \right ]
\end{equation}
and unit row vector
\begin{equation}
\mathbf{j}^T_3 = [1 \ 1 \ 1].
\end{equation} 
To compute the mean of each column vector in $\mathbf{A}$ (here, each column has length $n = 3$), we note that
\begin{equation}
\mathbf{\bar{x}} _c = \frac{1}{3} \mathbf{j}^T_3 \mathbf{A}, \mbox{ and in general } \mathbf{\bar{x}} _c = \frac{1}{n} \mathbf{j}^T_n \mathbf{A}.
\end{equation}	
For our example, we find
\begin{equation}
\mathbf{\bar{x}}_c = \frac{1}{3} \left[ 1 \  1\  1 \right ] \cdot
\left[ \begin{array}{c}
1 \ 2 \ 3 \\ 4 \ 5 \ 6 \\ 7 \ 8 \ 9 \end{array} \right ] 
= \frac{1}{3} \left[ 12 \ 15 \ 18 \right] =  \left[ 4 \ 5 \ 6 \right].
\end{equation}
To compute the mean of each row vector in $\mathbf{A}$ (here, each row has length $m = 3$), let
\begin{equation}
\mathbf{\bar{x}}_r = \frac{1}{3} \mathbf{Aj}_3, \mbox{ and in general }  \mathbf{\bar{x}}_r = \frac{1}{m} \mathbf{Aj}_m .
\end{equation} 
Again, for our example, we find
\begin{equation}
\mathbf{\bar{x}}_r = \frac{1}{3}
\left [ \begin{array}{c}
1 \ 2 \ 3 \\
4 \ 5 \ 6 \\
7 \ 8 \ 9 
\end{array} \right ] \cdot 
\left [ \begin{array}{c}
1\\
1\\
1
\end{array} \right ] = \frac{1}{3}
\left[
\begin{array}{c}
6 \\
15\\
24 \end{array} \right ] 
=
\left[
\begin{array}{c}
2 \\
5\\
8 \end{array} \right ].
\end{equation}
Given these terms, how can we compute normal scores for a data table (or matrix)?  What we want in 
each cell are the elements
\begin{equation}
z_{ij} = \frac{a_{ij} - \bar{a}_j}{s_j}.
\end{equation} 
In matrix terminology we would first need to form the difference matrix, $\mathbf{D}$, given as
\begin{equation}
\mathbf{D = A} - \frac{1}{n}\mathbf{JA},
\end{equation}
where $\mathbf{J}$ is the $n \times n$ unit matrix (all entries equal 1).
Given the diagonal matrix $\mathbf{S}$ containing the standard deviation of each column defined as
\begin{equation}
\mathbf{S} = 
\left [ \begin{array}{cccc}
s_1 & 0 & \ldots & 0 \\
0 & s_2 & \ldots & 0 \\
\vdots & \vdots &  \ddots  & \vdots\\
0 & 0 & \ldots & s_m
\end{array} \right]
\end{equation}
we get the normal scores (here, $\mathbf{I}$ is of size $n \times n$) as
\begin{equation}
\mathbf{Z = DS}^{-1} = \left ( \mathbf{A} - \frac{1}{n} \mathbf{JA} \right) \mathbf{S}^{-1} = \left ( \mathbf{I} - \frac{1}{n} \mathbf{J} \right)
\mathbf{AS}^{-1}.
\end{equation}
\index{Normal scores|)}

\section{Solution of Simultaneous Linear Equations}
\index{Solution of simultaneous linear equations|(}

A system of four simultaneous linear equations in four unknowns $x_1, x_2, x_3, x_4$ can be written
\begin{equation}
\begin{array}{c}
a_{11}x_1+a_{12}x_2+a_{13}x_3+a_{14}x_4=b_1\\
a_{21}x_1+a_{22}x_2+a_{23}x_3+a_{24}x_4=b_2\\
a_{31}x_1+a_{32}x_2+a_{33}x_3+a_{34}x_4=b_3\\
a_{41}x_1+a_{42}x_2+a_{43}x_3+a_{44}x_4=b_4\\
\end{array}
\end{equation}
or, in matrix form, 
\begin{equation}
\mathbf {Ax=b},
\end{equation}	 
where
\begin{equation}
\mathbf{A}=
\left[\begin{array}{cccc}
a_{11} & a_{12} & a_{13} & a_{14}\\	 
a_{21} & a_{22} & a_{23} & a_{24}\\
a_{31} & a_{32} & a_{33} & a_{34}\\
a_{41} & a_{42} & a_{43} & a_{44}\\	 
\end{array}
\right]
\end{equation}
is called the coefficient matrix,
\begin{equation}
\mathbf{x}= \left[\begin{array}{c}x_1\\x_2\\x_3\\x_4\\
\end{array}\right]
\end{equation}
is the unknown vector, and
\begin{equation}
\mathbf{b}=\left[\begin{array}{c}b_1\\b_2\\b_3\\b_4\\
\end{array}
\right]
\end{equation}
is the right hand side (i.e., the observations).  Premultiplying both sides by $\mathbf{A}^{-1}$ yields
\begin{equation}
\mathbf{A}^{-1}\mathbf{Ax}=\mathbf{A}^{-1}\mathbf{b},
\end{equation}	 
hence
\begin{equation}
\mathbf{Ix=x=A}^{-1}\mathbf{b}
\end{equation}	 
gives the solution for values of $x_1, x_2, x_3, x_4$ which solve the system. For simplicity, the following example 
solves for two simultaneous equations only. Consider two equations in two unknowns (e.g., equations of lines 
in the $x_1-x_2$ plane):
\begin{equation}
\begin{array}{c}
5x_1 + 7x_{2} = 19\\
3x_1 - 2 x_2 = -1
\end{array} .
\end{equation}
In matrix form this system translates to
\begin{equation}
\left[ \begin{array}{cc}
5 & 7\\
3 & -2 
\end{array}
\right ] 
\left[
\begin{array}{c}
x_1\\
x_2
\end{array}
\right ] =
\left[ \begin{array}{c}
19\\
-1
\end{array}
\right ]
\end{equation}
or
\begin{equation}
\mathbf{A \cdot x = b}.
\end{equation}
To solve this matrix equation we need the inverse of $\mathbf{A}$, which is simply
\begin{equation}
\mathbf{A}^{-1} = \frac{1}{-10 - 21} \quad
\left[ \begin{array}{cc}
-2 & -7\\
-3 & 5
\end{array}\right ] = \left[ \begin{array}{cc}
\frac{2}{31} & \frac{7}{31}\\[4pt]
\frac{3}{31} & \frac{-5}{31}
\end{array} \right ] .
\end{equation}	 
Then, $\mathbf{x = A}^{-1}\cdot \mathbf{b}$, where
\begin{equation}
\mathbf{x = A}^{-1}\mathbf{b} = \left[ \begin{array}{cc}
\frac{2}{31} & \frac{7}{31}\\[4pt]
\frac{3}{31} & \frac{-5}{31}
\end{array}
\right] 
\left[ \begin{array}{c}
19\\
-1
\end{array} \right ] = \left [ \begin{array}{c}
\frac{38}{31} - \frac{7}{31} \\[4pt]
\frac{57}{31} + \frac{5}{31}
\end{array} \right ]
= \left[ \begin{array}{c}
1 \\
2 \end{array} \right ] .
\end{equation}
So, the values $x_1 = 1$ and $x_2 = 2$ solve the above system, or
\begin{equation}
\mathbf{x} = \left[ \begin{array}{c}
x_1\\ x_2 \end{array} \right]
=
\left[ \begin{array}{c}
1\\ 2 \end{array} 
\right ] .
\end{equation}	 

While this approach may seem burdensome, it is good because it is extremely general and 
allows for a straightforward solution to very large systems. However, it is 
true that direct (elimination) methods to the solution are in fact quicker for fully populated 
matrices:

\begin{enumerate}
\item 	A solution using the inverse matrix approach involves $n^3$ multiplications for the inversion and $n^2m$ 
more multiplications to finish the solution, where $n$ is the number of equations per set, and $m$ 
is the number of sets of equations (each of the same form but different $\mathbf{b}$ vector). The total 
number of multiplications is $n^3 + n^2m$.
\item	A solution by directly solving the linear equations involves $n^3/3 + n^2m$ multiplications.
\end{enumerate}

Hence, while the matrix form is easy to handle, one should not necessarily always use it blindly. We 
will consider many situations for which matrix solutions are ideal. For sparse or symmetrical 
matrices, the above relationships may not hold.
\index{Solution of simultaneous linear equations|)}

\subsection{Simple regression and curve fitting}
\index{Simple regression|(}
\index{Regression!simple|(}
\index{Curve fitting|(}

\PSfig[h]{Fig1_L2_error}{Graphical representation of the regression errors used in least-squares procedures.
We measure misfit vertically in the $y$-direction from data point to regression curve.}
	Whereas an interpolant fits each data point exactly, it is frequently advantageous to produce a 
smoothed fit to the data --- not exactly fitting each point, but producing a ``best'' fit. A popular (and 
convenient) method for producing such fits is known as the \emph{method of least squares}.
\index{Method of least squares}
\index{Least squares method}

	The method of least squares produces a fit of a specified (usually continuous) basis to a set of 
data points which minimizes the sum of the squared misfit (error) between the fitted curve 
and the data. The misfit can be measured vertically, as in Figure~\ref{fig:Fig1_L2_error}.
\index{Regression}
This \emph{regression} of $y$ on $x$ is the most commonly used method. Less common methods (i.e., more work 
involved) is the regression of $x$ on $y$ and even orthogonal regression (which we will return to later;
see Figure~\ref{fig:Fig1_y_and_ortho_error}).

\PSfig[h]{Fig1_y_and_ortho_error}{Two other regression methods: regressing $x$ on $y$ and orthogonal regression.
Here we measure misfits horizontally from data point to regression line or orthogonally onto the regression line,
respectively.}

	Consider fitting a single ``best'' linear curve to $n$ data points. This can be a scatter plot of $x(t)$, $d(t)$ 
plotted at similar values of $t$, or a simple $d = f(x)$ relationship. At any rate, $d$ (our data) are considered a 
function of $x$ (which may be a spatial coordinate or time). We wish to fit a line of the form
\begin{equation}
d(x) = m_1 + m_2 (x-x_0)
\end{equation}
and must therefore determine values for the model coefficients $m_1$ and $m_2$ that produce a line that minimizes the sum 
of the squared misfits (here, $x_0$ is a constant specified beforehand). In other words,
\begin{equation}
\mbox{minimize } \sum ^n _{i=1} \left [(d_{\mbox{computed}}(x_i) - d_{\mbox{observed}}(x_i) \right ]^2.
\end{equation}	 
Ideally, for each observation $d_i$ at location $x_i$ we should have
\begin{equation}
\begin{array}{c}
m_1 + m_2(x_1 - x_0) = d_1\\
m_1 + m_2(x_2 - x_0) = d_2\\
m_1 + m_2(x_3 - x_0) = d_3\\
\vdots\\
m_1 + m_2 (x_n - x_0) = d_n
\end{array}
\end{equation}
There are many more equations ($n$ --- one for each observed value of $d$) than unknowns (2 --- $m_1$ and 
$m_2$). Such a system is \emph{overdetermined} and there exists no unique solution (unless all the $d_i$'s do 
lie exactly on a single line, in which case any two equations will uniquely determine $m_1$ and $m_2$).  
In matrix form,
\index{Overdetermined system of equations}
\begin{equation}
\left[
\begin{array}{cc}
1 & (x_1 - x_0) \\
1 & (x_2 - x_0) \\
\vdots & \vdots \\
1 & (x_n - x_0) 
\end{array} \right]
\ \left [ \begin{array}{c}
m_1\\
m_2
\end{array} \right ] =
\left [ \begin{array}{c}
d_1\\
d_2\\
\vdots\\
d_n
\end{array} \right ],
\end{equation}
i.e., $\mathbf{G \cdot m = d}$.  Here, $\mathbf{G}$ represents how predictions of $\mathbf{d}$ are related to the model $\mathbf{m}$ and is often
called the \emph{design matrix}.
However, since $\mathbf{G}$ is a not square it has no inverse, hence the equation cannot be inverted and solved as is. 
Consider instead the \emph{misfit}, $e_i$, at each point, between prediction and observation:
\begin{equation}
\begin{array}{c}
m_1 + m_2 (x_1 - x_0) - d_1 = e_1\\
m_1 + m_2 (x_2 - x_0) - d_2 = e_2\\
\vdots\\
m_1 + m_2(x_n - x_0) - d_n = e_n
\end{array}
\end{equation}
We wish to determine the values for $m_1$ and $m_2$ that minimize
\begin{equation}
	\index{Misfit function}
E(m_1, m_2) = \sum^n_{i=1} e^2_i = {\mathbf e}^T{\mathbf e},
\end{equation}
where $\mathbf{e}^T = (e_1, e_2, ..., e_n)$ is the \emph{misfit vector}.       
This condition will minimize the variance of the residuals about the regression line and give the desired least-squares fit.
Thus, $E(m_1 ,m_2)$ and the minimum of this function (with respect to the two unknown coefficients) 
can be determined using simple differential calculus where, at the desired minimum, we require
\PSfig[h]{Fig1_3D_misfit}{(left) The solution we seek minimizes the misfit function $E(\mathbf{m}) = E(m_1, m_2)$,
which portrays a surface in 3-D.  Because of the functional (quadratic) form of $E$ we are guaranteed a unique global
minimum. (right) Two orthogonal cross-sections of $E$ along the axes $m_1$ and $m_2$.  We seek the solutions for these
two parameters so that the respective slopes in $E$ are zero simultaneously.}
\begin{equation}
\frac{\partial E(m_1,m_2)}{\partial m_1} = \frac{\partial E (m_1, m_2)}{\partial m_2} = 0.
\end{equation}
Thus, the \emph{slopes} of the misfit function with respect to each parameter must be zero (see Figure~\ref{fig:Fig1_3D_misfit}).  We find
\begin{equation}
\begin{array}{rcl}
\displaystyle
\frac{\partial E}{\partial m_1} & = & \displaystyle \frac{\partial}{\partial m_1} \left ( \sum^n_{i=1} \ e^2_i \right ) = \frac{\partial}{\partial m_1} \left \{ \sum^n_{i=1} \left [ m_1 + m_2 (x_i - x_0) - d_i \right ] ^2 \right \}\\*[4ex]
 &  = & \displaystyle 2 \sum^n_{i=1} \left [ m_1 + m_2 (x_i - x_0) - d_i \right ] = 0
\end{array}
\end{equation}
\begin{equation}
\begin{array}{rcl}
\displaystyle
\frac{\partial E}{\partial m_2} & = & \displaystyle \frac{\partial}{\partial m_2} \left ( \sum^n_{i=1} \ e^2_i \right ) = \frac{\partial}{\partial m_2} \left \{ \sum^n_{i=1} \left [ m_1 + m_2 (x_i - x_0) - d_i \right ] ^2 \right \}\\*[4ex]
 & = & \displaystyle 2 \sum^n_{i=1} \left [ m_1 + m_2 (x_i - x_0) - d_i \right ](x_i - x_0) = 0.
\end{array}
\end{equation}
These two equations can now be expanded into their individual terms, forming what are known as 
the \emph{normal equations}.  This system of two equations with two unknowns can be uniquely 
solved. Rearranging, we find
\index{Normal equations}
\begin{equation}
nm_1 + m_2 \sum^n_{i=1} (x_i - x_0) = \sum^n_{i=1} d _i,
\label{eq:normeq1}
\end{equation}
\begin{equation}
m_1 \sum^n_{i=1} (x_i - x_0) + m_2 \sum^n_{i=1} (x_i - x_0)^2 = \sum^n_{i=1} d_i (x_i - x_0).
\label{eq:normeq2}
\end{equation}
Notice that all sums involve known values that add to simple constants. 
Specifically, we must compute the sums
\begin{equation}
S_y = \sum^n_{i=1} d_i, \ S_{xy} = \sum^n_{i=1}d_i (x_i - x_0), S_x = \sum^n_{i=1}(x_i - x_0), \mbox{ and} \ 
S_{xx} = \sum^n_{i=1}(x_i - x_0)^2.
\end{equation}
Substituting these symbols into (\ref{eq:normeq1}) and (\ref{eq:normeq2}), we obtain
\begin{equation}
nm_1 + m_2 S_x = S_y
\end{equation}
\begin{equation}
m_1 S_x + m_2 S_{xx} = S_{xy}
\label{eq:slope_equation}
\end{equation}
Solving for the intercept yields
\begin{equation}
m_1 = \frac{1}{n} S_y - \frac{m_2}{n} S_x.
\label{eq:intercept_equation}
\end{equation}
We substitute $m_1$ into (\ref{eq:slope_equation}) and find
\begin{equation}
\left [ \frac{1}{n} S_y - \frac{m_2}{n} S_x \right ] S_x + m_2 S_{xx} = S_{xy}.
\end{equation}
Now solve for $m_2$:
\begin{equation}
\frac{1}{n} S_y S_x - \frac{m_2}{n} S_x^2 + m_2 S_{xx} = S_{xy},
\end{equation}
\begin{equation}
m_2 \left ( S_{xx} - \frac{1}{n} S_x^2 \right ) = S_{xy} - \frac{1}{n} S_y S_x.
\end{equation}	 
Finally,
\begin{equation}
m_2 = \left ( S_{xy} - \frac{1}{n} S_y  S_x \right ) / \left ( S_{xx} - \frac{1}{n} S_x^2 \right )  =\frac{n S_{xy} - S_x  S_y}{n S_{xx} - S_x^2},
\label{eq:slope_solution}
\end{equation}	 
and we substitute $m_2$ into (\ref{eq:intercept_equation}) to find
\begin{equation}
m_1 = \frac{S_{xx} S_y - S_x  S_{xy}}{n S_{xx} - S_x^2}.
\label{eq:intercept_solution}
\end{equation}	 
In matrix form the normal equations are
\begin{equation}
\left [
\begin{array}{cc}
n & \displaystyle \sum^n_{i=1}(x_i - x_0)\\
\displaystyle \sum^n_{i=1}(x_i - x_0) & \displaystyle \sum^n_{i=1}(x_i - x_0)^2
\end{array} \right ]
\left [
\begin{array}{c}
m_1\\ m_2
\end{array} \right ] =
\left[
\begin{array}{c}
\displaystyle \sum^n_{i=1}d_i\\
\displaystyle \sum^n_{i=1}d_i(x_i - x_0)
\end{array}
\right ],
\end{equation}
which may be simplified to
\begin{equation}
\left [
\begin{array}{cc}
n & S_x \\
S_x & S_{xx}
\end{array} \right ]
\left [
\begin{array}{c}
m_1\\ m_2
\end{array} \right ] =
\left[
\begin{array}{c}
S_y\\
S_{xy}
\end{array}
\right ] .
\end{equation}
Therefore, $\mathbf{Nm = v}$, and since $\mathbf{N}$ is square, symmetric and of full rank, this equation is solved in 
the standard manner:
\begin{equation}
\mathbf{N}^{-1} \mathbf{Nm = m = N}^{-1} \mathbf{v}.
\end{equation}	 
This problem was simple enough $(2 \times 2)$ to solve for $m_1$ and $m_2$ by brute force. For larger systems,
this approach becomes impractical and instead a matrix solution to the rectangular $\mathbf{G\cdot m = d}$ equation must be 
sought. We will next look at the general linear least-squares problem and find a solution in 
matrix notation.
\index{Simple regression|)}
\index{Regression!simple|)}
\index{Curve fitting|)}

\subsection{General linear least squares method, version 1}
\index{General linear least squares method|(}

	We have looked at a few special cases where we have sought to fit a model to data in a 
least-squares sense.  Fitting a straight line to the $x-d$ points was a very simple example of this 
technique. We will now look at the more general problem of finding the coefficients for \emph{any} 
linear combination of a chosen set of basis functions that fits a data set in a least squares sense. There are 
numerous situations where this is needed; some are listed in Table~\ref{tbl:LLS_examples}.

\begin{table}[h]
\centering
\begin{tabular}{|l|p{1.5in}|l|}
\hline
\bf{Situation}  & \multicolumn{1}{c|}{\bf{Model Parameters}} & \multicolumn{1}{c|}{\bf{Data}} \\ \hline
Curve Fitting & 
Coefficients of polynomials, Fourier series, etc. &
Points in $x-y$ plane \\ \hline
Gravity modeling &
Densities of subsurface polygons & 
Gravity observations\\ \hline
Hypocenter location & 
Small perturbations to hypocenter location & Seismic arrival times \\ \hline
\end{tabular}
\caption{Examples of situations where linear least squares solutions are used.}
\label{tbl:LLS_examples}
\end{table}

While the basis functions in these cases are all vastly different, they are all used in linear 
combinations to fit the observed data. We will therefore take time to investigate how such a 
problem is set up, and how the setup can be simplified with matrix algebra.
Some typical basis functions are given in Table~\ref{tbl:basis_funcs}.

\begin{table}[h]
\centering
\begin{tabular}{|c|c|}
\hline
\bf{Polynomial basis} & \bf{Fourier sine basis}\\ \hline
$g_1=x^0$  & $g_1=\sin(2\pi x/T)$\\ \hline
$g_2=x^1$  & $g_2=\sin(4\pi x/T)$\\ \hline
$g_3=x^2$  & $g_3=\sin(6\pi x/T)$\\ \hline
$\vdots$ & $\vdots$\\ \hline
$g_k=x^{k-1}$  & $g_k=\sin(2k\pi x/T)$ \\ \hline
\end{tabular}
\caption{Examples of basis functions used for modeling of data.}
\label{tbl:basis_funcs}
\end{table}

Consider the least squares fitting of any continuous basis of the form
\begin{equation}
g_1(x), g_2(x), g_3(x), \cdots , g_k(x).
\end{equation}
For example, we desire to fit a model with $k$ terms
\begin{equation}
d(x) = m_1g_1(x) + m_2g_2(x) + \cdots + m_k g_k(x)
\label{eq:lsgenmodel}
\end{equation}
to a data set of $n$  data points, where $n > k$, by minimizing $E(\mathbf{m})$ given by
\begin{equation}
E(\mathbf{m}) = E(m_1, m_2, \cdots, m_k) = \sum ^n _{i=1}  (e_i)^2  = \sum ^n _{i=1} (m_1 g_1(x_i) + m_2g_2(x_i) + \cdots + m_k g_k(x_i) - d_i)^2,
\end{equation}
or simply
\begin{equation}
E(\mathbf{m}) = \sum ^n _{i=1} (m_1g_{i1} + m_2g_{i2} + \cdots + m_k g_{ik} - d_i)^2,
\label{eq:L2_misfit}
\end{equation}
where $d_i$ is the observed value and $g_{ij}$ is the $j'$th basis function, evaluated at the location (or time) $x_i$. In other words,
$g_{ij} = g_j(x_i)$.

There are \emph{four} concepts of vital importance in a general linear least squares modeling problem:
\begin{enumerate}
	\item The \emph{observed data}, $(x_i, d_i), i = 1, n$, where $n$ is the number of observations.  These are all known quantities.
	\item The \emph{general linear model} (linear in $\mathbf{m} = m_j, j = 1,k$, with $k$ unknown parameters), given by (\ref{eq:lsgenmodel}).
	\item The $m$ chosen \emph{basis functions}, $g_j(x), j = 1, k$.  We can evaluate these for any $x$.
	\item The \emph{least squares misfit criteria}, given by (\ref{eq:L2_misfit}).
\end{enumerate}
We can write a linear system of equations for the misfit at each data point:
\begin{equation}
\begin{array}{c}
m_1 g_{11} + m_2 g_{12} + \cdots + m_k g_{1k} - d_1 = e_1\\
m_1 g_{21} + m_2 g_{22} + \cdots + m_k g_{2k} - d_2 = e_2\\
\vdots\\
m_1 g_{n1} + m_2 g_{n2} + \cdots + m_k g_{nk} - d_n = e_n\\
\end{array}.
\end{equation}     
To minimize $E$, we require
\begin{equation}
\displaystyle
\frac{\partial E(\mathbf{m})}{\partial m_{j}} = 0, j = 1,k.
\label{eq:L2_criteria}
\end{equation}
Considering the first term (case $j = 1$), we see
\begin{equation}
\begin{array}{rcl}
\displaystyle \frac{\partial E(\mathbf{m})}{\partial m_1} & = & \displaystyle \frac{\partial}{\partial m_1}\sum ^n _{i=1}(m_1g_{i1} + m_2g_{i2} + \cdots m_k g_{ik} -d_i)^2 \\
 & = & \displaystyle 2 \sum ^n _{i=1}(m_1g_{i1} + m_2g_{i2} + \cdots m_k g_{ik} -d_i)g_{i1}= 0, \\
\end{array}
\end{equation}
while for the second term (case $j = 2$), we find
\begin{equation}
\begin{array}{rcl} 
\displaystyle \frac{\partial E(\mathbf{m})}{\partial m_2} & = & \displaystyle \frac{\partial}{\partial m_2}\sum ^n _{i=1}(m_1g_{i1} + m_2g_{i2} + \cdots m_k g_{ik} -d_i)^2 \\
& = & 2 \displaystyle \sum ^n _{i=1}(m_1g_{i1} + m_2g_{i2} + \cdots m_k g_{ik} -d_i)g_{i2}= 0. \\
\end{array}
\end{equation}
Consequently, for the $j$'th parameter,
\begin{equation}
\frac{\partial E(\mathbf{m})}{\partial m_j} = 2 \displaystyle \sum ^n _{i=1}(m_1g_{i1} + m_2g_{i2} + \cdots m_k g_{ik} -d_i)g_{ij}= 0.
\end{equation}
Rearranging these normal equations gives the square $k \times k$ system
\begin{equation}
\begin{array}{c}
m_1 \displaystyle \sum ^n _{i=1} g^2_{i1} + m_2 \displaystyle \sum ^n _{i=1} g_{i2} g_{i1} + \cdots + m_k 
 \displaystyle \sum ^n _{i=1} g_{ik}g_{i1} =  \displaystyle \sum^n _{i=1} d_i g_{i1}\\
m_1 \displaystyle \sum ^n _{i=1} g_{i1}g_{i2} + m_2 \displaystyle \sum ^n _{i=1} g^2_{i2} + \cdots + m_k  \displaystyle \sum ^n _{i=1} g_{ik}g_{i2} =  \displaystyle \sum^n _{i=1} d_i g_{i2}\\
\vdots \\
m_1 \displaystyle \sum ^n _{i=1} g_{i1}g_{ik} + m_2 \displaystyle \sum ^n _{i=1} g_{i2} g_{ik} + \cdots + m_k 
 \displaystyle \sum ^n _{i=1} g^2_{ik} =  \displaystyle \sum^n _{i=1} d_i g_{ik}
\end{array}
\end{equation}
or equivalently,
\begin{equation}
m_1 \displaystyle \sum ^n _{i=1} g_{i1}g_{ij} + m_2 \displaystyle \sum ^n _{i=1} g_{i2} g_{ij} + \cdots + m_k 
 \displaystyle \sum ^n _{i=1} g_{ik}g_{ij} =  \displaystyle \sum^n _{i=1} d_i g_{ij}, j=1,k.
\end{equation}
This setup provides a \emph{closed system} of $k$ normal equations.  In matrix form,
\begin{equation}
\left [  \begin{array}{cccc}
\displaystyle \sum ^n _{i=1} g^2_{i1} & \displaystyle \sum ^n _{i=1} g_{i2}g_{i1} & \cdots &
\displaystyle \sum ^n _{i=1} g_{ik}g_{i1} \\
\displaystyle \sum ^n _{i=1} g_{i1}g_{i2} &  \displaystyle \sum ^n _{i=1} g^2_{i2} & \cdots &
\displaystyle \sum^n _{i=1} g_{ik}g_{i2} \\
\vdots & \vdots & \ddots & \vdots\\
\displaystyle \sum ^n _{i=1} g_{i1}g_{ik} &  \displaystyle \sum ^n _{i=1} g_{i2}g_{ik} & \cdots & \displaystyle \sum ^n _{i=1} g^2_{ik} 
\end{array}   \right ]	\left[ \begin{array}{c} m_1 \\ m_2 \\ \vdots \\ m_k     \end{array}  \right ]  = \left[  \begin{array}{c}\displaystyle \sum ^n _{i=1} d_{i}g_{i1}\\
\displaystyle \sum ^n _{i=1} d_{i}g_{i2}\\
\vdots \\
\displaystyle \sum ^n _{i=1} d_{i}g_{ik}\\
   \end{array} \right].
\label{eq:normalwsums}
\end{equation}
Hence, we simply have
\begin{equation}
\mathbf{N \cdot m = v},
\label{eq:Nxvsolution}
\end{equation}
where $\mathbf{N}$ is the (known) coefficient matrix, $\mathbf{m}$ the vector with the unknowns $m_j$, and $\mathbf{v}$ contains 
weighted sums of known (observed or computable) quantities. Solving for the $\mathbf{m}$ vector (since $\mathbf{N}$ is square, symmetric and 
of full rank) yields
\begin{equation}
\mathbf{N}^{-1} \cdot \mathbf{N \cdot m} = \mathbf{m} = \mathbf{N}^{-1} \cdot \mathbf{v}.
\end{equation}	 
The resulting $m_j$ values are the ones which satisfy (\ref{eq:L2_criteria}) and 
therefore the same ones, when combined with the chosen basis, that produce the ``best'' fit to the 
$n$ data points such that (\ref{eq:L2_misfit}) is minimized.

\subsection{General linear least squares method, version 2}

We will now look at a simpler approach to the same problem using matrix algebra. We have $\mathbf{e = G \cdot m - d}$, or
\begin{equation}
\left [ \begin{array}{c} e_1\\ e_2 \\ \vdots \\ e_n  \end{array} \right ] =
\left [
\begin{array}{cccc}
g_{11} & g_{12} & \cdots & g_{1k} \\	 
g_{21} & g_{22} & \cdots & g_{2k} \\
\vdots & \vdots & \ddots & \vdots\\
g_{n1} & g_{n2} & \cdots & g_{nk} \\
\end{array} \right ]
\cdot
\left [ \begin{array}{c} 
m_1\\ 
m_2\\
 \vdots\\ 
m_k   \end{array}   \right]
-
\left [ \begin{array}{c} d_1\\ d_2 \\ \vdots \\ d_n  \end{array} \right ].
\end{equation}
We note that each column vector of $\mathbf{G}$ is simply a single basis function evaluated at all our
observation points.  In fact, we could write $\mathbf{G}$ as
\begin{equation}
\mathbf{G} = \left [
\begin{array}{cccc}
\mathbf{g}_1 & \mathbf{g}_2 & \cdots & \mathbf{g}_k \\	 
\end{array} \right ],
\label{eq:Aasvectors}
\end{equation}
where
\begin{equation}
\mathbf{g}_j = \left [
\begin{array}{cccc}
g_j(x_1) & g_j(x_2) & \cdots & g_j(x_n) \\	 
\end{array} \right ]^T.
\end{equation}
We wish to find the $m_j$ values that minimize $E = \mathbf{e}^T\mathbf{e}$.
Minimizing the misfit with respect to the unknown model parameters $\mathbf{m}$ means we must solve
the $k$ linear equations that result from setting all partial derivatives of $E$ to zero (i.e., \ref{eq:L2_criteria}).
Using matrix algebra, we express the \emph{predicted} solution as $\hat{\mathbf{d}} = \mathbf{G \cdot m}$.
We may now express the misfit
between model and observations as $\mathbf{e} = \hat{\mathbf{d}} - \mathbf{d} = \mathbf{G \cdot m} - \mathbf{d}$
and use this expression to evaluate the misfit as
\begin{equation}
E(\mathbf{m}) = \sum ^n _{i=1}  (e_i)^2  = \mathbf{e}^T \cdot \mathbf{e} = \left(\hat{\mathbf{d}} - \mathbf{d}\right)^T\cdot \left(\hat{\mathbf{d}} - \mathbf{d}\right) = \left(\mathbf{G \cdot m} - \mathbf{d}\right)^T \cdot \left(\mathbf{G \cdot m} - \mathbf{d}\right).
\end{equation}
Expanding terms, we find
\begin{equation}
E(\mathbf{m}) =  \left(\mathbf{m}^T\mathbf{G}^T - \mathbf{d}^T\right) \cdot \left(\mathbf{G \cdot m} - \mathbf{d}\right) = \
\mathbf{m}^T\mathbf{G}^T\mathbf{Gm} - \mathbf{m}^T\mathbf{G}^T\mathbf{d} - \mathbf{d}^T\mathbf{Gm} + \mathbf{d}^T\mathbf{d},
\end{equation}
where we have used (\ref{eq:transposerule}) to handle the transpose of a matrix product.
Note that since $E$ is a scalar then each of these terms must evaluate to scalars as well.  To find the solution, we set
\begin{equation}
\frac{\partial E(\mathbf{m})}{\partial m_j} = \mathbf{\dot{m}}^T\mathbf{G}^T\mathbf{Gm} + \mathbf{m}^T\mathbf{G}^T\mathbf{G\dot{m}} - \mathbf{\dot{m}}^T\mathbf{G}^T\mathbf{d} - \mathbf{d}^T\mathbf{G\dot{m}} = 0, j = 1,m,
\end{equation}
where the ``dot'' over a vector represents the derivative of that vector with respect to $m_j$.
We note the first and second terms are transposes of each other, as are the third and fourth terms.
However, since they all evaluate to scalars the two transposes must be identical and hence this
repetition simply constitutes a factor of two, which we delete by retaining only the first and third term:
\begin{equation}
\frac{\partial E(\mathbf{m})}{\partial m_j} = \mathbf{\dot{m}}^T\mathbf{G}^T\mathbf{Gm} - \mathbf{\dot{m}}^T\mathbf{G}^T\mathbf{d} = 0, j = 1,k.
\end{equation}
What is the mysterious ``dot''-derivative, written as
\begin{equation}
\mathbf{\dot{m}}^T = \frac{\partial}{\partial m_j} \left (\mathbf{m}^T \right), j = 1,k?
\end{equation}
We illuminate this term by trying some values of $j$, remembering $\mathbf{m}^T = [m_1\ m_2 \ \cdots \ m_k ]$:
\[\begin{array}{*{20}{c}}
{{\rm{Case }}j = 1:{\frac{\partial}{\partial m_1}\mathbf{m}} = {{\left[ {\begin{array}{*{20}{c}}
1&0& \cdots &0
\end{array}} \right]}^T}}\\[4pt]
{{\rm{Case }}j = 2:{\frac{\partial}{\partial m_2}\mathbf{m}} = {{\left[ {\begin{array}{*{20}{c}}
0&1& \cdots &0
\end{array}} \right]}^T}}\\[4pt]
 \vdots \\[4pt]
{{\rm{Case }}j = k:{\frac{\partial}{\partial m_k}\mathbf{m}} = {{\left[ {\begin{array}{*{20}{c}}
0&0& \cdots &1
\end{array}} \right]}^T}}
\end{array}\]
Thus, the $k$ linear equations can be combined into a single matrix equation, noting that all these derivatives (each producing a row vector)
combine to form the identity matrix, $\mathbf{I}$:
\begin{equation}
\frac{\partial }{{\partial {m_j}}}\left( {{{{\mathbf{m}}}^T}} \right), j = 1,k \to \left[ {\begin{array}{*{20}{c}}
1&0& \cdots &0\\
0&1& \cdots &0\\
 \vdots & \vdots & \ddots & \vdots \\
0&0& \cdots &1
\end{array}} \right] = {\mathbf{I}}.
\end{equation}
Hence, we may write
\begin{equation}
\mathbf{I}\mathbf{G}^T\mathbf{Gm} - \mathbf{I}\mathbf{G}^T\mathbf{d} = \mathbf{0},
\end{equation}
or by rearranging,
\begin{equation}
\mathbf{G}^T\mathbf{Gm} = \mathbf{G}^T\mathbf{d}.
\end{equation}
Because the $\mathbf{G}^T\mathbf{G}$ matrix is square and symmetric and thus can be inverted, we simply multiply by its inverse
and obtain the general least squares solution as
\begin{equation}
\mathbf{m} = \left [\mathbf{G}^T\mathbf{G}\right ]^{-1}\mathbf{G}^T\mathbf{d}.
\label{eq:lsgensolution}
\end{equation}
Comparing (\ref{eq:lsgensolution}) with (\ref{eq:Nxvsolution}) we see clearly that $\mathbf{N} = \mathbf{G}^T\mathbf{G}$
and $\mathbf{v} = \mathbf{G}^T\mathbf{d}$.  Furthermore, given (\ref{eq:Aasvectors}) we may write
$\mathbf{G}^T\mathbf{G}$ using the product
\begin{equation}
\mathbf{N} = \mathbf{G}^T\mathbf{G} = \left [
	\begin{array}{c}
	\mathbf{g}_1^T \\[6pt]
	\mathbf{g}_2^T \\[6pt]
	\vdots \\[6pt]
	\mathbf{g}_k^T \\	 
	\end{array} \right ] \cdot
	\left [
		\begin{array}{cccc}
		\mathbf{g}_1 & \mathbf{g}_2 & \cdots & \mathbf{g}_k \\	 
		\end{array}
	\right ] =
	 \left [
	\begin{array}{cccc}
	\mathbf{g}_1^T\mathbf{g}_1 & \mathbf{g}_1^T\mathbf{g}_2 & \cdots & \mathbf{g}_1^T\mathbf{g}_k \\[6pt]	 
	\mathbf{g}_2^T\mathbf{g}_1 & \mathbf{g}_2^T\mathbf{g}_2 & \cdots & \mathbf{g}_2^T\mathbf{g}_k \\[6pt]	 
	\vdots & \vdots & \ddots & \vdots \\[6pt]
	\mathbf{g}_k^T\mathbf{g}_1 & \mathbf{g}_k^T\mathbf{g}_2 & \cdots & \mathbf{g}_k^T\mathbf{g}_k \\	 
	\end{array} \right ],
	\label{eq:gdotg}
\end{equation}
which makes it clear that each element of $\mathbf{N}$, such as $n_{pq}$, is the dot product between two basis vectors $\mathbf{g}_p^T$ and $\mathbf{g}_q$, and
\begin{equation}
\mathbf{v} = \mathbf{G}^T\mathbf{d} = \left [
	\begin{array}{c}
	\mathbf{g}_1^T \\[6pt]
	\mathbf{g}_2^T \\[6pt]
	\vdots \\[6pt]
	\mathbf{g}_k^T \\	 
	\end{array} \right ] \cdot \mathbf{d} =
	 \left [
	\begin{array}{c}
	\mathbf{g}_1^T\mathbf{d} \\[6pt]	 
	\mathbf{g}_2^T\mathbf{d}  \\[6pt]	 
	\vdots \\[6pt]
	\mathbf{g}_k^T\mathbf{d} \\	 
	\end{array} \right ],
	\label{eq:gdotd}
\end{equation}
which shows each element of $\mathbf{v}$ is the dot product between each basis function $\mathbf{g}_j^T$ and the data vector $\mathbf{d}$.
This is simply what we found the hard way earlier (i.e., \ref{eq:normalwsums}).
Thus, to solve a general linear least squares problem, all we have to do is to evaluate $\mathbf{G}$ via (\ref{eq:Aasvectors}) and
the rest is taken care of by (\ref{eq:lsgensolution}).
\begin{example}
We are given a data set with four data pairs (2,1), (4,4), (6,3) and (8,4) ($n = 4$) and asked to determine the
coefficients for a \emph{quadratic} curve that best describes the data.  Except for special situations, we know that the
three-parameter curve will not pass through all the four points, so we decide to seek a least squares solution.

We write down the functional form for our quadratic curve as $d = m_1 + m_2 x + m_3 x^2$ and use it to
form the matrix equation
\begin{equation} \begin{array}{c}
m_1 + m_2 x_1 + m_3 x_1^2 = d_1\\[6pt]
m_1 + m_2 x_2 + m_3 x_2^2 = d_2\\[6pt]
m_1 + m_2 x_3 + m_3 x_3^2 = d_3\\[6pt]
m_1 + m_2 x_4 + m_3 x_4^2 = d_4\\[6pt]
\end{array},
\end{equation}
which for our data yields the linear system
\begin{equation}
 \left [
\begin{array}{ccccc}
1 & 2 & 4 \\	 
1 & 4 & 16 \\
1 & 6 & 36 \\
1 & 8 & 64 \\
\end{array} \right ]
\cdot
\left [ \begin{array}{c} 
m_1\\ m_2\\ m_3  \end{array}   \right]
=
\left [ \begin{array}{c} 1\\ 4 \\ 3 \\ 4  \end{array} \right ] .
\end{equation}
Using (\ref{eq:lsgensolution}) we find the solution to be
$d = -1.5 + 1.65 x -0.125x^2$.  Figure~\ref{fig:Fig1_GenLS} shows our data as well as the best quadratic curve.
\PSfig[h]{Fig1_GenLS}{Fitting a three-parameter quadratic curve to four data points by minimizing the least squares misfit.}
\end{example}
\index{General linear least squares method|)}

\subsection{Weighted least squares solution}
\index{Weighted least squares solution|(}

	What if some data constraints are more reliable than others? We may simply give those residuals more 
weight than the others, i.e.,
\begin{equation}
\mathbf{e} = \left[ \begin{array}{c} e_1 \\ 2e_2\\ \vdots \\ e_n \end{array} \right ] .
\end{equation}	 
In general, we can assign weights $w_i$ to each misfit so that the new weighted misfits become $e_i' = e_i\cdot w_i$.
Very often, the weights will simply be $s_{ii} = 1/s_i$, where $s_i$ is the one-sigma uncertainty in the $i$'th measurement $d_i$.
We implement such weighting by introducing a diagonal weight matrix,
\begin{equation}
{\mathbf S} = \left[\begin{array}{cccc}
s_{11} \\ & s_{22}\\ & & \ddots\\ & & & s_{nn} \end{array}
  \right ],
\end{equation}
which means the weighted residuals are $\mathbf{S}\cdot\mathbf{e}$ and the sum of the squared errors, $E$, becomes
\begin{equation}
E = \left (\mathbf{S}\cdot\mathbf{e}\right )^T\left (\mathbf{S}\cdot\mathbf{e}\right ) = \mathbf{e}^T \cdot \mathbf{S}^T \cdot \mathbf{S \cdot e = e}^T \cdot \mathbf{W \cdot e},
\end{equation}	 
where we have introduced $\mathbf{W = S}^T\mathbf{S}$. Since $\mathbf{S} \cdot \mathbf{e = S\cdot(G \cdot m - d)}$ we obtain
\begin{equation}
\begin{array}{rcl}
E(\mathbf{m}) & = & \mathbf{(S \cdot G \cdot m - S\cdot d)}^T \cdot \mathbf{(S \cdot G \cdot m - S \cdot d)} =
(\mathbf{m}^T \cdot \mathbf{G}^T \cdot \mathbf{S}^T - \mathbf{d}^T \cdot S\mathbf{^T) \cdot (S \cdot G \cdot m - S \cdot d)}\\
& = & \mathbf{m}^T \cdot \mathbf{G} ^T \cdot \mathbf{S}^T \cdot \mathbf{S \cdot G \cdot  m - m}^T \cdot \mathbf{G}^T \cdot \mathbf{S}^T \cdot \mathbf{S \cdot d - d}^T \cdot \mathbf{S}^T \cdot \mathbf{S \cdot G  \cdot m + d}^T \cdot \mathbf{S}^T \cdot \mathbf{S \cdot d}.
\end{array}
\end{equation}
We substitute $\mathbf{W = S}^T\mathbf{S}$, take the partial derivatives, and obtain
\begin{equation}
\frac{\partial E(\mathbf{m})}{\partial m_j }
= \mathbf{  0 =  \dot{m}}^T \cdot \mathbf{G}^T \cdot \mathbf{W  \cdot G \cdot  m + m}^T \cdot \mathbf{G}^T  \cdot \mathbf{W  \cdot G  \cdot \dot{m} - \dot{m}}^T \cdot \mathbf{G}^T \cdot \mathbf{W  \cdot  d - d}^T \cdot \mathbf{W  \cdot G  \cdot \dot{m}}, j = 1,k.
\end{equation}	 
Since $\mathbf{m}$ only contains the $m_j$, we know $\mathbf{\dot{m}}^T = \mathbf{\dot{m} = I}$. We again find the $k$ normal equations can be written more compactly as the single matrix equation
\begin{equation}
\mathbf{G}^T \cdot \mathbf{W  \cdot G \cdot  m + m}^T \cdot \mathbf{G}^T  \cdot \mathbf{W  \cdot G  -  G}^T \cdot \mathbf{W  \cdot  d - d}^T \cdot \mathbf{W  \cdot G = 0}.
\end{equation}
As before, the second and fourth terms are the transposes of the first and third terms, and as they all represent scalars our equation reduces to
\begin{equation}
\mathbf{G}^T \cdot \mathbf{W \cdot G \cdot m - G}^T \cdot \mathbf{W \cdot d = 0}.
\end{equation}	 
Thus, the weighted linear least squares solution is
\begin{equation}
\mathbf{m} = \left [ \mathbf{G}^T \cdot \mathbf{W \cdot G} \right ] ^{-1} \mathbf{G}^T \cdot \mathbf{W \cdot d}.
\label{eq:weightedLSgeneral}
\end{equation}
This solution is universal and applies to \emph{any} linear least squares problem one can imagine.
In the particular case when all $s_{ii} = 1$ the solution reduces to (\ref{eq:lsgensolution}).
\begin{example}
\PSfig[h]{Fig1_grav_model}{Observed and modeled gravity anomalies over a dense ore body.
While the model is nonlinear in $x$ and $z$, it is \emph{linear} in the coefficients $p_i$.}
We will try the least squares machinery on an example taken from exploration geophysics.
Figure~\ref{fig:Fig1_grav_model} shows how observed gravity anomalies ($d_i$, solid circles) vary
over a buried dense ore body as a function of location $x_i$.
Based on the inferred geometry of the ore (from subsurface geology seen in mine shafts, etc.) we
expect that a first-order approximation to the ore body could be a sphere buried at a depth of 5 km, with
a radius of 2.5 km, and located at 15 km to the right of the origin.  We would like to determine
the density of that ore body.  Exploration geophysics textbooks tell us that the gravity anomaly over
a buried sphere of radius $r$ and \emph{unit} density is
\begin{equation}
g_{sp}(x, z, r) = \gamma \frac{\frac{4}{3}\pi r^3 z}{(x^2 + z^2)^{3/2}},
\end{equation}	 
where $z$ is the depth to the center of the sphere, $x = 0$ is where the sphere is located, and $\gamma$
is the universal gravitational constant ($6.674\cdot10^{-11}$ m$^3$kg$^{-1}$s$^{-2}$).  However, 
inspection of the data suggests that the anomaly due to the ore body is superimposed on a regional field
with some curvature to it (i.e., dashed trend in Figure~\ref{fig:Fig1_grav_model}).  Therefore, we decide to model
these anomalies as a sum of a quadratic background (regional) field and the attraction of the sphere;
this is accomplished with the four-parameter linear model
\begin{equation}
g(x) = m_1 + m_2 x + m_3 x^2 + m_4 g_{sp}(x-15, 5, 2.5),
\end{equation}	 
where $m_4 = \Delta \rho$, the density contrast between the ore and the host rock.  In the parlance of
the previous sections, our basis functions $g_j(x)$ are \{1, $x$, $x^2$, and $g_{sp}(x)$\}.  To solve the
problem we need to evaluate the matrix equation $\mathbf{G \cdot m = d}$, i.e.,
\begin{equation}
\left[ \begin{array}{cccc}
1 & x_1 & x_1^2 & g_{sp}(x_1-15,5, 2.5) \\[5pt]
1 & x_2 & x_2^2 & g_{sp}(x_2-15,5, 2.5) \\[5pt]
1 & x_3 & x_3^2 & g_{sp}(x_3-15,5, 2.5) \\[5pt]
\vdots	&	\vdots & \vdots & \vdots \\
1 & x_n & x_n^2 & g_{sp}(x_n-15,5, 2.5)
\end{array}   \right]	 
\cdot
\left [ \begin{array}{c} 
m_1\\ m_2\\ m_3 \\ m_4  \end{array}   \right]
=
\left [ \begin{array}{c} d_1\\ d_2 \\ d_3 \\ \vdots \\ d_n  \end{array} \right ],
\end{equation}
whose solution becomes
\begin{equation}
\mathbf{m} = [ \mathbf{G}^T \cdot \mathbf{G} ]^{-1} \mathbf{G}^T \cdot \mathbf{d}.
\label{eq:LS_grav}
\end{equation}

Thus, we have solved a fairly complicated least squares modeling problem (solution is
the solid line in Figure~\ref{fig:Fig1_grav_model}).
\end{example}
\index{Weighted least squares solution|)}

\clearpage
\section{Problems for Chapter \thechapter}

\begin{problem}
If the major product moment $\mathbf{A}^T\mathbf{A} = \mathbf{0}$, show that all elements in
$\mathbf{A}$ must be zero, i.e. $a_{ij} = 0$. (Hint: what does an element in $\mathbf{A}^T\mathbf{A}$ represent?)
\end{problem}

\begin{problem}
An analytical experiment yields the following data pairs $t$ and $d(t)$:
\begin{table}[H]
\centering
\begin{tabular}{|c|c|} \hline
$t$ & $d$ \\ \hline
-0.82 & -0.86 \\ \hline
0.23 & -0.58 \\ \hline
1.35 & 0.54 \\ \hline
2.25 & 1.30 \\ \hline
3.33 & 2.20 \\ \hline
\end{tabular}
\end{table}
\begin{enumerate}[label=\alph*)]
\item Using the solutions for least-squares regression lines, i.e., (\ref{eq:intercept_solution}) and (\ref{eq:slope_solution})
above, determine the slope and intercept of the best-fitting line.
\item Plot this line and the data points.
\item Redo the problem using matrix algebra.  Write down the various components in the matrix equation
$\mathbf{G \cdot m = d}$.  What does $\mathbf{G \cdot m}$ represent?
\item Evaluate the matrix solution (i.e., $\mathbf{m} = [ \mathbf{G}^T\mathbf{G} ]^{-1}\mathbf{G}^T\mathbf{d}$)
and make sure it matches your answer in (a).
\item What is the total misfit, $E$?
\end{enumerate}
\end{problem}

\begin{problem}
A detailed study of seafloor heat flow was conducted along a transect perpendicular
to a mid-ocean ridge.  The results are reported as distance (km), heat flow (mWm$^{-2}$) pairs in
the file \emph{heatflow.txt}.  Because of hydrothermal circulation, we choose to model the
decay of heat flow with distance from the ridge as
\[
q(x) = q_0 + q_1x + \frac{q_2}{x}.
\]
\begin{enumerate}[label=\alph*)]
\item Using matrix algebra, what are the least-squares values for the model coefficients?
\item Plot the data as points and the model as a solid line evaluated every km from 25 km to 250 km
on the same graph.
\item What is the r.m.s (root-mean-square) misfit (r.m.s = $\sqrt{E/n}$)? Be sure to give units.
\item Using your error analysis skills, what is the estimated
heat flow at a distance of $x = 25 \pm 1$ km from the ridge axis? (Hint: Consider the uncertainty
in the function $q(x)$.)
\end{enumerate}
\end{problem}

\begin{problem}
The file \emph{faultstep.txt} contains distance (m) and relief (m) for a topographic profile across
an small normal fault.  We want to estimate the total vertical offset across the fault.  There is also a
linear trend in the data because the the fault sits on a gently dipping monocline.
\begin{enumerate}[label=\alph*)]
\item  Make a linear model that includes the trend and the step.  You may want to use the Heaviside 
step function (\texttt{heaviside} in MATLAB), defined as
$$
H(x) = \left \{ \begin{array}{rr} 
0, & x < 0 \\
\frac{1}{2}, & x = 0 \\
1, & x > 0
\end{array} \right.
$$
What is your model equation for the topographic profile?
\item  Let your best guess for the fault location be $x_0 = 145$ m and use matrix algebra
to solve for the model parameters.  Write the first few terms of the matrix equation
$\mathbf{G \cdot m = d}$ so you can see the pattern.  
Plot your model prediction on top of the data and indicate the values of the parameters.  What is the
rms misfit? What is the size of the fault offset?
\item  Guessing $x_0$ is not very robust.  Determine the best choice for $x_0$ that minimizes the total 
misfit $E$ by trying a range of $x_0$ (hint: evaluate $E(x_0)$ for a dense set of $x_0$ values
covering the likely range, plot $E$ versus $x_0$, and find the value of $x_0$ where $E$ has its minimum).
What is your final model parameters for the fault profile? Plot it on top of the data.
What is your minimum r.m.s. value?  What is the final size of the fault offset?
\end{enumerate}
\end{problem}

\begin{problem}
As sedimentary layers become buried they compact, reducing the porosity of the unit.
The empirical relation between depth ($z$) and porosity ($\theta$) in well-compacted sediments is called \emph{Athy's law}:
$$
\theta = \theta_0 e^{-\alpha z},
$$
where $\theta_0$ and $\alpha$ are constants that vary with sediment type and location, and $z$ is depth of burial.  
\begin{enumerate}[label=\alph*)]
\item	Given the data set below, find the weighted least-squares estimates of $\theta_0$ and $\alpha$ using matrix algebra,
taking the uncertainties in $\theta$ into account (note: Your first idea on how to do that is likely to be wrong!).
Plot the data and your Athy's law.  (Hint:  Athy's law as given
is not linear!  You must transform it first.)
\item	What is the predicted porosity ($\pm$ uncertainty) at a depth of 6 km ($\pm 200$m)?
\end{enumerate}
\begin{table}[H]
\centering
\begin{tabular}{|c|c|} \hline
\bf{Depth} (m) &	\bf{Porosity (\%)} \\ \hline
650	& $38 \pm 5.0$ \\ \hline
1000	& $35 \pm 4.0$ \\ \hline
2050	& $24 \pm 2.5$ \\ \hline
2950	& $18 \pm 2.0$ \\ \hline
4075	& $14 \pm 1.5$ \\ \hline
5030	& $9.8 \pm 1.0$ \\ \hline
\end{tabular}
\end{table}
\end{problem}

\begin{problem}
The rim of Halemaumau crater within Kilauea caldera has been digitized and its UTM coordinates ($x_i, d_i$) are
listed in \emph{halemaumau.txt}.  We wish to approximate this shape by a perfect circle with parameters
$(x_0, d_0, r)$.
\begin{enumerate}[label=\alph*)]
	\item What is the misfit function, $E$, that we want to minimize using the least-squares criterion?
	\item Determine the three parameters for the circular model by minimizing $E$.
	\item Estimate the area of the crater and use the standard deviation of the radial residual misfits to assign
	  one-sigma confidence bounds on the area.
\end{enumerate}
\end{problem}

\begin{problem}
	\newcounter{noisyc}
	\setcounter{noisyc}{\thechapter}
	\newcounter{noisyp}
	\setcounter{noisyp}{\theproblem}
The file \emph{noisy.txt} contains observations of a phenomenon known to oscillate at a single frequency, $\omega$.
This signal is superimposed on a linear trend in the presence of some random noise.
\begin{enumerate}[label=\alph*)]
	\item Assume you know the frequency $\omega$. What is the form of the model you could use to fit the data?
	Hint: $A\cos(\omega t - \phi) = a \cos(\omega t) + b \sin(\omega t)$.
	\item Plot the data and eye-ball the period and use it to determine the frequency.  Use least-squares to fit your
	model and plot it.  What is the misfit, $E$, for this trial model?
	\item Try a range of frequencies centered on your best guess and determine the frequency $\omega_0$ that minimizes
	the misfit.  Plot the optimal model, and in a separate diagram plot the misfits you obtained versus the frequencies
	you explored.
\end{enumerate}
\end{problem}

\begin{problem}
Having crash-landed on an alien planet you worry you may not have enough fuel to reach escape velocity.  To find out you
need to know the planet's gravitational attraction, $g_p$ (which on Earth is 9.81 m s$^{-2}$).  To get an estimate you decide
to drop rocks off cliffs of various heights and clock the time it takes for them to reach the valley floor.  Your data
are listed in the file \emph{drops.txt}.  You vaguely remember your high-school physics exploits where you learned that the vertical drop in vacuum
is related to drop time given by $h = \frac{1}{2}g_p t^2$.  Plotting the data you realize that the height measurements obtained with your damaged laser
rangefinder is biased by some unknown but constant offset.  Aargh, there is always something....
\begin{enumerate}[label=\alph*)]
	\item What is a suitable linear model that will relate your observations and unknowns?
	\item Determine the planet's gravity and the bias in your instrument using the linear least squares method.
\end{enumerate}
\end{problem}

\begin{problem}
An elongated sedimentary basin is estimated to have a width of 20 km and a depth of 4 km.  The gravity anomalies measured across the
basin are given in \emph{basingrav.txt} and show a broad negative anomaly due to the lower density sediments relative to
the surrounding higher density igneous bedrock, but some lateral regional trend is also apparent.  You decide to model the anomalies as
a linear regional trend plus the attraction of a
prism-shaped two-dimensional basin of given dimensions.  You may use the MATLAB function \texttt{g\_basin.m} to compute the gravity of the prism.
\begin{enumerate}[label=\alph*)]
	\item Determine the least-squares solution for the density contrast.  If the bedrock density is 2670 kg m$^{-3}$,
	what is your estimate of the sediment density?
	\item A seismic crew collects data that throw some doubt on your depth estimate of 4 km.  You decide to repeat your
	modeling for a range of depths (from 3 to 6 km in steps of 100 m) and keep track of the misfit $E$ as a function
	of the chosen depth.  Find the optimal depth that minimizes the misfit and report this depth and the corresponding density contrast.
\end{enumerate}
\end{problem}
