%  $Id: DA1_Chap6.tex 688 2019-06-09 09:24:42Z pwessel $
%
\chapter{REGRESSION}
\label{ch:regression}
\epigraph{``The invalid assumption that correlation implies cause is probably among the two or three most serious and common errors of human reasoning.''}{\textit{Stephen Jay Gould, Paleontologist}}
Regression refers to a subset of data modeling where we fit a simple model with a linear trend in one (or more) dimensions.  Usually we also include a constant (intercept) term.  Entire books have been written about regression
and all the various methods, norms, and misfit-minimizations possible.  A summary of some of these developments will be given below.
\section{Line-Fitting Revisited}
\index{Regression!weighted least squares|(}

We will again consider the best-fitting line problem $y = a + bx$, this time with errors $s_i$ in the observed $y$-values.  We 
want to measure how well the model agrees with the data, and for this purpose we will use the $\chi^2$ 
function, i.e.,
\begin{equation}
\chi^2(a,b) = \sum^n_{i=1} \left ( \frac{y_i - a - bx_i}{s_i} \right ) ^2.
\label{eq:L2_line_chi2}
\end{equation}
Minimizing $\chi^2$  will give the best weighted least squares solution.  Again, we set the partial 
derivatives to zero and obtain
\begin{equation}
\begin{array}{c}
\displaystyle \frac{\partial \chi^2}{\partial a} = 0 = -2 \sum^n_{i=1} \left( \frac{y_i -a -bx_i}{s^2_i}\right),  \\
\ \\
\displaystyle \frac{\partial \chi^2}{\partial b} = 0 = -2 \sum^n_{i=1} \left( \frac{y_i -a -bx_i}{s^2_i}\right) x_i.
\end{array}
\label{eq:L2_line_ddx}
\end{equation}
Let us define the following terms (unless noted, all sums go from $i = 1$ to $n$):
\begin{equation}
S = \sum \frac{1}{s^2_i}, \quad S_x = \sum \frac{x_i}{s^2_i}, \quad S_y = 
\sum \frac{y_i}{s^2_i}, \quad S_{xx} = \sum \frac{x^2_i}{s^2_i}, \quad
S_{xy} = \sum \frac{x_i y_i}{s^2_i}.
\end{equation}
Then, (\ref{eq:L2_line_ddx}) reduces to
\begin{equation}
\begin{array}{rcl}
\displaystyle
aS  + bS_x & = & S_y, \\*[2ex]
aS_x + bS_{xx} & = & S_{xy}.
\end{array}
\end{equation} 	 
Introducing 
\begin{equation}
\Delta = SS_{xx} - S ^2 _x
\end{equation}
we find
\begin{equation}
\begin{array}{rcl}
a & = & \displaystyle \frac{S_{xx}S_y - S_x S_{xy}}{\Delta},\\*[2ex]
b & = & \displaystyle \frac{S S_{xy}- S_x S_{y}}{\Delta}.
\end{array}
\label{eq:L2_sol_ab}
\end{equation}
\PSfig[h]{Fig1_linefit_x}{The uncertainty in the line fit depends to a large extent on the
distribution of the $x$-positions as well as the uncertainties in the $y$-values.}
   All this is swell but we must also estimate the uncertainties in $a$ and $b$.  For the same $s_i$ we 
may get large differences in the uncertainties in $a$ and $b$ (e.g., Figure~\ref{fig:Fig1_linefit_x}).
As shown in Chapter~\ref{ch:EDA}, consideration of the propagation of errors (e.g., \ref{eq:uncert_func}) shows
that the variance $\sigma_f^2$ in the value of any function is
\begin{equation}
\sigma^2_f = \sum \left ( \frac{\partial f}{\partial y_i} \sigma_i \right )^2,
\label{eq:L2_line_error}
\end{equation}
where we now consider $f$ a function of all the $n$ independent parameters $y_i$. For our model,
$f$ is either $a$ or $b$ so the partial derivatives become
\begin{equation}
\begin{array}{rcl}
\displaystyle \frac{\partial a}{\partial y_i} & = & \displaystyle \frac{S_{xx} - S_x x_i}{s ^2_i \Delta},\\*[2ex]
\displaystyle \frac{\partial b}{\partial y_i} & = & \displaystyle \frac{Sx_{i} - S_x}{s ^2_i \Delta}.
\end{array}
\end{equation}
Inserting in turn these terms into (\ref{eq:L2_line_error}) now gives
\begin{equation}
\begin{array}{rcl} 
\displaystyle
\displaystyle s^2_a & = & \displaystyle \sum s^2_i \left [ \frac{S_{xx} - S_{x} x_i}{s ^2_i \Delta} \right ] ^2 = \sum \frac{S^2_{xx} - 2S_{xx}S_{x}x_i + S^2_x x^2_i}{s^2_i \Delta ^2}\\*[2ex]
& = & \displaystyle \frac{S^2_{xx}}{\Delta^2} \sum \frac{1}{s^2_i} - \frac{2S_{xx}S_x}{\Delta^2}
\sum \frac{x_i}{s^2_i} + \frac{S^2_x}{\Delta^2} \sum \frac{x^2_i}{s^2_i}= \frac{S^2_{xx}S}{\Delta^2} - \frac{2S_{xx}S^2_x}{\Delta ^2} + \frac{S_{xx}S^2_x}{\Delta^2} \\*[2ex]
& = & \displaystyle \frac{S_{xx} (S_{xx}S-S^2_x)}{\Delta^2} = \frac{S_{xx}}{\Delta}
\end{array}
\end{equation}
and	
\begin{equation}
\begin{array}{rcl}
\displaystyle s^2_b & = & \displaystyle \sum s^2_i \left [ \frac{Sx_i - S_x}{s^2_i \Delta} \right ] ^2 =
\sum \frac{S^2x^2_i - 2S \ S_x x_i + S^2_x}{s^2_i \Delta^2}\\*[2ex]
 &  = & \displaystyle \frac{S^2}{\Delta^2}\sum \frac{x^2_i}{s^2_i} - 
\frac{2S \ S_x}{\Delta^2} \sum \frac{x_i}{s^2_i} + \frac{S^2_x}{\Delta^2} \sum \frac{1}{s^2_i} = \frac{S^2S_{xx}}{\Delta^2} - \frac{2S \ S^2_x}{\Delta^2} + \frac{S S^2_x}{\Delta ^2}\\*[2ex]
& = & \displaystyle \frac{S(S_{xx}S - S^2_x)}{\Delta^2} = \frac{S}{\Delta}.
\label{eq:err_in_slope}
\end{array}
\end{equation}
Similarly, we can find the covariance $s_{ab}$ from
\begin{equation}
s^2_{ab} = \sum s ^2_i \left( \frac{\partial a}{\partial y_i} \right)
\left( \frac{\partial b}{\partial y_i}\right) = -\frac{S_x}{\Delta}.
\end{equation}
Thus, the correlation between $a$ and $b$ becomes
\begin{equation}
r_{ab} = \frac{-S_x}{\sqrt{SS_{xx}}}.
\end{equation}
It is therefore useful to shift the origin to $\bar{x}$ so that $r_{ab} = 0$, leaving our estimates for slope and intercept uncorrelated.  

	Finally, we must check if the fit is significant.  We determine 
critical $\chi ^2_\alpha$ for $n - 2$ degrees of freedom and test if our computed $\chi^2$ exceeds the critical limit.  If it 
does not, then we may say the fit is \emph{significant} at the $\alpha$ level of confidence.

\subsection{Confidence interval on regression}
\index{Regression!confidence interval}
\PSfig[h]{Fig1_Draper}{Solid line shows the least squares regression fit to the data points (blue circles), with
color bands reflecting different confidence levels.  Short vertical lines are the residual errors which are squared and summed
in (\ref{eq:reg_errstd}).}
The formalism in the previous section allowed us to derive
solutions for slope and intercept given via (\ref{eq:L2_sol_ab}). Let us for a moment consider the case where there
are no individual uncertainties $\sigma_i$ associated with the data.  We write the least squares
fit as $\hat{y} = a + bx$, and by substituting $a = \bar{y} - b\bar{x}$ we obtain
\begin{equation}
	\hat{y} = \bar{y} + b(x-\bar{x}).
\end{equation}
Here, both the mean $y$-value ($\bar{y}$) and slope ($b$) are subject to error and these in turn affect $\hat{y}$.  For some
chosen location $x_0$ the prediction for the regression would be
\begin{equation}
	\hat{y}_0 = \bar{y} + b(x_0-\bar{x}).
\end{equation}
In this formulation, with a local origin at $(\bar{x}, \bar{y})$, the correlation between $\bar{y}$ and $b$ is zero.
This fact allows us to compute the expected variance $V$ of the \emph{mean predicted value} thus:
\begin{equation}
	V(\hat{y}_0) = V(\bar{y}) + V(b)(x-\bar{x})^2 = \frac{s^2}{n} + \frac{s^2(x_0 - \bar{x})^2}{\sum(x_i-\bar{x})^2}.
\end{equation}
Here, $s$ is our sample estimate of the standard deviation of the regression residuals,
\begin{equation}
	s^2 = \frac{\sum (y_i - \bar{y})^2}{n-2}.
	\label{eq:reg_errstd}
\end{equation}
Because $\bar{y}$ and $b$ are uncorrelated we obtain the variance of an individual
observation by adding the independent variance of $y$ about the mean, which is $s^2$, and find the regression
variance for the individual prediction at $x = x_0$ to be
\begin{equation}
s^2_0 = s^2 \left [ 1 + \frac{1}{n} +  \frac{(x_0 - \bar{x})^2}{\sum(x_i-\bar{x})^2} \right ] .
\label{eq:reg_conf}
\end{equation}
To obtain confidence intervals on the linear regression we simply scale (\ref{eq:reg_conf}) by a critical
Student's $t$-value for the degrees of freedom $\nu$ and chosen confidence level (Table~\ref{tbl:Critical_t}), i.e.,
\begin{equation}
	\hat{y}_0 \pm t(\nu,\alpha/2) s \sqrt{ 1 + \frac{1}{n} +  \frac{(x_0 - \bar{x})^2}{\sum(x_i-\bar{x})^2}}.
\end{equation}
For larger data sets the Student's $t$ values approach the normal distribution critical values $|z_{\alpha/2}|$.
Figure~\ref{fig:Fig1_Draper} illustrates how confidence bands on a least-squares regression fit takes
on a parabolic shape around the best-fit line.
\index{Regression!weighted least squares|)}

\section{Orthogonal Regression}
\index{Regression!orthogonal|(}
\subsection{Major axis}
\index{Regression!major axis|(}

	It is often the case that the uncertainties in our $(x,y)$ data affect both coordinates.  
Examples where this is the case include situations where both $x$ and $y$ are observed quantities 
(and hence are known to have errors).  It is also  applicable when $y$ is a function of $x$, but $x$ (e.g., distance 
or time) itself has uncertainties.  In these cases, orthogonal regression is the correct way to 
determine linear relationships between $x$ and $y$ (Figure~\ref{fig:Fig1_MA_misfit}).
\PSfig[h]{Fig1_MA_misfit}{The misfit is measured in the direction perpendicular to the line.  Note that
$(x_i, y_i)$ denote our data points (black circles) and $(X_i, Y_i)$ are the coordinates of their orthogonal
projections (white circles; only one is shown here) onto the regression line.}
We will use the least squares principle and minimize the sum of the squared \emph{perpendicular} 
distances $d_i^2$ from the data points $(x_i,y_i)$ to the regression line.  The function we want to minimize is
\begin{equation}
E = \sum_{i=1}^n \left [ (X_i - x_i)^2 + (Y_i - y_i)^2 \right ],
\label{eq:E_major_axis}
\end{equation}	 	
where lowercase $(x_i, y_i)$ again are our observations and uppercase $(X_i, Y_i)$ are the ``adjusted'' 
coordinates we want to find.  These are, of course, required to  lie on a straight line described by
\begin{equation}
Y_i = a + bX_i,
\label{eq:line_constraint}
\end{equation}	
or equivalently
\begin{equation}
f_i = a + bX_i - Y_i = 0.
\label{eq:Lagrange_major_axis}
\end{equation}
Thus, we cannot simply find \emph{any} set of $(X_i, Y_i)$ as they also have to lie on a straight line.  The problem 
of minimizing the function (\ref{eq:E_major_axis}) under the specified constraints (\ref{eq:Lagrange_major_axis}) can be solved by a method 
known as \emph{Lagrange's multipliers}.  This method says we should form a new function $F$ by adding the 
\index{Lagrange's multipliers}
original function (\ref{eq:E_major_axis}) and all the constraints (\ref{eq:Lagrange_major_axis}), with each constraint scaled by an unknown 
(Lagrange) multiplier $\lambda _i$.  Since (\ref{eq:Lagrange_major_axis}) is actually $n$ constraints, we find
\begin{equation}
F = E + \lambda_1 f_1 + \lambda_2 f_2 + \ldots + \lambda_n f_n = E + \sum ^n _{i=1} \lambda_i f_i.
\label{eq:MA_partials}
\end{equation}	 
We may now set the partial derivatives of $F$ to zero and solve the resulting set of equations:
\begin{equation}
\frac{\partial F}{\partial X_i} = \frac{\partial F}{\partial Y_i} = \frac{\partial F}{\partial a} = \frac{\partial F}{\partial b} = 0,
\end{equation}
or, when viewed separately (with all sums over $i = 1$ to $n$ unless explicitly stated), 
\begin{equation}
\frac{\partial F}{\partial X_i} = \sum_j^n \frac{\partial}{\partial X_i} ( X_j - x_j)^2  + \sum_j^n \frac{\partial}{\partial X_i}
(\lambda_j bX_j) = 0 \Rightarrow 2(X_i - x_i) + b \lambda_i = 0,
\end{equation}
\begin{equation}
\frac{\partial F}{\partial Y_i} = \sum_j^n \frac{\partial}{\partial Y_i} ( Y_j - y_j)^2  - \sum_j^n \frac{\partial}{\partial Y_i}
(\lambda_j Y_j) = 0  \Rightarrow 2(Y_i - y_i) -  \lambda_i = 0,
\end{equation}
\begin{equation}
\frac{\partial F}{\partial a} = \sum  \frac{\partial}{\partial a} (\lambda_i a) = 0  \Rightarrow \sum \lambda_i  = 0,
\label{eq:lambda_major}
\end{equation}
\begin{equation}
\frac{\partial F}{\partial b} = \sum \frac{\partial}{\partial b}(\lambda_i bX_i) = 0  \Rightarrow \sum \lambda_i X_i = 0.
\label{eq:lambda_x_major}
\end{equation}
Since each $i$ represents a separate equation, we find
\begin{equation}
2(X_i - x_i) = -b \lambda_i \Rightarrow X_i = x_i - b \lambda_i/2,
\label{eq:Xisolution}
\end{equation}
\begin{equation}
2(Y_i - y_i) = \lambda_i \Rightarrow Y_i = y_i + \lambda_i/2.
\end{equation}	 
Substituting these expressions for $X_i$ and $Y_i$ into (\ref{eq:line_constraint}), we obtain
\begin{equation}
y_i + \lambda_i/2 = a + b \left (x_i - b \lambda_i/2 \right ) = a + bx_i - b^2 \lambda_i/2
\end{equation}	 
or
\begin{equation}
\lambda_i = \frac{2}{1 + b^2}\left (a + bx_i - y_i\right ).
\label{eq:lambda_i_major}
\end{equation}
Now, (\ref{eq:lambda_major}), (\ref{eq:lambda_x_major}), and (\ref{eq:lambda_i_major}) gives us
$n+2$ equations in $n+2$ unknowns (all the $\lambda_i$ plus $a$ and $b$).  Combining 
(\ref{eq:lambda_i_major}) and (\ref{eq:lambda_major}) gives
\begin{equation}
\sum	  \frac{1}{1+b^2} \left ( a + b x_i - y_i\right ) =  0
\label{eq:step1_major}
\end{equation}	
and (\ref{eq:lambda_x_major}) using (\ref{eq:Xisolution}) gives
\begin{equation}
\sum \lambda_i x_i - b\lambda ^2_i/2 = 0,
\end{equation}	 
into which we substitute (\ref{eq:lambda_i_major}) and find
\begin{equation}
\sum \frac{1}{1 + b^2}\left (ax_i + bx^2_i - y_i x_i\right ) - \sum \frac{b}{(1+b^2)^2}\left (a+bx_i - y_i\right ) ^2 = 0.
\label{eq:step2_major} 
\end{equation}
These two equations (\ref{eq:step1_major} and \ref{eq:step2_major}) relate the parameters $a$ and $b$ to the given data values $x_i$ and $y_i$.  We find 
the solution by solving the equations simultaneously.  Noting that the denominator in (\ref{eq:step1_major}) cannot be zero, we find
\begin{equation}
\sum (a+ b x_i - y_i) = 0 \Rightarrow na + b \sum x_i = \sum y_i
\end{equation}	 
or
\begin{equation}
a = \bar{y} - b \bar{x},
\end{equation}
%	(4.16)
where $\bar{x}$  and $\bar{y}$ are the mean data values.  This expression for the intercept can now be substituted into 
(\ref{eq:step2_major}) so we may solve for the slope.  We multiply through by $(1 + b^2)^2$ and obtain
\begin{equation}
\sum (1 + b^2) \left ( \bar{y}x_i - b \bar{x} x_i + b x^2_i - x_i y_i \right ) -b \sum  \left ( \bar{y} - b \bar{x} + b x_i - y_i\right ) ^2 = 0
\end{equation}
which simplify to
\begin{equation}
\displaystyle (1 + b^2) \sum x_i \left ( \bar{y} - y_i + b (x_i - \bar{x}) \right ) -b \sum \left ( b ( x_i - \bar{x}) - (y_i - \bar{y})\right ) ^2 = 0,\\*[2ex]
\end{equation}
We now introduce the residuals $u_i = x_i - \bar{x}$ and $v_i = y_i - \bar{y}$.  Then
$$
\begin{array}{c}
\displaystyle (1 + b^2) \sum (u_i + \bar{x}) \left ( bu_i - v_i\right ) - b \sum \left (bu_i - v_i\right )^2 = 0, \\*[2ex]
\displaystyle (1 + b^2) \sum \left ( bu^2_i - u_i v_i  + \bar{x} bu_i - \bar{x} v_i \right ) - b\sum \left (b^2 u^2_i - 2b u_iv_i + v^2_i\right ) = 0, \\*[2ex]
\displaystyle \sum \left ( bu^2_i + b^3 u^2_i - u_iv_i - b^2 u_iv_i - b^3 u^2_i + 2b^2 u_iv_i - bv^2_i \right ) = 0, \\*[2ex]
\displaystyle \sum \left ( b^2u_iv_i + b (u^2_i - v^2_i)-u_i v_i \right ) = 0.
\end{array}
$$
where we have used the properties $\sum u_i = \sum v_i = 0$.  These steps finally give the solution for the slope as
\begin{equation}
b = \frac{\sum v^2_i - \sum u^2_i \pm \sqrt{(\sum u^2_i - \sum v^2_i)^2 + 4(\sum u_iv_i)^2}}{2 \sum u_iv_i}.
\end{equation}
This equation gives two solutions for the slope and we choose the one that minimizes $E$.  The other solution maximizes $E$ and
makes an angle of 90 degrees with the optimal solution.
\index{Regression!major axis|)}

\subsection{Reduced major axis (RMA) regression}
\index{Regression!reduced major axis (RMA)|(}

\PSfig[h]{Fig1_RMA_misfit}{RMA regression minimizes the sum of the \emph{areas} of the (white) rectangles
defined by the data points (black circles) and their orthogonal 
projection points (white circles; only one is shown here) on the regression line.}
In this alternative formulation of misfit we minimize the sum of the \emph{areas} of the rectangles defined by $\Delta x$ and $\Delta y$ (Figure~\ref{fig:Fig1_RMA_misfit}).
Hence, the function to minimize is
\begin{equation}
E = \sum (X_i - x_i)(Y_i - y_i).
\label{eq:E_RMA_axis}
\end{equation}
The constraints remain the same, i.e. $Y_i = a + bX_i$.  The Lagrange's multiplier method leads to a 
system of equations similar to those discussed in the previous section (\ref{eq:MA_partials}), but now we find
the $2n + 2$ equations
\begin{equation}
\frac{\partial F}{\partial X_i} = \frac{\partial}{\partial X_i} \sum_j^n (X_j - x_j)(Y_j - y_j) + \frac{\partial}{\partial X_i} \sum_j^n \lambda_j b X_j = 0 \Rightarrow Y_i - y_i + b \lambda_i = 0  \quad i = 1,n,
\end{equation}	 	
\begin{equation}
\frac{\partial F}{\partial Y_i} = \frac{\partial}{\partial Y_i} \sum_j^n (X_j - x_j)(Y_j - y_j) - \frac{\partial}{\partial Y_i} \sum_j^n \lambda_j Y_j = 0 \Rightarrow X_i - x_i - \lambda_i = 0 \quad i = 1,n,
\end{equation}
\begin{equation}
\frac{\partial F}{\partial a} = \frac{\partial}{\partial a} \sum \lambda_i a = 0 \Rightarrow \sum \lambda_i = 0,
\end{equation}
\begin{equation}
\frac{\partial F}{\partial b} = \frac{\partial}{\partial b} \sum \lambda_i b X_i = 0 \Rightarrow \sum \lambda_i X_i = 0.
\label{eq:RMA_lxi}
\end{equation}
We find     
\begin{equation}
X_i - x_i - \lambda_i = 0 \Rightarrow X_i = x_i + \lambda_i,
\end{equation}
\begin{equation}	           
Y_i - y_i + b \lambda_i = 0 \Rightarrow Y_i = y_i - b\lambda_i.
\end{equation}
Substituting these values into the equation for the line (i.e., $Y_i = a + bX_i$) gives
\begin{equation}
y_i - b \lambda_i = a + b (x_i + \lambda_i)
\end{equation}
or			       
\begin{equation}
\lambda_i = \frac{y_i - a - bx_i}{2b}.
\end{equation}	 
Since $\sum \lambda_i = 0$, we find again
\begin{equation}
a = \bar{y} - b \bar{x}.
\end{equation}	 
Substituting $\lambda_i, X_i$ and $a$ into (\ref{eq:RMA_lxi}) gives
$$
\sum \left ( x_i + \frac{y_i - \bar{y} + b \bar{x} - b x_i}{2b} \right ) \left ( 
\frac{y_i - \bar{y} + b \bar{x} - b x_i}{2b} \right) = 0.
$$
We let $u_i = x_i - \bar{x}$ and $v_i = y_i - \bar{y}$ as before and obtain 
$$
\sum \left (b (u_i + \bar{x} ) + v_i - bu_i \right ) (v_i - bu_i) = 0,
$$

$$
\sum (2b \bar{x} + bu_i + v_i) (v_i - bu_i) = 0,
$$

$$
\sum (2b \bar{x} v_i - 2b ^2  \bar{x} u_i + b u_i v_i - b^2 u^2_i + v^2_i - b u_i v_i ) = 0,
$$

$$
\sum v^2_i - b^2 \sum u^2_i = 0.
$$
where we again have used the property that sums of $u_i$ and $v_i$ are zero.  Finally, we obtain
\begin{equation}
b = \pm \sqrt{ \sum v^2_i / \sum u^2_i} = \pm \frac{s_y}{s_x},
\end{equation}
i.e., the best slope equals the ratio of the $y$ and $x$ observations' separate standard deviations.
The sign is indeterminate but is inferred from the sign of the correlation coefficient (a negative
correlation means a negative slope).

\index{Regression!orthogonal|)}

\section{Robust Regression}
\index{Regression!robust|(}
\index{Robust!regression|(}

	In simple regression one assumes a relation of the type
\index{Explanatory variable}
\index{Response variable}
\begin{equation}
y_i = a + b x_i + \epsilon_i,
\label{eq:robregress}
\end{equation}
in which $x_i$ is called the \emph{explanatory variable} or \emph{regressor}, and $y_i$ is the \emph{response variable}.
Again, we seek to estimate $a$ and $b$ (intercept and slope) from the data 
$(x_i , y_i)$.  It is commonly assumed 
that the deviations $\epsilon_i$ are normally distributed.
Fortunately, in simple regression the observations $(x_i, y_i)$ are 2-D so they can be plotted.  It 
is always a good idea to do that first to see if any unusual features are present and to make sure 
the data are roughly linear.
Applying a regression estimator to the data $(x_i, y_i)$ will result in the two regression 
coefficients $\hat{a}$  and $\hat{b}$.  They are not the \emph{true} parameters $a$ and $b$, but our ``best'' \emph{estimates} of 
them.  We can insert those into (\ref{eq:robregress}) and find the predicted estimate as
\begin{equation}
\hat{y}_i = \hat{a} + \hat{b} x_i,\quad i = 1,n.
\end{equation}	 
The residual is then the difference between the observed and estimated values, yielding
\begin{equation}
e_i = y_i - \hat{y}_i,\quad i = 1,n.
\end{equation} 	
Note that there is a difference between $e_i$ (the misfit) and 
$\epsilon_i$ (the deviation), because $\epsilon_i = y_i - a - b x_i$ are
evaluated with the true unknown $a,b$.  We can compute $e_i$, but not $\epsilon_i$.

	The most popular regression estimator dates back to the early 1800's (to our old friends Gauss and Legendre) 
and is called the ``least-squares'' (LS) method since it seeks to minimize
\begin{equation}
E = \sum^n_{i=1} e^2_i.
\end{equation}	 
The rationale was to make the residuals very small.  Gauss preferred the least-squares 
criterion to other objective functions because in this way the regression coefficients could be 
computed explicitly from the data (no computers back in the day, at least not mechanical or electronic ones).
Later, Gauss introduced the  normal, or Gaussian, distribution for which least-squares is optimal.
	
\index{Regression!outlier|(}
More recently, people have realized that real data often do not satisfy the Gaussian 
assumption and this ``failure to comply'' may have a dramatic effect on the LS results.  In Figure~\ref{fig:Fig1_LS_pitfalls1} we have five points that lie almost 
on a straight line.  Here, the $LS$ line fits the data very well.  However, let us see what happens if 
we get a wrong value for $y_4$ because of a recording or copying error:
	 
\PSfig[h]{Fig1_LS_pitfalls1}{Pitfalls of least-squares regression, part I.  An outlying point in
the $y$-direction will effect the regression line considerably.}

\index{Regression!outlier in the y-direction}
The bad point $y_4$  is called an \emph{outlier in the y-direction}, and it has a dramatic effect on the $LS$ line 
which now is tilted away from the trend of the remaining data.  Such outliers have received the most 
attention because most experiments are set up to expect errors in $y$ only.  However, in 
observational studies it often happens that outliers occur in the $x_i$ data as well.

\PSfig[h]{Fig1_LS_pitfalls2}{Pitfalls of least-squares regression, part II.  Here, an outlying point in
the $x$-direction can have a huge effect on the regression line.}

Figure~\ref{fig:Fig1_LS_pitfalls2} illustrates the effect of an outlier in the $x$-direction.
\index{Regression!outlier in the x-direction}
It has an even more dramatic 
effect on $LS$ since it now is almost perpendicular to the actual trend.  Because this single point 
has such a large influence we denote it as a \emph{leverage} point.
\index{Regression!leverage point}
\index{Leverage point}
This is because the residual $e_i$ 
(measured in the $y$-direction) is enormous with regard to the original $LS$ fit.  The second $LS$ fit 
reduces this enormous error at the expense of increasing the errors at all other points.  In 
general, we call the $k$'th  point a leverage point if $x_k$ lies far from the bulk of the $x_i$.  Note that this 
definition does not take $y_i$ into account.  For instance, Figure~\ref{fig:Fig1_leverage} shows a ``good'' leverage 
point since it lies on the linear trend set by the majority of the data.  Thus, a leverage point only 
refers to its \emph{potential} for influencing the coefficients $\hat{a}, \hat{b}$.
When a point $(x_i, y_i)$ deviates from the linear relation of the majority it is called a \emph{regression outlier},
taking into account both $x_i$ and $y_i$ simultaneously.  As such, it is a vertical outlier or a bad 
leverage point.
	
It is often stated that regression outliers can be detected by looking at the $LS$ residuals. This is 
not always true.  The bad leverage point 1 has tilted the line so much  that its residual is very 
small.  Consequently, a rule that says ``delete points with highest $LS$ residual'' would first find points 
number 2 and 5, thereby deleting the good points first.  Of course, in simple $x-y$ regression we have the 
benefit of being able to plot the data so this is not often a problem, except when the number of 
data sets and points are large.
\index{Regression!outlier|)}

\PSfig[H]{Fig1_leverage}{The effect of leverage points in regression can be enormous, whether the data
point is a valid observation or a bad outlier.}

From the simple examples we have just seen, we find that the breakdown point for $LS$ 
regression is merely $1/n$ since one point is enough to ruin the day --- analogous to the breakdown 
point for the mean, which was also based on $LS$.
	
A first step toward a more robust regression was taken more than 100 years ago when 
Edgeworth suggested that one could instead minimize
\begin{equation}
E = \sum^n_{i=1} |e_i|,
\end{equation}	 
which we will call $L_1$ regression.  Unfortunately, while $L_1$ regression is robust with respect to 
outliers in $y$, it offers no protection toward bad leverage points.  Thus, the breakdown point is 
still only $1/n$.
	
While there are many methods that offer a higher breakdown point than $L_1$ and $L_2$, we will 
concentrate our presentation on one particular method.  Again, let us look at the $LS$ formulation:
$$
\begin{array}{cc}
\mbox{minimize} & \displaystyle E = \sum^n_{i=1} e^2_i. \\*[-2ex]
\hat{a}, \hat{b} \end{array}
$$
	 
\PSfig[H]{Fig1_LMS_regress}{Robust regression, such as LMS, is very tolerant of outlying points in 
both the $x$ and $y$ directions.}

At first glance, you would think that a better name for $LS$ would be least \emph{sum} of squares.  
Apparently, few people objected to removing the word ``sum'' as if the only sensible thing to do 
with $n$ numbers would be to add them.  Adding the $e_i^2$ terms is the same as using their mean (dividing 
by $n$ does not affect the minimization).  Why not replace the mean (i.e., the sum) by a median, which we know is 
very robust?  This yields the ``least median of squares'' (LMS) criterion:
\begin{equation}
\begin{array}{cc}
\mbox{minimize} & \mbox{median } \ e^2_i. \\
    \hat{a}, \hat{b} \end{array}
\end{equation}	 
It turns out the LMS fit is very robust with respect to outliers in $y$ as well as in $x$.  Its breakdown 
point is 50\%, which is the most we can ask for.  If more than half your data are bad then you cannot tell
which part is good unless you have additional information.
Figure~\ref{fig:Fig1_LMS_regress} shows what we get
using this method on the data that made the LS technique fail so badly.
	
The LMS line also has an intuitive geometric interpretation: it lies at the center of the narrowest 
strip that covers half of the data points.  By half of the points we mean $n/2 + 1$, and the thickness of 
the strip is measured in the vertical direction (i.e., Figure~\ref{fig:Fig1_LMS_geometry}).
	 
\PSfig[H]{Fig1_LMS_geometry}{Geometrical meaning of the LMS regression: the narrowest strip that
covers half the data points.}

An example of LMS regression comes from astronomy.  Astronomers often look for a linear 
relationship between the logarithm of the light intensity and the logarithm of the surface temperature of stars.  A 
scatter plot of observed quantities may look like Figure~\ref{fig:Fig1_astronomy}.  Here, the LMS line defines what is known 
as the \emph{main sequence}; the four outlying stars turned out to be red giants that do not follow the general 
trend.  The $LS$ fit produces a rather worthless compromise solution.  Here, the outliers are not so 
much errors as \emph{contamination from a different population}.

\PSfig[H]{Fig1_astronomy}{Example of robust regression in astronomy.  We see a Hertzsprung-Russell diagram of the star cluster
CYG OB1 with the least squares (dashed line) and LMS (solid line) fit.  The red giants are distorting the LS fit.  Data
taken from Rousseeuw, P. J., and A. M. Leroy (1987), \emph{Robust regression and outlier detection}, 329 pp., John Wiley and Sons, New York.}
\index{Regression!reduced major axis (RMA)|)}

\subsection{How to estimate LMS regression}
\index{LMS regression|(}
\index{Regression!LMS|(}

We can rewrite the minimization criterion as follows:
\begin{equation}
\begin{array}{cccc}
\mbox{minimize} \{ \mbox{median } \ e^2_i \}& = & \mbox{minimize} \{  \mbox{median} & ((y_i - \hat{b} x_i) - \hat{a})^2\} \\
\hat{a}, \hat{b} & & \hat{a}, \hat{b} & \end{array}
\end{equation}	 	
in the form
\begin{equation}
\begin{array}{ccc}
\mbox{minimize} \! \!  & \! \! \! \left \{ \! \! \! \phantom{\displaystyle \frac{\sum }{\ }} \! \!\! \mbox{minimize median} \right. & \! \! \! \left. (( y_i - \hat{b}x_i) - \hat{a})^2 \! \! \!\phantom{\displaystyle \frac{\sum }{\ }}\! \! \! \right \}\\*[-1ex]
\hat{b} & \hat{a} & \end{array} .
\end{equation}	 	
We will treat the two minimizations here separately.  The innermost minimization is the easy part, 
because for any given $\hat{b}$ it becomes essentially a 1-D problem, i.e., we want to find the value for $\hat{a}$ that 
minimizes the median
\begin{equation}
\begin{array}{cc}
\mbox{minimize} \! \! & \! \! \!  \left \{ \phantom{\displaystyle \frac{\sum }{\ }} \! \! \! \! \! \mbox{median } (u_i - \hat{a})^2 \right \}, \\*[-1ex]
\hat{a} & \end{array}
\end{equation}
where $u_i$ is calculated as $u_i = y_i - \hat{b} x_i$ (remember, we assumed that $\hat{b}$ was given).  This minimization problem is 
the same one we found earlier to give a good estimate of the mode.  Thus, this operation finds the mode 
of the $u_i$ data set.  We therefore need to find the $\hat{b}$ for which
\begin{equation}
E (\hat{b}) = \mbox{median} \ [(y_i - \hat{b}x_i) - \hat{a}]^2
\end{equation}	 
is minimal.  This is simply the minimization of a 1-D function $E(\hat{b})$ which is continuous but 
not everywhere differentiable.

To find this minimum we make the observation that the slope in 
the $x-y$ plane must be in the $\pm 90^{\circ}$ range (when expressed as an angle $\beta$, with $b = \tan \beta$).
We then simply perform a search for the optimal angle.  Starting 
with $\beta = -90 ^{\circ}$, we form the resulting $u_i$ and solve the 1-D minimization problem for $\hat{a}$, i.e., 
finding the LMS mode estimate $\hat{a}$.  We now increment the angle $\beta$ by $d\beta$ to, say, $-89 ^{\circ}$, and 
repeat the process.  At each step we keep track of what $E(\beta)$ is, and repeat these steps for all angles 
through $\beta = 90^{\circ}$.

\PSfig[H]{Fig1_LMS_bestslope}{Determining the best regression slope, $\hat{b} = \tan \hat{\beta}$.  The misfit function is not smooth but usually has
a minimum for the optimal slope. It is also likely to reveal several local minima that could trick a simpler search.}

Having found the slope $\hat{b}$ that gave the smallest misfit we may improve on this estimate by using a smaller step size $d\beta$ in this region 
to pinpoint the best choice for $\hat{b}$.  A plot of $E(\beta)$, shown in Figure~\ref{fig:Fig1_LMS_bestslope},
is very useful since it may tell us how 
unique the LMS regressions are: if more than one minimum are found they may indicate a possible 
ambiguity as there may be two or more lines that fit the data equally well.

	We will elaborate on the breakdown point for simple regression and illustrate it with 
a simple experiment.  Consider a data set that contains 100 good data points that exhibit a 
strong linear relation, computed from
\begin{equation}
y_i = 1.0 x_i + 2.0 + \epsilon_i \quad 1 \leq x_i \leq 4,
\end{equation}
with $\epsilon_i$ normally distributed, with $\mu = 0$ and $\sigma =0.2$.

\PSfig[h]{Fig1_LS_breakdown}{(a) Best LS fit to synthetic data set computed from a linear model with Gaussian noise.  
(b) Synthetic data set computed from a linear model with Gaussian noise, but now contaminated by points 
from another (bivariate) distribution centered on (7,2).  The LS line is pulled way off by these bad leverage points.
We can then plot the slope value as a function of the percentage of contamination.}

	Any regression technique, including $L_2$, will of course recover estimates of the slope and 
intercept that are very close to the true values 1 and 2 (Figure~\ref{fig:Fig1_LS_breakdown}a).  Then, we will start to contaminate the data 
by replacing ``good'' points with bad ones, the latter coming from a bivariate normal distribution 
with $\mu = (7, 2)$ and $\sigma_r = 0.5$.  We systematically substitute one bad point for a good point and 
recompute the regression parameters after each step.  What we find is that the LS estimate goes 
bad right away (Figure~\ref{fig:Fig1_LS_breakdown}b).  The bad points are basically bad leverage points, which we know the LS process 
cannot handle.  We keep track of the slope estimate after each substitution and graph the results
in Figure~\ref{fig:Fig1_breakdown}.
\index{Regression!breakdown point}

\PSfig[H]{Fig1_breakdown}{Breakdown plot for several regressors.  The LMS only breaks down when 50\% of the data are outlying.  
LS breaks down immediately because every point matters.}

	We call the percentage at which the slope starts to deviate significantly from the true value the \emph{breakdown point}.
The clear winner of this test is the LMS regression which 
keeps finding the correct trend until half the points have been replaced.  We should add that 
while LMS always have a breakdown point of 50\%, it is found that the effect of the outliers often
depends on the quality of the good data.  In cases where 
the good data exhibit a strong correlation the outliers do less damage than in a case where there 
is little or no correlation (Figure~\ref{fig:Fig1_contamination}).  Of course, when the correlation among the good data is minimal it is 
probably not very useful to insist that the data really exhibit a linear trend in the first place.

\PSfig[H]{Fig1_contamination}{Two data sets with the same degree of contamination.  However, one exhibit a much stronger correlation 
between the ``good'' points than the other.}
\index{LMS regression|)}
\index{Regression!LMS|)}

\subsection{How to find LMS 1-D Location (single mode)}
\index{Mode|(}

When we discussed estimates of central location we briefly mentioned that the value $\hat{x}$
that minimized
\begin{equation}
\begin{array}{cc}
\mbox{minimize} & \! \! \left \{ \mbox{median} \ (x_i - \hat{x})^2 \right \} \\
\hat{x} & \end{array}
\end{equation}
was called the LMS location and that it was a good approximation to the \emph{mode}, but how do we 
determine the LMS estimate?  It turns out that it is rather simple.  The following recipe 
will do:
\begin{enumerate}
\item	Sort the data into ascending order.
\item	Determine the shortest half of the sample, i.e., find the value for $i$ that yields the smallest of the differences $(x_{h+i} 
- x_i)$ with $h = n/2 + 1$.
\item	The LMS estimate is the midpoint of the shortest half:
\end{enumerate}
\begin{equation}
\mbox{LMS} \ = \frac{1}{2} (x_{h +i} + x_i).
\end{equation}	 	
You can empirically try this out by letting $\hat{x}$ take on all values within the data range, compute the median
of all $(x_i - \hat{x})^2$ values, and graph the curve and find its minimum (Figure~\ref{fig:Fig1_LMS}).

\PSfig[h]{Fig1_LMS}{LMS defines the mode while L2 defines the mean; both are the locations where their respective objective 
functions are minimized.}
\index{Mode|)}

\subsection{Making LMS ``analytical'' --- finding outliers}
\index{Regression!reweighted least squares|(}

	There is one problem with using the robust LMS parameters: the method is not \emph{analytical} so it 
does not lend itself easily to standardized statistical tests.  We will look into how we can 
overcome this obstacle.
	
The main problem comes from the fact that the outliers cause L$_2$ estimates to become unreliable.  We 
will avoid this problem altogether by using the best from both worlds (L$_2$ \emph{and} LMS):  We will use robust 
LMS techniques to find the best parameters in the regression model and then use the robust residuals to 
detect outliers.  Finally, we recompute weighted L$_2$ parameters with outliers given 
zero weights and other points given unit weights.  These L$_2$ estimates now represent only the 
``good" portion of the data and confidence limits and statistical tests may be based on the 
behavior of these good values.  We call this technique ``Re-weighted Least Squares'' (RLS).

\index{Outlier!identifying}
	First, we need a robust scale estimate for the residuals that will make them nondimensional.
It is customary to choose the preliminary scale estimate
\begin{equation}
s^0 =1.4826 \left( 1 + \frac{5}{n-2} \right ) \sqrt{\mbox{median } e^2_i},
\end{equation}	 	
where $e_i = y_i - \hat{y}_i$ are the residuals (i.e., misfits) at each point.
With this scale estimate we can evaluate the normalized residuals as
\begin{equation}
z_i = e_i/s^0.
\end{equation}	 
Now use these numbers to design the weights:
\begin{equation}
w_i = \left \{ \begin{array}{cl}
1, & |z_i| \leq 2.5\\
0, & \mbox{otherwise}
\end{array} \right. .
\end{equation}	 
The final LMS regression scale estimate is then given by
\begin{equation}
s^* = \sqrt{\displaystyle \sum w_i e^2_i / (\sum w_i -2)}.
\end{equation}	 	
The RLS regression parameters are therefore obtained by minimizing the weighted, squared residuals:
\begin{equation}
\min E = \sum^n_{i=1} w_i e^2_i.
\end{equation}	 	
This is simply the L$_2$ solution when only the good data are used.  As shown when we discussed 
the weighted L$_2$ regression problem, this technique will provide confidence intervals on both the 
slope and intercept and it also allows us to use the $\chi^2$-test to check whether the RLS fit is significant or not.
	
The strength of the linear relationship can again be measured by the (Pearson) correlation 
coefficient.  The LMS estimate of correlation is now given by
\begin{equation}
r = \sqrt{ 1 - \left (\frac{\mbox{median } |e_i|}{\mbox{MAD } y_i} \right )^2}
\end{equation}	 
with
\begin{equation}
\mbox{MAD } y_i = \mbox{median } | y_i - \tilde{y} |.
\end{equation}	 	
Compare this to the L$_2$ case, where
\begin{equation}
r =  \frac{s_{xy}}{s_x s_y} =
     \sqrt{ 1 - \frac{\sum (y_i - \hat{y}_i)^2}{\sum (y_i - \bar{y})^2}} = \sqrt{ 1 - \left (\frac{\bar{e}}{s_y} \right )^2}.
\end{equation}
This comparison shows that the robust estimates for ``average'' and ``scale'' have replaced the traditional
least-squares estimates in the expression for correlation.	
\index{Regression!robust|)}
\index{Robust!regression|)}
\index{Regression!reweighted least squares|)}

\section{Multiple Regression}
\index{Multiple regression|(}
\index{Regression!multiple|(}

\subsection{Preliminaries}

	Multiple regression is an extension of the simple regression problem for including more than 
one explanatory variable.  The most straightforward extension from 1-D to 2-D results in the 
determination of the constants $m_1, m_2,$ and $m_3$ in
\begin{equation}
d = m_1 + m_2 x + m_3 y,
\label{eq:plane}
\end{equation}
which you will recognize as the equation for a \emph{plane}.  A very common application of finding the 
best-fitting plane is to define and remove a regional (planar) trend from 2-D data sets so that 
local variations can be inspected and compared.  Let us say we have $n$ data points $d_i(x_i, y_i)$ in the 
plane and we want to determine the regional trend using standard $L_2$ techniques.  Using (\ref{eq:plane}) 
gives
\begin{equation}
m_1 + m_2 x_i + m_3y_i = d_i,\quad i = 1,n
\end{equation}
or $\mathbf{G\cdot m=d}$.  We know how to solve these $L_2$ problems now and quickly state that the solution is 
given by the linear system:
\begin{equation}
\mathbf{G}^T \mathbf{Gm = G}^T \mathbf{d}
\end{equation}
or 
\begin{equation}
\mathbf{	N\cdot m = v}.
\end{equation}
We can find the $\mathbf{N}$ and $\mathbf{v}$ by performing the matrix operations (below, all sums are implicitly from $i = 1$ to $n$)
\begin{equation}
\left [ \begin{array}{ccc}
n & \displaystyle \sum x_i & \displaystyle \sum y_i \\*[2ex]
\displaystyle \sum x_i & \displaystyle \sum x^2_i & \displaystyle \sum y_i x_i \\*[2ex]
\displaystyle \sum y_i & \displaystyle \sum x_i y_i & \displaystyle \sum y^2_i \\*[2ex]
\end{array} \right ]
\left[ \begin{array}{c}
m_1 \\*[2ex]
m_2 \\*[2ex]
m_3 
\end{array} \right]  = 
\left[ \begin{array}{c}
\displaystyle \sum d_i \\*[2ex] \displaystyle \sum x_i d_i\\*[2ex] 
\displaystyle \sum y_i d_i
\end{array} \right ].
\end{equation}	 
To verify this system, note that
\begin{equation}
\mathbf{N} = \mathbf{G}^T \mathbf{G} = \left [ \begin{array}{ccccc}
1 & 1 & 1 & \cdots & 1 \\*[2ex]
x_1 & x_2 & x_3 & \cdots & x_n \\*[2ex]
y_1 & y_2 & y_3 & \cdots & y_n \end{array}  \right ] \cdot 
\left [ \begin{array}{ccc}
1 & x_1 & y_1 \\*[2ex]
1 & x_2 & y_2 \\*[2ex]
\vdots & \vdots & \vdots \\*[2ex]
1 & x_n & y_n
\end{array} \right ]
= \left [ \begin{array}{ccc}
n & \displaystyle  \sum x_i & \displaystyle \sum y_i \\*[2ex]
\displaystyle \sum x_i & \displaystyle \sum x^2_i & \displaystyle \sum x_i y_i \\*[2ex]
\displaystyle \sum y_i & \displaystyle \sum x_i y_i & \displaystyle \sum y^2_i
\end{array} \right ]
\end{equation}
and
\begin{equation}
\mathbf{v} = \mathbf{G}^T\mathbf{d} = \left [ \begin{array}{ccccc}
1 & 1 & 1 & \cdots & 1 \\*[2ex]
x_1 & x_2 & x_3 & \cdots & x_n \\*[2ex]
y_1 & y_2 & y_3 & \cdots  & y_n \end{array}  \right ] \cdot
\left [ \begin{array}{c}
d_1 \\
d_2 \\
\vdots \\
d_n
\end{array} \right ] = 
\left [ \begin{array}{c}
\displaystyle \sum d_i \\*[2ex]
\displaystyle \sum x_i d_i \\*[2ex]
\displaystyle \sum y_i d_i 
\end{array}
\right ] .
\end{equation}
In many situations we have a gridded data set where we have observations on an equidistant 
lattice.  We can then find the $\mathbf{N}$ matrix analytically since the geometry is so simple.
\begin{example} 
We will examine data set given in Table~\ref{tbl:xy_grid}.
\begin{table}[H]
\center
\begin{tabular}{|c||r|r|r|r|r|} \hline
\bf{X/Y} & -2 & -1 & 0 & 1 & 2 \\ \hline \hline
2 & -1 & 0 & 3 & 2 & 4 \\ \hline
1 & 1 & -1 & 2 & 2 & 3 \\ \hline
0 & 0 & 0 & 1 & 2 & 2 \\ \hline
-1 & -2 & 0 & 1 & 1 & 2  \\ \hline
-2 & -1 & -1 & 0 & -1 & 1 \\ \hline
\end{tabular}
\caption{Simple data set of measured values of $d(x,y)$ on an equidistant grid.}
\label{tbl:xy_grid}
\end{table}
In this case, we find $x_i$ and $y_i$  to be centered on 0.  This means both the terms $\sum x_i = \sum y_i = 0$ by 
definition.  Furthermore, since $\sum {x_i y_i}$ can be written
\begin{equation}
\sum^n_{i=1} x_i y_i = \sum^{x=2}_{x=-2} \ \sum^{y=2}_{y=-2} x y = \sum ^2_{x = -2} x \  \sum^{y=2}_{y=-2} y = 0 
\end{equation}	 
we are left with
\begin{equation}
\left [ \begin{array}{rrr}
25 & 0 & 0 \\
0 & 50 & 0 \\
0 & 0 & 50 \end{array} \right ]
\left [ \begin{array}{c}
m_1\\ m_2\\ m_3 \end{array} \right ] = \left [ \begin{array}{c}
20 \\ 38 \\ 25 \end{array} \right ],
\end{equation}	 
or directly
\begin{equation}
d = \frac{4}{5} + \frac{19}{25}x + \frac{1}{2} y.
\label{eq:example_solution}
\end{equation}
The data and the best-fitting plane are shown in Figure~\ref{fig:Fig1_multregress}.
\PSfig[h]{Fig1_multregress}{Data points (red cubes) used in Example~\thechapter.\theexample, with the least-squares solution
shown as a transparent green plane.  Small black dots indicate prediction from the solution in (\ref{eq:example_solution}),
while vertical lines connect data points with model predictions.}
\end{example}
It should surprise no one that in the general case (i.e., no organized grid structure) it is advantageous to translate 
the data to a new origin $\bar{x},\bar{y}$   and operate on the adjusted coordinates
\begin{equation}
u_i = x_i - \bar{x}, v_i = y_i - \bar{y}	 .
\end{equation}
The system to solve is then
\begin{equation}
\left [ \begin{array}{ccc}
n & 0 & 0 \\*[2ex]
0 & \displaystyle \sum u^2_i & \displaystyle \sum u_iv_i \\*[2ex]
0 & \displaystyle \sum u_i v_i & \displaystyle \sum v^2_i
\end{array} \right ] 
\left [ \begin{array}{c}
m_1\\ m_2\\ m_3 \end{array} \right ] = 
\left[ \begin{array}{c} \displaystyle \sum d_i \\*[2ex] \displaystyle \sum u_i d_i \\*[2ex] \displaystyle \sum v_i d_i \end{array} \right ].
\end{equation}	 
Since the first equation directly gives us $m_1 = \bar{d}$, we are left with the simple $2\times 2$ system
\begin{equation}
\left[ \begin{array}{cc}
\displaystyle \sum u^2_i & \displaystyle \sum u_i v_i \\*[2ex]
\displaystyle \sum u_i  v_i & \displaystyle \sum v^2_i
\end{array} \right ]
\left [ \begin{array}{c}
m_2\\ m_3 \end{array} \right ] = 
\left [ \begin{array}{c}
\displaystyle \sum u_i d_i \\*[2ex]
\displaystyle \sum v_i d_i 
\end{array}
\right ].
\end{equation}
	Going up one more dimension means we want to find a hyper plane in a 4-D coordinate 
system.  That situation become difficult to envision but easy to construct.  As an example, consider 
measurements of temperature of points described by  $(x,y,z)$ triplets.  One can easily find the 
equation of the hyper-plane
\begin{equation}
	T = m_1 + m_2 x + m_3 y + m_4 z
\end{equation}
that best fits the data in a least square sense.

\subsection{Multiple regression}

In general, we will be considering a linear regression 
with $m+1$ independent variables, and our regression model will be of the form
\begin{equation}
m_0 + m_1 x_1 +  m_2 x_2 + \cdots + m_m x_m + \epsilon = d.
 \end{equation}	 	
Our model states that the observations $d_i$ can be explained by a constant term $m_0$ plus a linear 
combination of $m$ variables, with each variable having its own ``slope''.  Thus, $m_4$ describes how 
fast $d$ changes when $x_4$ changes, holding all other $x_i$ fixed.  As usual, we explain the misfit as 
being due to deviations $\epsilon_i$, which we hope are normally distributed with zero mean.
	
In many situations it is obvious which values are the observations and which are the 
variables.  In our example with temperatures in $x-y-z$ we wanted to  find how $T$ varies as a 
function of position.  However, in other cases it may be less clear-cut.  E.g., if we measure (depth, 
porosity, permeability, water content) in a core, we will probably let depth be a variable but which 
of the others do we pick as our $d$ observation?  In the general case this becomes arbitrary.  It also becomes 
arbitrary to measure the misfit in the $d$-direction only.  The extension of orthogonal regression to 
multiple regression is then a solution. In that case we would want to minimize the shortest distance 
between the data points and the hyper plane.  To prevent unseemly bleeding from the ears, we will not attempt this approach here.
	
Once we have selected our $d$-values (and used the symbol $x_{ij}$ to represent the $i$'th observation
of the $j$'th variable), we set up the problem in matrix form as
\begin{equation}
\left [ \begin{array}{ccccc}
1 & x_{11} & x_{12} &  \cdots & x_{1m} \\
1 & x_{21} & x_{22} &  \cdots & x_{2m} \\
\vdots & \vdots & \vdots & \cdots & \vdots \\
1 & x_{n1} & x_{n2} & \cdots & x_{nm} \end{array}\right ]
\left [ \begin{array}{c}
m_0\\ m_1\\ \vdots \\ m_m \end{array} \right ]  = 
\left [ \begin{array}{c}
d_1 \\ d_2\\ \vdots \\  d_n
\end{array} \right ],
\end{equation}	 
or $\mathbf{G\cdot m = d}$.  This is nothing more than our general least squares problem, whose solution is known to be 
\begin{equation}
\mathbf{m} = [ \mathbf{G}^T\mathbf{G} ]^{-1} \mathbf{G}^T\mathbf{d}.
\end{equation}	 	
However, in multiple regression we want to determine the \emph{relative importance} of the independent variables 
as predictors of the dependent variable $d$.  The values of the regression coefficients tells us little, 
since they depend on the units chosen.  Also, if the $\bar{x}_j$'s are very different in magnitude we may lose 
precision due to round-off errors.  Consequently, we choose to transform our data into normal scores
\begin{equation}
x'_{ij} = \frac{x_{ij} - \bar{x}_j}{s_j} \mbox{ and } d'_i = \frac{d_i - \bar{d}}{s_d}.
\end{equation}
where
\begin{equation}
s_j = \sqrt{ \frac{1}{n-1} \sum ^n _{i=1} (x_{ij} - \bar{x}_j)^2} \mbox{ and } s_d = \sqrt{ \frac{1}{n-1} \sum ^n _{i=1} (d_i - \bar{d})^2}
\end{equation}	 
are the sample standard deviations of the $j'$th variable and $d$, respectively.  To populate the normal equation matrix we must form 
$\mathbf{G}^T\mathbf{G}$ and carry out the summations.  
You will remember that the form of the  normal matrix equations is
\begin{equation}
\left [ \begin{array}{ccccc}
n &  \sum x_1 &  \sum x_2 & \cdots &  \sum x_m \\[5pt]
 \sum x_1 &  \sum x^2_1 & \sum x_1 x_2 & \cdots &  \sum x_1 x_m \\[5pt]
 \sum x_2 &  \sum x_2 x_1 & \sum x^2_2 & \cdots &  \sum x_2 x_m \\[5pt]
\vdots & \vdots & \vdots & \cdots & \vdots \\[5pt]
 \sum x_m &  \sum x_m x_1  & \sum x_m x_2 & \cdots &  \sum x^2_m 
\end{array} \right ] 
\left [ \begin{array}{c}
m_0\\ m_1\\ \vdots \\ m_m \end{array} \right ]  = 
\left [
\begin{array}{c}
 \sum d \\[5pt]  \sum x_1 d \\[5pt]  \sum x_2 d \\[5pt] \vdots \\[5pt] \sum x_m d
\end{array} \right ] ,
\label{eq:mregsystem}
\end{equation}
with the sums implied to include all the data points (i.e., over $i = 1,n$).
The effect of normalizing is two-fold:
\begin{enumerate}
\item	Because the linear sums over $x_i$ and $d_i$ now are zero, $m_0' = 0$, and we must find only $m$ coefficients $m_1'$, ..., $m_m'$ instead of $m + 1$, as originally 
intended.  Consequently, the first row and first column of the system (\ref{eq:mregsystem}) are removed.
\item	The rest of the matrix becomes proportional to the correlation matrix, $\mathbf{C}$.
\end{enumerate}
For instance, examining the terms in the first equation in the normal system, i.e.,
\begin{equation}
\sum \frac{(x_{i1} - \bar{x}_1) ^2}{s^2_1} \quad \sum \frac{(x_{i1} - \bar{x}_1) (x_{i2} - \bar{x}_2)}{s_1s_2} \quad \cdots \quad \sum 
\frac{(x_{i1} - \bar{x}_1)(d_i - \bar{d})}{s_1s_d}
\end{equation}	 
we note they are simply
\begin{equation}
\frac{(n-1)s^2_1}{s^2_1}  \quad \frac{(n-1)s_{12}}{s_1 s_2} \quad \cdots \quad \frac{(n-1)s_{1d}}{s_1 s_d}.
\end{equation}	 
Since the constant $(n-1)$ appears in all terms we can delete it and find
\begin{equation}
\left [ \begin{array}{ccccc}
1 & r_{12} & r_{13} & \cdots & r_{1m} \\
r_{21} & 1 & r_{23} &  \cdots & r_{2m} \\
r_{31} & r_{32} & 1 &  \cdots & r_{3m} \\
\vdots & \vdots & \vdots & \ddots & \vdots \\
r_{m1} & r_{m2} & r_{m3} & \cdots & 1 \end{array} \right ]
\left [
\begin{array}{c}
m'_1 \\ m'_2 \\ \vdots \\ m'_m \end{array} \right ] =
\left [ \begin{array}{c}
r_{1d} \\ r_{2d} \\ r_{3d} \\ \vdots \\ r_{md}
\end{array} \right ]
\end{equation}	 	
or $\mathbf{C} \cdot \mathbf{m'} = \mathbf{r}$.  Solving $\mathbf{m' = C}^{-1}\mathbf{r}$ we recover the unscaled regression parameters
\begin{equation}
m_j = m'_j \frac{s_d}{s_j},
\end{equation}	 
and we can reconstruct the missing intercept via
\begin{equation}
m_0 = \bar{d} - m_1 \bar{x}_1 - m_2 \bar{x}_2 - \cdots - m_m \bar{x}_m.
\end{equation}	 
As an indicator of the goodness-of-fit we use the \emph{coefficient of multiple regression},
\begin{equation}
R^2 = SS_R/SS_T = \sum^n_{i=1} (\hat{d}_i - \bar{d})^2 / \sum^n_{i=1} (d_i - \bar{d})^2,
\end{equation}	 
where $\hat{d}_i = d(x_i)$ is the predicted value.  For simple 2-D regression,
\begin{equation}
R^2 = r^2 = \frac{s_{xd}}{s^2_x s^2_d}.
\end{equation}	 
	While a simple inspection of the $m_j'$ can tell you which parameters seem to be most important, 
it is generally not enough to determine the best regression equation.  Obviously, if any of the $m_j'$ 
are close to or equal to zero we can say that the corresponding variable $x_j$ is largely \emph{irrelevant} to the 
regression and remove it from the system.  At other times, some of the variables may be linearly 
related to each other, leading to \emph{redundant} variables that should be removed.  One approach is to 
evaluate all possible regressions using all combinations of $x_j$'s and determine how many variables are 
necessary.
\index{Redundant variable}
	We can use an ANOVA test to resolve whether a coefficient is significant or not.  Suppose we have 
carried out a multiple regression for $p$ independent variables.  Next, we add $q$ additional variables and 
repeat the regression.  We would like to test whether these new variables are significant.  We construct
the ANOVA table given as Table~\ref{tbl:F_ANOVA}.
\begin{table}[H]
\center
\begin{tabular}{|l|c|c|c|c|} \hline
\bf{Source of Variation} & \bf{Degrees of Freedom} & \bf{Sums of Squares} & \bf{Mean Square} & $F$ \\ \hline
{1st regression} & $p$ & $SS_{R1}$ & &\\[3pt] \hline
{2nd regression} & $p + q$ & $SS_{R2}$ & & \\[3pt] \hline
{Difference between} & $q$ & $\Delta SS_R = $ & $MSR = \Delta SS_{R}/q$ & $\frac{MSR}{MSE}$ \\[3pt]
{1st and 2nd regression} & & $SS_{R2} - SS_{R1}$ & & \\[3pt] \hline
{Residuals} & $n - p - q - 1$ & $SSE$ & $MSE = SSE/(n - p - q - 1)$ & \\[3pt] \hline
{Total} & $n - 1$ & $SS_T$ & & \\ \hline
\end{tabular}
\caption{ANOVA table used to determine whether or not the addition of $q$ extra parameters results in
a significant reduction in misfit (i.e., a significant improvement in variance explained).}
\label{tbl:F_ANOVA}
\end{table}
\index{\emph{F}-test}
\index{Test!\emph{F}}
\noindent
The observed $F$ value is
\begin{equation}
F = \frac{\Delta SS_{R}/q}{SSE/(n-p-q-1)}.
\end{equation}	 
With explained $SS$ in \% defined as $ESS = 100 R^2$ we find
\begin{equation}
F = \frac{(ESS_{p+q} - ESS_p)/q}{(100 - ESS_{p+q})/(n-p-q-1)}.
\end{equation}	 	
Comparing the observed $F$ statistic to a table of critical values (see Appendix~\ref{sec:Ftables})
will tell us if the new $q$ parameters produce a significant improvement or not.
\begin{example}	
\PSfig[h]{Fig1_CanadaGold}{Geological map of part of Superior province, Canadian shield (After G.S.C.-Map no 1250A by R.J.W. Douglass).
Area of interest is indicated by the heavy outline.}
We will use multiple regression to look for a linear relation between the occurrences of gold 
and various lithological units in the area west of Quebec (Agterberg, 1974).  By overlying a grid on the geological 
map, we can estimate the density of gold occurrence per grid cell and find the area percentage of 
the rock units in the same cell.  The gold densities are our ``observed'' data $d_i, i=1,n$ and the rock types 
our ``variables'' $x_j, j = 1,m$.  We chose six types of rocks ($m=6$) for the mapping, as listed in Table~\ref{tbl:rocks}.
\begin{table}[H]
\center
\begin{tabular}{|c|l|} \hline
\bf{Variable} & 	\bf{Lithology} \\ \hline
$x_1$ & 	Granitic rocks, gneisses, acidic intrusive rocks \\ \hline
$x_2$ & 	Mafic intrusive rocks (gabbros and diorites)\\ \hline
$x_3$ &		Ultramafics\\ \hline
$x_4$ & 	Early Precambrian sedimentary rocks\\ \hline
$x_5$ &  	Acidic volcanics (rhyolites and pyroclastics)\\ \hline
$x_6$ & 	Mafic volcanics (basalts and andesites)\\ \hline
\end{tabular}
\caption{List of variable names and the corresponding rock types used in the regression for gold densities.}
\label{tbl:rocks}
\end{table}
The counting resulted in $n = 113$ cells, so $\mathbf{G}$ has dimensions $113 \cdot 7$ (i.e., six factors plus the intercept).  However, 
because of the constraint that
\begin{equation}
\sum^6_{j=1} x_{ij} = 100
\end{equation}	 
for all blocks (required to equal a total of 100\%), the rock type percentages are linearly dependent 
and we will not be able to invert $\mathbf{G}^T\mathbf{G}$.  We will instead obtain all multiple regression 
solutions where we regress $d$ on $x_1$, $d$ on $x_2$, etc., for all six variables.  Next, we try all possible 
combinations of two variables, then combinations of three, four, and finally five (using all six is not possible
since they are dependent.)  We find that we need to perform $N_r$ individual regression analyses, given by
\begin{equation}
N_r = \left ( \begin{array}{c}
6 \\ 1 \end{array} \right ) + \left ( \begin{array}{c}
6 \\ 2 \end{array} \right ) + \left ( \begin{array}{c}
6 \\ 3 \end{array} \right ) + 
\left ( \begin{array}{c}
6 \\ 4 \end{array} \right ) + 
\left ( \begin{array}{c}
6 \\ 5 \end{array} \right )  = \sum ^5 _{j=1} 
\left ( \begin{array}{c}
6 \\ j \end{array} \right ) = 62.
\end{equation}	 
This number $N_r$ goes up rapidly with the number of coefficients.  We make a table of the 
results where we tabulate the parameter numbers and the associated ESS (Table~\ref{tbl:mregress62}).
\begin{table}[H]
\center
\begin{tabular}{|cr|cr|cr|cr|} \hline
$x_j$	& 	\bf{ESS} (\%) &	$x_j$	& 	\bf{ESS} (\%) &	$x_j$	& 	\bf{ESS} (\%) &	$x_j$		& 	\bf{ESS} (\%) \\ \hline
1	&	8.64	&	3,5	&	16.29	&	2,3,5	&	16.43	&	1,3,4,6		&	25.61 \\ \hline
2	&	1.45	&	3,6	&	0.33	&	2,3,6	&	1.79	&	1,3,5,6		&	28.37 \\ \hline
3	&	0.19	&	\bf{4,5}	&	\bf{28.15}	&	2,4,5	&	28.18	&	1,4,5,6		&	28.61 \\ \hline
4	&	10.48	&	4,6	&	10.56	&	2,4,6	&	12.46	&	\bf{2,3,4,5}		&	\bf{29.40} \\ \hline
\bf{5}	&	\bf{15.56}	&	5,6	&	16.07	&	2,5,6	&	16.14	&	2,3,4,6		&	12.78 \\ \hline
6	&	0.17	&	1,2,3	&	8.95	&	\bf{3,4,5}	&	\bf{29.32}	&	2,3,5,6		&	16.84 \\ \hline
1,2	&	8.85	&	1,2,4	&	14.33	&	3,4,6	&	10.96	&	2,4,5,6		&	28.18 \\ \hline
1,3	&	8,76	&	1,2,5	&	19.21	&	3,5,6	&	16.72	&	3,4,5,6		&	29.33 \\ \hline
1,4	&	13.54	&	1,2,6	&	22.60	&	4,5,6	&	28.15	&	\bf{1,2,3,4,5}	&	\bf{29.41} \\ \hline
1,5	&	18.86	&	1,3,4	&	13.79	&	1,2,3,4	&	14.56	&	1,2,3,4,6	&	29.41 \\ \hline
1,6	&	22.52	&	1,3,5	&	14.40	&	1,2,3,5	&	19.83	&	1,2,3,5,6	&	29.41 \\ \hline
2,3	&	1.58	&	1,3,6	&	22.58	&	1,2,3,6	&	22.65	&	1,2,4,5,6	&	29.41 \\ \hline
2,4	&	12.42	&	1,4,5	&	28.19	&	1,2,4,5	&	28.24	&	1,2,4,5,6	&	29.41 \\ \hline
2,5	&	15.65	&	1,4,6	&	24.87	&	1,2,4,6	&	27.45	&	2,3,4,5,6	&	29.41 \\ \hline
2,6	&	1.68	&	1,5,6	&	28.34	&	1,2,5,6	&	29.37	&			&	      \\ \hline
3,4	&	10.85	&	2,3,4	&	12.72	&	1,3,4,5	&	29.32	&			&	      \\ \hline
\end{tabular}
\caption{Percentage explained sum of squares (\emph{ESS}) for 62 possible regressions, with density of gold
occurrences regressed on up to five lithological variables. Bold entries are the best combinations (from \emph{Agterberg}, 1974).}
\label{tbl:mregress62}
\end{table}
Examining all the regression results we will focus on each single combination that gave the highest ESS for that number of
parameters (bold entries in Table~\ref{tbl:mregress62}); these are isolated in the separate Table~\ref{tbl:xj_fits}.
\begin{table}[H]
\center
\begin{tabular}{|l|c|} \hline
$x_j$ combination & \bf{ESS} (\%) \\ \hline
5 & 15.56\\ \hline
4, 5 & 28.15 \\ \hline
3, 4, 5 & 29.32 \\ \hline
2, 3, 4, 5 & 29.40 \\ \hline
1, 2, 3, 4, 5 & 29.41 \\ \hline
\end{tabular}
\caption{Each row shows the best parameter combination of 1--5 variables and the corresponding percentage of
explained variation (ESS).}
\label{tbl:xj_fits}
\end{table}
We are now in the position to test the solutions for significance.  First, we will take a look at the best
one-term $(x_5)$ solution.  With $p = 0, q = 1, n = 113$, we find
\index{\emph{F}-test}
\index{Test!\emph{F}}
\begin{equation}
F = \frac{(15.56 - 0)/1}{(100 - 15.56)/(113 -1 -1)} = 20.46.
\end{equation}	 
From the table, $F_{0.95,1,111} = 3.93$, so it is clear that $x_5$ (acidic volcanics) is a significant component.  Will the 
addition of $x_4$ lead to a significant improvement?  Now, $p = 1, q = 1$, so 
\begin{equation}
F = \frac{(28.15 - 15.56)/1}{(100 - 28.15)/(113 - 1 -1 -1 )} = 19.72.
\end{equation}
This also far exceeds $F_{0.95,1,110} = 3.93$.  Going one step further, we check if $x_3$ should be 
included.  Here, $p = 2$ and $q = 1$, so
\begin{equation}
F = \frac{(29.32 - 28.15)/1}{(100 - 29.32)/(113 - 2 - 1 -1 )} = 1.80.
\end{equation}	 
The critical $F_{0.95,1,109}$ is still 3.93, so we must conclude that $x_3$ is an irrelevant variable, and 
obviously there is no need to consider any other of the remaining $x_j$'s that gave poorer fits still.  The result of this analysis 
is that the density of gold occurrences may be explained by a linear model that depends on the 
presence of acidic volcanics and Precambrian sediments in the area.  
However, the correlation is not all that impressive $(R^2 = 28.15\%)$, suggesting that rock type 
alone is but one possible indicator of prospective regions.  Applying this technical concept to a
unexplored area may produce some disappointment.
\end{example}
\index{Multiple regression|)}
\index{Regression!multiple|)}

\clearpage
\section{Problems for Chapter \thechapter}

\begin{problem}
The table \emph{hawaii.txt} contains triplets of (distance, age, $\Delta$age) values
obtained along the Hawaiian seamount chain.  Distance is small-circle distance from Kilauea (in km),
age is radiometric seamount/island dates (in My) with associated one-sigma uncertainties (in My).
You will also need the \texttt{regress\_ls.m} function for this problem.
\begin{enumerate}[label=\alph*)]
\item  Plot the data points with error bars (hint: see help \texttt{errorbar}).
\item  Find the simple least squares regression line, state the parameter values, and overlay the predictions on your plot.
\item  Find the weighted least squares regression line, state the parameter values, and overlay the predictions on your plot.
\item What is the geological meaning of the slopes and intercepts?  Comment on the values.
\end{enumerate}
\end{problem}

\begin{problem}
During a calibration, nine pairs of readings were obtained with a UV spectro-photometer used to measure chemical oxygen demand
in waste water (in mg/l; see file \emph{COD.txt}) versus UV absorbance.  Remove the mean from each coordinate set and
determine the least-squares regression fit.
Plot the data and the least-squares regression line.  Compute the
uncertainty in the slope using (\ref{eq:err_in_slope}) and plot the two regression lines going through the origin that
correspond to this range in slopes.  Finally, determine the orthogonal regression solutions (both Major Axis and Reduced Major Axis)
and plot them as well.  Do these fall within the
range of acceptable least-squares regression trends?
\end{problem}

\begin{problem}
	\newcounter{Conrad}
	\setcounter{Conrad}{\theproblem}
	\newcounter{Conradchap}
	\setcounter{Conradchap}{\thechapter}
The file \emph{c2407.txt} contains distance $x$ (in km), bathymetry $z$ (in m), and crustal age $t$ (in Myr) along a
segment across the Mid-Atlantic Ridge.  
\begin{enumerate}[label=\alph*)]
\item	Find the deterministic component given by the theoretical prediction
$$
z = d_r + c \sqrt{t}.
$$ 
	and report the parameters.  Note $z$ is a function of age, not distance.
\item	Plot the bathymetry and your best-fitting deterministic component on the same plot as functions of distance (not age).
\end{enumerate}
\end{problem}

\begin{problem}
\newcounter{RSL}
\setcounter{RSL}{\theproblem}
The files \emph{hilo.txt} and \emph{honolulu.txt} contain tide gauge 
readings (the first column is time (in years), the second is monthly average values in mm relative to some
arbitrary origin) from Hilo and Honolulu harbors, respectively.  It is 
believed that the observations represent the sum of two phenomena: (1) eustatic (global) sea-level 
variations, and (2) tectonic subsidence of the Big Island.
\begin{enumerate}[label=\alph*)]
\item Plot the data.  Use \texttt{regress\_ls.m} to determine the regression lines for the two series.  Superimpose 
these lines on the graphs and label the plots with the values of the regression slopes.  What do 
these slopes represent?
\item Assuming Oahu is tectonically stable, what is the tectonic subsidence rate at Hilo?
\item Alternatively, we only use the time period that the two data series have in common and directly 
subtract the Honolulu series from the Hilo series.  What is the tectonic subsidence rate based on 
these differences, and how does it compare to your answer in (2)?
\end{enumerate}
\end{problem}

\begin{problem}
	\index{Reweighted least squares (RLS)}
Data sets showing the separate weights of an animal and its brain present a strong correlation in log-log space (\emph{bb\_weights.txt}).
Use the methodology of ``reweighted least squares'' (RLS) to determine this regression and calculate residual $z$-scores
relative to the RLS regression. With a threshold of 2.5, which animals depart from the simple trend? How would you
explain these outliers?
\end{problem}

\begin{problem}
	The file \emph{sherwood.txt} contains a series of petrophysical measurements on samples
	of the Triassic Sherwood sandstone [from Lourenfemi, M.O., 1985, \emph{Math Geol., 17}, 845--452.]
	The five columns contain permeability (mm/s), porosity ($\phi$), matrix
	conductivity, true formation factor, and induced polarization.  Using multiple regression,
	how many of the last four parameters are significant in a regression to explain the permeability,
	assuming a 95\% of confidence?
\end{problem}
