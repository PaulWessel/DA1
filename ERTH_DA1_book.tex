%------------------------------------------
%	Introduction to Statistics and Data Analysis - Part I
%	Copyright 1992-2023.
%	Paul Wessel
%------------------------------------------
%
% This almost empty document includes all the actual
% files with material.  It is the master
% file that sets up the general layout of the book

\ifdefined\mypdfbook
   \documentclass[10pt,letter]{report}
   \usepackage[margin=1in]{geometry}
\else
   \documentclass[10pt,oneside,openany,a5paper]{book}
   \usepackage[margin=0.25in]{geometry}
   %\usepackage[utf8x]{inputenc}
\fi

\pdfminorversion=7
\usepackage{ifpdf}
\usepackage{amsmath}
\usepackage{epsfig}
\usepackage{makeidx}
\usepackage{float}
\usepackage{times}
\usepackage{mathptm}
\usepackage{afterpage}
\usepackage{xcolor}
%\usepackage{newtxmath}
\usepackage{enumitem}
\usepackage{epigraph}
\usepackage{url}
\usepackage{bm}
\usepackage{wrapfig}
\usepackage[labelfont=bf]{caption}
% Special commands & environments for this book
%
%	PSfig will insert an eps file, add a label, and set a caption
%
\ifpdf
   % Here for creating PDF output with pdfLaTeX
   \usepackage{hyperref}
   \DeclareGraphicsExtensions{.pdf}
   \newcommand{\PSfig}[3][tbp]{\begin{figure}[#1] \centering \epsfig{figure=pdf/#2} \caption{#3} \label{fig:#2} \end{figure}}
   \newcommand{\PSfigplace}[1]{\epsfig{figure=pdf/#1.pdf}}
   \newcommand{\PSfignarrow}[3]{\begin{wrapfigure}{#1}{0.5\textwidth} \setlength\intextsep{0pt} \centering \epsfig{figure=pdf/#2} \caption{#3} \label{fig:#2} \vspace{-12pt} \end{wrapfigure}}
   \pdfcompresslevel=9
   \hypersetup{%
      pdfauthor={Paul Wessel},
      pdftitle={INTRODUCTION TO STATISTICS \& DATA ANALYSIS},
      pdfsubject={Introduction to Statistics \& Data Analysis},
      pdfkeywords={DATA ANALYSIS, statistics, hypothesis testing, modeling},
      pdfcreator={pdfLaTeX},
      bookmarksopen=true,
      bookmarksnumbered=true,
      hypertexnames=true,
      breaklinks=true,
      %pdfstartview={FitH},
      %linkbordercolor={1 1 0},
      %urlbordercolor={1 0 0},
   }%
\else
   % Here for creating PS or HTML output with LaTeX
   \usepackage[margin=0.5in]{geometry}
   \DeclareGraphicsExtensions{.svg}
   \newcommand{\PSfig}[3][tbp]{\begin{figure}[#1] \centering \epsfig{figure=svg/#2.svg} \caption{#3} \label{fig:#2} \end{figure}}
   \newcommand{\PSfigplace}[1]{\epsfig{figure=svg/#1.svg}}
   %\DeclareGraphicsExtensions{.png}
   %\newcommand{\PSfig}[3][tbp]{\begin{figure}[#1] \centering \epsfig{figure=png/#2.png} \caption{#3} \label{fig:#2} \end{figure}}
   %\newcommand{\PSfigplace}[1]{\epsfig{figure=png/#1.png}}
\fi

%%------------ THESE COMMANDS TO BE USED FOR POSTSCRIPT --------
\newcommand{\DS}{$^{\circ}$}
\newcommand{\DSm}{^{\circ}}
\newcounter{example}[chapter]
\newcounter{problem}[chapter]
\renewcommand{\bf}{\textbf}
%\newcommand{\important}[1]{\emph{#1}\index{#1}}

\newenvironment{example}{\stepcounter{example}\\\noindent\rule{\textwidth}{.01in}\\\textbf{Example~\thechapter--\theexample}.}{\\\noindent\rule{\textwidth}{.01in}\linebreak}
\newenvironment{example2}{\stepcounter{example}\noindent\rule{\textwidth}{.01in}\\\textbf{Example~\thechapter--\theexample}.}{\\\noindent\rule{\textwidth}{.01in}\linebreak}

\newenvironment{problem}{\stepcounter{problem}\noindent\\\textbf{Problem~\thechapter.\theproblem}.}{}
\DeclareMathOperator\erf{erf}
\DeclareMathOperator\sinc{sinc}

%--------------------------------------------------------------------------

\makeindex
\usepackage[totoc]{idxlayout}

\begin{document}

\pagenumbering{roman}

%\maketitle

\include{DA1_Version}
%------------------------------------------
%	$Id: DA1_cover.tex 631 2018-07-28 21:03:49Z pwessel $
%
%------------------------------------------
%
\thispagestyle{empty}
\definecolor{mycolor}{rgb}{0.95,1,0.95}
\pagecolor{mycolor}\afterpage{\nopagecolor}
\begin{center}
\LARGE
\vspace{2\baselineskip}
\ifpdf
\textbf{INTRODUCTION TO STATISTICS \& DATA ANALYSIS}\par 
\else
\textbf{\title{INTRODUCTION TO STATISTICS \& DATA ANALYSIS}}\par
\fi
\Large
\vspace{4\baselineskip}

\PSfigplace{Fig1_Cover} 
\vspace{2\baselineskip}

\Huge
\ifpdf
\textbf{PAUL WESSEL}\par 
\else
\textbf{\author{Paul Wessel}}\par 
\fi
\clearpage
\Large
\vspace{3\baselineskip}
\textbf{Paul Wessel}\par 
\textbf{Department of Earth Sciences}\par 
\textbf{School of Ocean and Earth Science and Technology}\par 
\textbf{University of Hawai'i at M\={a}noa}\par 
\large
\vspace{3\baselineskip}
\epsfig{figure=orig/soest_oval_textblock_307C_white_LL3_no-pre.pdf,width=3in}\par
\vspace{3\baselineskip}
\textbf{Version 2.\DAversion\ --- Revised \DAmonth\ \DAday, \DAyear}\par 
\vspace{\baselineskip}
\normalsize
\textbf{Copyright \copyright\ 1992--\DAyear\ by P. Wessel}\par 
\end{center}
\vspace{1\baselineskip}
Dedicated to Jill, Erik and Malia for their kindness and willingness to let me work on interesting problems.


\clearpage

\pagestyle{plain}
\addcontentsline{toc}{chapter}{Contents}
\tableofcontents

%\clearpage
%\addcontentsline{toc}{chapter}{List of tables}
%\listoftables

%\clearpage
%\addcontentsline{toc}{chapter}{List of figures}
%\listoffigures

% $Id: DA1_Preface.tex 630 2018-07-28 20:59:29Z pwessel $
%
\chapter*{Preface}
\addcontentsline{toc}{chapter}{Preface}

I initially developed this book in support of an introductory course in statistics
and data analysis taken by many of our undergraduate majors and some graduate students in the Department of
Earth Sciences in the School of Ocean and Earth Science and Technology at
the University of Hawaii at M\={a}noa.  Over the years, I have
expanded them to support undergraduate students in the broad area of the natural sciences
anywhere.  There were several goals I tried to meet in designing this book:
\begin{enumerate}
\item I wanted to introduce students of science and environmental engineering to some
   of the most common methods used in the examination and analysis of simple data sets
   encountered in the sciences.  By learning these techniques your data analysis tool
   chest will expand and prepare you for further learning when you need it.
\item I sought to fill in many of the intermediate steps in derivations that traditional
   textbooks will skip.  Thus, it should be much easier to trace these derivations since there
   are no annoying messages of the type ``the derivation has been left as an exercise for the reader.''
   The more elaborate derivations also help potential instructors using this book to present such
   material at the level of detail they require.
\item I wished the book to be affordable.  Publishing it as a digital book directly by the author
   means it costs a fraction of a traditional, professionally produced textbook.  This method also enables timely updates
   and easy corrections, at no extra cost to the purchaser.  Of course, the flip-side of this decision
   is that the book has not benefited from the help of professional editors and graphics artists
   (but it \emph{has} been extensively reviewed by scientists familiar with the methodologies).
   In the end, the user will have to decide if the benefits outweigh any real or perceived drawbacks.
\item I hoped to make the book suitable for introductory college level courses in data analysis and statistics, with
   examples of analysis using real data sets.  Thus, I have added a
   variety of assignments at the end of each chapter, for which data sets are available from my website.
\end{enumerate}

As a science student or curious bystander, why should you consider this book?  While many
general answers may be given, including the relatively low cost, I believe some are particularly relevant:

\begin{enumerate}
\item The natural and environmental sciences have become very data intensive, with huge data sets readily accessible
   over the Internet or being collected by improved instrumentation capable of high sampling rates.  Consequently, old-fashioned
   hand analyses are no longer practical undertakings.
\item The job market requires quantitative skills, in particular for those pursuing a career in the natural
   or environmental sciences.  You will be dead in the water if you cannot juggle data to some extent.
\item All sciences need reproducibility of analyses for the testing of relevant hypotheses.
   This concept goes to the heart of what science and the scientific method are all about.
\item The increased sophistication of newer instruments is transforming visual
   characterizations into numbers that need to be analyzed quantitatively, hence new methods of analyses
   continue to be developed.
\end{enumerate}

What is the target audience for this book?  There are numerous candidates,
and I recommend the book specifically to:

\begin{enumerate}
\item Anyone ever planning to analyze data of any kind. This is of course a pretty broad statement but I believe it is true.
   Once you get into the habit of thinking and working quantitatively, there is no going back.
\item Anyone who wants to be prepared for a changing job market, considering that you are competing for
   opportunities with other students who likely have been exposed to similar material.
\item Budding scientists, engineers and technical personnel, especially those fearful of mathematics. Many students
   tend to avoid courses and treatises that expose them to mathematics, thus limiting what they
   can achieve as scientists.  I hope, by showing all the steps in the derivations and presenting the coupling of
   mathematics to data sets, that this book will help alleviate such fears.
\end{enumerate}

The main purpose of this book is to prepare you for facing and dealing with data, their limitations and
the methods of analysis that may be most suitable for different types of data.  I hope to achieve that goal by
a multiprong series of attacks:

\begin{enumerate}
\item Expose you to many different data analysis techniques and thus broaden your horizon. (No book can cover
   everything but being aware of other approaches allows you to pursue alternative
methodologies when the need arises.)
\item Make you appreciate why you should fully understand a technique's nuts and bolts before running a ``black
   box'' operator, such as most typical software packages, on your data.
\item Except for the simplest problems, instill a preference for using computers to solve data
   analysis problems, since many data sets may simply be too large and too numerous to process manually.
\item Make you comfortable with applied mathematics at an intermediate level.  The mathematics we employ
   in this book is mostly algebra, trigonometry and calculus, with an introduction to matrix algebra.
   There are numerous data analysis techniques that are simply exercises in applied matrix algebra.
\item Make you comfortable with the tools of the trade, such as MATLAB, R, Python, or Octave for your data analysis needs
   rather than depending on ``business software'', such a spreadsheets.
   However, this book is not a recipe collection of algorithms in these languages, but rather presents equations and the
   assumptions used to pursue specific analysis goals or tests.  The author's data analysis website
   (\url{http://www.soest.hawaii.edu/pwessel/DA}) contains links to all the
   data sets used in this book as well as any MATLAB example code discussed. (Additional modules may be added over time.)
\end{enumerate}

Finally, data analysis is a very broad subject and I am certainly not an expert in all the available
techniques. In fact, I am probably not an expert in any of them [:-(].  My goal is to give you a flavor
of what is available, show you how to find out more, and enable you to make sensible decisions on how
to approach and analyze your data.

This book is based on my collection of course notes that was inspired by several sources.  Here are the ones I have
found most useful:
\begin{enumerate}
\item Course notes from a class in data analysis at Columbia University during my graduate school days, developed by Doug Martinson
	   at Lamont-Doherty Earth Observatory, provided a clear introduction to spectral analysis.  His willingness to
	let me use some of his early material in this book is gratefully acknowledged.
\item John C. Davis' textbook, \emph{Statistics and Data Analysis in Geology, 3rd edition} available from John Wiley and Sons.
   His textbook has numerous useful problems complete with data sets that are now in the public domain
   (\url{http://www.kgs.ku.edu/Mathgeo/Books/Stat}). A few of these are used in this book.
\item \emph{Exploratory Data Analysis} by John W. Tukey, available from Addison-Wesley Publishing Company, outlines the basics
   of exploratory data analysis.
\item \emph{An Introduction to Error Analysis} by John R. Taylor, available from University Science Books, reviews the
   various rules for the propagation of errors in compound expressions.
\item \emph{Numerical Recipes} by Press et al., available from Cambridge University Press in multiple language flavors,
   clarifies many statistical procedures with a dose of nerd humor.
\item \emph{Robust Regression and Outlier Detection} by Peter. J. Rousseeuw and Annick M. Leroy, available from John Wiley and Sons, is a classic
   for understanding why using robust methods is imperative when dealing with actual data.
\item \emph{Applied Regression Analysis} by Norman R. Draper and Harry Smith, also available from John Wiley and Sons, deals with
   conventional least squares regressions.
\item \emph{Developments in Geomathematics} by Frederik P. Agterberg, available from Elsevier, deals with
   a variety of topics, including multiple regressions.
\item My cumulative experience with analyzing marine geophysical and plate kinematic data since 1985.
\end{enumerate}

Depending on the time available, this book may cover more material than can be presented in a single course.  However,
many of the topics may be of interest to you at a later stage, hence I hope the book may serve a valuable
purpose as a reference.

I am grateful to my SOEST colleagues, past and present (Fred Duennebier, Robert Dunn, Neil Frazier,
Garrett Ito, Julia Morgan, and Cecily Wolfe) who have used these notes when teaching
courses and provided me with valuable suggestions and error corrections.  Likewise, numerous former students have
also provided valuable feedback by pointing out typographical errors in the equations, odd or foul language,
stale jokes, or confusing illustrations.  Obviously, I alone am to blame for any remaining errors.

I am also grateful to our (now retired) secretary, Evelyn Norris,
who taught herself enough \LaTeX\ to help typeset large portions of this manuscript from printouts of clumsy Microsoft Word documents
with obsolete equation editing plug-ins.  As software come and go, \LaTeX\ remains solid and postprocessing tools allow
for new output formats, such as the digital version you are reading.  All illustrations are data-driven, original creations using the
Generic Mapping Tools (GMT; \url{http://gmt.soest.hawaii.edu}) via individual scripts that produce
publication-quality \emph{PostScript} documents.  These files are then converted to PDF (via \emph{Ghostscript}).  The entire
processing workflow from \LaTeX\ and GMT source code to PDF is automated.

Because it is relatively easy to update digital books as frequently as required, I hope to hear from you should you find any remaining errors
in the text or graphics.  By bringing them to my attention (\url{mailto:pwessel@hawaii.edu})
they can be corrected and updated as soon as possible.  Thank you for your interest in this book.

\vspace{2\baselineskip}
Paul Wessel, \DAmonth\ \DAyear.


\pagenumbering{arabic}
\pagestyle{headings}

% $Id: DA1_Chap1.tex 630 2018-07-28 20:59:29Z pwessel $
%
\chapter{EXPLORING DATA}
\label{ch:EDA}
\epigraph{``I never guess. It is a capital mistake to theorize before one has data. Insensibly one begins to twist facts to suit theories, instead of theories to suit facts.''}{\textit{Sherlock Holmes, Consulting Detective}}
Data come in all types and amounts, from a handful of hard-earned analytical quantities obtained after days of meticulous
work in a laboratory to terabytes of remotely sensed data simply gushing in from satellites and remotely operated vehicles.
It is therefore desirable to have a common language to describe data and to make initial explorations of trends.

\section{Classification of Data}
\index{Data!classification}
All observational sciences require data that may be analyzed and explored, which give rise to new ideas for
how the natural world works.  Such ideas may be developed into simple hypotheses that ultimately can be tested
against new data and either be rejected or live to fight another day.  New data crush hypotheses every day,
hence we have to be careful with and respectful of our data to a much greater extent than our hypotheses and models.
We start our exploration of data by considering the various ways we can categorize data,
discussing a few basic data properties, and introducing typical steps taken in exploratory data analysis.

\subsection{Data types}
\PSfig[H]{Fig1_dataclasses}{Classification of data types.  All data we end up analyzing in a computer program
are necessarily digitized and hence discrete, but they may represent a \emph{phenomenon} that produced \emph{continuous} output.}

Data represent measurements of either discrete or continuous quantities, often called \emph{variables}.  
\emph{Discrete} variables are those having discontinuous or individually distinct possible values (Figure~\ref{fig:Fig1_dataclasses}).  
Examples of such data include:
\index{Data!discrete}
\begin{itemize}
\item 	\emph{Counts}: The flipping of a coin or rolling of dice, or the enumeration of individual items or groups of items.
	Such data can always be manipulated numerically.
\item	\emph{Ordinal} data: These data can be ranked, but the intervals between consecutive items are not constant.
  For instance, consider Moh's hardness scale for minerals (Table~\ref{tbl:Moh}): While \emph{topaz}
(hardness 8) is harder than \emph{fluorite} (hardness 4), the values do not imply a doubling in hardness.
\index{Ordinal data}
\index{Data!ordinal}
\item	\emph{Nominal} data: These data cannot even be ranked.  Examples of nominal data include categorizations or classifications,
e.g., color of items (red \emph{vs} blue marbles), lithology of rocks (sandstone, limestone, granite, etc.), and similar categorical data.
\index{Nominal data}
\index{Data!nominal}
\end{itemize}
\begin{table}[h]
\centering
\begin{tabular}{|c|c|c|} \hline
\bf{Hardness} & \bf{Mineral} &  \bf{Chemical Formula} \\ \hline
 1 &  Talc      &    Mg$_3$Si$_4$O$_{10}$(OH)$_2$     \\
 2 &  Gypsum    &    CaSO$_4\cdot$2H$_2$O      \\
 3 &  Calcite   &    CaCO$_3$     \\
 4 &  Fluorite  &    CaF$_2$     \\
 5 &  Apatite   &    Ca$_5$(PO$_4$)$_3$(OH$^-$,Cl$^-$,F$^-$)      \\
 6 &  Feldspar  &    KAlSi$_3$O$_8$     \\
 7 &  Quartz    &    SiO$_2$       \\
 8 &  Topaz     &    Al$_2$SiO$_4$(OH$^-$,F$^-$)$_2$     \\
 9 &  Corundum  &    Al$_2$O$_3$      \\
 10 &  Diamond  &        C       \\ \hline
\end{tabular}
\caption{Traditional Moh's hardness scale.}
\label{tbl:Moh}
\end{table}
We will see in the following chapters that discrete data will often require specialized handling and testing.

	\emph{Continuous} variables are those that have an uninterrupted range of possible values (i.e., with 
no breaks).
\index{Data!continuous}
Consequently, they have an infinite number of possible values over a given range.  Our instruments, however,
necessarily have finite precision and thus yield a finite number of recorded values.  Examples 
of continuous data include:

\begin{itemize}
\item	Seafloor depths, wingspans of birds, weights of specimens, and thickness of sedimentary layers.
\item	Fault strikes, directions of wind and ocean currents.
\item	The Earth's geopotential fields, temperature anomalies, and dimensions of objects.
\item 	Percentages of components (such data, which are closed and forced to a constant sum, sometimes 
require special care and attention).
\end{itemize}

	In addition, much data of interest to scientists and engineers vary as a function of \emph{time} and/or \emph{space} 
(the independent variables).  Since time and space vary continuously themselves, our discrete or 
continuous variables will most often vary continuously as a function of one or more of these independent 
variables.  Such data represent continuous \emph{time series} data.  They may also be referred to as 
\emph{signals}, \emph{traces}, \emph{records}, and other names.  We may further subdivide continuous data into sub-categories:
\begin{itemize}
\item	\emph{Ratio scale} data: These data have a fixed zero point (e.g., weights, temperatures in degrees Kelvin).
\item	\emph{Interval scale} data: These data have an arbitrary zero point (e.g., temperatures in degrees Celsius or
Fahrenheit).
\item	\emph{Closed} data: These data are forced to attain a constant sum (e.g., percentages).
\item \emph{Directional} data: These data have components (e.g., vectors or orientations in two or higher dimensions).
\end{itemize}

	Data can also be classified according to how they are recorded for use:  
Analog signals are those signals which have been recorded continuously (even though one might 
argue that this is impossible due to instrument limitations).  Discrete data are those which have been
recorded at discrete intervals of the independent variable.
In either case, data must be discretized before they can be analyzed by a computer.  Consequently, all
data which are represented in computers are necessarily discrete.

\subsection{Data limits}
\index{Data!limits}
For a variety of reasons, such as lack of time or funds, our data tend to be limited in one or more ways.
In particular, limits typically will apply to three important aspects of any data set:
\begin{description}
\item	[Domain]: No phenomena can be observed over all time or over all space, hence data have 
a limited \emph{domain}.  The domain may be one-dimensional for scalar quantities and $N$-dimensional for
data in $N$ dimensions (e.g., a spatial vector data set such as the Earth's magnetic field is three-dimensional).
\index{Data!domain}
\item	[Range]: It is equally true that no measuring technique can record (or transmit) values 
that are arbitrarily large or small.  The lower limit on very small quantities is often set by 
the noise level of the measuring instrument (because matter is quantized, all instruments 
will have internal noise).  The \emph{Dynamic Range} $(DR)$ is the range over which the data can be 
measured (or exists).  This range is usually given on a logarithmic scale measured in \emph{decibels} (dB), i.e.,
\index{Data!range}
\index{Decibels}
\begin{equation}
		DR = 10 \log _{10} \left (\frac{\mbox{maximum power}}{\mbox{minimum power}} \right ) \mbox{ dB}.
\label{eq:DBpow}
\end{equation}

	One can see that every time $DR$ increases by 10, the ratio of the maximum to minimum 
values has increased by an order of magnitude.  Strictly speaking, (\ref{eq:DBpow}) is to be applied to data 
represented as a \emph{power} measurement (a squared quantity, such as variance or square of the 
signal amplitude).  If the data instead represent \emph{amplitudes}, then the formula should be
\begin{equation}
DR = 20 \log _{10} \left (\frac{\mbox{maximum amplitude}}{\mbox{minimum amplitude}} \right ) \mbox{ dB}.
\end{equation}
	For example, if the ratio between highest and lowest voltage (or current) is 10, then 
$DR = 20 \log_{10}(10) = 20$ dB.  If these same data were represented in watts (power), and since 
power is proportional to the voltage squared, the ratio would be 100, and $DR = 
10 \log_{10} (100) = 20$ dB.  Thus, regardless of the manner in which we express our data we 
get the same result (provided we are careful).  In most cases though (except for electrical data) the first formula given 
is the one to be used (so the data extrema should be expressed as powers). Few instruments have a dynamic range greater than 100 dB.  In any case, because of the 
limited range and domain of data, any data set, say $f(t)$, can be enclosed as
\begin{equation}
	t_0 < t < t_1 \mbox{ and } |f(t)| < M,
\end{equation}
for some constant $M$. Such functions are always integrable and manageable and can be subjected to further analysis
without any special treatment.

\item [Frequency]: Finally, most measuring devices cannot respond instantly to 
sudden change.  The resulting data are thus said to be \emph{band limited}.  This means the data will not 
contain frequency information higher (or lower) than the signal representing the fastest (or slowest) response of the 
recording device.
\index{Data!frequency}
\index{Data!band-limited}
\index{Band-limited}
\end{description}

\subsection{Noise}
\index{Data!noise}
\index{Noise}
	In almost all cases, real data contain information 
other than what is strictly desired (``desired'' is a key word here since we all know the saying that 
``one scientist's signal is another scientist's noise'').  We say such data consist of \emph{samples} of
\emph{random variables}.  This statement does not mean that the data are totally random, but instead imply that the value of any 
future observation can only be predicted in a \emph{probabilistic} sense --- it cannot be exactly predicted 
as is the case for a \emph{deterministic} variable, which is completely predictable by a known law.  In 
other words, because of inherent variability in natural systems and the imprecision of experiments or 
measuring devices, if we were to give an instrument the same input at two different times we 
would likely get two different outputs due to the difference in \emph{noise} at the two different times.  In analyzing data, one must 
never overlook or ignore the role of noise, as understanding the noise level provides the key to how subtle signals we
can reliably resolve in our data.
\index{Data!deterministic}
\index{Data!probabilistic}
	One of the main goals in data analysis is to detect the signal in the presence of noise or to reduce 
the degree of noise contamination.  We therefore try to enhance the \emph{signal-to-noise ratio} (S/N), 
defined (in decibels) as
\begin{equation}
	\mbox{S/N} = 10 \log _{10} \left (\frac{\mbox{power of signal}}{\mbox{power of noise}}\right ) \mbox{ dB}.
\end{equation}
Minimizing the influence of noise during data acquisition and stacking co-registered data are some of the
approaches used to enhance the S/N ratio.
\index{Data!signal to noise}
\subsection{Accuracy versus precision}
\index{Data!accuracy}
\index{Accuracy}
\index{Data!precision}
\index{Data!bias}
\index{Precision}
	An \emph{accurate} measurement is one that is very close to the true value of the phenomenon we 
are observing.  A \emph{precise} measurement is one that has very little scatter: Repeat measurements 
will give more or less the same values (Figure~\ref{fig:Fig1_acc_prec}).  If the measured data have high precision but poor accuracy, 
one may suspect that a systematic \emph{bias} has been introduced, e.g., we are using an instrument 
whose zero position has not been calibrated properly.  If we do not know the expected value of a 
phenomenon but are trying to determine just that, it is obviously better to have accurate 
observations with poor precision than very precise, but inaccurate values, since the former will 
give a correct, but imprecise estimate while the latter will give a wrong, but very precise result!
Fortunately, we will often have a good idea of what a result should be and can use that prior knowledge to
detect any bias in the measurements.
\PSfig[h]{Fig1_acc_prec}{Precision is a measure of repeatability, while accuracy refers to how close
the average observed value is to the ``true'' value.}

\subsection{Randomness}
\index{Data!randomness}
\index{Data!deterministic}
\index{Deterministic}
\index{Data!stochastic}
\index{Stochastic}
\index{Data!probabilistic}
\index{Probabilistic}

Phenomena, and hence the data that represent them, may range from those that are completely determined
by a natural law (\emph{deterministic} data) to those that seem to have no inherent structure or patterns (\emph{chaotic} or \emph{random}).
Most phenomena, and hence the data we collect, lie in-between these two extremes.  Here, there is a more-or-less
clear pattern or structure
to the data, but each individual measurement also exhibits a random component that makes it impossible to
predict the outcome of a future observation with 100\% certainty.  Such data are called \emph{probabilistic} or
\emph{stochastic}.  We will often fit simple, deterministic models to such data and hope that the residuals
reflect insignificant, uncorrelated noise.  We will also want to know if that hope is warranted
using specific statistical tests.

\subsection{Analysis}
\index{Analysis}

	Analysis means to separate something into its fundamental components in order to identify, interpret and/or study the 
underlying structure.  In order to do this properly we should have some idea of what the 
components are likely to be.  Therefore, we should have some concept of a model of the data in mind 
(whether this is a conceptual, physical, intuitive, or some other type of model is not important).  We 
essentially need guidelines to aid our analysis.  For example, it is not a good idea to take 
a data set and simply compute its Fourier series because you happen to know something about Fourier analysis.  One 
needs to have an idea as to what to look for in the data.  Often, this knowledge will grow with 
a set of well planned ongoing analyses, whose techniques and uses are the essence of this book.

The following steps are parts of most data analysis schemes:
\begin{enumerate}
	\item \emph{Collect} or obtain the data.
	\item Perform \emph{exploratory data analysis}.
	\item \emph{Reduce} the data to a few quantities (\emph{statistics}) that \emph{summarize} their bulk properties.
	\item Compare data to various \emph{hypotheses} using statistical \emph{tests}.
\end{enumerate}
We will briefly discuss step (1) while reviewing error analysis, which is the study and evaluation of 
uncertainty in measurements and how these propagate into our final statistical estimates.  The main point of
step (2) is to familiarize ourselves with the data set and its major structure.  
This acquaintance is almost always best done by graphing the data.  Only an inexperienced analyst will use a 
sophisticated ``black-box'' technique to compile statistics from data and accept the validity of these statistics 
without actually looking at the data.  Step (3) will usually include a model (simple or 
complicated) where the purpose is to extract a few representative parameters out of possibly 
millions of data points.  These statistics can then be used in various tests (4) to help us decide 
which hypothesis the data favor, or rather \emph{not} favor.  That is the curse of statistics: You can never 
prove anything, just disprove!  By disproving all possible hypotheses other than your pet theory, 
other scientists will eventually either grudgingly accept your views or they die of old age and 
then your theory will be accepted!  Hence, persistence and longevity are important characteristics of a successful researcher.
Joking aside, it is important to listen to your data as it is disturbingly easy to be convinced that your pet model
or theory is right, data be damned.  This phenomenon is called \emph{conviction bias}\index{Conviction bias}.

\section{Exploratory Data Analysis}
\index{Exploratory data analysis|(}
\index{Data!analysis}
\index{Analysis}

	As mentioned, the main objective of exploratory data analysis (EDA) is to familiarize yourself with 
your observations and determine their main structure.  Since simply staring at a table or computer printout of numbers will 
eventually lead to premature blindness or debilitating insanity, there are several standard techniques that we will 
classify under the broad EDA heading:

\begin{enumerate}
\item	Scatter plots --- show it all.
\item	Schematic plots --- simplify the sample.
\item	Histograms --- explore the distribution.
\item	``Smoothing'' of data --- reduce the noise.
\item	Residual plots --- determine the trends.
\end{enumerate}

We will briefly discuss each of these five categories of exploratory techniques.  For a complete 
treatment on EDA, see John Tukey's classic \emph{Exploratory Data Analysis} book; the reference is listed in the Preface.

\subsection{Scatter plots}
\index{Scatter plots}
\index{Plot!scatter}

\PSfig[h]{Fig1_scatter}{Scatter plots showing all individual data points --- the ``raw'' data --- 
are invaluable in identifying outlying data points and other potential problems.}

	If practical, consider plotting every individual data value on a single graph.  Such ``scatter'' plots show 
graphically the correlation between points, the orientation of the data, bad outliers, and the spread of
clusters (e.q., Figure~\ref{fig:Fig1_scatter}).  
We will later (in Chapter~\ref{ch:basics}) provide a more rigorous definition for what correlation is; at this stage it is just 
a visual appearance of a trend.  Scatter plots are particularly useful in two dimensions, but even three-dimensional
data are fairly easy to visualize.  For higher dimensions we may choose to view \emph{projections} of the data
onto lower-dimensional spaces and thus examine only 2--3 components at the time.

\subsection{Schematic plots}
\index{Schematic plots}
\index{Plot!schematic}
\index{Plot!box-and-whisker}
\PSfig[H]{Fig1_boxwhisker}{An example of a ``box-and-whisker'' diagram.  The five quartiles give a visual
representation of how one-dimensional data (or a single component of higher-dimensional data) are distributed.}

	The main objective here is to summarize a one-dimensional data distribution using a simple graph.  One very 
common method is the \emph{box-and-whisker} diagram, which graphically presents five informative 
measures of the sample.  These five quantities are the \emph{range} of the data (represented by the minimum and maximum 
values), the \emph{median} (a line at the half-way point), and the \emph{concentration} (represented by the 25\% and 75\% \emph{quartiles}) of a data 
distribution.  Schematically, these statistics can be conveniently illustrated as shown in Figure~\ref{fig:Fig1_boxwhisker}.

	As an example of the successful use of box-and-whisker diagrams, we shall return to the winter of 
1893--94 when Lord Rayleigh was investigating the density of nitrogen from various sources.
Some of his previous experiments had indicated that there seemed to be a discrepancy between 
the densities of nitrogen produced by removing the oxygen from a sample of air and the nitrogen produced by 
decomposition of different chemical compounds.  The 1893--94 results clearly established this difference 
and prompted further investigations into the composition of air, which eventually led him to the 
discovery of the inert gas \emph{argon}\index{Argon}.  Part of his success in convincing his peers has been attributed 
to his use of box-and-whisker diagrams to emphasize the difference between the two data sets he 
was investigating.  We will use Lord Rayleigh's data (reproduced in Table~\ref{tbl:nitrogen}) to make a scatter 
plot and two schematic plots:  The already mentioned box-and-whisker diagram and the \emph{bar} graph.
\index{Lord Rayleigh}

\begin{table}[h]
\centering
\begin{tabular}{|r|c|c|c|} \hline
\bf{Date} & \bf{Origin} &  \bf{Purifying Agent} & \bf{Weight} \\ \hline
29 Nov. 1893 &       NO     &    Hot iron     & 2.30143 \\
 5 Dec. 1893 &       "      &        "        & 2.29816 \\
 6 Dec. 1893 &       "      &        "        & 2.30182 \\
 8 Dec. 1893 &       "      &        "        & 2.29890 \\
12 Dec. 1893 &      Air     &        "        & 2.31017 \\
14 Dec. 1893 &       "      &        "        & 2.30986 \\
19 Dec. 1893 &       "      &        "        & 2.31010 \\
22 Dec. 1893 &       "      &        "        & 2.31001 \\
26 Dec. 1893 &    N$_2$O    &        "        & 2.29889 \\
28 Dec. 1893 &       "      &        "        & 2.29940 \\
 9 Jan. 1894 & NH$_4$NO$_2$ &        "        & 2.29849 \\
13 Jan. 1894 &       "      &        "        & 2.29889 \\
27 Jan. 1894 &      Air     & Ferrous hydrate & 2.31024 \\
30 Jan. 1894 &       "      &        "        & 2.31030 \\
 1 Feb. 1894 &       "      &        "        & 2.31028 \\ \hline
\end{tabular}
\caption{Lord Rayleigh's density measurements of nitrogen (Lord Rayleigh,
On an anomaly encountered in determinations of the density of nitrogen gas, \emph{Proc. Roy. Soc. Lond., 55}, 
340--344, 1894).}
\label{tbl:nitrogen}
\end{table}

\PSfig[h]{Fig1_Rayleigh_scatter}{Scatter plot of Lord Rayleigh's data on nitrogen.  The plot reveals distinct groups.}

We will first look at all the data using a scatter plot.  It may look something like 
Figure~\ref{fig:Fig1_Rayleigh_scatter}.
It is immediately clear that we probably have two different data groupings here.  Note that this is only apparent if you 
plot the ``raw'' data points.  Plotting all the values as one data set using the box-and-whisker 
approach would result in a confusing graph (Figure~\ref{fig:Fig1_Rayleigh_one_BW}), which tells us very little that is meaningful.  
Even the median, traditionally a stable indicator of ``average'' value, is questionable since it lies 
between the two data clusters.  Clearly, it is important to find out if our data consist of a single 
population or if it contains a mix of two (or even more) data components.
 
\PSfig[h]{Fig1_Rayleigh_one_BW}{Schematic box-and-whisker plot of Lord Rayleigh's nitrogen data.  While representing the
entire sample the graph obscures the pattern so clearly revealed by the scatter plot.}

Fortunately, in this example we know how to separate the two data sets based on their origins.  
It appears that we are better off plotting the data sets separately instead of treating them as a single population.  
However, the choice of diagram is also important.  Consider a simple bar graph (here indicating 
the average value) summarizing the data given in Table~\ref{tbl:nitrogen}.  It would simply look as 
shown in Figure~\ref{fig:Fig1_Rayleigh_bars}.
\PSfig[h]{Fig1_Rayleigh_bars}{Bar graph of the average values from Lord Rayleigh's nitrogen data.  Because the
average values are very similar the two bars look very similar and do not tell us much.}
	In this presentation, it is just barely visible that the weight of ``nitrogen'' extracted from the air 
is slightly heavier than nitrogen extracted from the chemical compounds.  Given the way they are shown, the 
data present no clear indication that the two data sets are \emph{significantly} different.  Part of the 
problem here is the fact that we are drawing the bars from an origin at zero, whereas all the 
variation actually takes place in the 2.29--2.32 interval (again evident from the scatter plot in Figure~\ref{fig:Fig1_Rayleigh_scatter}).
By expanding the scale and choosing
a box-and-whisker plot we concentrate on the differences and produce an illustration as the one 
shown in Figure~\ref{fig:Fig1_Rayleigh_two_BW}.
 
\PSfig[H]{Fig1_Rayleigh_two_BW}{Separate box-and-whisker diagrams of nitrogen weight given in Table~\ref{tbl:nitrogen}
based on origin.  Their separation and extent clearly convey the finding of two separate sources.}

It is obvious that the second box-and-whisker diagram allows a clearer interpretation than the bar graph. The 
diagram also benefits from the stretched scale, which highlights the different ranges of the data 
groupings.  In Lord Rayleigh's situation the convincing diagram was accepted as strong evidence for the 
existence of a new element (later determined to be \emph{argon}), and a Nobel prize in physics followed in 1904.

\subsection{Histograms}
\index{Histogram}
\index{Plot!histogram}
\PSfig[h]{Fig1_make_histogram}{A data set, here a function of distance, can be converted to a histogram by counting the frequency of 
occurrence within each sub-range.  The histogram only uses the $y$-values of the $(x,y)$ points shown in the left 
diagram.}

	Histograms convey an accurate impression of the data \emph{distribution} even if it is multimodal.  
One simply breaks the data range into equidistant sub-ranges and plots the frequency or occurrences for each range (e.g., Figure ~\ref{fig:Fig1_make_histogram}).  
Obviously, the width of the sub-range determines the level of detail you will see in the final 
histogram.  Because of this, it is usually a good idea to plot the discrete values as individual 
points since the ``binning'' throws away some information about the distribution.  If the amount 
of data is moderate, then one can plot the individual values inside the histogram bars.  Furthermore,
you should explore how the \emph{shape} of the distribution changes as you vary the \emph{bin width}.
Clearly, as the histogram bin width approaches zero you will end up with one point (or none) per bin, while at the
other extreme (a very wide bin width) you will simply have a single bin with all your data.
Try to select a width that yields a representative distribution, but at the same time try to understand
what is going on when your widths give different shapes.

\subsection{Smoothing}
\index{Smoothing}
\index{Data!smoothing}
\PSfig[h]{Fig1_smoothing}{Smoothing of data is usually done by filtering. The left panel shows a noisy
data set and a smooth Gaussian filtered curve (red), while the right panel shows the same data with a
glitch between $x = 5$ and $x = 6$.  For such data the Gaussian filter is unduly influenced by the outlying
data whereas a median filter (blue) is much more tolerant.}

\index{Data!median filter}
\index{Data!Hanning filter}
\index{Filtering!median}
\index{Median filter}
\index{Hanning filter}
\index{Filtering!Hanning}
	The purpose of \emph{smoothing} is to highlight the general trend of the data and suppress
high-frequency oscillations.  We will briefly look at two types of smoothing: The \emph{median} filter and the \emph{Hanning} 
filter. The median filter is typically a three-point filter and simply replaces a point 
with the median value of the point and its two immediate neighbors.  The filter then shifts one step 
further to the right and the process repeats.  This technique is very efficient at removing isolated 
spikes or \emph{outliers} in the data since the bad points will be completely ignored as they will never 
occupy the median position, unless they appear in groups of two or more (in that case a wider 
median filter, say 5-point, would be required).  In contrast, the Hanning filter is simply a moving average of 
three points where the center point is given twice the weight of the neighbor points, i.e.,
\begin{equation}	 
y'_i = \frac{y_{i-1} + 2y_i + y_{i + 1}}{4}.
\end{equation}
Note that while such a filter works well for data that have random high-frequency noise, it gives 
disastrous results for spiky data since the outliers are averaged into the filtered value and never 
simply ignored.  For noisy data with occasional outliers one might consider running the data first 
through a median filter, followed by a treatment of the Hanning filter.  In building automated data
procedures one should always have the worst-case scenario in mind and seek to protect the analysis by
preprocessing with a median filter.  Figure ~\ref{fig:Fig1_smoothing} illustrates the use
of simple smoothing with Gaussian filters, with or without protection from outliers by a median filter.
We will have more to say about filtering in Chapter~\ref{ch:spectralanallysis}.


\subsection{Residual plots}
\index{Residual plots}
\index{Plot!residual}
	We can always make the assumption that our data can be decomposed into two parts: A 
smooth trend plus noisier residuals.  This separation forms the basis for \emph{regional-residual} analysis.
The simplest trend (or regional) is just a straight line.  One can easily define such 
a line by picking two representative points $(x_1,y_1)$ and $(x_2,y_2)$ and then compute the trend as
\begin{equation}
T(x) = y_1 + \left ( \frac{y_2 - y_1}{x_2 - x_1}\right ) (x - x_1).
\end{equation}
We then can form residuals $r_i = y_i - T(x_i)$ (in Chapter~\ref{ch:regression} we will learn more rigorous ways to find 
linear trends in $x-y$ data).  If a significant trend still exists, one can try several standard 
transformations to determine the nature of the ``smooth'' trend, such as the family of trends represented by
\begin{equation}
y^n,..., y^1, y^{1/2}, \log(y), y^{-1/2}, y^{-1},..., y^{-n}.
\end{equation}
The residual plot procedure we will follow is:
\begin{enumerate}
	\item Take $\log (y)$ of the data and plot the values.
	\item If the line is concave then choose a transformation closer to $y^{-n}$.
	\item If the data is convex then choose a  transformation closer to $y^n$.
\end{enumerate}
While approximate, this method gives you a feel for how the data 
vary.  Note that if your $y$-values contain or straddle zero then you cannot explore a logarithmic relationship for that axis.

	We will conclude this section with a quote from J. Tukey's book ``Exploratory Data Analysis'',
which is worth contemplating: 

\begin{quote}
\emph{``Many people would think that plotting y against x for simple data is something 
requiring little thought.  If one wishes to learn but little, it is true that a very little 
thought is enough, but if we want to learn more, we must think more.''}
\end{quote}

The moral of it all is:  \emph{Always} plot your data.  \emph{Always}! \emph{Never} trust output from software performing statistical 
analyses without comparing the results to your data.  Very often, such software implements statistical
methods that are based on certain
assumptions about the data distribution, which may or may not be appropriate in your case.  Plotting your
data may be the easiest thing to do (for dimensionality less than or equal to three) or it may be quite
challenging.  Exploring sub-spaces of just a few dimensions is a simple starting point, while more
sophisticated methods will examine the dimensionality of the data to see if some dimensions simply
provide redundant information.
\index{Exploratory data analysis|)}

\clearpage
\section{Problems for Chapter \thechapter}
%See course website for any data sets.
Note: Problems that ask you to comment on what you observe in the data do not require any special knowledge
about geology, seismology, or any other discipline.

\begin{problem}
Perform exploratory data analysis on the data set \emph{sludge.txt}, which contains annual average concentrations
of total phosphorous in municipal sewage sludge from an unnamed city, given in percent of dry weight solids.
Make a few EDA plots to determine any trends or patterns captured by the data.
\end{problem}

\begin{problem}
Perform exploratory data analysis on the data set \emph{hypso.txt}, which contains topography bins in km and frequency in percent,
in other words the \emph{hypsometric} curve for the Earth.  Your EDA should include a few graphs such as a scatter plot, a box-and-whisker,
and a histogram).  Based on the graphical results, make some
preliminary conclusions about the data set (e.g., are there outliers? Are the distributions unimodal? Asymmetrical?).
\end{problem}

\begin{problem}
Perform exploratory data analysis on the data set \emph{porosity.txt} (porosity in percent for a few
sandstone samples).  Again, generate several typical EDA plots and reach some
preliminary conclusions about the data (e.g., any outliers? Unimodal vs. multimodal? Asymmetrical vs. symmetrical?).
\end{problem}

\begin{problem}
Scientists obtained free-air gravity anomalies along the track of the \emph{R/V Conrad} cruise 2308, which surveyed an area
around Oahu in the Hawaiian Islands.  Ideally, when the ship reoccupies the same location the free-air anomaly should be
the same, but instrument drift, inadequate correction for ship motion and erroneous navigation data all lead to discrepancies
called \emph{cross-over errors} (COE); here \emph{c2308\_xover.txt} contains these data.
This mismatch in values at the intersections of the track gives information on the uncertainty in the measurements.
Examine the CEOs using both histograms and box-and-whisker techniques.
\end{problem}

\begin{problem}
Make both a box-and-whisker diagram and a histogram of the bathymetry depths in \emph{pac\_bathy.txt}
(depths in m for a 1\DS\ radius region in the Western Pacific).  Try different binning intervals and discuss
the important aspects of the distribution.
\end{problem}

\begin{problem}
Examine the variations in the magnetic field (in nT) at the Honolulu station for September, 1998 reproduced in \emph{honmag\_1998\_09.txt}.
Show a scatter plot and a box-and-whisker diagram of the total field (HONF).
\end{problem}

\begin{problem}
\newcounter{EDA}
\setcounter{EDA}{\theproblem}
Perform EDA on the data given in \emph{seismicity.txt}, which
contains records with longitude, latitude, depth (in km), and magnitude for significant
earthquake hypocenters from the Tonga-Kermadec region.  Try various presentations and
decide on two plots that best highlight the features of the data
set.  Provide commentary on what you see as significant for each plot and especially
note any outliers or peculiarities about the data set (even if they do not make it onto the two
plots you select).
\end{problem}

\begin{problem}
Repeat the exercise in Problem~\thechapter.\theEDA\ on the data set \emph{hi\_ages.txt}, which contains a
list of longitude, latitude, radiometric age (in Myr), and distance from Kilauea (in km) along the Hawaiian
seamount chain.
\end{problem}

\begin{problem}
Repeat the exercise in Problem~\thechapter.\theEDA\ on the data set \emph{v3312.txt}, which documents the distance
(in km) and depth (in m) along the track of \emph{R/V Vema} cruise 3312 from Japan to Guam.
\end{problem}

\begin{problem}
Explore the record of Mississippi river daily discharge (m$^3$/s) from 1930--1941 listed in \emph{mississippi.txt}.
Select two graphs that tell the most complete story about this data set.
\end{problem}

\begin{problem}
The GISP-2 ice core $\delta^{18}$O variations in parts per thousand (ppt) from the last $\sim$10,000 years can be found in
the file \emph{icecore.txt}.  Select two graphs that reveal the most about this data set.
Note the times are relative to an origin at 1950.
\end{problem}

\begin{problem}
The data set \emph{trend.txt} contains ($x,y$) pairs exhibiting a trend.  Use the residual plot technique
to determine the likely trend exhibited by the majority of the data.  Are there any outliers?
\end{problem}

\begin{problem}
The Earth's magnetic field is known to have reversed direction over geologic time.  The file \emph{GK2007.txt} contains data from the
Gee and Kent (2007) magnetic time scale. It lists all normal and reversely magnetized chrons and gives the duration
of each interval (in Myr). Note: Examine the last letter in the chron: ``n'' means normalized and ``r'' stands for reversed polarity.
\begin{enumerate}[label=\alph*)]
	\item Make box-and-whisker plots for the entire data set as well as for normal and reverse polarities separately.
	\item Make a histogram of all intervals using a bin width of 0.1 Myr.
	\item The distribution seems to have a long tail.  Convert your data to logarithmic data (log$_{10}$) and
	make another histogram with a log-increment of 0.1.  Does the transformation reduce the tail of the data?
\end{enumerate}
\end{problem}

% DA1_Chap2.tex
%
\chapter{REVIEW OF ERROR ANALYSIS}
\label{ch:error}
\epigraph{``An error does not become truth by reason of multiplied propagation, nor does the truth become error because nobody will see it.''}{\textit{Mahatma Gandhi, India Independence Leader}}

\index{Error analysis|(}
	Error analysis is the study and evaluation of uncertainty in measurements of continuous data 
(discrete data may have no errors).  We know that no measurement, however carefully made, 
can be completely free of uncertainties.  Since science depends on measurements, it is 
crucially important to be able to evaluate these uncertainties and to keep them to a minimum.  
Thus errors are not mistakes and you cannot avoid them by being very careful, but you should strive to make them 
as small as possible.  Understanding uncertainties is critical for determining the \emph{significance} of models
that you may construct to explain your data.  In this chapter, we will review the basic rules that govern
the reporting of measurements with uncertainties and how these uncertainties propagate when used to
obtain derived quantities (such as sums and products) as well as how they affect the uncertainties in
more complicated situations (such as being arguments to nonlinear functions).

\section{Reporting Uncertainties}
\index{Reporting uncertainties}
\index{Uncertainty!reporting}
\index{Data!uncertainties}

	When reporting the value of a measurement, care should be taken to give the best possible 
estimate of the uncertainty or error in the measurement.  Values read off scales or measured with 
mechanical instruments can usually be bracketed between lower and upper limits.  E.g., we may know
that a temperature measurement $T$ is not less than 23\DS\ nor larger than 24\DS.  Hence, we
shall report the value of $T$ as
\begin{equation}
T = 23.5 \pm 0.5^{\circ},
\end{equation}
or, in general, $x \pm \delta x$.  For many types of measurements we can state with absolute certainty that 
$x$ must be within these bounds.  Unfortunately, very often we cannot make such a categorical 
statement.  To do so, we would have to specify unreasonably large values for $\delta x$ to be absolutely confident that 
the actual quantity lies within the stated interval.  In most cases, we will lower our confidence to, 
say, 90\% and use a smaller $\delta x$.  We need more detailed knowledge of the statistical laws that 
govern the process of measurement for finding a suitable $\delta x$, so we will return to this issue later in this book.

\subsection{Significant figures}
\index{Significant figures}
\index{Data!significant figures}

In general, the last significant figure of any stated answer should be of the same 
order of magnitude (i.e., same decimal position) as the uncertainty.
For intermediate calculations you should use one extra decimal (if calculating by 
hand) or all available decimals (if using calculators or computers).
\begin{example}
If a measured distance $d$ is 6051.78 m with an uncertainty of $\pm 30$ m, you should report the 
distance as $d = 6050 \pm 30$ m.  Similarly, if a time $t$ is measured to be 3 seconds and the
uncertainty is given as 1 part in 100, you should report the time as $t = 3.00 \pm 0.03$ s.
\end{example}
\section{Fractional Uncertainty}
\index{Uncertainty!fractional}
\index{Fractional uncertainty}
\index{Precision}
\index{Data!precision}

While the statement $x = x_0 \pm \delta x$ indicates the \emph{precision} of the measurement, it is clear that 
such a statement will have different meanings depending on the value of $x_0$ relative to $\delta x$.  
Clearly, with $\delta x = 1$ m, we imply a different precision for $x_0 = 3$ m than for $x_0 = 1000$ km.  Thus, we 
should consider the \emph{fractional uncertainty}, written as
\begin{equation}
\displaystyle \frac{ \delta x}{\left| x_0\right |}.
\end{equation}
Note that the fractional uncertainty is a \emph{nondimensional} quantity.
\begin{example}
We wish to report the uncertainty in the length of a 3m crocodile, which (due to a shortage of available arms)
we only were able to measure with a precision of 6 cm.  Using fractional uncertainty,
we report the uncertainty $\delta l = \pm 100 \frac{0.06}{3}$\% = $\pm2$\%.
\end{example}

\section{Uncertainty in Derived Quantities}
\index{Uncertainty!derived quantities}
It is common to obtain measurements (and estimate their uncertainties) and then use these values in expressions
that lead to derived quantities.  In such situations we must be careful to \emph{propagate} the uncertainties
of the initial measurements so that the derived quantities can be reported with their combined uncertainties.
The answers will differ depending on whether or not the individual measurements are
\emph{independent} or \emph{dependent} on each other.

\subsection{Uncertainty in sums and differences}

Consider the two values $x \pm \delta x$ and $y \pm \delta y$.  We would like to determine the correct
expressions for the uncertainties in the four derived quantities
\begin{equation}
	s = x + y,
\end{equation}
\begin{equation}
d = x - y,
\end{equation}
\begin{equation}
p = x  \cdot y,
\end{equation}
\begin{equation}
q = x / y.
\end{equation}
For the sum, common sense would suggest that the maximum value of $s$ must be $s = x + y + \delta x + \delta y$, while the minimum value is 
$s = x + y - \delta x- \delta y$.  Thus
\begin{equation}
\index{Uncertainty!sums}
s \approx (x + y) \pm (\delta x + \delta y ) = s_0 \pm \delta s
\end{equation}
and similarly for differences we reason that
\begin{equation}
\index{Uncertainty!differences}
d \approx (x - y) \pm (\delta x + \delta y) = d_0 \pm \delta d.
\end{equation}
Here, the subscript $0$ indicates the nominal or ``best'' value of the results.
We use the approximate sign $\approx$ since we anticipate that the stated uncertainties $\delta s$ and $\delta d$ 
probably overestimate the true uncertainties in the sum and difference, respectively.
Note that the uncertainties in both the sum and the difference of $x$ and $y$ are identical.
\subsection{Uncertainty in products and quotients}
For the product, we first use fractional uncertainties and rewrite $x$ and $y$ as
\begin{eqnarray*}
x & = & x_0 \left ( 1 \pm \frac{\delta x} {\left | x_0\right |} \right ),\\[10pt]
y & = & y_0 \left ( 1 \pm \frac{\delta y} {\left | y_0\right |} \right ).
\end{eqnarray*}
Then, the maximum value of $p$ is
\begin{equation}
\index{Uncertainty!products}
p_{high} = x_0 \left( 1 + \frac{\delta x} {\left | x_0 \right |} \right)
\cdot y_0  \left( 1 + \frac{\delta y} {\left | y_0\right |} \right),
\end{equation}
which becomes
\begin{equation}
\index{Uncertainty!products}
p_{high} = p_0  \left( 1 + \frac{\delta x} {\left | x_0\right |} +
\frac{\delta y} {\left | y_0\right |} + \ldots \right),
\end{equation}
where the higher order term proportional to $\delta x \cdot \delta y$ has been ignored.  We note that the best value is
\begin{equation}
p_0 = x_0 \cdot y_0.
\end{equation}
The minimum value is found by reversing the signs of $\delta x$ and $\delta y$.  Both expressions yield the uncertainty in 
$p$ as
\begin{equation}
\frac{\delta p} {\left | p_0\right |} \approx \frac{\delta x} {\left | x_0\right |} +
\frac{\delta y} {\left | y_0 \right |}.
\end{equation}	 
 For quotients, the maximum value will be
\begin{equation}
q_{high} = \frac{   x_0 \left ( 1 + \frac{\delta x} {\left | x_0\right |}\right) }
 { y_0 \left( 1 -  \frac{\delta y} {\left | y_0 \right |} \right)} =
q_0 \frac{1 + a}{1 -b}. 
\end{equation}
Using the binomial theorem, we expand $(1-b)^{-1}$ as 
$1+b+b^2+b^3\ldots$, hence
\begin{equation}
\frac{1+a}{1-b} \approx (1+a) (1+b)= 1+a + b + ab \approx 1 + a + b,
\end{equation}	 
where we again ignore higher-order terms in $b$ by assuming that $(\delta x/|x_0|) \ll 1$ and $(\delta y/|y_0|) \ll 1$.
Similarly, for the minimum value, find
\begin{equation}
\frac{1 - a}{1+b} \approx (1- a) (1- b)= 1 - a - b + ab \approx 1 - ( a + b).
\end{equation}	 
It therefore follows that
\begin{equation}
	\index{Uncertainty!quotients}
	q = q_0 \left[ 1 \pm \left( \frac{\delta x} {\left | x_0 \right |} +
	\frac{\delta y }{\left | y_0 \right |} \right) \right] =
	q_0 \left[ 1 \pm \frac{\delta q}{q_0} \right]
\end{equation}
and that the fractional uncertainty for both products and quotients are the same.

\subsection{Uncertainty for general expressions}
The stated uncertainties are the maximum values possible, but we suspect these are likely to be exaggerated.
Later, we will show that if we assume our errors to be \emph{normally} distributed (i.e., we have 
``Gaussian'' errors) and our measurements are \emph{independent}, then a better estimate of the uncertainty in 
sums and differences is 
\begin{equation}
\delta s = \delta d = \sqrt{ (\delta x)^2 + (\delta y) ^2},
\end{equation}
	and for products and quotients it becomes
\begin{equation}
\frac{\delta p }{p_0} = \frac{\delta q}{q_0} = 
\sqrt{ \left (  \frac{\delta x}{x_0} \right)^2 + \left( \frac{\delta y}{y_0}  \right)^2 }.
\end{equation}	 
Note that in the case $s = nx$, where $n$ is a constant, we must use $\delta s = n\delta x$ since all the $x$ values are the 
same and obviously \emph{not independent} of each other.  Similarly, the product $p = x^n$ will have the 
fractional uncertainty
\begin{equation}
\frac{\delta p }{p_0} = n \frac{\delta x}{x_0},
\label{eq:poweruncertainty}
\end{equation}
since the (repeated) measurements are \emph{dependent}.  In conclusion, if $s = x + y + nz - u - v - mw$, then
\begin{equation}
\delta s = \sqrt{ (\delta x)^2 + (\delta y)^2 + (n \delta z)^2 +(\delta u)^2 + (\delta v)^2
+ (m \delta w)^2    }.
\end{equation}
Even if our assumption of independent measurements is incorrect, $\delta s$ cannot exceed 
the ordinary sum
\begin{equation}
\delta s = \delta x + \delta y + n \delta z + \delta u + \delta v + m \delta  w.
\label{eq:uncert_sum}
\end{equation}
Similarly, if
\begin{equation}
q = \frac{x \cdot y \cdot z^n}{u \cdot v \cdot w ^m},
\end{equation}
then we find the fractional uncertainty in $q$ to be
\begin{equation}
\frac{\delta q }{q_0} = \sqrt{ \left ( \frac{\delta x}{x_0}   \right )^2 +
\left ( \frac{\delta y}{y_0}   \right )^2 + 
\left ( n \frac{\delta z}{z_0}   \right )^2 + \left ( \frac{\delta u}{u_0}   \right )^2  + \left ( \frac{\delta v}{v_0}   \right ) ^2 + 
\left ( m \frac{\delta w}{w_0}   \right )^2         }.
\label{eq:uncert_prod}
\end{equation}
While this is the likely error for independent data, we can confidently say that the fractional 
uncertainty will be less than the linear sum
\begin{equation}
\frac{\delta q}{q_0} = \frac{\delta x}{x_0} + \frac{\delta y }{y_0} +
n  \frac{\delta z}{z_0} +  \frac{\delta u}{u_0} +  \frac{\delta v}{v_0} +
m  \frac{\delta w}{w_0}.
\end{equation}
\begin{example}
As most students who have taken an introductory physics lab in mechanics will know, measuring
the period $T$ of a pendulum of length $\ell$ allows one to estimate the 
acceleration of gravity as 
\begin{equation}
g = \frac{4\pi^2 \ell}{T^2}.
\end{equation}	 
We wish to determine the uncertainty in this estimate given measurements of $\ell$ and $T$
and their uncertainties.  Being independent measurements we use (\ref{eq:uncert_prod}) and find
\begin{equation}
\frac{\delta g}{\left| g_0    \right |} = 
\sqrt{ \left( \frac{\delta \ell}{\ell}   \right)^2 + \left( 2 \frac{\delta T}{T}   \right)^2      },
\end{equation}	 
since the constant $4\pi$ has no uncertainty.  Given the measurements
$\ell  = 92.95 \pm 0.10$ cm and $T = 1.936 \pm 0.004$ s, we obtain
\begin{equation}
g_0 = \frac{4 \pi^2 0.9295\mbox{ m}}{1.936^2\mbox{ s}^2} = 9.79035 \mbox{ m s}^{-2}.
\end{equation}	 
We can now evaluate the fractional uncertainty as     
\begin{equation}
\frac{\delta g}{\left| g_0    \right |} = 
\sqrt{ \left( \frac{0.1}{92.95}   \right)^2 + \left( 2 \frac{0.004}{1.936}   \right)^2 } \approx 0.4\%.
\end{equation}	 
The answer, therefore, is
\begin{equation}
g = 9.790 \pm 0.042 \mbox{ m s}^{-2},
\end{equation}
where we have only used three significant decimals.
\end{example}

\PSfig[h]{Fig1_Kelvin}{Lord Kelvin's model for the vertical temperature profile of the Earth
(the \emph{geotherm}\index{Geotherm}) at a time
$t_0$ since its initial formation at a constant temperature $T_0$ (dashed vertical geotherm).  The
tangent to the curve at the surface represents the vertical temperature gradient ($G$), which could
be estimated from temperature measurements in mines.}

Another case study revisits the debate that raged in the 19th century regarding the age of the Earth.
Observing the slow process of erosion, Charles Darwin\index{Darwin, C.} had implied that perhaps the Earth might be
as old as 300 million years.  Lord Kelvin, the preeminent physicist of his times, strongly objected
and set out to calculate the age using the conductive cooling of the Earth (this and many other fascinating
stories from the development of the geological sciences are portrayed in the classic book, \emph{Great Geological Controversies} by
A. Hallam [Oxford University Press]).
\index{Great geological controversies}
\begin{example}
\index{Lord Kelvin}
Lord Kelvin assumed the whole Earth
was once at a uniform temperature $T_{0}$ and had since cooled at the surface to $\sim 0^{\circ}$C (as indicated in Figure~\ref{fig:Fig1_Kelvin}).
Then, the physics of heat conduction in solids dictates that
\begin{equation}
	t_{0} = \frac{T_{0}^{2}}{\pi \kappa G^{2}},
\end{equation}
with initial temperature $T_{0}  \approx 2000 \pm 200\mbox{ }^{\circ}\mbox{K}$, thermal diffusivity
$\kappa = 1 \pm 0.25$ mm$^{2}$s$^{-1}$, and observed near-surface temperature gradient
$G = 25 \pm 5\mbox{ }^{\circ}\mbox{K}$ km$^{-1}$.  Given the procedures established earlier, we first determine
\begin{equation}
	\frac{\Delta t_{0}}{t_{0}} = \left[ \left(2\frac{\Delta T_{0}}{T_{0}}\right)^{2} + \left(\frac{\Delta \kappa}{\kappa}\right)^{2} + \left(2\frac{\Delta G}{G}\right)^{2} \right]^{\frac{1}{2}}.
\end{equation}
Inserting the estimated parameters, we find

\begin{equation}
	\frac{\Delta t_{0}}{t_{0}} = \left[ \left(2\frac{200}{2000}\right)^{2} + \left(\frac{0.25}{1}\right)^{2} + \left(2\frac{5}{25}\right)^{2} \right]^{\frac{1}{2}} \approx 51\%.
\end{equation}
We evaluate Kelvin's estimate of the age of the Earth to be
\begin{equation}
	t_{0} = \frac{(2000\mbox{ }^{\circ}\mbox{K})^{2}}{\pi \cdot 10^{-6} \mbox{m}^{2} \mbox{s}^{-1} (25\mbox{ }^{\circ}\mbox{K} \cdot 10^{-3} \mbox{m}^{-1})^{2}} \approx 2 \cdot 10^{15} \ \mbox{s} \ = 65 \ \mbox{Myr}.
\end{equation}
Given the fractional uncertainty, we obtain $t_{0} = 65 \pm 33$ Myr.  As the debate raged on,
positions hardened and Lord Kelvin continued to revise his
estimates downwards, finally settling on 25 Myr.  Modern science estimates that the Earth is closer to
4.6 \emph{billion} years old.  Where did Kelvin go wrong?
\end{example}

\PSfig[h]{Fig1_digline}{Example of a coastline segment whose length we attempt to estimate using both
a compass and via digitizing.}

	As a final case, let us imagine we are measuring the length of the coastline segment 
in Figure~\ref{fig:Fig1_digline} using two 
different methods: (1) Set a compass to a fixed aperture $\Delta x = 1 \pm 0.025$ cm and march along the 
line counting the steps, and (2) use a digitizing tablet and sample the line approximately every $\Delta x = 1 \pm 0.1$ cm.  
Let us assume that it took $N = 50$ clicks or steps so the measured line length in both cases is 50 cm.  What is the 
uncertainty in the length for the two methods?  First, let us state that there will be an uncertainty 
for both methods that has to do with the undersampling of short-wavelength coastline ``wiggles'' (also, as
you learn about the \emph{fractal} nature of coastlines then perhaps our simple approach
here will seem a bit naive).  That 
aside, we can see that the errors accumulate very differently.  For the compass length-sum the 
errors are all dependent (since the aperture is fixed) and we must use the summation rule to find the uncertainty $\delta l = 
N\cdot 0.025$ cm $= 1.25$ cm.  For the digitizing operations all the uncertainties associated with points 2 
through 49 largely cancel and we are left with the uncertainty of the endpoints.  Those are clearly
independent and hence the uncertainty is $\delta l = (0.1^2 + 0.1^2)^{1/2}$ cm $= 0.14$ cm.  The systematic errors 
using the compass accumulate while the errors in digitizing only affect the end-points.  This 
discussion of digitizing errors is a bit oversimplified, but it does illustrate the difference between 
the two types of errors and how they accumulate.

\subsection{Uncertainty in a function}
\index{Uncertainty!function}

\PSfig[h]{Fig1_func_uncertainty}{As $\delta x$ becomes very small, the derivative of any well-behaved function can be approximated by a
\emph{tangent} at the point $(x_0, y_0 = y(x_0))$; this is the second term in Taylor's expansion.}

Many solutions to scientific or engineering problems require the evaluation of functions
with our uncertain measurements as arguments.
If $x$ is measured with uncertainty $\delta x$ and is used to evaluate the function $y = f(x)$, then the 
uncertainty $\delta y$ is related to the \emph{derivative} of the function at $x$, i.e.,
\begin{equation}
\delta y = \left | \frac{df}{dx}  \right |_{x_0} \cdot \delta x,
\label{eq:funcdir}
\end{equation}
where the derivative is evaluated at $x_0$ (e.g., Figure~\ref{fig:Fig1_func_uncertainty}).
\begin{example}
Let  $y(x) = \cos x$ and $x = 20 \pm 3^{\circ}$.  Following (\ref{eq:funcdir}),
\begin{equation}
\delta y = \left| \frac{d\cos (x)}{dx} \right |_{x=20^{\circ}} \left( \frac{\pi}{180^{\circ}}\right) 3^{\circ} = \left|-\sin 20^{\circ} \right| 
\frac{3\pi}{180} = 0.342 \cdot 0.0524 \approx 0.02,
\end{equation}	 
where we have converted the angle from degrees to radians (why?).  The final answer then becomes
\begin{equation}
y = \cos(x) = 0.94 \pm 0.02.
\end{equation}
\end{example}
Finally, for a function of multiple variables, $f(x,...,z)$, we extend our analysis to find
\begin{equation}
\delta f = \sqrt{ \left ( \frac{\partial f }{\partial x} \delta x \right) ^2 + \cdots +
\left( \frac{\partial f }{\partial z} \delta z  \right)^2 }
\label{eq:uncert_func}
\end{equation}
when $x,..., z$ are all random and independent.  As before, $\delta f$ cannot exceed the ordinary sum
\begin{equation}
\delta f \leq \left | \frac{\partial f}{\partial x} \right | \delta x + \cdots + 
\left | \frac{\partial f}{\partial z}\right | \delta z,
\end{equation}
which is suitable for dependent measurements.
\begin{example}
Consider the spherical function
\begin{equation}
f (r,\theta, \phi ) = \frac{1}{2} r^2 \cos^2 \theta \sin \phi.	 \label{eq:spherical}	\end{equation}
We measured the parameters and found $r = 10 \pm 0.1, \theta = 60 \pm 1^{\circ}$, and 
$\phi = 10 \pm 1^{\circ}$.  Using (\ref{eq:uncert_func}), the 
uncertainty in the evaluated expression in (\ref{eq:spherical}) is found as
\begin{equation}
\delta f = \sqrt{ (r \cos ^2 \theta \sin \phi \delta r )^2 + 
(-r^2 \cos \theta \sin \theta\sin \phi \delta \theta )^2 + 
\left( \frac{1}{2} r^2 \cos^2 \theta \cos \phi \delta \phi\right )^2},
\end{equation}
which means our final estimate of $f(r, \theta, \phi)$ evaluates to
\begin{equation}
f(r, \theta, \phi) = 2.17 \pm 0.26.
\end{equation}
\end{example}
While the simple expressions for uncertainty in a derived quantity (\ref{eq:uncert_sum} or \ref{eq:uncert_prod}) generally apply, one must
be careful with expressions where one or more of the measurements appear in different \emph{subgroups} within the
expression.  In such cases one must treat the entire expression as a function of several variables and apply the
general expression given in (\ref{eq:uncert_func}), as our next example illustrates.
\begin{example}
Given the multivariate function
\begin{equation}
	f(a,b) = 2ab^2 + \pi b + 1,
\end{equation}
we wish to find its value and uncertainty for $a = 0.3 \pm 0.02$ and $b = 1 \pm 0.01$.
Because we have more than one term that depends on $b$ we cannot easily employ the rules
in (\ref{eq:uncert_sum}) and (\ref{eq:uncert_prod}) but must use (\ref{eq:uncert_func}) instead.
We first evaluate $f(0.3,1)$ to be 5.7416... and
next evaluate the uncertainty using (\ref{eq:uncert_func}):
\begin{equation}
	\delta f = \sqrt{\left (2b^2 \delta a \right)^2 + \left [ (4ab + \pi)\delta b \right ]^2} = 0.05903....
\end{equation}
Consequently, this yields a final estimate of
\begin{equation}
	f = 5.74 \pm 0.06,
\end{equation}
where we have rounded the answer to two decimals only.
\end{example}
\index{Error analysis|)}

\clearpage
\section{Problems for Chapter \thechapter}
%See course website for any data sets.

\begin{problem}
	The distance to a building is estimated from a map to be $2550\pm25$ m.  With a theodolite,
a student determines the angle between the horizontal plane and the building roof to be 1\DS 21' $\pm 1$'.
What is the height of the building and the uncertainty in this estimate?
\end{problem}

\begin{problem}
	The area of a cornfield is being estimated from aerial photographs.  Because of an unknown stretching factor
	the uncertainty in linear distances has been set to 1\% and hence uncertainties are \emph{dependent}.  What is the
	area and uncertainty of a field with measured dimensions 235.5 m by 115.6 m?
\end{problem}

\begin{problem}
Some neighborhood kid is driving his scooter way too fast while passing your house. Annoyed, you set out to apply
basic high-school physics in order to compute his speed: You measure a fixed length section
in front of your house and, while hiding in the bushes, use a stop-watch to time how long he takes to cover that distance.
Your old measuring tape reports $18.20 \pm 0.05$ m and you clock him covering that distance in
$0.82 \pm 0.10$ s.  In your passive-aggressive letter to the kid's parents, what is the speed and uncertainty
that you report?
\end{problem}

\begin{problem}
From compositional data you infer that two volcanic rock samples represent the initial and final
pulse of activity, respectively.  Your two samples have been radiometrically dated at
$25.53 \pm 0.1$ Ma and $29.66 \pm 0.2$ Ma, respectively.
What is the likely duration of volcanic activity (including the error)?
\end{problem}

\begin{problem}
You have carefully measured the densities of an exposed ore body and the surrounding bed rock and determined
$\rho_o = 3.15 \pm 0.05$ g cm$^{-3}$ and $\rho_r = 2.67 \pm 0.05$ g cm$^{-3}$, respectively.  A gravimetric survey over the region
will be sensitive to variations in lateral density only, in other words the density contrast between the
ore and the host rock.  Assuming your samples are independent, what is the density contrast and its uncertainty?
\end{problem}

\begin{problem}
Assigned to a crummy summer internship on Mars, you are tasked with the low-tech job of measuring the perimeters
of circular craters using a \$69.99 measuring wheel from Home Depot.  You obtain the circumferences of two craters
before you realize you are in fact the subject of a psychological experiment.  Although upset, you nevertheless decide
to complete the estimates for the two craters.  The measuring wheel reports 12,311 and 9,045
clicks, respectively, and the manufacturer says each click represents a distance increment of 25 cm (i.e., the circumference of the
measuring wheel).  Previous studies suggest that this method is accurate to 1\% of the circumference.
What is the area (and its uncertainty) of each crater in m$^2$?
\end{problem}

\begin{problem}
The Bouguer\index{Bouguer, P.} equation for gravity due to a constant thickness slab is given by $g = 2 \pi \rho \gamma h$, where
$\gamma = 6.6738\cdot10^{-11}$ m$^3$ kg$^{-1}$ s$^{-2}$ is the universal gravitational constant with fractional uncertainty
$1.2\cdot10^{-4}$, $\rho = 2850$ g cm$^{-3}$ is the slab density, and $h = 230\pm1$ m is the thickness of the slab.  
\begin{enumerate}[label=\alph*)]
	\item To achieve a precision of 1\% in the Bouguer calculation, how well do you need to know the density
	value (i.e., what is the maximum uncertainty you can tolerate)?
	\item Report the Bouguer value for the slab, including its uncertainty, in mGal ($= 10^{-5} \mbox{m s}^{-2}$).
\end{enumerate}
\end{problem}

\begin{problem}
The subsidence of young ($< 80$ Myr) oceanic crust due to lithospheric cooling has been
shown to follow approximately a linear $\sqrt {\mbox{age}}$ relationship, given by
$$
z = z_r + \frac{2 \rho_m \alpha_v T_m}{(\rho_m - \rho_w)} \sqrt{\frac{\kappa t}{\pi}}.
$$
Given estimates of thermal diffusivity $\kappa = 1.00 \pm 0.04 \mbox{ mm}^2s^{-1}$, water density
$\rho_w = 1.027 \pm 0.001$ g cm$^{-3}$, mantle density $\rho_m = 3.30 \pm 0.01$ g cm$^{-3}$,
volumetric thermal expansion coefficient $\alpha_v = (3.00 \pm 0.02) \cdot 10^{-5}$ $^{\circ}\mbox{K}^{-1}$,
average ridge depth $z_r = 2500 \pm 200$ m, and mantle temperature $T_m = 1300 \pm 25$ $^{\circ}$K,
determine the predicted depth and its uncertainty for a location where rocks of age
$t = 29.7 \pm 0.5$ Myr were recovered.  Which term dominates the final uncertainty?
\end{problem}

\begin{problem}
\index{Elastic plate thickness}
\index{Young's modulus}
\index{Poisson's ratio}
The flexural parameter $\alpha$ encountered in studies of elastic plate flexure is given by
$$
\alpha = \left [ \frac{4D}{(\rho_m - \rho_w)g} \right ]^\frac{1}{4},
$$
where $D$ is the flexural rigidity and is related to the \emph{elastic plate thickness}, $h$, via
$$
D = \frac{E h^3}{12 (1 - \nu^2)},
$$
where $E$ and $\nu$ are Young's modulus and Poisson's ratio, respectively.
It can be shown that the distance from a 2-D line load (approximating an island chain, perhaps) to
the maximum peripheral uplift (the ``bulge'') is given by $x_b = \pi \alpha$.  Given the measurements
$\rho_m = 3300 \pm 50$ kg m$^{-3}$, $E = 70 \pm 7$ GPa, $\nu = 0.25 \pm 0.01$, assuming no
uncertainty in $g = 9.81$ m s$^{-2}$ and $\rho_w = 1027$ kg m$^{-3}$, and observing $x_b = 252 \pm 10$ km,
what is the corresponding elastic plate thickness (and its uncertainty)?
\end{problem}

\begin{problem}
When collecting gravity measurements on-board a moving platform, such as a ship or aircraft, one must account
for the \emph{E{\"o}tv{\"o}s effect}, which will reduce or increase the apparent gravity depending on the moving platform's
heading ($\alpha$), latitude ($\theta$) and speed ($v$).  This effect (in mGal) is
$$
E = 14.585 v \cos \theta \sin \alpha + 0.015696 v^2,
$$
where speed $v$ is given in m/s.
\begin{enumerate}[label=\alph*)]
\item If the ship is at the equator and moving at $10.0 \pm 0.5$ knots with heading $\alpha = 35 \pm 2$\DS,
what is the E{\"o}tv{\"o}s effect and its uncertainty ? (You can assume $\alpha$ and $v$ have independent errors and
that the uncertainty in latitude is negligible).  Hint: Consider $E$ to be the function $E(v, \alpha)$ and use the
rule for uncertainty in a function (i.e., \ref{eq:uncert_func}).
\item Consider an aero-gravity survey using a plane capable of flying at 200 knots.  We wish to run
parallel lines in a certain direction $\alpha$ such that the \emph{changes} in $E$ with small changes in
$v$ are minimized.  Find the optimal survey line orientation.  You may again assume that the uncertainty
in latitude is negligible and that we are flying at a latitude of 45 degrees north.
\end{enumerate}
\end{problem}

% $Id$
%
\chapter{BASIC STATISTICAL CONCEPTS}
\label{ch:basics}
\epigraph{``The most important questions of life are, for the most part, really only problems of probability.''}{\textit{Pierre Simon de Laplace, Mathematician}}

Probability is mostly organized common sense.  However, being able to be specific about what probability is
enables us to more accurately calculate probabilities and to employ theoretical statistical distributions to
address confidence limits on data-derived quantities.

\section{Probability Basics}
\index{Probability!basics|(}

	In data analysis and hypothesis testing we are concerned with separating the probable from 
the possible.  First, let us have a look at possibilities.  In many situations we can either list all the 
possibilities or say how many such outcomes there are.  In evaluating possibilities, we are often 
concerned with finding all the possible choices that are offered.  Studying these choices leads us 
to the ``multiplication of choices'' rule:
\index{Multiplication of choices}
\begin{quote}
If a choice consists of $k$ steps, of which the first can be made in $n_1$ ways and 
the $k^{th}$ in $n_k$ ways, the total number of choices is $\Pi n_{i}, i = 1,k$.
\end{quote}
This can often be seen most clearly with a tree diagram (Figure~\ref{fig:Fig1_choices}).
The number of choices here are $3 \times 4 = 12$.

\PSfig[h]{Fig1_choices}{Tree diagram for illustrating all possible choices.}

\subsection{Permutations}
\index{Permutations}
How many ways can we arrange $r$ objects selected from a set of $n$ distinct objects?
This question applies to numerous statistical and probabilistic situations.  We will first
consider a simple example.
\begin{example}
We have a tray with 20 water samples.  How many ways can you select three samples 
from  the 20?  The first sample can be any of 20, the second will be any of the remaining 19, while the third is one of 
the remaining 18.  The total ways must therefore be $20 \times 19 \times  18 = 6840$.
\end{example}
We can write the number of choices as $20 \times (20-1) \times (20-2)$, and by induction we find 
\begin{equation}
\mbox{ways} = n(n-1)(n-2)\ldots(n - r + 1 ) = {}_n P_r.
\label{eq:choices}
\end{equation}
It is convenient to introduce the factorial $n!$, defined as 
\begin{equation}
n! = \prod^n_{i=1} i.
\end{equation}
For convenience, we also define $0!$ to equal 1.  We can then rewrite (\ref{eq:choices}) as
\begin{equation}
_{n}P_{r} = \frac{n (n-1) (n-2) \ldots (n-r + 1) (n-r) (n-r -1) \dots 1}{(n-r) (n-r-1) \ldots 1} = \frac{n!}{(n-r)!}.	
\end{equation}
This quantity is called the number of \emph{permutations} of $r$ objects selected from a set of $n$ distinct 
objects.
\begin{example}
	We wish to determine how many different hands one can be dealt in a game of poker.
With $n = 52$ (total number of cards in the deck) and $r = 5$ (number of cards in a hand), we find
\begin{equation}
_{52} P_5 = \frac{52!}{(52-5)!} = \frac{52!}{47!} = 48 \cdot 49 \cdot 50 \cdot 51 \cdot 52 = 
3 \cdot 10^8.
\end{equation}	 
However, this calculation assumes that the \emph{order} in which you receive the cards is important.
\end{example}

\subsection{Combinations}
\index{Combinations}

	In many situations we do not care about the exact ordering of the $r$ objects, i.e., $abc$ is the 
same choice as $acb$ for our purpose.  In general, $r$ objects can be arranged in $r!$ different ways 
($_{r}P_r = r!$).  Since we are only concerned about \emph{which} $r$ objects have been selected and not their 
order, we can use ${}_nP_r$ but must now normalize the result by $r!$, i.e., 
\begin{equation}
_{n} C_{r} = \frac{_{n} P_{r}}{r!} = \frac{n!}{r!(n-r)!} = \binom{n}{r}.
\end{equation}
The quantity $_{n} C_{r}$ is called the number of \emph{combinations}, and
the factors $\binom{n}{r}$ are called the 
\emph{binomial coefficients}.
\index{Binomial coefficients}
After picking the $r$ objects, $n - r$ objects are left, so consequently there are as many ways of 
selecting $n - r$ objects from $n$ as there are of selecting $r$ objects, i.e.,
\begin{equation}
\binom{n}{r} = \binom{n}{n-r}.
\label{eq:binom_inverse}
\end{equation}
\begin{example}
How many ways can you select three tide gauge records from 10 available stations?
This is a question of combinations:
\begin{equation}
{}_{10} C_3 = \binom{10}{3} =
\frac{10!}{3!7!} = \frac{8\cdot 9 \cdot 10}{1\cdot 2 \cdot 3} = 8 \cdot 3 \cdot 5 = 120.
\end{equation}	
Likewise, per (\ref{eq:binom_inverse}), there are also 120 ways to select 7 tide gauge records from the same 10 stations. 
\end{example}

\subsection{Probability}
\index{Probability}

	So far we have studied only what is \emph{possible} in a given situation.  We have listed all 
possibilities or determined how many possibilities there are.  However, to be of use to us we 
need to be able to judge which of the possibilities are \emph{probable} and which are \emph{improbable}.
	The basic concept of probability can be stated thus: If there are $n$ possible outcomes or 
results, and $s$ of those are regarded as favorable (or as ``successes''), then the probability of
success is given by
\begin{equation}
P = s/n.
\end{equation}
This classical definition applies only when all possible outcomes are \emph{equally likely}.
\begin{example}
What is the probability of drawing an ace from a deck of cards?
\emph{Answer}:  $P = 4/52 = 1/13 = 7.7\%$.
How about getting a 3 \emph{or} a 4 with a balanced die?
\emph{Answer}: $s = 2$ and $n = 6$, so $P = 2/6 = 33\%$
\end{example}
While equally likely possibilities are found mostly in games of chance, the classical probability 
concept also applies to random selections, such as making selections to reduce a large set of data
down to a manageable quantity without introducing sampling bias.
\begin{example}
If three of 20 water samples have been 
contaminated and you select four random samples, what is the probability of picking one of the 
bad samples?

\emph{Answer}:  We have 
$\binom{20}{4} = 3 \cdot 5 \cdot 17 \cdot 19 = 4845$
ways of making the selection of our four samples.  The number of 
``favorable'' outcomes is $\binom{17}{3}$ [we pick three good samples of the 17 good ones] times $\binom{3}{1}$ 
[we pick one of  the three bad samples] = 2040.  It then follows that the probability is
$P = s/n = 2040/4845 = 42\%$.
Here we used the rule of multiplicative choices.
\end{example}

	Obviously, the classical probability concept will not be useful when some outcomes are more 
likely than others.  A better definition would then be

\begin{quote}
\emph{The probability of an event is the proportion of the time that events of the same 
kind will occur in the long run.}
\end{quote}
So, when the National Weather Service says that the chance of rain on any day in June is 0.2, it is based 
on past experiences that on average June had 6 days of rain.  Another important probability 
theorem is the \emph{law of large numbers}, which states
\index{Law of large numbers}
\begin{quote}
\emph{If a situation, trial, or experiment is repeated again and again, the proportion of 
successes will tend to approach the probability that any one outcome will be a 
success.}
\end{quote}
which is basically our probability concept in reverse.

Coin tosses illustrate the law of large numbers nicely.  We toss the coin and keep track of how many
times we get ``heads'' versus the total number of tosses.  For a nice symmetric coin we expect the
proportion of heads to total tosses to approach 0.5 over the long haul, but initially we are not surprised
that there can be large departures from this expectation.  Figure~\ref{fig:Fig1_coin} shows how the
proportion may oscillate for a small number of tosses but eventually it will approach the expected value.
\PSfig[h]{Fig1_coin}{Proportion of heads in a series of coin tosses.  The more tosses we complete,
the closer the ratio of heads to total tosses will approach 0.5. Shown are five separate sequences.
They differ considerably for small numbers but all converge on the expected proportion.}

\subsection{Some rules of probability}
\index{Probability!rules}
\index{Event}
	In statistics, the set of all possible outcomes of an experiment is called the \emph{sample space}, 
usually denoted by the letter $S$.  Any subset of $S$ is called an \emph{event}.  An event may contain more 
than one item.  Sample spaces may be finite or infinite.  Two events that have no elements in 
common are said to be \emph{mutually exclusive}, meaning they cannot both occur at the same time.

There are only positive (or zero) probabilities, symbolically written
\begin{equation}
P(A) \geq 0
\end{equation}
for any event $A$.
Every sample space has probability 1, so that
\begin{equation}
P(S) = 1,
\end{equation}
where $P = 1$ means absolute certainty.
If two events are mutually exclusive, the probability that one \emph{or} the other will occur equals the 
sum of their probabilities
\begin{equation}
P(A \cup B) = P (A) + P(B).
\label{eq:add_probe}
\end{equation}
Regarding the notation, $\cup$ means \emph{union} (which we read as ``OR''),  $\cap$ means \emph{intersection} (``AND''), and $'$ 
(the prime symbol) means \emph{complement} (``NOT''). We can furthermore state that
\begin{equation}
	P(A)\leq 1,
\end{equation}
since absolute certainty is the most we can ask for.  Also,
\begin{equation}
P(A) + P(A') = 1,
\end{equation}
since it is certain that an event either will or will not occur.


\subsection{Probabilities and odds}
\index{Odds}
\index{Probability!odds}

Bookmakers in London use a slightly different system of reporting probabilities.
If the probability of an event is $p$, then the \emph{odds} for its occurrence are
\begin{equation}
a : b = \frac{p}{1-p}.
\end{equation}
The inverse relation gives
\begin{equation}
p = \frac{a}{a+b}.	 
\end{equation}
If you are still reading this book then odds are you will pass this course!

\subsection{Addition rules}
\index{Probability!addition}
\index{Probability!Venn}
\index{Venn diagram}
\index{Plot!Venn}

\PSfig[h]{Fig1_Venn}{A Venn diagram illustrating the probabilities of finding hydrocarbons.  The overlapping magenta
wedge graphically represents the probabilities of finding \emph{both} oil and gas.}

	The addition rules demonstrated above only holds for \emph{mutually exclusive events}.  Let us now 
consider a more general case.
The sketch in Figure~\ref{fig:Fig1_Venn} is a \emph{Venn diagram}, a handy graphical way of illustrating the various 
combinations of possibilities and probabilities.  The diagram illustrates the probabilities 
associated with finding hydrocarbons during a hypothetical exploration campaign. We see from 
the diagram that
\begin{equation}
\begin{array}{rcl}
P(\mbox{oil}) & = &0.18 + 0.12 = 0.3,\\
P(\mbox{gas}) & = & 0.24 + 0.12 = 0.36, \\
P(\mbox{oil} \cup \mbox{gas} ) & = & 0.18 + 0.12 + 0.24 = 0.54.
\end{array}
\end{equation}	 
Now, if we used the simple addition rule (\ref{eq:add_probe}), we would find
\begin{equation}
P(\mbox{oil} \cup \mbox{gas} ) = P (\mbox{oil}) + P \mbox{(gas)} = 0.3 + 0.36 = 0.66.
\end{equation}
This value overestimates the probability, because finding oil and finding gas are \emph{not} mutually 
exclusive since we might find both.  We can correct the equation by writing 
\begin{equation}
P (\mbox{oil} \cup \mbox{gas}) = P\mbox{(oil)} + P \mbox{(gas)} - P(\mbox{oil} \cap \mbox{gas}) = 0.3 + 0.36 - 0.12 = 0.54.
\end{equation}	 
The general addition rule for probabilities thus becomes
\begin{equation}
P(A\cup B) = P(A) + P(B) - P(A \cap B).
\end{equation}
Note that if the events \emph{are} mutually exclusive then 
$P(A \cap B) = 0$ and we recover the original rule.

\subsection{Conditional probability and Bayes basic theorem}
\index{Probability!conditional}
\index{Conditional probability}

	We must sometimes evaluate the probability of an event \emph{given that another event already has occurred}.
We write the probability that $A$ will occur given that $B$ already has occurred as
\begin{equation}
	P(A | B) = \frac{P(A \cap B)}{P(B)}.
	\label{eq:cond_prob}
\end{equation}
In our exploration example, we can find the probability of finding oil given that gas already has 
been found as
\begin{equation}
	P(\mbox{oil}|\mbox{gas}) = \frac{P(\mbox{oil} \cap \mbox{gas})}{P(\mbox{gas})} = \frac{0.12}{0.36} = \frac{1}{3}.
\end{equation}	 
We can now derive a general multiplication rule from (\ref{eq:cond_prob}) by multiplying it by $P(B)$ and
exchange \emph{A} and \emph{B}, which gives
\begin{equation}
\begin{array}{rcl}
P(A \cap B) & = & P(B) P (A | B)\\
P(A \cap B) & = & P(A) P (B | A)
\end{array}
\label{eq:Bayes_basic}
\end{equation}
and implies that the probability of both events $A$ and $B$ occurring is given by the probability of 
one event occurring multiplied by the probability that the other event will occur given that the first one 
already has occurred (occurs, or will occur).  This rule is called the \emph{joint probability} or \emph{Bayes 
basic theorem}.
\index{Probability!joint}
\index{Joint probability}
\index{Bayes basic theorem}
\index{Probability!Bayes basic theorem}
	Now, if the events $A$ and $B$ are independent events, then the probability that $A$ will take place 
is not influenced by whether $B$ has taken place or not, i.e.
\begin{equation}
	P(A|B) = P(A).
\end{equation}
Substituting this expression into (\ref{eq:Bayes_basic}) we obtain
\begin{equation}
P(A\cap B) = P(A) \cdot P(B).
\label{eq:jointindependent}
\end{equation}	 
That is, the probability that two independent events $A$ and $B$ both will occur equals the product of their probabilities.  In 
general, for $n$ independent events with individual probability $p_i$, the probability that all $n$ events 
occur is
\begin{equation}
P = \prod ^n_{i=1} p_i.
\end{equation}
\begin{example}
What is the probability of rolling three ones in a row with a balanced die?

\emph{Answer}: With $n = 3$ and $p =1/6$,
\begin{equation}
P = \frac{1}{6} \cdot  \frac{1}{6} \cdot  \frac{1}{6} \approx 0.005.
\end{equation} 	 
\end{example}
While $P(A | B)$ and $P(B|A)$ may look similar, they can be vastly different.  For example, let $A$ be the event 
of a death on the Bay Bridge connecting San Francisco and Oakland, and $B$ the event of a magnitude 8 earthquake in the area.  
Then,  $P(A|B)$ is the probability of a fatality on the Bay Bridge \emph{given} that a large earthquake has 
taken place nearby, while $P(B|A)$ is the probability that we will have a magnitude 8 quake \emph{given} that a 
death has been reported on the bridge.  Clearly $P(A|B)$ seems more likely than $P(B|A)$ since we know the former to 
have happened in the past.  On the other hand, we can list many causes of fatalities on the freeway other than 
earthquakes (e.g., traffic accidents, heart attacks, old age, road rage, talk radio rants, and so on).

	We can arrive at a relation between $P(B|A)$ and $P(A|B)$ by equating the two expressions for $P(A\cap B)$ in 
(\ref{eq:Bayes_basic}).  We obtain $P(A) \cdot P (B|A) = P (B) \cdot P (A|B)$, or
\begin{equation}
P(B | A) = \frac{P(B) \cdot P (A | B)}{P(A)}.
\label{eq:relate_cond_prob}
\end{equation}
This is a useful relation since we may sometimes know one conditional probability but are 
interested in the inverse relationship.  For example, we may know that salt domes
(known as potential traps for hydrocarbons) often are associated with 
large curvatures in the gravity field.  However, we may be more interested in the converse: 
Given that large curvatures in the gravity field exist, what is the probability that salt domes are 
the cause of such anomalies?

\subsection{Bayes general theorem}
\index{Bayes general theorem}
\index{Probability!Bayes general theorem}

	If there are more than one event $B_i$ (all mutually exclusive) that are conditionally related to an
event $A$, then $P(A)$ is simply the sum of the conditional probabilities of the events $B_i$ times their individual probabilities, i.e.
\begin{equation}
P(A) = \sum^n_{i=1} P (A|B_i) \cdot P (B_i).	 	
\label{eq:cond_prob_sum}
\end{equation}
Substituting (\ref{eq:cond_prob_sum}) into (\ref{eq:relate_cond_prob}) gives, for any of the $n$ events $B_i$,
\begin{equation}
P(B_i |A) = \frac{P (B_i) \cdot P (A|B_i)}{\displaystyle \sum ^n _{j=1} P (A|B_j)\cdot P(B_j)}.
\label{eq:Bayes_theorem}
\end{equation}

\PSfig[h]{Fig1_fossil_site}{Location of a fossil discovery with respect to the two drainage basins from which it
must have originated.  Bayes theorem provides a formal way to assign likelihood to the possible origins.}
\noindent
This is the general \emph{Bayes theorem}.
\begin{example}
Let us assume that an unknown marine fossil 
fragment was found in a dry stream bed in northern Sahara.  Excited, a paleontologist would like to send out an
expendable graduate student field party to search for a more complete specimen of the unknown species.
Unfortunately, the source of the 
fragment cannot be identified uniquely since it was found several kilometers below the junction of two dry stream 
tributaries (Figure~\ref{fig:Fig1_fossil_site}).  The drainage basin $B_1$ of the larger stream covers
407.5 km$^2$, while the other basin ($B_2$) covers only 207.5 
km$^2$.  Based on this difference in basin size alone we might expect the probabilities that the fragment came from one of 
the basins are
\begin{equation}
\begin{array}{c}
P(B_1) = \frac{407.5} {615} = 0.66,\\*[1ex]
P(B_2) = \frac{207.5} {615} = 0.34,
\end{array}
\end{equation}
based solely on the proportion of each basin's area to the combined area.  However, inspecting an ancient British-produced geological map 
reveals that only 31\% of the outcropping rocks in the larger basin $B_1$ are marine, whereas almost 85\% of 
the outcrops in basin $B_2$ are marine.  We can now state two conditional probabilities:\\

	$P(A|B_1) = 0.31$  (Probability of a marine fossil, given it was derived from basin $B_1$.)

	$P(A|B_2) = 0.85$  (Probability of a marine fossil, given it was derived from basin $B_2$.)\\

\noindent
With these probabilities and Bayes general theorem (\ref{eq:Bayes_theorem}) we can find the conditional probability that 
the fossil came from basin $B_1$ given that the fossil is marine:
\begin{equation}
P(B_1|A) =
\frac{P(A|B_1) \cdot P (B_1)} {P(A|B_1) \cdot P (B_1) + P (A|B_2) \cdot P(B_2)}
=
\frac{0.31 \cdot 0.66}{0.31 \cdot 0.66 + 0.85 \cdot 0.34} = 0.41.
\end{equation}	 
Consequently, the probability of the fossil coming from the smaller basin $B_2$ is the complimentary probability
\begin{equation}
P(B_2|A) = 0.59.
\end{equation}	 
It therefore seems somewhat more likely that the smaller basin was the source of the fossil and that this area should be
the initial target for the student-led expedition.
However, $P(B_1|A)$ and $P(B_2|A)$ are not dramatically different and depends to some extent on the assumptions used to 
select $P(B_i)$ and $P(A|B_i)$ in the first place.
Bayes general theorem is extensively used in such search and find scenarios and the probabilities that go into
the procedure are constantly being revised as more is learned during the search.
\end{example}
\index{Probability!basics|)}

\section{The M\&M's of Statistics}

	When discussing exploratory data analysis we mentioned that it is useful to be able to present 
large data sets using just a few parameters.  We saw the box-and-whisker diagram graphically 
summarized a data distribution.  However, it is often desirable to represent a data set by 
a \emph{single} number which, in its way, is descriptive of the entire data set.  We will see there are 
several ways to select this ``representative'' value.  We will mostly be concerned with measures 
that somehow describe the center or middle of the data set.  These are called estimates of 
\emph{central location}\index{Central location}.

\subsection{Population and samples}
\index{Data!population}
\index{Population}
\index{Data!sample}
\index{Sample}

	If a data set consists of all conceivably possible (or hypothetically possible) observations of a 
certain phenomenon then we call it a \emph{population}.  A population can be finite or infinite.  Any subset 
of the population is called a \emph{sample}.  Thus, a series of 12 coin-tosses is a sample of the potentially 
unlimited number of tosses in the population.  We will most often find that we are analyzing 
samples taken from a much larger population, and our aim will be to learn something about the 
population by studying the smaller sample set (Figure~\ref{fig:Fig1_outcrop}).

\PSfig[h]{Fig1_outcrop}{We must always try to select an unbiased sample from the population.  In this example we are sampling 
the weathered outcrop of a sedimentary layer, which most likely is not representative of the entire formation.}

\subsection{Measures of central location (mean, median, mode)}

	The best known estimate of central location is called the \emph{arithmetic mean}, defined as 
\begin{equation}
	\index{Sample!mean}
	\index{Mean}
	\index{Arithmetic mean}
\bar{x} = \frac{1}{n} \sum^n_{i=1} x_i.
\label{eq:arith_mean}
\end{equation}
The mean is also loosely called the ``average.''  Resist being that sloppy! When reporting the mean value, always say ``mean'' and 
not ``average'' so that the reader knows exactly what you have done.  We call $\bar{x}$  the \emph{sample mean} to 
distinguish it from the true mean of the population, denoted
\begin{equation}
\mu = \frac{1}{N} \sum^N_{i=1} x_i,
\end{equation}
which likely will remain unknown to us.  The mean has many useful properties, which explains its common use:
\begin{itemize}
\item	It can always be calculated for any numerical data, i.e., it always exists.
\item	It is unique and straightforward to calculate.
\item It is relatively stable and does not fluctuate much from sample to sample taken from the same 
population.
\item 	It lends itself to further statistical treatment:  several $\bar{x}$ estimates from subgroups can later be combined into an overall grand
mean.
\item	It takes into account every data value.
\end{itemize}
However, the last property can sometimes be a liability.  Should a few points deviate excessively 
from the bulk of the data then it does not make sense to include them in the sample.  A better 
estimate for the central location may then be the \emph{sample median}:
\begin{equation}
	\index{Sample!median}
	\index{Median}
\mbox{median } x_i = \tilde{x} = \left \{ \begin{array}{cl}
x_{ n/2 + 1}, & n \mbox{ is odd}\\*[1ex]
\displaystyle \frac{1}{2} (x_{n/2 + 1} + x_{n/2} ), & n \mbox{ is even}
\end{array} \right.
\end{equation}
Here, the data first must be sorted into ascending (or descending) order.  We then choose the middle 
value (or mean of the two middle values for even $n$) as our median estimate.
\index{Robust estimation}

	Consider this sample of sandstone densities: \{2.30, 2.20, 2.35, 2.25, 2.30, 23.0, 2.25\}, $n = 7$.  
The median density can be found to be $\tilde{x}  = 2.30$, a reasonable value, while the mean density $\bar{x} = 5.24$, 
which is a rather useless estimate since it is clearly far outside the bulk of the data \emph{and} outside
the range of known sandstone densities anywhere.  For this reason we 
say that the median is a \emph{robust} estimate of central location.  Here it is rather obvious that the value 
23.0, which probably is a clerical error, threw off the mean and we could correct for that by excluding
it from the calculation and find $\bar{x} = 2.28$ 
instead.  However, in many cases our data set will be very large and we must anticipate that some 
values may be erroneous.

	The disadvantage of the median is the need to sort the data, which can be slow. (Do you think this is really
a valid reason not to use it?).  However, 
like the mean, the median always exists and is unique.

\index{Sample!mode}
\index{Mode}
	Our final traditional estimate for central location is the \emph{mode}.  The mode is defined as 
the observation that occurs the most frequently.  For defining the central location the mode is at a 
disadvantage since it may not exist (perhaps no two values are the same) or it may not be unique (our 
densities actually have two modes).  Of course, if our data set is expected to have more than one ``peak,'' 
modal estimates are important, and we will return to that later.  The mode will be denoted as $\hat{x}$.  
The mean, median and mode of a distribution typically are related as indicated in
Figure~\ref{fig:Fig1_mmm}.
 
\PSfig[h]{Fig1_mmm}{The relationship between the mean, median, and mode estimates of central location for
a skewed data distribution.  These
estimates will all coincide for a perfectly symmetric and unimodal distribution.}

	Returning to the mean, it is occasionally the case that some measurements are considered 
more important than others.  It could be that some observations were made with a more precise 
instrument, or simply that some values are not as well documented as others.  These are 
examples of situations where we should use a \emph{weighted mean}
\index{Mean!weighted}
\index{Weighted mean}
\begin{equation}
\bar{x} = \sum^n_{i=1} w_i x_i \left / \sum^n_{i=1} w_i \right.,
\label{eq:weighted_mean}
\end{equation}
where $w_i$ is the weight of the $i$'th data value.  If all $w_i = 1$ then we recover the original definition for the 
mean (\ref{eq:arith_mean}).  This general equation is also convenient when we need to compute the overall, or
\emph{grand mean} based on the individual means from several data sets.  The grand mean based on $m$ data sets may be 
written as
\index{Grand mean}
\index{Mean!grand}
\begin{equation}
\bar{\bar{x}} = \frac{\sum^m _{i=1} n_i \bar{x}_i}{\sum^m_{i=1} n_i},
\end{equation}
where the sample sizes $n_i$ take the place of the weights in (\ref{eq:weighted_mean}).

\subsection{Measures of variation}

	While a measure of central location is an important attribute of our data, it says little 
about how the data are distributed.  We need some way of representing the \emph{variation} of our 
observations about the central location.  In the EDA section, we used the \emph{range} and \emph{hinges} to 
indicate data variability.  Another way to define the variability would be to compute the 
deviations from the mean,
\begin{equation}
\Delta x_i = x_i - \bar{x},
\end{equation}
and take the average of the sum of deviations, $\frac{1}{n}\displaystyle \sum^n_{i=1} \Delta x_i$.
Sadly, it turns out that this sum is 
always zero, which makes it rather useless for our purposes.  A more useful quantity might be the 
mean of the absolute value ($AD$) of the deviations:
\begin{equation}
	\index{AD (Absolute value of deviation)}
	\index{Deviation!absolute value}
	\index{Absolute value of deviation (AD)}
AD = \frac{1}{n} \sum^n _{i=1} | \Delta x_i |.
\label{eq:AD}
\end{equation}
Because of the absolute value sign this function is nonanalytic and often completely ignored by 
statisticians.  You will find very superficial treatment of medians and absolute deviations in most 
elementary statistics books.  However, when dealing with real data that include occasional bad 
values, the $AD$ is useful, just as the median can be more useful than the mean.  However, the most 
common way to describe variation of a population is to define it as the average \emph{squared} 
deviation.  Hence, the population \emph{variance} is
\begin{equation}
	\index{Variance}
	\index{Data!variance}
	\index{Population!variance}
\sigma^2 = \frac{1}{N} \sum^N _{i=1} (x_i - \mu)^2,
\end{equation}
and the population \emph{standard deviation} is therefore
\index{Standard deviation}
\begin{equation}
\sigma = \sqrt{ \frac{1}{N} \sum^N_{i=1} (x_i - \mu) ^2}.
\end{equation}
Most often we will be working with samples rather than entire populations, and we hope (and will later test)
that the sample is representative of the population.
The sample standard deviation $s$ is given by
\begin{equation}
	\index{Sample!variance}
s = \sqrt{ \frac{1}{n-1} \displaystyle \sum^n_{i=1} (x_i - \bar{x}) ^2}.
\label{eq:stdev}
\end{equation}
Note that we are dividing by $n - 1$ rather than by $n$.  This is done because $\bar{x}$  must first be \emph{estimated} from the 
sample rather than being a \emph{given} parameter of the population, such as $\mu$ and $N$.  This reduces the degrees 
of freedom by one; hence we divide by $n - 1$ (we will have more to say about degrees of freedom in Section~\ref{sec:freedom}).
	We can now show one property of the mean:  It is clear that $s^2$ depends on the choice for $\bar{x}$.  
Let us find the value for $\bar{x}$   in (\ref{eq:stdev}) that gives the smallest value for $s^2$.  Consider
\begin{equation}
f(\bar{x}) = s^2 = \frac{1}{n-1} \sum^n_{i=1} (x_i - \bar{x})^2.
\end{equation}
The function $f$ has a minimum where $df/d\bar{x}= 0$  and $d^2f/d\bar{x}^2 > 0$, so we find
\begin{equation}
\frac{df}{d\bar{x}} = \displaystyle \frac{\displaystyle  \sum ^n_{i=1} - 2 (x_i - \bar{x})} {n-1} =
\frac{-2}{n-1} \sum ^n _{i=1} (x_i - \bar{x}) =  0,
\end{equation}	 
which gives
\begin{equation}
\sum^n_{i=1} (x_i - \bar{x}) = 0.
\end{equation}	 
We can solve this equation and find
\begin{equation}
\bar{x} = \frac{1}{n} \sum^n _{i=1} x_i.
\end{equation}	 
Since
\begin{equation}
\frac{d^2f}{dx^2} = \frac{2n}{n-1} > 0,
\end{equation} 
we know that $f$ has a minimum for this value of $\bar{x}$.  Thus, we have shown that the value $\bar{x}$
that minimizes the standard deviation equals 
the mean we defined earlier in (\ref{eq:arith_mean}).  This is a very useful and important property of the mean. 
Because $\bar{x}$ 
minimizes the squared ``misfit'', it is also called the \emph{least-squares estimate} of central location 
(or L$_2$ estimate for short).  When computing the mean and standard deviation on a computer we 
do not normally use (\ref{eq:stdev}) since it requires two passes through the data: One to compute the $\bar{x}$ and 
another to solve (\ref{eq:stdev}).  Rather, we rearrange (\ref{eq:stdev}) to give
\begin{equation}
\begin{array}{ll}
s & = \displaystyle \sqrt{ \sum^n _{i=1} \frac{(x_i - \bar{x})^2} {n-1} } =
\sqrt{ \sum^n_{i=1} \frac{x^2_i - 2x_i \bar{x} + \bar{x}^2} {n-1} }\\*[3ex] \\ 
 \ & = \displaystyle \sqrt{\frac{n \displaystyle \sum x^2_i - 2 n \bar{x} \displaystyle \sum x_i + n \sum \bar{x}^2}
{n(n-1)} } = \sqrt{\frac{n \displaystyle \sum x^2_i - (\sum x_i)^2}{n(n-1)} }.
\end{array}
\end{equation}

\subsection{Robust estimation}
\label{sec:zscore}
\index{Robust!estimation|(}

	We found that the arithmetic mean is the value that minimizes the sum of the squared deviations from the 
central value.  Can we apply the same argument to the mean absolute deviation and find what the 
best value for  $\tilde{x}$ may be?  In other words, let
\index{Median}
\begin{equation}
\frac{d}{d\tilde{x}} \left( \frac{1}{n}\displaystyle \sum^n_{i=1} \left |x_i - \tilde{x}\right | \right) = -\frac{1}{n} \sum^n_{i=1} \frac{x_i - \tilde{x}}{|x_i - \tilde{x}|} = 0.
\label{eq:dAD-d*}
\end{equation}
The term inside the summation can only take on the values $-1$, $0$, or $+1$.  Thus, the only $\tilde{x}$  that can 
satisfy (\ref{eq:dAD-d*}) is a value chosen such that half the $x_i$ are smaller (giving $-1$) and half the $x_i$ are 
larger (giving $+1$), and for odd sample sizes we also get one or more exact zeros.  Thus, we
have proven that the median is the location estimate that minimizes 
the mean absolute deviation.  The 
median is also called the L$_1$-estimate of central location.

	The discussion of mean and median brings up the general issue of \emph{robust estimation}: How to 
calculate a stable and reasonable estimate of central location in the presence of contaminated 
data?  As an indicator of how robust a method is, we will introduce the concept of ``breakdown 
point.''  It is the \emph{smallest fraction} of the observations that must be replaced by outliers in order to throw 
the estimator outside reasonable bounds.  

	We have already seen that even a single bad value is enough to throw the mean way off.  For our 
densities of sandstone, we had $\rho = \{2.2, 2.25, 2.25, 2.3, 2.3, 2.35, 23.0\}$, with $n = 7$.  If we 
realized that 23.0 should be 2.3, we find $\bar{\rho} = 2.28 \pm 0.05$, while if we included $\rho_7 = 23.0$ we 
would find $\bar{\rho} = 5.24 \pm 7.8$.  The second estimate is obviously far outside the 2.20--2.35 range we first 
determined.  We can therefore say that the least squares estimate (i.e., the mean) has a breakdown value of 
$1/n$; it only takes one outlier to ruin our day.  On the other hand, note that the median is $\sim 2.3$ in 
both cases, well inside the acceptable interval.  It is found that the breakdown point of the 
median is 
50\%:  We would have to replace half the data with bad outliers to move the estimate of 
the median outside the range of the original (good) data values.

	Apart from the central location estimator, we also want a robust estimate of the spread of the 
data.  Clearly, the classical standard deviation is problematic since only one bad value will make it 
biased due to the $x^2$  effect.  From the success of taking the median of a string of numbers rather 
than summing them up, could we do something similar with the deviations?  Consider what 
value of $\tilde{x}$ would minimize the median of $\{|x_i - \tilde{x}|\}$.
You can probably see for yourselves that the $\tilde{x}$ must equal 
our old friend the median.  Because of the robustness of the median operator, we will often use 
the quantity called the \emph{median absolute deviation} ($MAD$) as our robust estimate of ``spread'' or variation.
Note: Many textbooks and software packages (such as MATLAB) use $MAD$ to indicate \emph{mean absolute deviation}
instead, as defined in (\ref{eq:AD}) and called $AD$ in these notes.  Thus, we define
\index{Mean absolute deviation}
\begin{equation}
	\index{MAD}
	\index{Median absolute deviation}
MAD = 1.4826 \mbox{ median } |x_i - \tilde{x} |,
\end{equation}
where the factor 1.4826 is a correction term that makes the $MAD$ equal to the standard deviation 
of normally distributed data\footnote{This factor equals $1/P^{-1}_c(0.75)$, where $z = P^{-1}_c(p)$ is
the inverse cumulative normal distribution.}.  Like the median, the $MAD$ has a breakdown point of 50\%.  The $MAD$ 
for our  example was 0.07 and it remained unchanged by using the contaminated value.
	Having robust estimates of central location and scale, we can attempt to identify \emph{outliers}.  We may 
compute the robust \emph{standard units}
\begin{equation}
	\index{Normal scores!robust}
	\index{Standard scores!robust}
z_i = \frac{x_i - \tilde{x}} {MAD}
\end{equation}
and compare them to a cutoff value: If $|z_i| > z_{cut}$ we say we have detected an outlier.  The choice 
for $z_{cut}$ is to a certain extent arbitrary.  It is, however, quite standard to choose $z_{cut} = 2.5$.  Chances 
that any $z_i$ will exceed $z_{cut}$  is very small if the $z_i$'s came from a normal distribution.  Our 
normalized densities (including the contaminated value) using $\bar{x}$ and $s$ to compute $z_i$ gives 
\begin{equation}
z_{\scriptscriptstyle L_{\scriptscriptstyle 2}} = \left \{ -0.39, -0.38, -0.38, -0.377, -0.377, -0.37, 2.28\right \},
\end{equation}
where none of the values qualify as an outlier.  Using the median and $MAD$ instead, we find
\begin{equation}
z_{\scriptscriptstyle L_{\scriptscriptstyle 1}}
 = \left \{ -1.35, -0.68, -0.68, 0.0, 0.0, 0.68, 280.0 \right \},
\end{equation}
and we see that the bad observation gives a huge $z$-value two orders of magnitude larger than any other.  
Clearly, the least-squares technique alone is not trustworthy when it comes to detecting bad 
points.  The outlier-detecting scheme presents us with an elegant two-step technique:  First find and remove 
the outliers from the data, then use classical \emph{least-squares} techniques on the remaining data 
points.  The resulting statistics are called the \emph{least trimmed squares} estimates (LTS).  We 
will return to the concept of robustness when discussing regression in Chapter~\ref{ch:regression}.
\index{Least trimmed squares (LTS)}
\index{Robust!estimation|)}

\subsection{Central limit theorem}

	How well does our sample mean, $\bar{x}$, compare to the true population mean, $\mu$?  An important 
theorem, called the \emph{central limit theorem}, states 
\begin{quote}
	\index{Central limit theorem}
\emph{If $n$ (the sample size) is large, the theoretical sampling distribution of the mean 
can be approximated closely with a normal distribution.}
\end{quote}
This is rather important since it justifies the use of the normal distribution in a wide range of 
situations.  It simply states that the sample mean $\bar{x}$ is an \emph{unbiased estimate} of the population 
mean and that the scatter about $\mu$ is \emph{normally distributed}.  It can be shown that the standard 
deviation of the sampling mean, $s_{\bar{x}}$, is related to the population deviation, $\sigma$, by
\begin{equation}
s_{\bar{x}} = \frac{\sigma} {\sqrt{n}}
\label{eq:samp_dev_int}
\end{equation}
or
\begin{equation}
	\index{Sample!mean}
s_{\bar{x}} = \frac{\sigma} {\sqrt{n}} \sqrt{\frac{N-n}{N-1}}
\label{eq:samp_dev_int2}
\end{equation}
depending on whether the population is infinite (\ref{eq:samp_dev_int}) or finite of size $N$ (\ref{eq:samp_dev_int2}).  Thus, as $n$ 
grows large, $s_{\bar{x}} \rightarrow 0$.   Furthermore, the sample variance $s^2$ has the mean value $\sigma^2$ with 
standard deviation
\begin{equation}
	\index{Sample!variance}
\sigma^2_s = \frac{2\sigma^4 }{n-1},
\end{equation}
which also $\rightarrow 0$ for large $n$.  For our analysis we will substitute the sample standard deviation
$s$ \emph{in lieu} of the unknown population standard deviation $\sigma$, since $s$ is an \emph{unbiased estimator} of $\sigma$.

\subsection{Covariance and correlation}
\label{sc:cc}
We found earlier that the sample variance was defined as 
\begin{equation}
s^2_x = \frac{\displaystyle \sum^n_{i=1} (x_i - \bar{x})^2}{n-1} =
 \frac{\displaystyle \sum^n_{i=1} (x_i - \bar{x})(x_i - \bar{x}) }{n-1}.
\end{equation}	 
It is often the case that our data set consists of pairs of properties, such as sets of (depth, pressure), 
(time, temperature), concentrations of two elements, and more.  Denoting the paired properties by $x$ and $y$, 
we can compute the variance of each quantity separately.  For instance, for $y$ we find
\begin{equation}
s^2_y = \frac{\displaystyle \sum ^n_{i=1} (y_i - \bar{y})^2} {n-1}=
\frac{\displaystyle \sum^n_{i=1} (y_i - \bar{y}) (y_i - \bar{y})} {n-1}.
\end{equation}
We can now define the \emph{covariance} between $x$ and $y$ in a similar way as
\begin{equation}
	\index{Sample!covariance}
	\index{Covariance}
s_{xy} =  \frac{\displaystyle \sum^n_{i=1} (x_i - \bar{x})(y_i - \bar{y})} {n-1}.
\end{equation}
While $s_x$ and $s_y$ tell us how the $x$ and $y$ values are distributed \emph{individually}, $s_{xy}$  tells us how 
the $x$ and $y$ values vary \emph{together}.

	Because the value of the covariance clearly depends on the units of $x$ and $y$, it is difficult to 
state what covariance values are meaningful.  This difficulty is overcome by defining the Pearson
\emph{correlation coefficient} $r$, which normalizes the covariance to yield correlations in the $\pm 1$ range, i.e.,
\begin{equation}
	\index{Sample!correlation}
	\index{Correlation}
r = \frac{s_{xy}} {s_x s_y}.
\end{equation}
If $|r|$ is close to 1, then the variables are strongly correlated or anti-correlated.  Values of $r$ close to 0 mean that there 
is little significant correlation between the data pairs.  Figure~\ref{fig:Fig1_correlations} shows some examples of 
data pairs and their correlations.
We see that in general, $r$ will tell us how well the data are ``clustered'' in some direction.  Note in 
particular example (f), which presents data that are clearly correlated (i.e., all pairs lie on a circle), yet $r 
= 0$.  This occurs because $r$ is a measure of a \emph{linear} relationship between values; a nonlinear 
relationship may not register a significant correlation.  Thus, we must be careful with how we use $r$ to draw conclusions 
about the interdependency of paired values.  For example, if our ($x,y$) data are governed by a $y = \sqrt{x}$ law then we 
may find a fairly good correlation between $x$ and $y$, but we would be wrong to conclude that $x$ and $y$ 
have a \emph{linear} relationship (plotting $y$ versus $\sqrt{x}$  \emph{would give} a linear relationship and a much higher 
value of $r$).  We will return to correlation under the rubrics of curve fitting and multiple regression in 
Chapter~\ref{ch:regression}.
\PSfig[h]{Fig1_correlations}{Some examples of data sets and their correlation coefficients.  Note that the perfect
circular correlation in (f) gives a zero linear correlation coefficient.  While clearly $x$ and $y$ are correlated,
their relationship is not \emph{linear}.}

\subsection{Moments}

	Returning to the L$_2$ estimates, we will briefly introduce the concept of \emph{moments}.  In general, 
the $r$'th moment is defined as
\begin{equation}
	\index{Moments}
m_r = \frac{1}{n} \sum^n_{i = 1} (x_i - \mu)^r,
\end{equation}
except for $r = 1$ where it is customary to use the ``raw moment'' about zero instead.  From this definition it can be seen
that the mean and variance are the first (raw) and second (central) moments, 
respectively.  We will look at two higher order (central) moments that one may encounter in the literature.  
The first is called the \emph{skewness} ($SK$) and it is the third central moment, given by
\begin{equation}
	\index{Skewness}
	\index{Data!skewness}
SK = \frac{1}{n} \sum ^n_{i=1} \left ( \frac{x_i - \bar{x}} {s} \right) ^3 = \frac{1}{n} \sum ^n_{i=1} z_i^3,
\end{equation}
where we normalize by $s$ to get dimensionless values for $SK$.  The skewness is used to investigate 
our data sets' \emph{degree of symmetry} about the mean.  A positive $SK$ means we have a longer tail 
to the right of the mean than to the left, and vice versa for a negative $SK$ (Figure ~\ref{fig:Fig1_skewness}).

\PSfig[h]{Fig1_skewness}{Examples of data distributions with positive and negative skewness.  The
sign of the skewness indicates which side of the distribution is long-tailed.}

\noindent
Unfortunately, if the data contain outliers then the $SK$ will be very sensitive to these values and 
consequently be of little use to us.  A more robust estimate of skewness is the \emph{Pearson 
coefficient of skewness},
\begin{equation}
	\index{Pearson skewness}
	\index{Skewness!Pearson}
SK_p = \frac{3(\bar{x} - \tilde{x})} {s},
\end{equation}
where we basically compare the mean and the median.  An even higher-order central moment is the 
\emph{kurtosis},
\begin{equation}
	\index{Kurtosis}
	\index{Data!kurtosis}
K  = \left \{ \frac{1}{n} \sum^n_{i=1} \left ( \frac{x_i - \bar{x}}{s} \right) ^4 \right \} -3 = \left \{ \frac{1}{n} \sum ^n_{i=1} z_i^4 \right \} - 3.
\end{equation}
The correction term $-3$ makes $K = 0$ for a normal distribution, which we will discuss shortly.  The kurtosis $K$ attempts to 
quantify a data distribution's ``sharpness'' ($K > 0$) or ``flatness'' ($K < 0$; Figure ~\ref{fig:Fig1_kurtosis}).
However, for most real data $K$ can be almost infinite and should be used only with 
``well-behaved'' data.

\PSfig[h]{Fig1_kurtosis}{Examples of distributions with different kurtosis.  Distributions with negative $K$ are
called \emph{platykurtic}\index{Platykurtic}, while a positive $K$ is called \emph{leptokurtic}\index{Leptokurtic}.  You will of course be immensely pleased to learn
that an intermediate case is called \emph{mesokurtic}\index{Mesokurtic}.}

\section{Discrete Probability Distributions}
\index{Probability distributions|(}
\index{Probability distribution!discrete}

	An important concept in statistics and probability is the notion of a \emph{probability distribution}.  It is a 
function $P(x)$, which indicates the probability that the event $x$ will take place.  $P(x)$ can be a 
discrete or continuous function.  As an example of a discrete function, consider the function $P(x), x=1, 2,..,6$, that gives 
the probability of throwing an $x$ with a balanced die:
\begin{equation}
P(x) = 1/6,\quad x = 1,2, \ldots, 6,
\end{equation}	 
or for flipping a coin:
\begin{equation}
P(x) = 1/2,\quad x = \left \{ H, T \right \}.
\end{equation}	 
Staying with the throws of the die, we can relate $P(x)$ to the area under the curve in Figure ~\ref{fig:Fig1_die_probability}.

\PSfig[h]{Fig1_die_probability}{Probability of throwing any number on a die is a constant $1/6$, unless
the die is ``loaded''.}

Two important properties shared by all discrete probability distributions are
\begin{equation}
0 \leq P (x_i) \leq 1, \mbox{ for all } x_i,
\end{equation}
\begin{equation}
\sum^n_{i=1} P(x_i) = 1.
\label{eq:Pdiscretesum}
\end{equation}	 
\subsection{Binomial probability distribution}
\label{sec:binom}
\index{Probability distribution!binomial}
\index{Binomial probability distribution}
Often we are more interested in knowing the probability of a certain outcome after $n$ repeated 
tries, such as ``what is the probability of receiving junk mail three days in one week?''  To derive such a 
function, we will assume that each event is independent and has the same probability, $p$.  Then, the 
probability that an event \emph{does not} occur is the complement, $q = 1 - p$.  Consequently, the probability of 
getting $x$ successes in $n$ tries (and thus $n - x$ failures) is
\begin{equation}
P_1(x) = p^x q^{n-x}.
\end{equation}
However, this probability applies to a \emph{specific order} of all possible outcomes.  Since we may not care about 
the order in which the successful $x$ events occurred, we must scale $P_1(x)$ by the number of 
possible combinations of $x$ successes in $n$ tries.  We already know this amount to be given by $\binom{n}{x}$,
so our discrete probability function becomes
\begin{equation}
P_{n,p}(x) = \binom{n}{x} p^x q^{n-x} = \binom{n}{x} p^x (1 - p)^{n-x}, \quad x =0, 1, \ldots, n. 
\label{eq:binomial_dist}
\end{equation}
This expression is known as the binomial probability distribution or simply the \emph{binomial distribution}
(Figure~\ref{fig:Fig1_binom_dist}) and it is used to predict the probability that $x$ events out of $n$
tries will be successful, given that each  independent $x$ has the probability $p$ of success.
\PSfig[h]{Fig1_binom_dist}{Binomial probability distribution $P_{n,p}(x)$, which shows the probability of having $x$ successful
outcomes out of a total of $n$ tries, when each try has the probability $p$ of success (and $q = 1 - p$ of failure).
Here, $p = 0.25$ and $n = 8$.}
\begin{example}
What are the chances of drawing three red cards in six tries from a deck (assuming we place the card back 
into the deck after each try)?  Here $p = 1/2$, so 
\begin{equation}
P_{6,0.5}(3) = \frac {6!}{3!3!} \left ( \frac{1}{2} \right ) ^3 \left ( \frac{1}{2} \right )^{6-3} = 0.31.
\end{equation}
One might have thought that getting half red and half black cards would have a higher probability, but 
remember that we require \emph{exactly} 3 reds.  If we compute the probability of getting 1, 2, or 3 reds 
separately and used the summation rule to compute the probability that we would draw 1, 2, or 3 
red cards then $P$ would be much higher.
\end{example}
The binomial probability distribution can also be used to assess the likelihood of more serious scenarios, such as the
next example presents.
\begin{example}
	A silver-tonged con artist approaches you on a street in New York City with a simple proposition: He has
	10 beads --- 9 black and one white.  You get to pick one bead from his bag.  You are
	given six opportunities to draw a bead (the bead is returned to the bag after each try), and
	if anytime during the six tries you pick the white bead then you have won and he will give you \$20.
	However, if you
	have not picked the white bead after six tries then you owe him \$20 instead.  Is this a good deal?
	Answer: Clearly, the probability of picking the white bead is fixed at $p = 0.1$. To lose
	the bet you will have to come up empty-handed six times in a row.  For $n = 6$ and $r = 0$ the
	chances of that is simply
\begin{equation}
P_{6,0.1}(0) =  \binom{6}{0} 0.1^0(1-0.1)^6 = 0.53.
\end{equation}
So while it is close to 50--50 the con-artist will most likely win, at least in the long run.
You probably should also be concerned that there might be something else going on as well, such as sleight-of-hand
removal of the white bead before each try...
\end{example}

\subsection{The Poisson distribution}
\index{Poisson distribution}
\index{Probability distribution!Poisson}
\index{Rare events}
\index{Binomial probability distribution!approximation}
	In some situations, the binomial distribution can be approximated by simpler expressions.
One such case arises when the probability $p$ for one event is 
very small and $n$ is large.  Such events are called \emph{rare}, and the discrete distribution may then be approximated by 
\index{Rate of occurrence}
\begin{equation}
P(x) = \frac{\lambda^x e^{-\lambda}}{x!},\quad x = 0, 1, 2, \ldots, n
\end{equation}
where $\lambda = np$ is the \emph{rate of occurrence}.  The Poisson distribution can be used to evaluate the 
probabilities for the occurrence of rare events such as large earthquakes, volcanic eruptions, and reversals of the 
geomagnetic field.  For instance, the number of floods occurring in a 50-year period has been shown to 
follow a Poisson distribution with $\lambda = 2.2$.  What is the probability that we will have at least 
one flood in the next 50 year period?  Here, $P = 1 - P_0$, the probability of having no flood.  
Plugging in for $x = 0$ and $\lambda = 2.2$ we find $P_0 = 0.1108$, so $P = 0.8892$.
\begin{example}
A student is monitoring the radioactive decay of a certain sample that is expected to
undergo three decays per minute.  The student observes the number of decays over 100
individual one-minute periods and constructs the summary shown in Table~\ref{tbl:decay1}.
\begin{table}[h]
\centering
\begin{tabular}{|l||c|c|c|c|c|c|c|c|c|c|} \hline
\bf{Decays}   & 0 &  1 &  2 &  3 &  4 &  5 & 6 & 7 & 8 & 9 \\ \hline
\bf{Observed} & 5 & 19 & 23 & 21 & 14 & 12 & 3 & 2 & 1 & 0 \\ \hline
\end{tabular}
\caption{Number of decays observed in one-minute interval.}
\label{tbl:decay1}
\end{table}
Does the data support the expected decay rate?  We make a histogram of the data
by normalizing the observed frequencies by the total count
and superimposing the Poisson distribution for the expected rate.  The result (Figure~\ref{fig:Fig1_poisson})
shows a very good fit.	
\PSfig[h]{Fig1_poisson}{Histogram of observed decay rate frequencies (bars) and
the theoretical Poisson distribution (circles) for the expected rate $\lambda = 3$.}
\end{example}

\section{Continuous Probability Distributions}

	While many populations are of a discrete nature (e.g., outcomes of coin tosses, numbers of 
microfossils in a core, etc.), we are very often dealing with observations of a phenomenon that 
can take on any of a continuous spectrum of values.  We may sample the phenomenon at certain 
points in space-time and thus have discrete observations.  Nevertheless, the underlying probability 
distribution is continuous (e.g., Figure~\ref{fig:Fig1_cont_pdf}).

\PSfig[h]{Fig1_cont_pdf}{Example of a continuous probability density function (pdf).  The area under any pdf
must equal 1.  The finite probability identified in (\ref{eq:probfinite}) is indicated in dark gray.}

	Continuous distributions can be thought of as the limit for discrete distributions when the 
``spacing'' between events shrinks to zero.  Hence, we must replace the summation in (\ref{eq:Pdiscretesum}) with the integral
\index{Probability distribution!continuous}
\index{Continuous probability distribution}
\index{pdf (probability density function)}
\index{Probability density function (pdf)}
\begin{equation}
\int^\infty _{-\infty} p (x) d x = 1.
\label{eq:pdf}
\end{equation}
Because of their continuous nature, functions such as $p(x)$ in (\ref{eq:pdf}) are called \emph{probability} 
\emph{density functions} (pdf).  The probability of an event is still defined by the area under the curve, but 
now we must integrate to find the area and hence the probability.
E.g.,  the probability that a random variable will take on a  value between $a - \Delta$  and $a +\Delta$ is 
\begin{equation}
P(a\pm \Delta) =  \int ^{a+\Delta} _{a - \Delta} p(x) dx.
\label{eq:probfinite}
\end{equation}	 
As $\Delta \rightarrow 0$ we find that the probability goes to zero.  Thus, the probability of getting exactly $x = a $
is nil.

	The \emph{cumulative distribution function} (cdf) gives the probability that an observation less than or 
equal to $a$ will occur.  We obtain the integral expression for this distribution by replacing the 
lower limit by $-\infty$ and the upper limit by $a$, finding
\begin{equation}
	\index{Probability distribution!cumulative}
	\index{Cumulative probability distribution}
P_c(a) = \int^a _{-\infty} p (x) dx.
\end{equation}
Obviously, as $a \rightarrow \infty, P_c(a)\rightarrow 1$.  Given the cumulative distribution function we can
revisit (\ref{eq:probfinite}) and instead state
\begin{equation}
P(a\pm \Delta) =  P_c(a+\Delta) - P_c(a - \Delta).
\label{eq:probfinite2}
\end{equation}

\subsection{The normal distribution}
\index{Normal distribution|(}
\index{Gaussian distribution|(}

	So far the function $p(x)$ has been arbitrary. Any continuous function with unit area under 
the curve (i.e., \ref{eq:pdf}) would qualify.  We will now turn our attention to the best known and most frequently 
used pdf: the \emph{normal distribution}.  Its study dates back to  18th 
century investigations into the nature of experimental error.  It was found that repeat 
measurements of the same quantity displayed a surprising degree of regularity.  In particular, the German scientist K. 
F. Gauss played a major role in developing the theoretical foundations for the normal distribution,
hence its other name: the \emph{Gaussian} distribution.  It is given by
\begin{equation}
p(x) = \frac{1}{\sigma \sqrt{2 \pi}} e^{- \frac{1}{2} \left( \frac{x-\mu}{\sigma} \right) ^2 },
\label{eq:gnorm}
\end{equation}
where $\mu$ and $\sigma$ have been defined previously.  The constant term before the exponential normalizes the 
area under the curve to unity (Figure~\ref{fig:Fig1_normal_pdf}).  As discussed in Section~\ref{sec:zscore},
it is often convenient to transform your data into so-called 
\emph{standard scores}:
\begin{equation}
	\index{Standard scores}
	\index{Normal scores}
z_i = \frac{x_i - \mu} {\sigma},
\end{equation}
in which case (\ref{eq:gnorm}) reduces to 
\begin{equation}
p(z) = \frac{1}{ \sqrt{2\pi}} e^{-\frac{1}{2}z^2},
\end{equation}
which has zero mean and unit standard deviation.

\PSfig[h]{Fig1_normal_pdf}{A normally distributed data set will have almost all of
its values within $\pm 3\sigma$  of the mean (this corresponds to 99.73\% of the data; see legend for percentages corresponding to other multiples of $\pm \sigma$).}

Given the functional form of $p(z)$ we can evaluate the probability that an observation 
$z$ will be $\leq a$:
\begin{equation}
P_c(a) = \int ^a_{-\infty} p (z) = \int ^0 _{-\infty} p (z) + \int^a_0 p(z) = \frac{1}{2} + \frac{1}{\sqrt{2\pi}} \int ^a_0
e^{- \frac{z^2}{2}} dz.
\end{equation}	 
Let
\begin{equation}
u^2 = \frac{z^2}{2}, \mbox{ hence } dz = \sqrt{2} du,
\end{equation}	 
then
\begin{equation}
P_c(a) = \frac{1}{2} + \frac{1}{\sqrt{\pi}} \int^{\frac{a}{\sqrt{2}}}_{0} e^{-u^2} du = 
\frac{1}{2} + \frac{1}{\sqrt{\pi}} \frac{\sqrt{\pi}}{2} \erf{\left ( \frac{a}{\sqrt{2}} \right)} =
\frac{1}{2} \left [ 1 + \erf \left( \frac{a}{\sqrt{2}} \right) \right ].
\label{eq:erf}
\index{Error function ($\erf$)}
\index{$\erf$ (error function)}
\end{equation}	 
It follows that, for any value $z$, the cumulative distribution function is
\begin{equation}
	\index{Cumulative normal distribution}
P_c(z) = \frac{1}{2} \left [ 1 + \erf \left ( \frac{z}{\sqrt{2}} \right) \right]. 
\end{equation}
Here, $\erf$ represents the \emph{error function} and it is defined by the definite integral in (\ref{eq:erf})
and tabulated in Table~\ref{tbl:Critical_z}.
Furthermore, the probability that $z$ falls between $a$ and $b$ must necessarily be 
\begin{equation}
P_c (a \leq z \leq b) = P_c (b) - P_c (a)
 = \frac{1}{2} \left[ \erf  \left ( \frac{b}{\sqrt{2}}\right) -
\erf \left( \frac{a}{\sqrt{2}}\right)\right].
\label{eq:cumpart}
\end{equation}
\begin{example}
Investigations into the strength of olivine have provided estimates of Young's modulus ($E$) 
that follow a normal distribution given by $\mu = 1.0\cdot10^{11}$ Pa and $\sigma = 1.0 \cdot 10^{10}$ Pa.  What is the 
probability that a single estimate $E$ will lie in the interval  $9.8 \cdot 10^{10}$ Pa $<$ E $< 1.1\cdot 10^{11}$ Pa?  We 
convert the limits to normal scores and find they correspond to the interval $-0.2 \leq z \leq 1.0$.  Using 
these values for $a$ and $b$ in (\ref{eq:cumpart}) (or using Table~\ref{tbl:Critical_z}) we find the probability to be 0.4206.
\end{example}

\subsubsection{Approximate binomial distribution}
\index{Binomial probability distribution!approximation}
	Like the Poisson distribution, the normal distribution may also serve as an approximation to the binomial distribution
when $n$ is large. More specifically, this approximation holds when both $np$ and $(1 - p)n$ exceed 5.  Under those circumstances,
the mean and standard deviation of the approximate normal distribution become
\begin{equation}
	\index{Binomial probability distribution!approximation}
\mu = np, \quad \sigma = \sqrt{np(1-p)},
\label{eq:binomial_approx}
\end{equation}	 
leading to the simplified distribution
\begin{equation}
P_b(x) = \frac{1}{\sqrt{2 \pi np (1- p)}} \exp{\left \{ \frac{-(x-np)^2}{2 np (1-p)}\right \}}.
\label{eq:binomial_approx_norm}
\end{equation}
\begin{example}
What is the probability that at least 70 of 100 sand grains will be larger than 0.5 mm if 
the probability that any single grain is that large is $p = 0.75$?  Using the approximation (\ref{eq:binomial_approx}) we 
find $\mu = np = 75$ and $s   = \sqrt{np(1-p)} = 4.33$.  Converting 69.5 (halfway between 69 and 70) to a $z$ 
score gives $-1.27$, and we find via Table~\ref{tbl:Critical_z} that the probability becomes 0.898 or about 90\%.
\end{example}
\index{Gaussian distribution|)}
\index{Normal distribution|)}

\subsection{The exponential distribution}
\index{Exponential distribution}
\index{Probability distribution!exponential}

	Another important probability density distribution is the \emph{exponential} distribution.  It is given by
\begin{equation}
p_e(x) = \lambda e ^{-\lambda x}
\end{equation}
for some constant $\lambda$.  However, most of the time we will see it used as a cumulative distribution function:
\begin{equation}P_c(x) = 1-e^{-\lambda x}.
\label{eq:cum_exp_dist}
\end{equation}
Eq.\ (\ref{eq:cum_exp_dist}) gives the probability that the observation $a$ will be in the range $0 \leq a \leq x$.
\begin{example}
It has been reported that the heights ($z$) of Pacific seamounts follow an 
exponential distribution defined as 
\begin{equation}
P_c(z \leq h) = 1 - e^{-h/340},
\label{eq:Pac_segments}
\end{equation}
which gives the probability that a seamount is shorter than $h$ meters.  Equation 
(\ref{eq:Pac_segments}) then predicts that we might expect that 
\begin{equation}
P_c(1000) = 1 - e^{-1000/340} \approx 95\%
\end{equation}
of them are less than one km tall.
\end{example}

\subsection{Log-normal distribution}
\index{Log-normal distribution}
\index{Probability distribution!log-normal}

	Many data sets, such as grain-sizes of sediments and geochemical concentrations, have very 
skewed and long-tailed distributions (e.g., Figure~\ref{fig:Fig1_lognormal}).  In general, such distributions arise when the observed 
quantities have errors that depend on \emph{products} rather than \emph{sums}.  It therefore follows that the \emph{logarithm} of the 
data may be normally distributed.  Hence, taking the logarithm of your data may make the 
transformed distribution look normal.  If this is the case, you can apply standard statistical 
techniques applicable to normal distributions to the logarithm of your data and convert the results 
(e.g., mean, standard deviation) back to get the proper units.  The log-normal probability density distribution is therefore given by
\begin{equation}
	p(x) = \frac{1}{\sigma \sqrt{2 \pi}} e^{- \frac{1}{2} \left( \frac{\log x-\mu}{\sigma} \right) ^2 }.
	\label{eq:lognorm}
\end{equation}
\PSfig[h]{Fig1_lognormal}{(left) The concentration of Pb in soil is very long-tailed and clearly not normally distributed.
The red squares indicate individual sample values. (right) The same distribution after taking the logarithm of the data values.
The resulting distribution is approximately normal, hence a log-normal distribution might be suitable to describe the data.}
\index{Probability distributions|)}

\section{Inferences about Means}
\index{Confidence interval!sample mean}
\index{Sample!mean!confidence interval}

\index{Central limit theorem}
	The central limits theorem states that the mean of a large sample taken from any distribution will be 
normally distributed even if the data themselves are not normally distributed, and furthermore it says
that the sample mean is an unbiased estimator of the population mean.  We can then use our 
knowledge of the normal distribution to quantify our faith in the precision of our sample mean.  
We already know that $s_{\bar{x}} = \sigma/ \sqrt{n}$,  so we can state with probability $1-\alpha$
that $\bar{x}$ will differ from $\mu$ by at most $E$, which is given by
\begin{equation}
E = z_{\alpha/2} \cdot \frac{s}{\sqrt{n}},
\label{eq:sample_error}
\end{equation}
where $s$ is our estimate of $\sigma$.  In other words, the chance that $\bar{x}$ exceeds
the $\pm z_{\alpha/2}$ confidence interval is $\alpha$.
These error estimates apply to large samples ($n \geq 30$) and infinite populations.  In those cases we 
can use our sample standard deviation $s$ in place of $\sigma$, which we usually do not know.  Here, (\ref{eq:sample_error}) can 
be inverted to yield the sample size necessary to be confident that the error in our sample mean is 
no larger than $E$, and we find
\begin{equation}
n = \left( \frac{z_{\alpha/2} \cdot s}{E} \right)^2.
\end{equation}

\PSfig[h]{Fig1_normal_tails}{Probability is $\alpha$ that a value will fall in one of the two tails of
the normal distribution, and $\alpha/2$ that it will fall in a specific tail.}

The normal score for our sample mean is
\begin{equation}
z = \frac{\bar{x} - \mu} {s_{\bar{x}}} = \frac{\bar{x} - \mu} {s/ \sqrt{n}}.
\end{equation}	 
Since this statistic is normally distributed we know that the probability is $1 - \alpha$ that $z$ will take on 
a value in the interval $-z _{\alpha/2} < z < +z_{\alpha/2}$.  Plugging in for the limits on $z$,
\begin{equation}
-z_{\alpha/2} < \frac{\bar{x} - \mu}{s/\sqrt{n}} < +z_{\alpha/2}
\end{equation} 	  
or							
\begin{equation}
\bar{x} - z _{\alpha/2} \cdot \frac{s}{\sqrt{n}} < \mu < \bar{x} + 
z _{\alpha/2} \cdot  \frac{s}{\sqrt{n}}.
\label{eq:conf_inter}
\end{equation}
Rearranging, we find
\begin{equation}
\mu = \bar{x} \pm z _{\alpha/2} \cdot \frac{s}{\sqrt{n}}.
\label{eq:conf_inter_mu}
\end{equation}
    Eq.\ (\ref{eq:conf_inter_mu}) shows the \emph{confidence interval} on $\mu$ at the $1 - \alpha$ confidence level.
Very often, our confidence levels will be 95\% ($\sim 2 \sigma$) or 99\% ($\sim3\sigma$).

\subsection{Small samples}
\index{Small sample}
\index{Sample!small}

\index{Central limit theorem}
	The previous section dealt with large ($n \geq 30$) samples, where we could assume that $\bar{x}$ would be 
normally distributed as dictated by the central limits theorem.  For smaller samples we must 
assume instead that the \emph{population we are sampling} is normally distributed.  We can then base our 
inferences on the statistic
\begin{equation}
	\index{Student's $t$ distribution}
	\index{Test!Student's $t$}
	\index{Probability distribution!Student's $t$}
t = \frac{\bar{x} - \mu}{s_{\bar{x}}} = \frac{\bar{x} - \mu}{s/ \sqrt{n}},
\end{equation}
whose distribution is called the \emph{Student's} $t$-distribution (Figure~\ref{fig:Fig1_normal_tdist}).  It is similar to the normal distribution but 
its shape depends on the degrees of freedom, $\nu = n -1$.  For large $n$ (and hence $\nu$) the $t$ 
statistics approach the $z$ statistics.  As for $z$ statistics, one can find tables with $t$ values 
for various combinations of confidence levels and degrees of freedom (see Table~\ref{tbl:Critical_t}).  For insight into what the $t$-distribution
and others really are, get \emph{Numerical Recipes} by Press et al.  This excellent book gives both 
theory and computer code (in C++, C, FORTRAN, Java and a host of legacy languages).
\index{Numerical recipes}
\PSfig[h]{Fig1_normal_tdist}{The same normal distribution and critical tails as in Figure~\ref{fig:Fig1_normal_tails}, overlain
by the Student's $t$-distribution for $\nu = 3$ degrees of freedom (red line).  For small samples the
probability distribution becomes wider.}
\begin{example}
Given our sandstone density estimates from earlier, i.e., \{2.2, 2.25, 2.25, 2.3, 2.3, 2.3, 2.35\}, what is 
the 95\% confidence interval on the population mean?

\emph{Answer}:  We have $\bar{x} = 2.28$ with $s = 0.05$, and $\alpha = 1 - 95\% = 0.05$.  The degrees of freedom $\nu = n - 1 = 6$.
Table~\ref{tbl:Critical_t} gives $t_{\alpha/2,\nu} = t_{0.025,6} = 2.447$.
Using (\ref{eq:conf_inter}), we find our sample mean brackets the population mean, thus (with $t_{\alpha/2}$
instead of  $z_{\alpha/2}$ and $s$ instead of $\sigma$)
\begin{equation}
2.28 - 2.447 \cdot \frac{0.05}{\sqrt{7}} < \mu < 2.28 + 2.447 \cdot \frac{0.05}{\sqrt{7}}
\end{equation}	 
or (since the bounds are symmetrical)
\begin{equation}
2.234 < \mu < 2.326
\label{eq:conf_interval}
\end{equation}	 
or 
\begin{equation}
\mu = 2.280 \pm 0.046.
\end{equation}
\end{example}

\clearpage
\section{Problems for Chapter \thechapter}

\begin{problem}
A student prepares for an exam in a data analysis class by studying a list of 10 specific topics.  She is
confident that she can answer any question related to six of these topics, but is ill prepared to handle the
remaining four.  For the exam, the instructor selects five topics at random from the same list of 10 topics.
What is the probability that the student can solve all five problems on the exam?
\end{problem}

\begin{problem}
During an expedition to Antarctica a research team collects 22 oriented rock cores to be used to determine
the paleo-magnetic field.  Among the 22 samples there are 7 of utmost importance: three are from an exciting new
basaltic outcrop and four were recovered from another area with no prior samples.  During the flight back to Punta Arenas
the Principal Investigator samples too much Chilean wine and proceeds to trip over the box with the rock samples,
causing 8 of the rock cores to fall out and break.
\begin{enumerate}[label=\alph*)]
\item What is the probability of total disaster (i.e., all 7 important samples are destroyed)?
\item What is the probability that the 7 samples are all intact?
\item What is the probability that \emph{at least} two samples from each of the two exciting areas have been ruined?
\end{enumerate}
\end{problem}

\begin{problem}
Returning from extensive fieldwork
in the Congo, a biologist calmly discovers that the glue behind the
labels on her glass specimen jars with spiders has dissolved and all the labels have
separated from their corresponding jars.  Of a total of 28 specimens, 8 of the specimens
are previously unknown spiders; the remaining 20 are known to be harmless to humans.  Given past
experience that half of newly discovered spiders are venomous,
what is the probability that, in randomly selecting four specimens:
\begin{enumerate}[label=\alph*)]
\item One of the specimens is a venomous spider?
\item All four are nonvenomous?
\end{enumerate}
Note: Unlike the biologist, the technician
making the selection has no knowledge of spider species.
\end{problem}

\begin{problem}
Tuco owns a tire manufacturing plant in South America that makes
automobile tires for American-produced cars.  For each batch of 100 tires, Tuco's quality control
team goes to work:  They randomly select four tires, mount them on a vehicle, and
give each tire a solid kick.  If any one of the four tires fall apart then the whole batch is sent
back for reprocessing.  How many defective tires can one batch have and still have at least
a 50/50 chance of passing the test? [Hint: Plot the probability of passing the
test as a function of the number of defective tires and graphically find the answer.]
\end{problem}

\begin{problem}
A manufacturer of compasses used in geological field mapping
has three physical plants that make the compasses.  Plant A produces 55\%, plant B 30\%, and plant
C 15\% of the total production.  If 0.4 \% of the compasses from plant A are defective, and the corresponding
numbers for plants B and C are 0.6 \% and 1.2\%, what is the probability that a defective
compass purchased by mail-order was produced by plant A?

\end{problem}

\begin{problem}
In the poor kingdom of Parador the levels of pollutants in water wells are measured using older,
imprecise instruments.  Here, 25\% of all wells have excessive amounts of pollutants.  When tested, 99\% of all
wells that have excessive amounts of pollutants will fail the test, but 17\% of the wells that do \emph{not} have
excessive amounts of pollutants will also fail due to the poor quality of the instruments.  What is the
probability that a well failing the test actually has excessive amounts of pollutants?
\end{problem}

\begin{problem}
A sample of 64 specimens of a particular fossil gives a mean length of 52.8 mm, with a standard deviation of 4.5 mm.
Find the 99\% confidence intervals for the mean length.
\end{problem}

\begin{problem}
An expedition measuring the heat flux, $q$, out of the seafloor near a mid-ocean ridge returned
with 13 independent measurements (in mW m$^{-2}$), given by

\[
q = \{45.2, 47.4, 55.1, 39.2, 51.2, 46.3, 49.9, 42.9,
75.3, 53.1, 48.8, 58.8, 42.2\}.
\]

\begin{enumerate}[label=\alph*)]
\item Estimate the sample mean and sample standard deviation, as well as
the median and the median absolute deviation (MAD).
\item Using robust scores, are there any outliers?  If so, identify them,
remove them from the data, and redo the answers from (a).
\end{enumerate}
\end{problem}

\begin{problem}
Walter, a youngster growing up in rural North Dakota, spends two winter weeks in bed with chicken pox.
To stave off boredom he decides to measure the lowest temperatures (in \DS C) reached during each night.
He obtains the following series of 14 measurements:
\begin{table}[H]
	\centering
	\begin{tabular}{|c|c|c|c|c|c|c|} \hline
	-51.32 & -49.34 & -42.41 & -56.72 & -45.92 & -50.33 & -47.09 \\  \hline
	-53.39 & -24.23 & -27.41 & -44.21 & -48.08 & -39.08 & -54.02 \\  \hline
	\end{tabular}
\end{table}
\begin{enumerate}[label=\alph*)]
\item Estimate the sample mean and sample standard deviation, as well as
the median and the median absolute deviation (MAD).
\item Using robust scores, are there any outlying data that suggest 
Walter's fever may have affected his measurements?  If so, identify them,
remove them from the data, and redo your analysis.
\end{enumerate}
\end{problem}

\begin{problem}
The data set \emph{depths.txt} contains the bathymetric depths (in meters) for part of an ocean basin.
\begin{enumerate}[label=\alph*)]
\item Find the mean depth, the standard deviation, and the 95\% confidence interval on the mean depth.
\item What is the probability that a random depth measurement will be shallower than $-4000$m?
\item Determine the median and median absolute deviation (MAD).
\item Given the criteria for outliers ($|z_i| > 2.5$) using the median and MAD, find the robust confidence limits in meters.
How many measurements are considered outliers?
\end{enumerate}
\end{problem}

\begin{problem}
Previous sampling of the salinity in tap water from a local water company has
revealed that the salinity content is well described as following a normal distribution,
with $\mu = 10$ ppm and $\sigma = 5$ ppm.  When randomly testing households in this
neighborhood for the salinity of the drinking water, what is the probability that
the salinity will be:
\begin{enumerate}[label=\alph*)]
\item Less than 4 ppm?
\item Between 8 and 16 ppm?
\end{enumerate}
Make sure to illustrate the two cases graphically.
\end{problem}

\begin{problem}
In Texas, past experience has shown that, on average, only one in 10 exploratory wells drilled discovers oil.
Let $n$ be the number of holes drilled until the first success (i.e., oil is struck).  Assume that
the exploratory wells represent independent events.
\begin{enumerate}[label=\alph*)]
\item Find $P(1)$, $P(2)$, and $P(3)$.
\item Derive a formula for $P(n)$.
\item Plot $P(n)$ for $n = 1$ to $30$.
\item Find $P_c(n)$, the cumulative probability that we will
find oil in $n$ \emph{or less} tries, and plot it.
\item How many holes should we expect to drill
in order to have a 90\%  probability of finding oil?
\end{enumerate}
\end{problem}

%  $Id: DA1_Chap4.tex 670 2018-12-20 20:37:20Z pwessel $
%
\chapter{TESTING OF HYPOTHESES}
\label{ch:testing}
\epigraph{``The great tragedy of science -- the slaying of a beautiful hypothesis by an ugly fact.''}{\textit{Thomas Huxley, Biologist}}

Much of statistics is concerned with testing hypotheses for certain properties of entire populations
based on sample data taken from these populations.  There are numerous standard 
techniques used to perform these tests, and we will broadly group them into two sets:
\begin{enumerate}
	\item Parametric tests
	\item Nonparametric tests
\end{enumerate}
This chapter will discuss the assumptions behind both kinds of tests and how they are performed.

\section{The Null Hypothesis}
\index{Null hypothesis}

	At the core of all tests lies the concept of the ``null hypothesis'' (also
known informally as the ``boring hypothesis'').  The null hypothesis,
denoted $H_0$, is stated and we will use our tests to see if we can reject it.  For instance, if 
we want to test whether two rock types (i.e., two separate populations) have different densities, we 
obtain samples from each population and form the null hypothesis 
that they have equal densities (a boring result); we then test if we can reject $H_0$.
We will illustrate this approach with an 
example:
\begin{example}
	It is claimed that the density of a particular sandstone unit is 
2.35 g cm$^{-3}$.  We are handed a sample 
of 50 specimens from an outcrop in the same area and decide to set the criteria that the sample 
comes from another lithological unit if the sample mean is less than 2.25 or larger than 2.45.  In other
words, our null hypothesis $H_0$ is $2.25 < \mu < 2.35$.  This statement is 
a clear-cut criterion for accepting or rejecting the claim that the sample originates from the same unit, 
but it is not infallible.  Since our decision will be based on a sample, there is the possibility that 
the sample mean $\bar{x}$ may indeed satisfy $\bar{x} < 2.25$ or $\bar{x} > 2.45$ \emph{even though} the
population mean $\mu$ \emph{is} 2.35.  We will 
therefore want to know what the chances are that we could make a wrong decision and reject $H_0$.
Clearly, we must investigate what is the probability that $\bar{x} < 2.25$ or $\bar{x} > 2.45$ when $\mu$ in fact is $2.35$.  
Here, $s = \sigma = 0.42$.  This probability is given by the area under the two tails in 
Figure~\ref{fig:Fig1_type1_error}.

\PSfig[h]{Fig1_type1_error}{Type I error: Possibility of erroneously rejecting a correct hypothesis.}

Since $n = 50 \gg 30$, we will treat our sample as of infinite size.  We find the uncertainty in the sample mean to be
\begin{equation}
s_{\bar{x}} = \frac{s}{\sqrt{n}} = \frac{0.42}{\sqrt{50}} = 0.06.	 
\end{equation}
We can now evaluate the normal scores of the two limits as
\begin{equation}
z_0 = \frac{2.25 - 2.35}{0.06} = -1.67,
\end{equation}
\begin{equation}
z_1 = \frac{2.45 - 2.35}{0.06} = + 1.67.
\end{equation}
We find the area under each tail to be $\frac{\alpha}{2} = \frac{1}{2} \left[ 1 + \ \erf \left ( -1.67 / \sqrt{2}\right) \right] = 0.0475$.
Consequently, the probability of getting a sample 
mean that falls in the distal tails of the distribution is
\index{Test!one sample mean}
\begin{equation}
\alpha = 2 \cdot 0.0475 = 0.095 \mbox{ or } 9.5\%.
\end{equation}	 
This result means there is a 9.5\% chance we will erroneously reject the hypothesis that $\mu = 2.35$ 
when it is in fact \emph{true}.  In statistics, we say we have committed a
\emph{type I error}\footnote{Its name suggests there are more ways to make mistakes...}.
\index{Type I error}
\end{example}

\PSfig[h]{Fig1_type2_error}{Type II error: Possibility of erroneously accepting an incorrect hypothesis.}

	Let us look at another possibility, where our test will fail to detect that $\mu$ is \emph{not} equal to 2.35.
\begin{example}
Suppose, for the sake of argument, that the true mean $\mu = 2.53$.  Then, the probability of getting a 
sample mean in the range $2.25 - 2.45$, and hence erroneously accept the claim that $\mu = 2.35$, is 
now given by the tail area in Figure~\ref{fig:Fig1_type2_error}.
 As before, $s_{\bar{x}} = 0.06$ so the normal scores become
\begin{equation}
z_0 = \frac{2.25 - 2.53}{0.06} = -4.67, \quad \quad z_1 = \frac{2.45-2.53}{0.06} = -1.333.
\end{equation}	           
It follows that the probability $\beta = \frac{1}{2} \left[ \erf\left( -1.333 / \sqrt{2} \right)- \ \erf \left( -4.67/ \sqrt{2} \right) \right ]   = 0.092$ or $9.2\%$.
This is the risk we run of accepting the incorrect hypothesis $H_0$.  We call this committing a \emph{type II error}.
\index{Type II error}
\end{example}

Therefore, we recognize that there are several possible outcomes when testing a null hypothesis, as
shown in Table~\ref{tbl:typerror}.
\begin{table}[h]
\centering
\begin{tabular}{|l|l|l|} \hline
&  \bf{Accept} $H_0$  & \bf{Reject} H$_{0}$ \\ \hline
$H_0$ is true & Correct Decision & Type I Error \\ \hline
$H_0$ is false & Type II Error & Correct Decision \\ \hline
\end{tabular}
\caption{The four possible decision scenarios when testing an hypothesis.  Of these, we always
seek to avoid making a Type II error.}
\label{tbl:typerror}
\end{table}
If the hypothesis is true, but is rejected, we have committed a Type I error, and the probability of 
doing so is designated $\alpha$.  In our example, $\alpha$ was 0.095.  If our hypothesis is incorrect, but we 
still accept it, then we have committed a Type II error, and the probability of doing so is 
designated $\beta$.  In our case, with $\mu = 2.53, \beta$  was 0.092.

	We saw in our example that the type II error probability \emph{depended on the value of} $\mu$.  Since $\mu$ 
is typically not known and we cannot evaluate $\beta$, it is common to simply either reject $H_0$ or \emph{reserve judgment} (i.e., never 
accept $H_0$).  This way we avoid committing a type II error altogether, at the expense of never 
accepting $H_0$.  We call this a \emph{significance test} and say that the results are \emph{statistically significant} 
if we can reject $H_0$.  If not, the results are \emph{not} statistically significant, and we attempt no further 
decisions.  Of course, we may be wrong in rejecting $H_0$ but we can always state the likelihood of being wrong as $\alpha$.
Hence, in statistics we can only disprove hypotheses, but never prove them.
\index{Statistically significant}

\section{Parametric Tests}
\index{Parametric tests|(}

Parametric tests are used to make decisions based on parameters derived by assuming the data are
approximately described by a known probability distribution.  The parameters typically used
are properties of the distribution, such as the mean and standard deviations.  Note that while we obtain our statistical parameters
from the \emph{sample}, our hypothesis testing applies to the \emph{parent} population.

\subsection{What are the degrees of freedom?}
\label{sec:freedom}
\index{Degrees of freedom|(}
When dealing with statistical tests we encounter the concept of ``degrees of freedom'' (usually
denoted by the variable $\nu$) and it is
often confusing to know what it is for various cases.  We need to understand what $\nu$ is in
order to perform the tests.  Typically, we will compute a \emph{statistic} from our sample data.
It may represent our best estimate of one \emph{parameter} of the theoretical population distribution we are
investigating via our limited sample.  For instance, we may wish to compute a statistical quantity that requires
us to use the mean of our data set of $n$ points.  While we will initially use $\nu = n$ in evaluating the mean
(since all the points are independent), once we use
the mean in \emph{subsequent} calculations we must reduce $\nu$ by one, since any individual data point can
now be obtained from that mean and any $n-1$ data points.  In general, we say we lose one degree of
freedom for each parameter we have estimated from the data.  We will also see that certain transformations
of the data, say, grouping the data into a set of $k$ bins, will also affect $\nu$.  While you may have had
$n$ data points to start with, after binning you now only have $k$ items (the bin counts), so any testing involving these bins
will have to consider a $\nu$ that may only be $k-1$ (you still lose one since the bin counts must sum to $n$).  However,
the degrees of freedom may be even less than $k-1$ if any other parameters had to be estimated in order to
facilitate the binning in the first place.  In this Chapter, we will illuminate these situations using relevant examples.
\index{Degrees of freedom|)}

\subsection{Differences between sample means (equal variance)}
\index{Sample!mean!differences}
\index{Differences between sample means}
\index{Test!two sample means}
\label{sec:twomeans}
	We will often want to know if an observed difference between two sample means can be attributed to 
chance.  We will again rely on the Student's $t$ distribution.  Here, it is assumed that the two populations
have the \emph{same variance} but possibly \emph{different means}.  

	We are interested in the distribution of $\bar{x}_1 - \bar{x}_2$, the difference in sample means.
Of course, we only have one such estimate, but if the two
samples are independent and random then the hypothetical difference distribution will be approximately normal,
with mean  $\mu_1 - \mu_2$  and standard deviation
\begin{equation}
\sigma_e = \sigma_p \sqrt{ (1/n_1 + 1/n_2)}.
\end{equation}
Here, $\sigma_p$ is the \emph{pooled} standard deviation,
\begin{equation}
	\index{Pooled standard deviation}
	\index{Standard deviation!pooled}
\sigma_p = \sqrt{ \frac{(n_1 - 1 ) \sigma^2_1 + (n_2 - 1) \sigma^2_2}{n_1 + n_2 - 2}     }.	 
\end{equation}
We find the $t$-statistic by evaluating
\begin{equation}
t = \frac {\bar{x}_1 - \bar{x}_2} { \sqrt{ \frac{ (n_1 - 1) s^2_1 + (n_2 - 1) s^2_2 }
{n_1 + n_2 -2} \left( \frac{1}{n_1} + \frac{1}{n_2} \right)  }  } = \frac{\bar{x}_1 - \bar{x}_2}{s_e}
\label{eq:t_two_means}
\end{equation}
and test the hypothesis $H_0$: $\mu_1 = \mu_2$  based on the $t$-distribution for  $\nu = n_1 + n_2 - 2$ degrees of 
freedom.  Here, we have lost one degree of freedom for each sample mean we computed.
For large $n_1, n_2$, the $t$-distribution becomes very close to a normal distribution and we 
may instead use $z$-statistics based on
\begin{equation}
z = \frac{\bar{x}_1 - \bar{x}_2} { \sqrt{ \frac{s^2_1} {n_1} + \frac{s^2_2}{n_2} }}.
\label{eq:twosamplez}
\end{equation}
We will illustrate the two-sample $t$-test with an example.
\begin{example}
We have obtained random samples of 
magnetite-bearing rocks from two separate basalt outcrops.  The measured magnetizations (in Am$^2$kg$^{-1}$) are
\begin{quote}
	Outcrop 1: \{87.4, 93.4, 96.8, 86.1, 96.4\}   $n_1 = 5$

	Outcrop 2: \{106.2, 102.2, 105.7, 93.4, 95.0, 97.0\}   $n_2 = 6$
\end{quote}
We state our null hypothesis $H_0$: $\mu_1 = \mu_2$; the alternative hypothesis is of course H$_1:\mu_1 \neq \mu_2$.  
We decide to use a 95\% confidence level, so the level of significance $\alpha = 0.05$.
In this case, $\nu = 5 + 6 - 2 = 9$, and Table~\ref{tbl:Critical_t}
shows that the critical $t$ value is 2.262.  We will reject $H_0$ if our calculated
$t$ exceeds this critical value.  Using the data, we find
\begin{equation}
\bar{x}_1 = 92.0 \mbox{ with } s_1 = 5.0,
\end{equation}
\begin{equation}
\bar{x}_2 = 99.9 \mbox{ with } s_2 = 5.5.
\end{equation}
Using (\ref{eq:t_two_means}) we obtain
\begin{equation}
t = \frac{99.9 - 92.0}{\sqrt{  \frac{4\cdot 5^2 + 5\cdot5.5^2}{9}  \left(  \frac{1}{5} + \frac{1}{6}   \right) }} = 2.5.
\end{equation}
Since $t > 2.262$ we must reject $H_0$. We conclude that the magnetizations at the two outcrops are 
not the same.
\end{example}

\subsection{Differences between sample means (unequal variance)}
\index{Sample!mean!differences}
\index{Differences between sample means}
\index{Test!two sample means}
\index{Test!Welch's $t$}
\index{Welch's $t$-test}
\label{sec:twomeansw}

In situations where the two populations \emph{do not} have equal variance (which we will test for in Section~\ref{sec:twostd}) one
can use \emph{Welch's} $t$-test instead.  It is similar to the regular two-sample $t$-test but now the observed
$t$ value is obtained directly from (\ref{eq:twosamplez}).  However, the estimation of the
degrees of freedom differs and is given by
\begin{equation}
	\displaystyle
	\nu \approx \frac{\left( \frac{s^2_1} {n_1} + \frac{s^2_2}{n_2} \right)^2}{\frac{s^4_1} {n^2_1 \nu_1} + \frac{s^4_2}{n^2_2 \nu_2}}.
\label{eq:Welch2nu}
\end{equation}
Rounding $\nu$ to the nearest integer allows you to look up critical $t$ from Table~\ref{tbl:Critical_t}.

So far, we have put confidence limits on sample means and compared sample means 
to investigate whether two populations have different means.  We will now turn our attention to 
inferences about the standard deviation.

\subsection{Inferences about the standard deviation}

\PSfig[h]{Fig1_chi2_dist}{A typical chi-square probability density function, with mean value $\nu$ and mode at $\nu-2$.}

The most popular way of estimating $\sigma$ is to compute the sample standard deviation.  When 
investigating properties of $\sigma$ using $s$ as a proxy we will rely on the ``chi-square'' statistic, given by
\begin{equation}
	\index{$\chi^2$ statistic (``chi-squared'')}
	\index{Chi-squared statistic ($\chi^2$)}
	\index{Probability distribution!Chi-squared}
	\index{Probability distribution!$\chi^2$}
\chi^2=\frac{(n-1)s^2}{\sigma^2}.
\label{eq:chi2_stat}	 	
\end{equation}
The $\chi^2$ distribution depends on the degrees of freedom, $\nu = n - 1$, and it is restricted to positive 
values because of the squared terms.  It portrays how the sample variances would be 
distributed if we selected random samples of $n$ items from a normal distribution with a standard deviation of $\sigma$.
Figure~\ref{fig:Fig1_chi2_dist} shows a typical $\chi^2$ distribution.
In the same way we used $z_\alpha$ and $t_\alpha$, we now use $\chi^2_\alpha$ as the value for which the area to the right 
of $\chi^2_\alpha$ equals $\alpha$.  Because the distribution is not symmetrical, we must evaluate the
critical values for $\alpha/2$ and $1 - \alpha/2$ separately.
Furthermore, in the same way we placed confidence intervals on $\mu$, we now 
use (\ref{eq:chi2_stat}) to place confidence intervals on the variance, i.e.,

\begin{equation}
\chi^2_{1-\frac{\alpha}{2}}<\frac{(n-1)s^2}{\sigma^2}<\chi^2_\frac{\alpha}{2}
\end{equation}	 
or
\index{Confidence interval!sample standard deviation}
\index{Sample!standard deviation!confidence interval}
\begin{equation}	 
\frac{(n-1)s^2}{\chi^2_\frac{\alpha}{2}}<\sigma^2<\frac{(n-1)s^2}{\chi^2_{1-\frac{\alpha}{2}}},
\end{equation}
which gives the $\alpha$ confidence interval on the variance.  For large samples $(n > 30)$ this can be 
simplified to 
\begin{equation}
\frac{s}{1+{z_\frac{\alpha}{2}/\sqrt{2n}}}<\sigma<\frac{s}{1-{z_\frac{\alpha}{2}/\sqrt{2n}}}.
\end{equation}
Note that the confidence interval is not symmetrical about the sample standard deviation.

\subsection{Testing a sample standard deviation}
\index{Sample!standard deviation!test}
\index{Test!standard deviation (one sample)}

	We might want to test whether our sample standard deviation $s$ is equal to or different from a 
given population standard deviation $\sigma$.  In such a case the null hypothesis becomes H$_0: s = \sigma$, with the alternative 
hypothesis H$_1: s \neq \sigma$.  As usual, we select our level of significance to be $\alpha$ = 0.05.
\begin{example}
	We have 15 estimates of temperatures with $s = 1.3^\circ$ C and we want to know if $s$ is 
significantly different from $\sigma = 1.5^\circ$ C based on past experience.  From $\alpha$ = 0.05 and $\nu$ = 14 (since we
lose one degree of freedom by first computing $\bar{x}$ to obtain $s$), we find the 
critical $\chi^2$ values from Table~\ref{tbl:Critical_chi2} to be $\chi^2_{0.975} = 5.63$ and $\chi^2_{0.025} = 26.119$.
Based on our sample statistic, we compute
\begin{equation}
\chi^2=\frac{14\cdot1.3^2}{1.5^2}=10.5.
\end{equation}
We see that we cannot reject $H_0$ at the 0.05 significance level, so we simply
reserve judgment.  This was a \emph{two-sided} test since we had to check that $\chi^2$  did not land in 
either of the two tails.
\end{example}

	For large samples ($n \geq 30$), the $\chi^2$ critical values do not vary much with $\nu$ and we may use the simpler statistic
\begin{equation}
z=\frac{s-\sigma}{\sigma/\sqrt{2n}}
\end{equation}
and use the standard $z$-statistics, i.e., the $n = \infty$ entry in Table~\ref{tbl:Critical_t}.

\subsection{Differences between sample standard deviations}
\index{Test!standard deviation (two samples)}
\index{Sample!standard deviation!test}
\label{sec:twostd}
\PSfig[h]{Fig1_F_dist}{A typical $F$ probability density function, for $\alpha = 0.10$, $\nu_1 = 20$ and $\nu_2 = 12$.}

	In the $t$-test for differences between two means (Section~\ref{sec:twomeans})
we \emph{assumed} that the standard deviations of the two samples were the same.
Often this is not the case  and one should first test whether this 
assumption is valid.  We want to know whether the two variances are different or not.  The 
statistic that is most appropriate for such tests is called the $F$-statistic, defined as the ratio
\index{Probability distribution!$F$}
\begin{equation}
F=\left \{ \begin{array}{cc}{s_1}^2/{s_2}^2 & ,s_1>s_2\\
{s_2}^2/{s_1}^2 & ,s_2>s_1
\end{array}\right..
\label{eq:F}
\end{equation}
An example of this distribution as shown in Figure~\ref{fig:Fig1_F_dist}.
For normal distributions, this variance ratio is a continuous distribution called the $F$-distribution.  
It depends on the two degrees of freedom, $\nu_1 = n_1 - 1$ and $\nu_2 = n_2 - 1$.  As before, we will reject 
the null hypothesis H$_0:\sigma_1 = \sigma_2$ at the $\alpha$ level of significance and (possibly) entertain the alternative 
$H_1$: $\sigma_1 \neq \sigma_2$  when our observed $F$-statistic exceeds the critical value $F_{\alpha/2,\nu_1,\nu_2}$.
Note that because of the way (\ref{eq:F}) forces $F$ to be equal to or larger than unity, any $F$-ratio that would
have been located into the left tail is now instead mapped to the right tail.  This is why we use $\alpha/2$ in determining
the critical value as it is still a \emph{two-sided} test.
\begin{example}
In our case of rock magnetizations we assumed that the two standard deviations were approximately 
the same.  Let us now show that this was actually justified.  We find our observed statistic to be
\begin{equation}
	\index{\emph{F}-test}
	\index{F-test}
	\index{Test!\emph{F}}
F=\frac{5.5^2}{5.0^2}=1.21.
\end{equation}	 
From Table~\ref{tbl:Critical_F975} we find $F_{0.025}(\nu_1 = 5, \nu _2 = 4) = 9.36$. Hence, we cannot reject $H_0$ and conclude instead
that the difference in sample standard deviations is not statistically significant at the 95\% level.  In fact, $s_1$ would
have to be over three times larger than $s_2$ ($\sqrt{9.36} = 3.06$ to be exact) before we would be able to reject $H_0$.
\end{example}

\subsection{Testing distribution shape: The $\chi^2$ test}
\index{Test!$\chi^2$ (``chi-squared'')}
\index{Test!chi-squared ($\chi^2$)}
\index{$\chi^2$ test (``chi-squared'')}
\index{Chi-squared test ($\chi^2$)}

	The next parametric test we shall be concerned with is the chi-squared test.  It is a sample-based
statistic using normal scores that are squared and summed up:
\begin{equation}
\chi^2=\sum^n_{i=1}z_i^2=\sum^n_{i=1}\left(\frac{x_i-\bar{x}}{s}\right)^2.
\label{eq:ch:2_test}
\end{equation}
If we draw all possible samples of size $n$ from a normal population and plotted a histogram of the resulting $\sum z^2$ they would 
approximate the $\chi^2$ distribution mentioned earlier.  The $\chi^2$ test is used to compare the \emph{shape} of our data 
distribution to a distribution of known shape (which is usually a normal distribution but need not be).
\PSfig[h]{Fig1_Whitewater}{By desiring equal probabilities ($p = 0.2$) we obtain $z$-values for the bin boundaries that are not equidistant.
This ensures that each bin has an acceptable number of observations and simplifies the calculation of the expected
values.}
\index{Data!binned}
	The $\chi^2$ test is most often used on data that have been categorized or \emph{binned}.  Assuming that 
our observations have been binned into $k$ bins, the test statistic is found as 
\begin{equation}
\chi^2=\sum^k_{j=1}\frac{(O_j-E_j)^2}{E_j},
\label{eq:chi2_test}
\end{equation}
where $O_j$ and $E_j$ are the number of observed and expected values in the $j$'th bin.  At first glance
you may think we have a typographical error in (\ref{eq:chi2_test}) --- we do not.  Note that this $\chi^2$ 
still is nondimensional since we are using counts, even if the denominator is not squared.  With 
counts, the probability that $m$ out of $n$ counts will fall in a given bin $j$ is determined by the 
binomial distribution approximated by (\ref{sec:binom}), with
\begin{equation}
\bar{x}_j = E_j = np_j
\end{equation}
and
\begin{equation}
s_j = \sqrt{np_j (1- p_j)} \approx \sqrt{np_j} = \sqrt{E_j}. 
\end{equation} 	
Here, $p_j$ is the probability that any value will fall in the $j$'th bin.  Plugging in for $\chi^2$, we find
\begin{equation}
\chi^2 = \sum^k_{j=1} \left ( \frac{x_j - \bar{x}_j}{s_j} \right) ^2 =
\sum^k_{j=1} \left( \frac{O_j - E_j}{\sqrt{E_j}} \right) ^2 = \sum^k_{j=1}
\frac{(O_j - E_j)^2}{E_j}.	 
\end{equation}
as stated in (\ref{eq:chi2_test}).  Since $\chi^2 = 0$ would mean a perfect match between observations and
expectations we realize that this $\chi^2$ test is \emph{one-sided}, hence we will only check if our observed $\chi^2$
statistic exceeds the critical $\chi^2_{\alpha, \nu}$ value.
\begin{example}
Consider the 48 measurements of salinity from Whitewater Bay in Florida (Table~\ref{tbl:salinity}). We would like to
know if these observations came from a normal distribution or not.  The 
answer might have implications for models of mixing salt and fresh water in the bay.

	The first step is to transform the data into normal scores.  We find $\bar{x}  = 49.54$ and $s = 9.27$, and
thus convert all values via
\index{Standard units}
\index{Standard scores}
\index{Normal scores}
\begin{equation}
z_i = \frac{x_i - 49.54}{9.27}.
\end{equation}	 
We choose to bin the data into five bins whose boundaries were chosen so that the area under the curve for each bin is the 
same, i.e., 0.2.  This choice ensures that the expected value will be the same for all bins (Figure~\ref{fig:Fig1_Whitewater}).
Using Table~\ref{tbl:Critical_z}, we find that the corresponding $z$-values 
for the intervals are (-$\infty$, -0.84), (-0.84, -0.26), (-0.26, +0.26), (0.26, 0.84), and (0.84, $\infty$).  Counting 
the values in Table~\ref{tbl:salinity} we find the observed number of samples for each of the five bins to be 10, 11, 
10, 5, and 12.  These are the observed $O_j$'s.  The expected values $E_j$ are all the same, i.e.,
\begin{equation}
E_j = \frac{n}{k} = \frac{48}{5} = 9.6.
\end{equation}
Using (\ref{eq:chi2_test}) we find the observed value $\chi^2 = 3.04$.
\index{Degrees of freedom}

	The $\chi^2$ distribution depends on $\nu$, the degrees of freedom, which normally would be 
$\nu = k - 1 = 4$ in 
our case (we lose one since the bin counts must sum to $n$).  However, we also used our observations to
compute $\bar{x}$, then $s$, in order to \emph{determine the bin boundaries}.  These estimations further reduce $\nu$ by two, leaving 
just two degrees of freedom.  From the relevant Table~\ref{tbl:Critical_chi2} we find critical $\chi^2$ for $\nu = 2$ and $\alpha = 0.05$ to be 
5.99.  Since this is much larger than our computed value we conclude that we cannot, at the 0.05 level of confidence, reject the 
null hypothesis that the salinities were drawn from a normal distribution. 
We stress that while we used a normal distribution for comparison in this example, the $E_j$ could have 
been derived from any other distribution we may wish to compare our data to. 
\end{example}

\begin{table}[h]
\centering
\small
\begin{tabular}{|c|c|c||c|c|c|} \hline
\bf{Sample} &  \bf{Original}  & \bf{Standardized} & \bf{Sample} & \bf{Original}  & \bf{Standardized} \\
\bf{Number} & \bf{Sample} & \bf{Sample}  & \bf{Number} & \bf{Sample} & \bf{Sample} \\ \hline
1 & 46.00 & -0.38 \, & 25 & 35.00 & -1.57 \, \\ \hline
2 & 37.00 & -1.35 \, & 26 & 49.00 & -0.06 \, \\ \hline
3 & 62.00 & 1.34 & 27 & 48.00 & -0.17 \,  \\ \hline
4 & 59.00 & 1.02 & 28 & 39.00 & -1.14  \, \\ \hline
5 & 40.00 & -1.03 \, & 29 & 36.00 & -1.46 \, \\ \hline
6 & 53.00 & 0.37 & 30 & 47.00 & -0.27 \, \\ \hline
7 & 58.00 & 0.91 & 31 & 59.00 & 1.02 \\ \hline
8 & 49.00 & -0.06 \, & 32 & 42.00 & -0.81 \,  \\ \hline
9 & 60.00 & 1.13 & 33 & 61.00 & 1.24 \\ \hline
10 & 56.00 & 0.70 & 34 & 67.00 & 1.88 \\ \hline
11 & 58.00 & 0.91 & 35 & 53.00 & 0.37 \\ \hline
12 & 46.00 & -0.38 \, & 36 & 48.00 & -0.17 \, \\ \hline
13 & 47.00 & -0.27 \, & 37 & 50.00 & 0.05 \\ \hline
14 & 52.00 & 0.27 & 38 & 43.00 & -0.71 \, \\ \hline
15 & 51.00 & 0.16 & 39 & 44.00 & -0.60 \, \\ \hline
16 & 60.00 & 1.13 & 40 & 49.00 & -0.06 \, \\ \hline
17 & 46.00 & -0.38 \, & 41 & 46.00 & -0.38 \, \\ \hline
18 & 36.00 & -1.46 \, & 42 & 63.00 & 1.45 \\ \hline
19 & 34.00 & -1.68 \, & 43 & 53.00 & 0.37 \\ \hline
20 & 51.00 & 0.16 & 44 & 40.00 & -1.03 \, \\ \hline
21 & 60.00 & 1.13 & 45 & 50.00 & 0.05 \\ \hline
22 & 47.00 & -0.27 \, & 46 & 78.00 & 3.07 \\ \hline
23 & 40.00 & -1.03 \, & 47 & 48.00 & -0.17 \, \\ \hline
24 & 40.00 & -1.03 \, & 48 & 42.00 & -0.81 \, \\ \hline
\end{tabular}
\normalsize
\caption{Standardized scores of salinity measurements from Whitewater Bay.}
\label{tbl:salinity}
\end{table}

\subsection{Test for a correlation coefficient}
\index{Test!correlation coefficient}
\index{Correlation!parametric test}

We recall that the conventional correlation coefficient was defined as
\begin{equation}
r = \frac{s_{xy}}{s_x s_y} = 
\frac{\sum^n_{i=1} ( x_i - \bar{x}) (y_i - \bar{y})} {\sqrt { \sum^n_{i=1} (x_i - \bar{x})^2  \sum^n _{i=1} (y_i - \bar{y})^2}}.
\label{eq:corrcoef}
\end{equation}
Often, we need to test if an observed $r$ is \emph{significant}.  In such tests, $r$ is our sample-derived estimate of $\rho$, the 
actual correlation of the population pairs.  The most useful null hypothesis is simply H$_0: \rho = 0$.  It can be 
shown that the sampling distribution of correlations for a population that has zero correlation ($\rho = 0$) is a normal distribution 
with mean $\mu = 0$ and $\sigma = \sqrt{(1 - r^2)/(n-2)}$.  Hence, a $t$ statistic can be calculated as
\begin{equation}
t = \frac{r - \mu}{\sigma} = \frac{r}{\sqrt{(1-r^2)/(n-2)}} = \frac{r\sqrt{n-2}}{\sqrt{1-r^2}}.
\label{eq:t_stat_corr}
\end{equation}
The degrees of freedom, $\nu$, is $n - 2$ since we needed to compute both $\bar{x}$ and $\bar{y}$ first to determine $r$.
\begin{example}
Suppose we roll a pair of dice, one red and one green 
(Table~\ref{tbl:dice1}).  Using (\ref{eq:corrcoef}) we obtain $r = 0.66$ which seems quite high, especially since there is no 
reason to believe a correlation should exist at all.  Let us run a test to determine if the correlation is 
significant.  Choosing $\alpha = 0.05$, Table~\ref{tbl:Critical_t} shows that critical $t_{\alpha/2,3} = 3.182$.  Applying
(\ref{eq:t_stat_corr}) gives the observed 
$t = 1.52$, hence the correlation of 0.66 is most likely caused by random fluctuations found in small 
samples and (gratefully) we cannot reject $H_0$.
\begin{table}[h]
\centering
\begin{tabular}{|c|c|} \hline
\bf{Red} (x) & \bf{Green} (y) \\ \hline
4 & 5 \\ \hline
2 & 2 \\ \hline
4 & 6 \\ \hline
2 & 1 \\ \hline
6 & 4 \\ \hline
\end{tabular}
\caption{Result of tossing a pair of dice five times.}
\label{tbl:dice1}
\end{table}

\index{Type I error}
 	How high would $r$ have to be for us to find it significant and commit a type I error by 
rejecting the (true) null hypothesis?  We can solve for $r$ in
\begin{equation}
t_{\alpha/2,n-2} = \frac{r \sqrt{n-2}}{\sqrt{1 - r^2}} \Rightarrow  3.182^2 = \frac{3r^2}{1-r^2} \Rightarrow r =  \pm 0.88.
\end{equation}
So, if $r$ happens to equal or exceed $\pm 0.88$ we would find ourselves awkwardly concluding that red and green dice give 
correlated pairs of values, but we would only make that mistake once in twenty tries on average.
\end{example}
Going a bit further, we can use (\ref{eq:t_stat_corr}) to determine what the critical correlation, $r_c$, has to be
for a given sample size (i.e., number of rolls) before we would suspect the dice have been tampered with.  We
simply solve the equation for $r$ given a level of confidence and sample size.  The results are displayed in
Figure ~\ref{fig:Fig1_dice}.  High correlations are required when the number of rolls are relatively small.
\PSfig[h]{Fig1_dice}{The critical values of correlation, i.e., how large an observed correlation would have
to be before we must reject the null hypothesis.  The circle and dashed lines reflect the situation in Example~{\thechapter.\theexample}.}
\index{Parametric tests|)}

\subsection{Analysis of variance}
\index{Analysis of variance (ANOVA)}
\index{Variance!analysis of (ANOVA)}
	We found earlier that we could use the Student's $t$-test to decide if two 
samples had different means.  However, very often we are faced with the task of deciding 
whether observed differences among \emph{more} than two sample means can be attributed to chance, or 
whether there are real differences among the means of the populations sampled.
Statistical scientists developed an exploratory procedure to analyze such observations known as \emph{ANOVA},
which stands for ``ANalysis Of VAriance''.  ANOVA usually comes in two flavors: One-way and two-way.

\subsubsection{One-way ANOVA}
\index{One-way ANOVA|(}
\index{ANOVA!one-way|(}

	In one-way ANOVA, the basic idea is to express the total variation of the data as a 
sum of two terms, each of which can be attributed to a specific \emph{source}.  The two sources of variation are:
\begin{enumerate}
\item Actual differences (because of different means) among the populations that the samples represent.
\item Chance or experimental error.
\end{enumerate}
The measure of the total variation that we shall use here is the \emph{total sum of squares}, \emph{SST}, which is simply
\begin{equation}
SST = \sum ^k _{j=1} \sum ^n _{i=1} (x_{ij} - \bar{\bar{x}} ) ^2,
\label{eq:SST}
\end{equation}	 	
where $x_{ij}$ is the $i$'th observation from the $j$'th sample.  Thus, we have $k$ samples with $n$ observations 
each and $\bar{\bar{x}}$  is the grand mean of all observations.  Note that if we divided $SST$ by 
$kn-1$ we would obtain the variance 
of the combined data set.  Let $\bar{x}_j$  be the mean of the $j$'th sample.  By adding and subtracting
$\bar{x}_j$ we can then rewrite (\ref{eq:SST}) as
\begin{equation}
\begin{array}{rcl}
SST & = & \displaystyle \sum^k_j \sum^n_i [ ( \bar{x}_j - \bar{\bar{x}})   + (x_{ij} - \bar{x}_j)]^2 =  \sum^k_j \sum^n_i [(\bar{x}_{j} - \bar{\bar{x}})^2  + 2 (\bar{x}_j - \bar{\bar{x}}) (x_{ij} - \bar{x}_j) +  (x_{ij} - \bar{x}_j)^2]   \\
 & = & \displaystyle n \sum^k_j   ( \bar{x}_j - \bar{\bar{x}})^2  +  \sum^k_j \sum^n_i  (x_{ij} - \bar{x}_j)^2  +   2 \sum^k_j \sum^n_i (\bar{x}_j x_{ij} - \bar{x}^2_j - \bar{\bar{x}}x_{ij} + \bar{\bar{x}}  \bar{x}_j)   \\
 & = &   \displaystyle n \sum^k_j  ( \bar{x}_j - \bar{\bar{x}})^2 + \sum^k_j \sum^n_i  (x_{ij} - \bar{x}_j)^2 +     2  \sum^k_j  \bar{x}_j n \bar{x}_j - 2n        \sum^k_j   \bar{x}^2_j - 2\bar{\bar{x}}    \sum^k_j \sum^n_i  x_{ij} + 2 \bar{\bar{x}} n  \sum^k_j \bar{x}_j\\
& = & n \displaystyle  \sum^k_j  ( \bar{x}_j - \bar{\bar{x}})^2    + \sum^k_j \sum^n_i (x_{ij} - \bar{x}_j)^2 + 2n      \sum^k_j  \bar{x}^2_j - 2n \sum^k_j \bar{x}^2_j - 2 \bar{\bar{x}}kn \bar{\bar{x}} + 2 \bar{\bar{x}} nk \bar{\bar{x}} \\
& = &  n \displaystyle  \sum^k_j  ( \bar{x}_j - \bar{\bar{x}})^2    + \sum^k_j \sum^n_i
(x_{ij} - \bar{x}_j)^2.
\end{array}
\end{equation}	 
\index{Treatment sum of squares}
\index{Sum of squares!treatment}
Looking at the last two terms remaining we see that the first is a measure of the variation \emph{among} the sample 
means.  Similarly, the second term is a measure of the variation \emph{within} the individual samples.  It 
is customary to refer to the first term as the \emph{treatment sum of squares}, $SS(Tr)$
and to the second 
term as the \emph{error sum of squares}, $SSE$ (Note: these names originated from agricultural experiments, i.e., ``treatments'' of 
different fertilizers.)
\index{Error sum of squares}
\index{Sum of squares!error}
Our overall goal with the ANOVA test is to investigate 
if the sample means are all statistically equal, hence the null hypothesis is
\[
H_0 = \mu_1 = \mu_2 = \cdots    = \mu_k \mbox{ versus } H_1: \mbox{they are not all equal}.
\]
We usually state that $\mu_i = \mu + \epsilon_i$, so  $\sum \epsilon_i \equiv 0$.  Thus, the null hypothesis can also be 
written,
$$
H_0: \epsilon_1 = \epsilon_2 = \cdots    = \epsilon_k = 0.
$$
Let us introduce two assumptions that are critical to this test:
\begin{enumerate}
\item The populations we are sampling are approximately normal.
\item They all have the same variance, $\sigma^2$.
\end{enumerate}
If true, then we can look upon the $k$ samples as if they came from the same (normal) population and 
therefore consider the variance of their means $s^2_{\bar{\bar{x}}}$  as an estimate of $\sigma_{\bar{\bar{x}}}$.
Since $\sigma_{\bar{\bar{x}}} = \sigma / \sqrt{n}$  for infinite 
populations (recall the central limits theorem), we find that $ns^2_{\bar{\bar{x}}} = \sigma^2$.
This is, upon examination, the same as $SS(Tr)/(k-1)$.

	Since $\sigma^2$ is not known it must be estimated from the data.  Furthermore, since the populations are assumed to be normal then any 
one of the sample variances could be used.  To improve on the estimate we chose to take the mean of the sample variances,
\begin{equation}
\sigma^2 \approx s^2 = \frac{s^2_1 + s^2_2 + \cdots s^2_k}{k}.
\end{equation}	 
This turns out to equal the (normalized) second variance term, i.e., $SSE/(k(n-1))$.  We now have two competing estimates of $s^2$ for the population 
variance $\sigma^2$.  If the first estimate (based on variation among the sample means) is much larger 
than the second estimate (based on variations within the samples, i.e., variation due to chance) we 
should reject $H_0$.   This is simply the \emph{F}-test comparing two variances.  Consequently, we shall use the 
statistic
\begin{equation}
F = \frac{SS(Tr)/_{k-1}}{SSE/_{k(n-1)}}.
\end{equation}
where the numerator is our estimate of $\sigma^2$ from variation among the $\bar{x}_j$ while the
denominator is our estimate of $\sigma^2$ from variation within the samples.  The expected value of $F$ is therefore unity.
In practice, we construct an ANOVA table (Table~\ref{tbl:one_way_ANOVA}).
\begin{table}[H]
\center
\begin{tabular}{|l|c|c|c|c|} \hline
\bf{Source of} & \bf{Degrees of} & \bf{Sum of} & \bf{Mean Square} & $F$ \\ 
\bf{Variation} & \bf{Freedom} & \bf{Squares} & &  \\ \hline
\rule{0pt}{4ex}Treatments & $k - 1$ & $SS(Tr)$ & $MS(Tr) = \displaystyle \frac{SS(Tr)}{k-1}$ & $ \displaystyle \frac{MS(Tr)}{MSE}$\\[9pt]  \hline
\rule{0pt}{4ex}Error & $k(n - 1)$ & $SSE$ &  $MSE = \displaystyle \frac{SSE}{k(n-1)}$ &  \\[9pt] \hline
Total & $kn-1$ & $SST$ & & \\ \hline
\end{tabular}
\caption{Table used for a one-way analysis of variance (ANOVA).}
\label{tbl:one_way_ANOVA}
\end{table}
We will apply this procedure to a small data set of four samples with six observations each (Table~\ref{tbl:sandstone_data}).
\begin{example}
The porosity $\bar{\varphi}$ of a sandstone was measured at four different 
locales and the observed values are listed in Table~\ref{tbl:sandstone_data}.
\begin{table}[h]
\center
\begin{tabular}{|c|c|c|c|c|c|c|} \hline
\bf{Loc A} &  \bf{Loc B} & \bf{Loc C} & \bf{Loc D} \\ \hline
23.1 & 21.7 & 21.9 & 19.8 \\ \hline
22.8  & 23.0 & 21.3 & 20.4 \\ \hline
23.2  & 22.4 & 21.6 & 19.3 \\ \hline
23.4 & 21.1 & 20.2 & 18.5 \\ \hline
23.6 & 21.9 & 21.6 & 19.1 \\ \hline
21.7 & 23.4 & 23.8 & 21.9 \\ \hline
\end{tabular}
\caption{Four samples of sandstone porosity take from different locations.}
\label{tbl:sandstone_data}
\end{table}
At the 0.05 level of significance, is the porosity the same at all four locations?  The null 
hypothesis becomes
\begin{equation}
H_0: \bar{\varphi} _1 = \bar{\varphi}_2  = \bar{\varphi}_3 = \bar{\varphi}_4.
\end{equation}	 
We first find
\begin{equation}
\bar{x}_1 = 22.97, \bar{x}_2 = 22.25, \bar{x}_3 = 21.73, \bar{x}_4 = 19.83, \bar{\bar{x}} = 21.70.
\end{equation}
The resulting ANOVA table is given as Table \ref{tbl:sandstone_ANOVA}.
\begin{table}[h]
\center
\begin{tabular}{|l|c|c|c|c|} \hline
\bf{Source of Variation} & $\nu$ & \bf{SS} & \bf{MS} & $F$ \\ \hline	
Treatments & 3 & 32.35 & 10.78 & 10.78  \\ \hline
Error & 20 & 20.02 & 1.000 &   \\ \hline
Total & 23 & 52.37 & &  \\ \hline
\end{tabular}
\caption{One-way ANOVA table based on the statistics from the porosity data given in Table~\ref{tbl:sandstone_data}.}
\label{tbl:sandstone_ANOVA}
\end{table}
\index{\emph{F}-test}
\index{Test!\emph{F}}
From Table~\ref{tbl:Critical_F95} we find the critical $F$ value for 3 and 20 degrees of freedom at the 5\% level to 
be $F_{0.05,3,20} = 3.10$ (This is a \emph{one-sided} test since we only consider if $F > 1$ or not).  Since our 
observed $F$ is much larger than the critical value we must reject our null hypothesis:  The porosity at the four 
locations are not all the same.
\end{example}
To simplify the calculations, we rewrite the expressions for the various sums of squares as
\begin{equation}
SST = \left \{ \sum^k_{j=1} \sum^n_{i=1} x^2_{ij} \right \} - \frac{1}{kn} S^2,
\end{equation}
\begin{equation}
SS(Tr) = \left \{ \frac{1}{n}  \sum^k_{j=1} S^2_j \right \} - \frac{1}{kn} S^2,
\end{equation}
and
\begin{equation}
SSE = SST - SS(Tr).
\end{equation}
Here, $S_j$ is the sum of the values in the $j$'th sample, and $S$  is the total sum of all observations.

	So far we assumed that each sample has the same number of observations.  Instead, if there are $n_j$ 
observations in the $j$'th sample we get
\begin{equation}
SST = \left \{ \sum^k_{j=1} \sum^{n_j}_{i=1} x^2_{ij} \right \} - \frac{1}{N} S^2,
\end{equation}
\begin{equation}
SS(Tr) = \left \{ \sum^k_{j=1} \frac{S^2_j}{n_j} \right \} - \frac{1}{N} S^2,
\end{equation}
where $\displaystyle S_j = \sum^{n_j}_{i=1} x_{ij}$ and $\displaystyle N = \sum^k_{j=1} n_j$.
Also, $\nu_{\mbox{treat}} \ = k-1$ but $\nu_{\mbox{error}} = N-k$.
\index{One-way ANOVA|)}
\index{ANOVA!one-way|)}

\subsubsection{Two-way ANOVA}
\index{Two-way ANOVA|(}
\index{ANOVA!two-way|(}

	The previous ANOVA test was concerned only with the task of checking if the means of the 
populations were the same.   The two-way ANOVA procedure extends the test to whether there also are variations 
\emph{across} the populations.  In other words, the population mean for $j$'th treatment and $i$'th (agricultural) \emph{block} is expected to be
\begin{equation}
\mu _{ij} = \mu + \epsilon_j + \gamma_i.
\end{equation}
Thus, $\epsilon_j$ are the treatment effects that vary from  sample to sample, and $\gamma_i$ are called the \emph{block 
effects}  and vary within each sample.  Examples might be porosity at four locations where the $\epsilon_j$ 
represent differences among the locations and $\gamma_i$ may represent variations with depth across all 
samples.  Again, we test
$$
H_0: \epsilon_1 = \epsilon_2 = \ldots = \epsilon_k = 0,
$$	 
but now we also consider the second null hypothesis
$$
H_0: \gamma_1 = \gamma_2 = \ldots = \gamma_n = 0.
$$	 
To do this we must obtain a quantity, similar to the treatment sum of squares, which measures 
the variation among the block means.  If we let $S_i$ be the total of all values in the $i$'th block (e.g., 
depth) and substitute it for $S_j$, sum over $i$ instead of $j$, and swap $n$ and $k$, we find the \emph{block sum of 
squares} via
\index{Sum of squares!block}
\index{Block sum of squares}
\begin{equation}
SSB = \left \{ \frac{1}{k} \sum^n _{i=1} S^2_i \right \} - \frac{1}{kn} S^2.
\end{equation}
Hence, we compute $SST$ and $SS(Tr)$ as before, $SSB$ as just given, and $SSE$ now becomes
\begin{equation}
	SSE = SST - [SS(Tr) + SSB].
\end{equation}
Table~\ref{tbl:two_way_ANOVA} shows the extended ANOVA table for two-way analysis.
\begin{table}[H]
\center
\begin{tabular}{|l|c|c|c|c|} \hline
\bf{Source of}  & \bf{Degrees of}  & \bf{Sum of Squares} & \bf{Mean Square} & $F$ \\ 
\bf{Variation} & \bf{Freedom} & & & \\ \hline
\rule{0pt}{4ex}Treatments & $k - 1$ & $SS(Tr)$ & $MS(Tr) = \displaystyle \frac{SS(Tr)}{k-1}$ & $\displaystyle \frac{MS(Tr)}{MSE}$ \\[9pt] \hline
\rule{0pt}{4ex}Blocks  & $n -1$ & $SSB$ & $MSB = \displaystyle \frac{SSB}{n-1}$ & $\displaystyle \frac{MSB}{MSE}$ \\[9pt] \hline
\rule{0pt}{4ex}Error & $(k - 1)(n - 1)$ & $SSE$  & $MSE = \displaystyle \frac{SSE}{(k-1)(n-1)}$  & \\[9pt] \hline
Total & $kn - 1$ & $SST$ & & \\ \hline
\end{tabular}
\caption{Two-way table for the analysis of variance (ANOVA).}
\label{tbl:two_way_ANOVA}
\end{table}
\begin{example2}
We have measured the nickel concentration in a shale at four locations where we have obtained three 
observations at different depths in the unit.  Our data are given in parts per million (ppm), with depth
increasing downward in Table~\ref{tbl:two_way_nickel}.  We want to do a two-way ANOVA 
to see if the variations among the locations and among observations at the same depth (our ``blocks'') are similar 
at the 95\% level of confidence.  We state
$$
H_0: \epsilon_1 = \epsilon_2 = \epsilon_3 = \epsilon_4 = 0,$$
$$
\gamma_1 = \gamma_2 = \gamma_3 = 0.
$$
\begin{table}[H]
\center
\begin{tabular}{|c|r|r|r|r|r|} \hline
& 	  \bf{Loc 1} & \bf{Loc 2} & \bf{Loc 3} & \bf{Loc 4} & $S_i$ \\ \hline
\bf{Depth 1} & 71 & 44 & 50 & 67 & 232 \\ \hline
\bf{Depth 2} & 92 & 51 & 64 & 81 & 288 \\ \hline
\bf{Depth 3} & 89 & 85 & 72 & 86 & 332 \\ \hline
$S_j$ & 252 &  180 & 186 & 234 & 852  \\ \hline
\end{tabular}
\caption{Four samples of nickel concentrations from different locations, sorted by depth.}
\label{tbl:two_way_nickel}
\end{table}
\noindent
Following the procedure, we compute the total sum of squares to be
\begin{equation}
\sum \sum x^2 = 63,414.
\end{equation}
Combined with the sums in Table~\ref{tbl:two_way_nickel} we find
\begin{equation}
SST = 63,414 - \frac{1}{12}(852)^2 = 63,414 - 60,492 = 2922,
\end{equation}
\begin{equation}
SS(Tr) = \frac{1}{3}[252^2 + 180^2 + 186^2 + 234^2] -  60,494 = 1260,
\end{equation}
\begin{equation}
SSB = \frac{1}{4} [232^2 + 288^2  + 332^2 ] - 60,492 = 1256,
\end{equation}
\begin{equation}
SSE = 2922 - [1260 + 1256] = 406.
\end{equation}
We construct a two-way ANOVA table presented as Table~\ref{tbl:two_way_F}.
\begin{table}[H]
\center
\begin{tabular}{|c|c|c|c|c|}
\hline
\bf{Source of Variation} & $\nu$ & \bf{SS} & \bf{MS} & $F$ \\ \hline	
{Treatments} & 3 & 1260 & 420 & 6.21\\ \hline
{Blocks} & 2 & 1256 & 628 & 9.28 \\ \hline
{Error} & 6 & 406 & 67.67 & \\ \hline
{Total} & 11 & 2922 &  & \\ \hline
\end{tabular}
\caption{Two-way ANOVA table resulting from the statistics of the nickel concentrations given in Table~\ref{tbl:two_way_nickel}.}
\label{tbl:two_way_F}
\end{table}
\noindent
With the help of Table~\ref{tbl:Critical_F95} we reach the following decisions:
\begin{description}

\item [Treatments (locations):] Critical value $F_{0.05,3,6} = 4.76$, so we reject the hypothesis that $\epsilon_i = 0$.
\item [Blocks (depth):]  Critical value $F_{0.05,2,6} = 5.14$, so again we reject the hypothesis that all $\gamma_j = 0$.
\end{description}
\index{\emph{F}-test}
\index{Test!\emph{F}}
In other words, we conclude that the average nickel concentration is not the same at the four 
locations, and that it is not the same at all depths.
\end{example2}
\index{Two-way ANOVA|)}
\index{ANOVA!two-way|)}
\section{Nonparametric Tests}
\index{Nonparametric tests|(}

	The last section concluded the examination of standard parametric tests (i.e., the $t$-, $F$-, and $\chi^2$-tests.)  
We justified using these tests by \emph{either} having large samples and invoking the central limits 
theorem \emph{or} simply assuming that the distribution we have sampled is approximately normal.  
Sometimes, however, none of these conditions are met.  The two typical situations that can arise are:
\begin{enumerate}
\item You have a small sample $(n < 30)$ and you \emph{cannot} assume that the population it came from is normal.
\index{Small sample}
\index{Sample!small}
\item You have \emph{ordinal} data (which can be ranked, but not operated on numerically).
\index{Ordinal data}
\index{Data!ordinal}
\end{enumerate}
In those cases we must consider \emph{nonparametric} methods, which make no assumptions about the shape of 
the data distribution.  In particular, nonparametric tests \emph{do not} involve the calculation
of distribution parameters, such as the mean and standard deviation.

\subsection{Sign test for the one-sample mean or median}
\index{Sign test}
\index{Test!sign}

The nonparametric sign test is a robust alternative to the standard one-sample $t$-test.
It can be used when the distribution we have sampled has a continuous \emph{symmetrical} population.
This implies that the probability of getting a data value \emph{less} than the mean is the same as
getting one \emph{larger} than the mean: both probabilities equal 0.5.  However, if we cannot assume
that the population is symmetrical, the test should instead apply to the median value rather than the mean.
Since we will be testing whether or not our sample mean (or median) is statistically indistinguishable
from a specified hypothetical mean (or median), the procedure relies on properties of the binomial
distribution encountered in Chapter~\ref{ch:basics} and is reminiscent of the simple coin-toss analogy.
Values may be less than or larger than the hypothetical mean (median), and the probability of finding
$x$ values out of $n$ values to be less than the mean (median) follows directly from (\ref{eq:binomial_dist}), with $p = 0.5$.
To perform the sign test we need to evaluate the cumulative binomial distribution or consult pre-tabulated distributions.
\begin{example}
The following data constitute a random sample of 15 measurements of salinity
content (in ppt):
\begin{center}
	97.5, 95.2, 97.3, 96.0, 96.8, 100.3, 97.4, 95.3, 93.2, 99.1, 96.1, 97.6, 98.2, 98.5, 94.9
\end{center}
We will use the one-sample sign test to consider the null hypothesis, $H_0$: $\tilde{\mu} \geq 98.5$ against the
alternative hypothesis, $H_1$: $\tilde{\mu} < 98.5$ at the $\alpha = 0.01$ level of significance.  Because of the inequality
we have a \emph{one-sided} test.  We replace all values greater than 98.5 with a plus sign and all values less
than 98.5 with a minus sign.  Values that equal 98.5 exactly are discarded; in our case we lose one value,
thus $n = 14$, resulting in the following series:
\begin{center}
	- - - - - + - - - + - - - -
\end{center}
We find $x = 2$ values (represented by the two plus-signs) larger than the hypothetical median.  The probability of finding $x \le 2$ is given by
the binomial distribution (\ref{eq:binomial_dist}) by adding up the probabilities for $x = 0$, $x = 1$, and
$x = 2$.  We find
\begin{equation}
P = P_{14,0.5}(0) + P_{14,0.5}(1) + P_{14,0.5}(2) = C_{14,0.5}(2) = \sum_{x=0}^{2} P_{14,0.5}(x),
\end{equation}
which evaluates to
\begin{equation}
P = \binom{14}{0} \frac{1}{2}^{14} + \binom{14}{1} \frac{1}{2}^{14} + \binom{14}{2} \frac{1}{2}^{14} = 0.00006 + 0.0009 + 0.0056 \approx 0.0065.
\end{equation}
Since 0.0065 is less than 0.01, we must reject $H_0$. We conclude that the null hypothesis must be rejected as the
data suggest that the median salinity from the sampled
region is less than 98.5 ppt.  Note that in this test we did not compute a critical value for $x$ but compared
the probability for the observed case with a specified probability $\alpha$.
\end{example}

When both $np$ and $n(1 - p)$ are greater than 5 (here they are both equal to 7) we are allowed to use the normal approximation to
the binomial distribution.  Per (\ref{eq:binomial_approx_norm}), the sign test may then be based on the statistic

\begin{equation}
z = \frac{x - np}{\sqrt{np(1-p)}},
\end{equation}
which in our situation (with $p$ = 0.5) simplifies to
\begin{equation}
z = \frac{2x - n}{\sqrt{n}}.
\end{equation}
We may now simply compare the observed $z$ statistic with the chosen $z_{\alpha/2}$
critical value as in the standard parametric case (or $z_{\alpha}$ for a one-sided test like the present case).  Here,
$z_{\alpha} = -2.326$ while observed $z = -2.676$ and we again must reject the null hypothesis.

\subsection{Mann-Whitney test ($U$-test)}
\index{$U$-test|(}
\index{Mann-Whitney test|(}
\index{Test!Mann-Whitney|(}
\index{Test!U|(}

	This test is a nonparametric alternative to the two-sample Student's $t$-test.  It also goes by the 
names \emph{Wilcoxon}\index{Wilcoxon test}\index{Test!Wilcoxon} test and the $U$-test.  The Mann-Whitney test is performed by combining the two 
data sets we want to compare, sorting the combined set into ascending order, and assigning each point a \emph{rank}: 
the smallest value is given rank $= 1$, while the largest observation is ranked $n_1 + n_2$.  Should some of the 
observations be identical one assigns the mean rank to all tied values.  E.g., if the 7th and 8th 
sorted values were identical, we would assign to each the mean rank of 7.5.  The idea here is that if the samples 
consist of random drawings from the same population (i.e., when $H_0$ is true) then we would expect the ranks for both 
samples to be scattered more-or-less uniformly throughout the sequence.  This would be true regardless of the
distribution that characterizes the population.

	After sorting the data we add up the ranks for each data set separately into \emph{rank sums}, which we 
denote $S_1$ and $S_2$.  The sum of $S_1 + S_2$ must obviously equal the sum of the first 
$(n_1 + n_2)$ integers, which is
\index{Rank sum}
\begin{equation}
\frac{1}{2} (n_1 + n_2) (n_1 + n_2 + 1 ).
\end{equation}
Many early rank sum tests were based on $S_1$ or $S_2$ but now it is customary to use the statistic $U$,
defined as $U = \min {(U_1,U_2)}$, i.e., the smallest of $U_1$ and $U_2$, with
\begin{equation}
U_1 = S_1 - \frac{1}{2} n_1 (n_1 + 1)
\label{eq:U1}
\end{equation}
and
\begin{equation}
U_2 = S_2 - \frac{1}{2} n_2 (n_2 + 1).
\label{eq:U2}
\end{equation}	 	
This statistic can range from 0 to $n_1\cdot n_2$ and its 
sampling distribution is symmetrical about $n_1 \cdot n_2/2$.  The test, then, consists of these steps:
\begin{enumerate}
	\item Compute the $U$-statistic using the smallest value of (\ref{eq:U1}) and (\ref{eq:U2}).
	\item Evaluate critical $U_{\alpha,n_1,n_2}$, given the sample sizes and the desired level of significance $\alpha$.
	\item Compare the calculated $U$ statistic to the critical $U_{\alpha,n_1,n_2}$ and
		reject $H_0$ if $U$ is \emph{less than} the critical value.
\end{enumerate}
Note that for the $U$-test, $H_0$ is rejected when our $U$-value is \emph{less} than and not larger than
the critical value, as is common for most other tests we have discussed.
\begin{example}
We want to compare the grain size of sand obtained from two different locations on 
the moon on the basis of measurements of grain diameters (in mm), as follows:
\begin{table}[H]
\centering
\begin{tabular}{|l||c|c|c|c|c|c|c|c|c|c|l|} \hline
\bf{Location 1} & 0.37 & 0.70 & 0.75 & 0.30 & 0.45 & 0.16 & 0.62 & 0.73 & 0.33 &      & $n_1 = 9$ \\ \hline
\bf{Location 2} & 0.86 & 0.55 & 0.80 & 0.42 & 0.97 & 0.84 & 0.24 & 0.51 & 0.92 & 0.69 & $n_2 = 10$ \\ \hline
\end{tabular}
\end{table}
We do not know what type of distribution that grain sizes of sand on the moon might follow, so we choose the 
$U$-test to see if the mean grain size differ between the two samples.
Computing the sample means gives $\bar{x}_1 = 0.49$ and $\bar{x}_2 = 0.68$.  If we wanted to use the $t$-test we would have to 
assume that the underlying distributions are normal, since the samples are small.  The $U$-test requires no such assumptions.  
We start by arranging the data jointly into ascending order and keep track of which location each 
point originated from (Table~\ref{tbl:moonsand}).

\begin{table}[h]
\centering
\begin{tabular}{|c|c|c||c|c|c|} \hline
\bf{Data} & \bf{Location} & \bf{Rank} & \bf{Data} & \bf{Location} & \bf{Rank}\\ \hline
0.16 & 1 & 1  & 0.69 & 2 & 11 \\ \hline
0.24 & 2 & 2  & 0.70 & 1 & 12 \\ \hline
0.30 & 1 & 3  & 0.73 & 1 & 13 \\ \hline
0.33 & 1 & 4  & 0.75 & 1 & 14 \\ \hline
0.37 & 1 & 5  & 0.80 & 2 & 15 \\ \hline
0.42 & 2 & 6  & 0.84 & 2 & 16 \\ \hline
0.45 & 1 & 7  & 0.86 & 2 & 17 \\ \hline
0.51 & 2 & 8  & 0.92 & 2 & 18 \\ \hline
0.55 & 2 & 9  & 0.97 & 2 & 19 \\ \hline
0.62 & 1 & 10 &      &   &    \\ \hline
\end{tabular}
\caption{Sorted listing of the grain sizes of moon sand with auxiliary column containing the location of each specimen.}
\label{tbl:moonsand}
\end{table}
We first evaluate the rank sum for location 1, giving $S_1 = 69$, from which it follows that
\begin{equation}
S_2 = \frac{19 \cdot 20}{2} - S_1 = 190 - 69 = 121.
\end{equation}	 
We now form the null hypothesis H$_0: \mu_1 = \mu_2$, with H$_1: \mu_1 \neq \mu_2$, and state the level of 
significance $\alpha = 0.05$.  Table~\ref{tbl:Critical_U3} has critical values for $U$ and we find $U_{\alpha, 9, 10} = 20$.
Thus, we will reject the null hypothesis if $U$ is $\leq 20$.  From $S_1$ and $S_2$ we find 
\begin{equation}
U_1 = 69 - \frac{9\cdot 10}{2} = 24,
\end{equation}	  
\begin{equation}
U_2 = 121 - \frac{10 \cdot 11}{2} = 66,
\end{equation}
and hence $U = \min(24, 66) = 24$.  This is larger than the critical value of 20, suggesting we 
\emph{cannot reject} the null hypothesis.  In other words, the observed difference in mean grain size at the two locations is 
not statistically significant at the 95\% level of confidence.
\end{example}
	For large samples $(n_1, n_2 > 30)$ the procedure again simplifies and it can be shown that the mean and 
standard deviation of the $U$ sampling distribution approach

\begin{equation}
\mu_U = \frac{n_1 n_2}{2}, \ \ \  \sigma _U = \sqrt{\frac{n_1 n_2(n_1 + n_2 + 1)}{12}     },
\label{eq:U_approx}
\end{equation}
provided there are \emph{no tied ranks}.  We could then evaluate standard $z$-scores as $z = (U - \mu_U)/\sigma_U$
and use the familiar critical values $\pm z_{\alpha/2}$ from Table~\ref{tbl:Critical_t}.
\index{$U$-test|)}
\index{Mann-Whitney test|)}
\index{Test!Mann-Whitney|)}
\index{Test!U|)}

\subsection{Comparing distributions: The Kolmogorov-Smirnov test}
\index{Kolmogorov-Smirnov test|(}
\index{Test!Kolmogorov-Smirnov|(}

\PSfig[h]{Fig1_KS}{Solid line is a cumulative normal distribution for $\mu = 49.59$ ppm and $\sigma = 9.27$ ppm.
The stair-case curve is the observed cumulative distribution which has its maximum difference, $D$, from the theoretical
curve at the 53 ppm point.}

	Another very useful nonparametric method is the Kolmogorov-Smirnov test  (or K-S for short).  It is a 
test for goodness of fit or \emph{shape} and is often used instead of the $\chi^2 $-test.
We may use it to test the null hypothesis that two distributions have the same probability density function
(i.e., the same shape).  A big advantage of the 
K-S test over the $\chi^2$-test is that one does not have to bin the data, which is an arbitrary procedure 
anyway (how do you select bin size and why?).  In the K-S test we convert the data distribution 
to a cumulative distribution $C(x)$.  Clearly, $C(x)$ then gives the fraction of data points to the ``left'' of $x$.  
While different data sets will in general have different distributions, all cumulative distributions 
agree at the smallest $x \mbox{ } (C(x) \equiv 0)$ and at the largest $x \mbox{ } (C(x) \equiv 1)$.
Thus, it is the behavior \emph{between} 
these points that sets distributions apart (e.g., Figure~\ref{fig:Fig1_KS}).  There is of course an infinite number of ways to 
measure the overall difference between two cumulative distributions: we could look at the absolute 
value of the area between the curves, the mean square difference, etc.  The K-S statistic chooses 
a simple measure: It determines the maximum absolute difference between the two cumulative 
curves.  Thus, when comparing two cumulative distributions $C_1(x)$ and $C_2(x)$ our K-S statistic becomes

\begin{equation}
D = \max_{-\infty < x < \infty} |C_1 (x) - C_2 (x) |.
\end{equation}
Note that $C_2$ may be another data-derived cumulative distribution (the test is then a \emph{two-sample} test)
or a theoretical cumulative probability function like the cumulative normal 
distribution (we call this case a \emph{one-sample} test).  The distribution of the K-S statistic itself can be calculated under the assumption 
that $C_1$ and $C_2$ are drawn from the same distribution (i.e., $H_0$), thus providing critical values for $D$.  For the one-sample
scenario there are two different cases to consider:
\begin{enumerate}
	\item $C_2$ is a \emph{known} cumulative distribution function, i.e., its parameters are prescribed
	by the null hypothesis.
	\item $C_2$ is an \emph{unknown} cumulative distribution function, i.e., its parameters must first
	be computed from $C_1$.
\end{enumerate}
The former case is the problem studied by Kolmogorov and Smirnov. The latter case, however, clearly reduces the
degrees of freedom and the standard K-S critical values are overestimated.
A different set of critical values (called the \emph{Lilliefors} critical values) has been developed for the normal
distribution.  Hence, when we need to compute the mean and standard deviation first we say we are performing a
\emph{Lilliefors test} rather than a Kolmogorov-Smirnov test.

We will use this test on the salinity measurements we looked at previously (Table~\ref{tbl:salinity}).
We sort the salinity measurements, convert them to a cumulative distribution (e.g., $C_1$), and plot the cumulative function
on the same graph as that of a normal cumulative distribution with the same mean and standard deviation (e.g., $C_2$).
Inspecting Figure~\ref{fig:Fig1_KS} we find the maximum absolute 
difference to occur at the 53 ppt observation.  The $D$ estimate is $0.701 - 0.641 
= 0.06$.  Based on a significance level of $\alpha = 0.05$ and $n = 48$, the critical Lilliefors value for a two-sided test
is found in Table~\ref{tbl:Critical_KS2} to be $\sim 0.128$, which is much 
larger than observed.  Hence we cannot reject the null hypothesis that the samples were collected 
from a normally distributed population.  In this example, both the K-S and $\chi^2$ tests reached the same conclusion.
\index{Kolmogorov-Smirnov test|)}
\index{Test!Kolmogorov-Smirnov|)}

\subsection{Spearman's rank correlation}
\index{Spearman's rank correlation|(}
\index{Test!Spearman's rank correlation|(}
\index{Correlation!Spearman's rank|(}

	Finally, we will look at nonparametric correlation called Spearman's \emph{rank correlation},
denoted by $r_s$.  The rank correlation is carried out by ranking the $x_i$'s and $y_i$'s 
\emph{separately}, then computing the standard correlation coefficient (i.e., \ref{eq:corrcoef}) using the ranks \emph{in lieu} of the data values. Let
$u_i$ be the rank of the $i$'th pair's $x$-value and $v_i$ be the rank of the $i$'th pair's $y$-value.  Then,
Spearman's rank correlation depends on the covariance and variances of the \emph{ranks}: 
\begin{equation}
r_s = \frac{s_{uv}}{s_u s_v} = \frac{n \sum_{i=1}^n u_i v_i - \left(\sum_{i=1}^n u_i \right)\left(\sum_{i=1}^n v_i \right)}{\sqrt{\left [n \sum_{i=1}^n u_i^2 - \left(\sum_{i=1}^n u_i \right)^2 \right] \left [n \sum_{i=1}^n v_i^2 - \left(\sum_{i=1}^n v_i \right)^2 \right]}}.
\label{eq:spearman_r_exact}
\end{equation}
If there are runs of tied ranks then we assign those points their \emph{average} rank. Fortunately, for
situations where there are no ties (\ref{eq:spearman_r_exact}) simplifies greatly to
\begin{equation}
r_s = 1 - \frac{6\sum d^2_i}{n(n^2 - 1)},
\label{eq:spearman_r}
\end{equation}
where $d_i = u_i - v_i$ is the difference in rank for each $(x_i, y_i)$ pair.
In the case where the null hypothesis $H_0$: $\rho = 0$ is true, the sampling distribution of $r_s$ is approximately normal and
has zero mean ($\mu = 0$) and standard deviation $\sigma = 1/\sqrt{n-1}$.  We could therefore base our statistics on 
\begin{equation}
z = \frac{r_s - \mu}{\sigma} = \frac{r_s - 0}{1/\sqrt{n-1}} = r _s\sqrt{n-1}
\label{eq:z_spearman}
\end{equation}
and compare this observed $z$-value to critical $z_{ \alpha / 2}$ values.
However, it turns out that a better approximation is the one given by (\ref{eq:t_stat_corr}), which we utilized when testing
the standard correlation coefficient.  Even so, for small data sets ($n < 20$) either approximation deviates
from the true distribution and hence special tables are required (see Table~\ref{tbl:Critical_Spearman}).

\index{Test!correlation coefficient}
A comparison between the standard correlation and the Spearman's rank correlation reveals some interesting differences:
\begin{itemize}
	\item Spearman's rank correlation is more tolerant of outliers since only their outlying \emph{ranks} and not actual
	data values enter into the calculation.
	\item While the standard correlation measures the degree of \emph{linear} correlation between $x$ and $y$,
		Spearman's rank correlation measures the degree of \emph{monotonicity} of the two rank series.  Any data set whose ranks
		$u$ and $v$ vary monotonically will yield $r_s = \pm 1$, even if they do not form a linear trend.
\end{itemize}
In most other situations the two correlations will be similar.
\begin{table}[H]
\centering
\begin{tabular}{|c|c|c|c|c|} \hline
\bf{Red} (x) & \bf{Rank} x & \bf{Green} (y) & \bf{Rank} y & \bf{d} \\ \hline
4 & 3.5 & 5 & 4 & 0.5 \\ \hline
2 & 1.5 & 2 & 2 & 0.5 \\ \hline
4 & 3.5 & 6 & 5 & 1.5 \\ \hline
2 & 1.5 & 1 & 1 & -0.5 \\ \hline
6 & 5    & 4 & 3 & -2 \\ \hline
\end{tabular}
\caption{Evaluating the differences in ranks among $x-y$ pairs obtained by rolling red and green dice.
Notice there are two groups of $x$-values with tied ranks but none among the $y$-values.}
\label{tbl:dice2}
\end{table}

Ranking the dice data discussed earlier (Table~\ref{tbl:dice1}) gives the values
listed in Table~\ref{tbl:dice2}.
Using (\ref{eq:spearman_r}) we find $r_s = 0.65$ (surprisingly similar to the $r = 0.66$ we found using (\ref{eq:corrcoef})),
while the exact equation (\ref{eq:spearman_r_exact}) yields $r_s = 0.6325$, which may be close enough for government work.
For the simplified value the $z$-statistic from (\ref{eq:z_spearman}) becomes $z = 1.3$, which is well inside the 95\% confidence limits
$(z_{0.025} = \pm 1.96)$ for a normal distribution.  Likewise, using (\ref{eq:t_stat_corr}) we find $t = 1.41$ with critical
$t_{0.025,3} = 3.18$.  Hence, in either case we again arrive at the same conclusion
that we cannot reject $H_0$.  However, for such a small data set the approximations usually are quite poor;
Table~\ref{tbl:Critical_Spearman} states the critical correlation is 1, meaning it would take a perfect nonparametric correlation to reject $H_0$.

\index{Nonparametric tests|)}
\index{Spearman's rank correlation|)}
\index{Test!Spearman's rank correlation|)}
\index{Correlation!Spearman's rank|)}
\index{Correlation!nonparametric test}

In summary, there are numerous tests, both parametric and nonparametric, that can be applied to our data,
and there are many others not covered in these notes.  However, the ones presented here are the most common
hypothesis tests that all scientists should be aware of.  A simple guide to their use is given in Figure~\ref{fig:Fig1_HypothesisChart}.
\PSfig[h]{Fig1_HypothesisChart}{Simple decision chart for selecting standard parametric or nonparametric tests.  The
\emph{run test}\index{Run test} is a specific application of the sign test and will be discussed in Chapter~\ref{ch:sequences}.}

\clearpage
\section{Problems for Chapter \thechapter}

\begin{problem}
The following data were obtained in an experiment designed to check whether there is a systematic difference
in the weights (in grams) obtained with two different scales.
At the 0.01 level of significance, is there a systematic bias in the readings from the two scales?
Hint: Given the hypothesis, do we have one or two samples? (Embrace the hint!).
\begin{table}[H]
\centering
\begin{tabular}{|l||c|c|} \hline
\bf{Item} & \bf{Scale 1} & \bf{Scale 2} \\ \hline
1  & 2.13	& 2.17 \\ \hline
2  & 17.56	& 17.61 \\ \hline
3  & 9.33	& 9.35 \\ \hline
4  & 11.40	& 11.42 \\ \hline
5  & 288.62	& 288.61 \\ \hline
6  & 10.25	& 10.27 \\ \hline
7  & 23.37	& 23.42 \\ \hline
8  & 106.27	& 106.26 \\ \hline
9  & 12.40	& 12.45 \\ \hline
10 & 24.78      & 24.75 \\ \hline
\end{tabular}
\end{table}
\end{problem}

\begin{problem}
We have obtained two samples of magnesium concentrations (in ppm) from an igneous rock unit.
The samples, obtained at two different locations, exhibit the concentrations listed below.
At the 95\% level of confidence, do the samples support the hypothesis that the average
Mg concentration is the same at both locations? (Assume such data are normally distributed.).
\begin{table}[H]
\centering
\begin{tabular}{|l||c|c|c|c|c|c|c|c|c|} \hline
\bf{Location 1} & 123 &  151 & 162 & 130 & 156 & 120 & 139 & 133 & \\ \hline
\bf{Location 2} & 142 &  131 & 138 & 145 & 166 & 159 & 173 & 151 & 155 \\ \hline
\end{tabular}
\end{table}
\end{problem}

\begin{problem}
The following random samples are measurements of the heat-producing capacity (in millions of calories
per ton) of specimens of coal from two mines:
\begin{table}[H]
\centering
\begin{tabular}{|l||c|c|c|c|c|} \hline
\bf{Mine 1} & 8,400 &  8,230 & 8,380 & 7,860 & 7,930 \\ \hline
\bf{Mine 2} & 7,510 &  7,690 & 7,720 & 8,070 & 7,660 \\ \hline
\end{tabular}
\end{table}
At the 95\% level of confidence, is the difference in mean value different?
(You must first check if the standard deviations are similar).
\end{problem}

\begin{problem}
If the seismic velocities $v$ (in km/sec) in a limestone layer are normally distributed according to the probability density function
$$
p(v) = \frac{1}{1.45\sqrt{2\pi}} \exp{\left \{ -\frac{1}{2} \left (\frac{v - 5.15}{1.45} \right )^2 \right \}},
$$
then what is the probability that any single seismic refraction study will determine a velocity in the $5.5 < v < 6.0$ range?
\end{problem}

\begin{problem}
After three months of strenuous field work a graduate student returns to the lab with new data to analyze.
Water samples collected from two remote rivers are analyzed for mercury contamination.  The
results of the analysis (in ppm) are:
\begin{description}
\item [River A:] $9.86, 12.02, 12.96, 10.40, 12.43, 9.61, 11.12, 10.64, 10.22$.
\item [River B:] $11.36, 10.48, 11.06, 11.61, 13.28, 12.72, 13.91, 12.08, 12.38, 12.80$.
\end{description}
\begin{enumerate}[label=\alph*)]
\item At the 95\% confidence level, do these samples support the alarming hypothesis that the average
mercury contamination is higher in river B than in river A?
\item Given the statistical
parameters you have obtained, plot on a single graph the two normal probability density distributions showing
how the mercury samples are distributed, as well as the two \emph{theoretical} probability density distributions of
the sample means.  Label your illustration and comment on what you see.
\end{enumerate}
\end{problem}

\begin{problem}
	The water contents of soils (in volume \%) were measured at two different sites A and B
	and are reproduced in \emph{soilwater.txt}.
At the 99\% level of confidence, do the soils at the two sites have different water content?
\end{problem}

\begin{problem}
A sensitive instrument measuring nickel content (in ppm)  was tested on a set of reference
specimens with known concentrations.  The results (listed below in the ``Before'' columns) were
unbiased and no calibration was necessary.  After a thorough cleaning of the instrument the test
sequence was repeated, with results listed in the ``After'' columns below.
\begin{table}[H]
\centering
\begin{tabular}{|c|c||c|c|} \hline
\bf{Before} & \bf{After} &  \bf{Before} & \bf{After} \\ \hline
211	&	198	&	172	&	166 \\
180	&	173	&	155	&	154 \\
171	&	172	&	185	&	181 \\
214	&	209	&	168	&	164 \\
183	&	179	&	203	&	201 \\
194	&	192	&	180	&	175 \\
160	&	161	&	245	&	240 \\
181	&	182	&	146	&	142 \\ \hline
\end{tabular}
\end{table}
\noindent
At the 0.01 level of significance, is there a need to re-calibrate the instrument after the cleaning?
Hint: Consider the differences in the measurements.
\end{problem}

\begin{problem}
A soil scientist worrying about toxic waste wants to know if the standard deviation of a certain pollutant
coming from several nearby factories is less than 10 ppm.  Based on a sample of 15 values she finds
$s = 11.3$ ppm.  What can she conclude at the 95\% level of confidence?
\end{problem}

\begin{problem}
A designer of seismometers must know whether or not the standard deviation of
the time it takes the instrument to start recording an event after being triggered by
an earthquake is less than 0.010 s.  Use the 0.05 level of significance to test the
null hypothesis $H_0: \sigma \leq 0.010$ s against the alternative hypothesis
$H_1: \sigma > 0.010$ on the basis of a random sample of size $n = 16$ for which the
sample standard deviation was found to be $s = 0.012$ s.
\end{problem}

\begin{problem}
An examination designed to measure basic knowledge in geology was given to random samples
of freshmen at two major universities, and their scores were:
\begin{description}
\item [University A]: $77, 72, 58, 92, 87, 93, 97, 91, 70, 98, 76, 90, 62, 69, 90, 78, 96, 84, 73, 80$
\item [University B]: $89, 74, 45, 56, 71, 74, 94, 88, 66, 62, 88, 63, 88, 37, 63, 75, 78, 34, 75, 68$
\end{description}
Apply the $U$-test at the 0.05 level of significance to test the null hypothesis that there
is no significant difference in average knowledge of geology at the two universities.
(MATLAB hint: Fill in vectors g and source (A = 1, B = 2), use [gsorted, gkey] = sort (g)
to sort g, and use ranks1 = find (source(gkey) == 1) to get the ranks of entries from University A, etc.)
\end{problem}

\begin{problem}
Given 10 data pairs you find $r = -0.85$.  Is this correlation significant at the 99\% level?
\end{problem}

\begin{problem}
Given three data pairs $(x_i, y_i)$, how high correlation coefficient would you need to determine for it to be significant at the 95\% level of confidence?
\end{problem}

\begin{problem}
\newcounter{qfringec}
\setcounter{qfringec}{\thechapter}
\newcounter{qfringep}
\setcounter{qfringep}{\theproblem}
A fringe seismologist goes on TV and claims there are
more local earthquakes during certain months of the year when planetary constellations are
considered ``favorable''.  A wee bit skeptical, you decide to examine the earthquake catalog
for a whole year (see table \emph{quakedays.txt}) to test his claim.  The table gives
the day number (1--365) when a local earthquake occurred.  Assume this is not a leap year.
\begin{enumerate}[label=\alph*)]
\item State the null hypothesis and the alternative hypothesis.
\item What are the expected number of events for each month?
\item Using the $\chi^2$-test and a 95\% level of confidence,
do the data suggest that some months have more earthquakes than others?
\end{enumerate}
\end{problem}

\begin{problem}
An environmental scientist measures the sulfur dioxide emissions from an industrial plant
over an 80 day period.  The amounts (in tons per day) are given in the file \emph{sulfur.txt}.
\begin{enumerate}[label=\alph*)]
\item Bin the data using the categories less than 10, 10--15, 15--20, 20--25, 25 and above.
   Plot the histogram and indicate the counts.
\item The scientist wonders if the emissions are well described by the expected normal distribution.
  What are the mean and standard deviation for the raw data?
\item The scientist decides to use a $\chi^2$ test.  What are the expected counts in each bin?
\item Test whether or not the binned data are indistinguishable from a normal distribution
  at the 95\% level of confidence.  Should she reject $H_0$?
\end{enumerate}
\end{problem}

\begin{problem}
From past experience, we know there is a weak correlation between the the proportion
of manganese nodules on the seafloor and the age of the underlying crust.  How many data
pairs, $n$, do we need to sample before an inferred positive correlation of $r = 0.32$ is significantly
larger than zero at the 95\% level of confidence?  (MATLAB hint:  Use \texttt {icdf} to get
critical $t$-values -- try \texttt {help icdf} to see how to use it.)
\end{problem}

\begin{problem}
A senator from a ``blue state'' believes oil imported to the US from Norway has been infused with a nefarious
toxic chemical.  He theorizes that when drivers in the US use fuel from the imported oil they
become debilitated from the fumes and the chance of accidents at railroad crossings increases significantly.
He obtains two data sets from 1999--2009 showing the annual import of oil (in millions of barrels) and the number of drivers
killed in collisions with railway trains, reproduced in the file \emph{conspiracy.txt}.  Plot the two time series
on the same graph (with different vertical scales), then calculate the correlation between
the annual data pairs.  Is the correlation significant at the 95\% level?  Should the Senate punish the
Norwegians by authorizing US forces to invade their little kingdom and confiscate their oil rigs?
\end{problem}

\begin{problem}
A republican congressman is upset with the amount of funding the US National Science Foundation allocates to
support research in the social sciences.  Looking for arguments to reduce this wasteful spending, he
comes across data that seem to suggest that awarding too many doctorates in sociology may cause an increase in deaths
due to \emph{anticoagulants}.  He theorizes that the lack of rigor in the social sciences causes blood to
thin, leading to these unnecessary deaths.  As his intern, you are tasked with examining the data (see file \emph{PhD\_deaths.txt})
and determine if the correlation between the number of Ph.D.'s and deaths might be significant at the 99\% level.
What will your report to the Congressman conclude?
\end{problem}

\begin{problem}
We have made determinations of chromium content in four shale units.  For each unit we obtained six measurements;
the data are summarized below (values are given in ppm).  With a one-way ANOVA at the 0.05 level of significance, can the
differences among the four sample means be attributed solely to measurement uncertainties?

\begin{table}[H]
\centering
\begin{tabular}{|c|c|c|c|} \hline
\bf{Unit A} & \bf{Unit B} & \bf{Unit C} & \bf{Unit D} \\ \hline
181.3 & 160.6 & 163.2 & 132.3 \\ \hline
176.8 & 179.9 & 154.1 & 140.8 \\ \hline
182.7 & 170.9 & 158.6 & 124.5 \\ \hline
186.0 & 151.3 & 137.8 & 123.2 \\ \hline
188.4 & 163.4 & 158.7 & 121.1 \\ \hline
160.2 & 185.8 & 191.8 & 163.2 \\ \hline
\end{tabular}
\end{table}
\end{problem}

\begin{problem}
Four different temperature gauges recorded the temperature in a storage room.  The data are summarized
below.  At the 0.05 level of significance, can the differences among the four sample means be attributed to chance?
\begin{table}[H]
\centering
\begin{tabular}{|c|c|c|c|} \hline
\bf{Gauge A} & \bf{Gauge B} & \bf{Gauge C} & \bf{Gauge D} \\ \hline
23.1	&	21.7	&	21.9	&	19.8 \\ \hline
22.8	&	23.0	&	21.3	&	20.4 \\ \hline
23.2	&	22.4	&	21.6	&	19.3 \\ \hline
23.4	&	21.1	&	20.2	&	18.5 \\ \hline
23.6	&	21.9	&	21.6	&	19.1 \\ \hline
21.7	&	23.4	&	23.8	&	21.9 \\ \hline
\end{tabular}
\end{table}
\end{problem}

\begin{problem}
	Water samples were obtained from four different locations along a river to determine
	whether the quantity of dissolved oxygen, a proxy for water pollution, varies between
	the locations.  Locations A and B were selected above an industrial plant: one near
	the shore and another in midstream. Location C was adjacent to the industrial water
	discharge from the plant, and location D was slightly downriver in midstream.
	Water specimens were randomly selected at each location, but one specimen, from location
	D, was lost in the laboratory.  The data are shown in the table below, with lower values
	corresponding to higher levels of pollution.
	\begin{enumerate}[label=\alph*)]
	\item Do the data provide sufficient evidence to indicate a significant difference in mean dissolved oxygen
	content for the four locations?
	\item Compare the mean dissolved oxygen content in midstream above the plant with the mean content
	adjacent to the plant (i.e., locations B versus C).  Use a 95\% confidence interval.  What do you conclude?
	\end{enumerate}
	\begin{table}[H]
	\centering
	\begin{tabular}{|c||c|c|c|c|c|} \hline
	\bf{Location} & \multicolumn{5}{c|}{\bf{Mean Oxygen Content}} \\ \hline
	A & 5.9 &  6.1 & 6.3  & 6.1 & 6.0 \\ \hline
	B & 6.3 &  6.6 & 6.4  & 6.4 & 6.5 \\ \hline
	C & 4.8 &  4.3 & 5.0  & 4.7 & 5.1 \\ \hline
	D & 6.0 &  6.2 & 6.1  & 5.8 &     \\ \hline
	\end{tabular}
	\end{table}
\end{problem}

\begin{problem}
A laboratory technician measures the breaking (tensile) strength of each of five
samples of granite ($G_1$ -- $G_5$) using four different measuring instruments ($I_1$ -- $I_4$),
and obtains the following results (in MPa):

\begin{table}[H]
\centering
\begin{tabular}{|c||c|c|c|c|} \hline
 & $I_1$ & $I_2$ & $I_3$ & $I_4$ \\ \hline
$G_1$ & 18.6 &  18.1 & 17.6  & 19.6 \\ \hline
$G_2$ & 22.7 &  23.9 & 24.7  & 22.5 \\ \hline
$G_3$ & 23.2 &  20.8 & 19.2  & 22.1 \\ \hline
$G_4$ & 22.5 &  18.9 & 21.2  & 23.4 \\ \hline
$G_5$ & 17.3 &  18.9 & 19.8  & 18.8 \\ \hline
\end{tabular}
\end{table}

\begin{enumerate}[label=\alph*)]
\item Using a two-way ANOVA test, do the data provide sufficient evidence to indicate a difference in strength for the five samples?
(Use $\alpha = 0.1$)
\item Is there any statistical evidence of instrument bias? (Use $\alpha = 0.1$)
\end{enumerate}
\end{problem}

\begin{problem}
A study was conducted to compare automobile gasoline mileage for three brands of gasoline,
A, B, C.  Four cars, all of the same make and model, were employed in the experiment and
each gasoline brand was tested in each car.  The data, in miles per gallon, are as follows:
\begin{table}[H]
\centering
\begin{tabular}{|c||c|c|c|c|} \hline
	&	\bf{CAR 1}	&	\bf{CAR 2}	&	\bf{CAR 3}	&	\bf{CAR 4} \\ \hline
\bf{GAS A}	&	15.7	&	17.0	&	17.3	&	16.1  \\ \hline
\bf{GAS B}	&	17.2	&	18.1	&	17.9	&	17.7  \\ \hline
\bf{GAS C}	&	16.1	&	17.5	&	16.8	&	17.8  \\ \hline
\end{tabular}
\end{table}

\begin{enumerate}[label=\alph*)]
\item Using a two-way ANOVA, do the data provide sufficient evidence to indicate a difference in the
	mean mileage per gallon for the three gasoline brands?
(Use $\alpha = 0.1$)
\item Is there any statistical evidence of differences mean mileage for the four automobiles? (Use $\alpha = 0.1$)
\end{enumerate}
\end{problem}

\begin{problem}
Densities $\rho_i$ were obtained from a sample of 40 specimens from the
same lithological unit (see \emph{rho.txt}).
Use the sign test at the 0.01 level of significance to test the null hypothesis $\tilde{\rho} \leq 2.42$
against the alternative hypothesis $\tilde{\rho} > 2.42$.
\end{problem}

\begin{problem}
The science team on an oceanographic research vessel has measured the heat flux through
the ocean floor at two sites with similar crustal ages.  Site I was formed by seafloor spreading
with a spreading rate of 15 cm/yr while site II was formed at half the spreading rate.  The
heat flux values (in mWm$^{-2}$) recorded were:
\begin{table}[H]
\centering
\begin{tabular}{|c||c|c|c|c|c|c|c|c|} \hline
\bf{Site I} & 57.15 &  63.00 & 67.64 & 62.09 & 73.84 & 65.54 & 61.88 & 67.95 \\ \hline
\bf{Site II} & 56.82 & 58.88 & 65.93 & 56.34 & 59.92 & 50.52 & 65.37 & 53.62 \\ \hline
\end{tabular}
\end{table}
\begin{enumerate}[label=\alph*)]
\item Use the $U$-test at the 0.05 level of significance to test the hypothesis that the heat flux
from site II is different from that of site I. (MATLAB hint: Fill in vectors q and source, use [qsorted, key] = sort (q)
to sort q, and use ranks1 = find (source(key) == 1) to get the ranks of entries from site 1, etc.)
\item Upon reviewing the literature for the region, it is discovered that a previous cruise recorded
one heat flux measurement from Site I, yielding a value of 74.19 mWm$^{-2}$.  Does this change the results
obtained above?
\end{enumerate}
\end{problem}

\begin{problem}
The Earth's surface temperature (important in many agricultural as well as hydrological problems)
can be tediously measured on the ground or conveniently recorded by remote infrared sensors mounted on
airplanes or satellites.  However, the remotely sensed data appear to have a bias.  At the 95\%
level of confidence, is there such a bias in the data below?
\begin{table}[H]
\centering
\begin{tabular}{|c||c|c|} \hline
\bf{Location}	&	\bf{Ground (\DS C)}	&	\bf{Remote (\DS C)} \\ \hline
1	&	36.9	&	37.3 \\ \hline
2	&	35.4	&	38.1 \\ \hline
3	&	26.3	&	27.9 \\ \hline
4	&	21.0	&	22.7 \\ \hline
5	&	14.7	&	16.2 \\ \hline
\end{tabular}
\end{table}
\end{problem}

\begin{problem}
The textural properties of sandstones are believed to reflect environmental 
conditions when they first formed.  Textural maturity is defined as the degree to which a sand is 
both well \emph{sorted} and well \emph{rounded}.  These two characteristics are expected to be correlated.  
However, roundness and degree of sorting are not quantified on a ratio scale but assigned ordinal 
values such as ``moderately sorted'' or ``well sorted''.  Hence, we can rank the assessments but not 
calculate ordinary statistics from them.  Based on the table below, is there a significant (95\% 
level) correlation between roundness and degree of sorting? (P = poor, M = moderate, W = well sorted,
and A = angular, SA = subangular, SR = subrounded.)
\begin{table}[H]
\centering
\begin{tabular}{|c|r|c|r|} \hline
\bf{Sorting} &  \bf{Rank} & \bf{Roundness} & \bf{Rank} \\ \hline \hline
P & 4  & SR & 11 \\ \hline
W & 10 & SA & 9  \\ \hline
P & 2  & A  & 1  \\ \hline
M & 8  & SA & 4  \\ \hline
M & 6  & SA & 6  \\ \hline
W & 9  & SR & 12 \\ \hline
P & 3  & SA & 8  \\ \hline
M & 7  & SA & 3  \\ \hline
W & 11 & SA & 7  \\ \hline
W & 12 & SR & 10 \\ \hline
M & 5  & A  & 2  \\ \hline
P & 1  & SA & 5  \\ \hline
\end{tabular}
\end{table}
\end{problem}

\begin{problem}
A homeowner decides to test whether or not the level of salt in his drinking water
at home is significantly higher than that of the drinking water at his office.  Measuring the salinity (in ppm) gives:
\begin{table}[H]
\centering
\begin{tabular}{|c||r|r|r|r|r|r|r|r|} \hline
\bf{Home}	& 76.19	& 84.00	& 79.89	& 82.78	& 98.45	& 87.38	& 82.50	& 90.60 \\ \hline
\bf{Office}	& 75.76	& 78.51	& 87.91	& 75.12	& 90.19	& 67.36	& 87.16	&       \\ \hline
\end{tabular}
\end{table}
\begin{enumerate}[label=\alph*)]
\item Use the $U$-test at the 0.01 level of significance to test his hypothesis.
\item He decides to get an 8th measurement from the office and finds salinity to be 72.31.
  Does this change the results obtained above?
\end{enumerate}
\end{problem}

\begin{problem}
Permeabilities have been estimated in the lab for samples taken from two different
sandstone outcrops.  We obtained these values (in Darcy; 1 Darcy = $9.8697\times10^{-13}\mbox{m}^2$):

\begin{table}[H]
\centering
\begin{tabular}{|c||r|r|r|r|r|r|r|r|r|r|} \hline
\bf{A} & 18.76 & 13.24 & 3.83 & 10.12 & 8.40 & 8.60 & 18.18 & 15.04 & 9.62 & 13.22 \\ \hline
\bf{B} & 15.90 & 5.80 & 1.31 & 17.50 & 9.22 & 6.20 & 1.92 & 13.46 & 2.61 & 8.01 \\ \hline
\end{tabular}
\end{table}

Using the Kolmogorov-Smirnov test at the 95\% level of confidence, do these samples appear to come from the same
lithological unit (i.e., population)?  Use the MATLAB script \texttt{kolsmir.m} and the two-sample critical
values provided in Table~\ref{tbl:Critical_KS3}.
\end{problem}

\begin{problem}
We have obtained two samples of residual magnetization from a basaltic sill at two different sites.  The values are:
\begin{table}[H]
\centering
\begin{tabular}{|c||r|r|r|r|r|r|r|r|r|r|} \hline
\bf{Site 1} & 68.9 & 41.1 & -6.1  & 25.6 & 17.0 & 18.0 &  65.9 & 50.0 &  23.1 & 41.1 \\ \hline
\bf{Site 2} & 54.5 &  4.0 & -18.5 & 62.5 & 21.0 &  6.0 & -15.5 & 42.2 & -13.0 & 15.0 \\ \hline
\end{tabular}
\end{table}
Using the Kolmogorov-Smirnov test at the 95\% level of confidence, do these samples come from the same population?
\end{problem}

\begin{problem}
As discussed in Chapter~\ref{ch:EDA}, the Earth's magnetic field reverses direction. Table \emph{GK2007.txt} contains data from the
Gee and Kent (2007) geomagnetic time scale. It lists all normal and reversely magnetized chrons and gives the duration
of each interval in Myr.  Note: Examine the last letter in the chron: ``n'' means normalized and ``r'' stands for reversed polarity.
Using the Kolmogorov-Smirnov test at the 95\% level of confidence, are the distributions of intervals for the normal and
reverse polarities different?
\end{problem}

\begin{problem}
We will revisit Problem~\theqfringec.\theqfringep, but this time we will utilize the Kolmogorov-Smirnov test and completely
avoid any discussion about binning artifacts.
\begin{enumerate}[label=\alph*)]
	\item What is the cumulative distribution function you will compare your observed data to?
	\item Can you reject the null hypothesis at the 99\% level of confidence?
\end{enumerate}
\end{problem}

\begin{problem}
From a Carboniferous shale we obtain the following concentrations of chromium and nickel (in ppm):
\begin{table}[H]
\centering
\begin{tabular}{|c|c|} \hline
\bf{Cr (ppm)}	&	\bf{Ni (ppm)} \\ \hline
122	&	604 \\ \hline
340	&	311 \\ \hline
522	&	173 \\ \hline
61	&	503 \\ \hline
133	&	495 \\ \hline
235	&	444 \\ \hline
498	&	272 \\ \hline
371	&	362 \\ \hline
239	&	384 \\ \hline
\end{tabular}
\end{table}
\begin{enumerate}[label=\alph*)]
\item	What is the correlation coefficient, $r$.  Is the correlation significant at $\alpha = 0.01$?
\item	What is Spearman's rank correlation coefficient, $r_s$? Is it significant at $\alpha = 0.01$?
\item	Suppose one additional measurement was added to the table, with Cr = 698, Ni = 597.
	What are the new correlations $r$ and $r_s$?  Are these correlations significant at $\alpha = 0.01$?
\item	We suspect the last measurement to be an outlier.  Given the recipe of detecting outliers discussed
	previously based on the statistical properties of the first 9 values, are the Cr and Ni values
	for the additional measurement outliers in their respective groups?
\end{enumerate}
\end{problem}

\begin{problem}
The abundance (in \%) of four elements in seven samples of basalt from the Pacific has been recorded as
\begin{table}[H]
\centering
\begin{tabular}{|c|c|c|c|c|} \hline
\bf{Sample \#}	& \bf{Si} & \bf{Al} & \bf{Fe} & \bf{Mg} \\ \hline
1	& 22.5	& 9.6	& 6.6	& 3.4 \\ \hline
2	& 22.1	& 8.4	& 7.8	& 3.6 \\ \hline
3	& 25.9	& 8.7	& 4.8	& 4.0 \\ \hline
4	& 23.5	& 8.1	& 5.0	& 5.2 \\ \hline
5	& 21.7	& 10.0	& 8.2	& 4.9 \\ \hline
6	& 21.9	& 8.2	& 9.3	& 4.9 \\ \hline
7	& 23.7	& 7.2	& 9.5	& 3.3 \\ \hline
\end{tabular}
\end{table}
Compute the correlations between all pairs of elements.  Are any of the correlations significant at the 0.05 level?
\end{problem}

%  $Id: DA1_Chap5.tex 688 2019-06-09 09:24:42Z pwessel $
%
\chapter{LINEAR (MATRIX) ALGEBRA}
\epigraph{``Never send a human to do a machine's job.''}{\textit{Agent Smith} in \textit{The Matrix}}
\label{ch:matrix}
A large subset of data analysis techniques is simply the practical application of
linear algebra.  Undergraduate students in the natural sciences often lack any formal introduction
to this material, or they suffered through an overly theoretical presentation in a course
offered by a mathematics department.  In this book we will not dwell too much on the finer
theoretical points of linear algebra but instead present a simple overview of the aspects that are particularly
pertinent in data analysis.  There are of course an infinite number of books on matrix and
linear algebra that the eager reader could consult beyond this brief introduction.

\section{Matrix Algebra Terminology}

\index{Matrix!definition}
\index{Matrix!order}
	A \emph{matrix} is simply a rectangular array of elements arranged in a series of $m$  rows and $n$  
columns. The \emph{order} of a matrix is the specification of the number of rows by the number of 
columns. \emph{Elements} of a matrix are denoted $a_{ij}$, where index $i$ specifies the \emph{row} position 
and index $j$ specifies the \emph{column} position; thus $a_{ij}$ identifies the element at position $i,j$.

\index{Matrix!element}
	An element can be a number (real or complex), an algebraic expression, or (with some 
restrictions) another matrix or matrix expression. As an example of a real matrix,  consider
\begin{equation}
\mathbf{A} = \left[ \begin{array}{rcc}
12 & 4 & 10 \\
8 & 1 & 11\\
15 & 3 & 11\\
14 & 1 & 11
\end{array}   \right]	 
 \ = \
\left[  \begin{array}{ccc}
a_{11} & a_{12} & a_{13} \\
a_{21} & a_{22} & a_{23}\\
a_{31} & a_{32} & a_{33} \\
a_{41} & a_{42} & a_{43}
\end{array} \right].
\end{equation}
This matrix, $\mathbf A$, is of order $4 \times 3$, with elements $a_{11}$ = 12, $a_{21}$ = 8, and so on. The notation for matrices is 
not always consistent, but it is usually one of the following schemes:

\begin{description}

\item [Matrix:] Designated by a bold, uppercase letter (the most common scheme), brackets, or hat $(\hat{ \ })$,
sometimes with one or (more commonly) two underscores.
The order is also sometimes explicitly given.  E.g., $\mathbf{A}_{(4,3)}$ means 
$\mathbf A$ is of order $4 \times 3$.

\item [Order:] Always given as rows x columns but uses letters $n,k,p$ differently: $n$ (rows) x $k$ (columns) or 
$k$ (rows) x $n$ (columns).

\item [Element:] Most commonly $a_{ij}$, with $i =$ row; $j =$ column (sometimes other dummy indices like $l,p,q$ are used).
\end{description}

The advantages of matrix algebra mainly lies in the fact that it provides a concise and simple 
method for manipulating large sets of numbers or computations, making it ideal for computers. 
Furthermore,

\begin{enumerate}

\item The compact form of matrices allows a convenient notation for describing large tables of data.

\item Matrix operations allow complex relationships to be seen, which otherwise would be obscured by the 
shear size of the data (i.e., it aids in clarification).

\item Most matrix manipulations involve just a 
few standard operations for which standard subroutines are readily available.
\end{enumerate}

MATLAB, which stands for ``Matrix Laboratory'', is ideally suited to perform such manipulations,
as is its Open Source clone, Octave. Python with numPy, R or Julia are also good choices.
\index{MATLAB}
\index{Octave}

	As a convention with data matrices (i.e., when the elements are data values), the columns 
usually represent the different \emph{variables} (e.g., one column contains temperatures, another salinity, 
etc.) while rows contain the \emph{observations} (e.g., the values of the variables at different depths, times, or positions). Since 
there are usually more observations than variables, such data matrices are typically rectangular, having 
more rows $(n)$ than columns $(k)$, i.e., the matrix has an order $n \times k$ where $n > k$.

\section{Matrix Definitions}

Matrices whose smallest dimension equals one are called \emph{vectors} and are typically designated
by a bold, lowercase letter, but sometimes they may be typeset normally with either an
arrow above them or a single underscore beneath. By having only one dimension, one of the two indices (row or column) is dropped.
A \emph{column vector} is a matrix containing only a single column of elements, such as
\index{Column vector}
\index{Vector!column}
\begin{equation}
\mathbf{a} = \left[ \begin{array}{c}
a_1\\
a_2\\
\vdots\\
a_n
\end{array}  \right] . 
\end{equation}
\noindent
A \emph{row vector} is a matrix containing only a single row of elements, e.g.,
\index{Row vector}
\index{Vector!row}
\begin{equation}
\mathbf{a} = \left[ \begin{array}{cccc}
a_1 & a_2 & \cdots & a_n
\end{array}  \right] .
\end{equation}
The size of a vector is simply the number of elements it contains ($n$, in both examples above).
\index{Null matrix}
\index{Matrix!null}
The \emph{null matrix}, written as $\mathbf{0}$ or $\mathbf{0}_{(k,n)}$ has all its elements set equal to 0 --- it plays the
role of ``zero'' in matrix algebra.
A \emph{square matrix} has the same numbers of rows as columns, so its order is $n \times n$.
\index{Square matrix}
\index{Matrix!square}
A \emph{diagonal matrix} is a square matrix with zeros in all positions except along the principal (or 
leading) diagonal:
\begin{equation}
	\index{Diagonal matrix}
	\index{Matrix!diagonal}
\mathbf{D} = \left[ 
\begin{array}{ccc}
3 & 0 & 0 \\
0 & 1 & 0\\
0 & 0 & 6
\end{array} \right]	 
\end{equation}
or
\index{Matrix!identity}
\index{Identity matrix}
\begin{equation}
d_{ij} = \left \{ \begin{array}{cl}
0 & \mbox{for } i \neq j\\
\mbox{nonzero} & \mbox{for } i = j 
\end{array} \right.
\end{equation}	 
This type of matrix is important for scaling the rows or columns of other matrices.  The \emph{identity 
matrix} is a diagonal matrix with all of its nonzero elements equal to one. Written as $\mathbf{I}$ or $\mathbf{I}_n$,
it plays the role of ``one'' in matrix algebra.
A \emph{lower triangular matrix} ($\mathbf{L}$) is a square 
matrix with all elements equal to zero \emph{above} the principal diagonal:
\begin{equation}
	\index{Matrix!lower triangular}
	\index{Lower triangular matrix}
\mathbf{L} = \left [ \begin{array}{ccc}
1 & 0 & 0 \\
3 & 7 & 0 \\
8 & 2 & 6
\end{array}
\right ] =
\left[ \begin{array}{ccc}
1 \\
3 & 7 \\
8 & 2 & 6 \end{array} \right]
\end{equation}
or
\begin{equation}
\ell_{ij} = \left \{ \begin{array}
{cl}
0 & \mbox{for } i < j\\
\mbox{nonzero} & \mbox{for } i \geq j 
\end{array} \right .
\end{equation}	 
An \emph{upper triangular matrix} is thus a square matrix with all elements equal to zero \emph{below} the principal 
diagonal
\begin{equation}
	\index{Matrix!upper triangular}
	\index{Upper triangular matrix}
u_{ij} = \left \{ \begin{array}{cl}
0 & \mbox{for } i > j\\
\mbox{nonzero} & \mbox{for } i \leq j 
\end{array} \right .
\end{equation}
If one multiplies two triangular matrices of the same form, the result is a third matrix of the same 
form.

\index{Fully populated matrix}
\index{Matrix!fully populated}
\index{Matrix!sparse}
\index{Sparse matrix}
	We also have the \emph{fully populated matrix} which is a matrix with all of its elements nonzero, 
the \emph{sparse matrix} which is a matrix with only a small proportion of its elements nonzero, and the \emph{scalar} 
which simply is a number (i.e., a matrix of order 1x1, representing a single element).

   A \emph{matrix transpose} (or the transpose of a matrix) is obtained by \emph{interchanging} the rows and columns 
of the matrix. Thus, row $i$ becomes column $i$ and column $j$ becomes row $j$. As a consequence, the order of the matrix 
is reversed:
\begin{equation}
	\index{Matrix!transpose}
	\index{Transpose of matrix}
\mathbf{A} = \left[     \begin{array}{cc}
1 & 14 \\
6 & 7 \\
8 & 2 
\end{array} \right]
 \Rightarrow
\mathbf{A}^T = \left[ \begin{array}{ccr}
1 & 6 & 8 \\
14 & 7 & 2 \end{array} \right]
\end{equation}
As shown, taking the transpose is indicated by the superscript ``T''.
Repeated transposing yields the original matrix, i.e.,
\begin{equation}
(\mathbf{A}^T)^T = \mathbf{A}.
\end{equation} 
A diagonal matrix is its own transpose: $\mathbf{D}^T = \mathbf{D}$. In general, we find the transpose rule
\begin{equation}
a_{ij} \Leftrightarrow a_{ji}.
\end{equation}	 
   A \emph{symmetric matrix} is a square matrix that is symmetric about its principal diagonal, so 
$a_{ij} = a_{ji}$.  Therefore, a symmetric matrix is equal to its own transpose:
\begin{equation}
	\index{Matrix!symmetric}
	\index{Symmetric matrix}
\mathbf{A} = \left[ \begin{array}{ccc} 
1 & 2 & 5 \\
2 & 6 & 3 \\
5 & 3 & 4
\end{array}
\right]
= \mathbf{A}^T .
\end{equation} 	 
A \emph{skew symmetric matrix} is a matrix in which
\begin{equation}
a_{ij} = -a_{ji}.
\end{equation}
Therefore, $\mathbf{A}^T = - \mathbf{A}$.  Thus, $a_{ii}$, the principal diagonal elements, must all be zero.
The following matrix is skew symmetric:
\begin{equation}
	\index{Matrix!skew symmetric}
	\index{Skew symmetric matrix}
\mathbf{A} = \left[ \begin{array}{ccc} 
0 & 4 & -5 \\
-4 & 0 & 3 \\
5 & -3 & 0
\end{array}
\right].
\end{equation}
Any square matrix can be decomposed into the \emph{sum} of a symmetric and a skew-symmetric matrix:
\begin{equation}
\mathbf{A} = \frac{1}{2} (\mathbf{A} + \mathbf{A}^T) + \frac{1}{2} (\mathbf{A} - \mathbf{A}^T).
\end{equation}
The \emph{trace} of a square matrix is simply the sum of the elements along the principal diagonal. It 
is symbolized as tr ($\mathbf{A}$).\index{Matrix!trace}\index{Trace of matrix}
This property is useful in calculating various quantities from matrices. 
\index{Matrix!submatrix}
\index{Submatrix}
\index{Matrix!supermatrix}
\index{Supermatrix}
\emph{Submatrices} are smaller matrix partitions of the larger \emph{supermatrix}, i.e.,
\begin{equation}
\left [ \frac{\mbox{Supermatrix}}{\mathbf{F}} \right] = \left[	\frac{\mathbf{A} | \mathbf{B}}{\mathbf{C} | \mathbf{D}} \right].
\end{equation}
Such partitioning will frequently be useful.
	 
\section{Matrix Addition}
\index{Matrix!addition}

	Matrix addition and subtraction require matrices of the \emph{same order} since each operation 
simply involves the addition or subtraction of corresponding elements. So, if $\mathbf{C} = \mathbf{A} + \mathbf{B}$ then
\begin{equation}
\mathbf{A} = \left[ \begin{array}{cc}
a_{11} & a_{12}\\
a_{21} & a_{22}\\
a_{31} & a_{32} \end{array} \right]	 , \mathbf{B} = 
\left[ \begin{array}{cc}
b_{11} & b_{12}\\
b_{21} & b_{22}\\
b_{31} & b_{32} \end{array} \right]	 ,
\mathbf{C} = \left[ \begin{array}{cc}
a_{11}+  b_{11} & a_{12} +  b_{12}\\
a_{21}+  b_{21} & a_{22} +  b_{22}\\
a_{31} + b_{31} & a_{32} + b_{32} \end{array} \right]	 ,
\end{equation}
and (with apologies to ABBA fans)
\begin{equation}
\mathbf{A} + \mathbf{B} = \mathbf{B} + \mathbf{A},
\end{equation}
\begin{equation}
(\mathbf{A} + \mathbf{B}) + \mathbf{C} = \mathbf{A} + (\mathbf{B} + \mathbf{C}),
\end{equation}	 
where all matrices must be of the same order. \emph{Scalar multiplication} of a matrix is achieved by multiplying all elements of a matrix 
by a constant (the scalar):
\begin{equation}
\beta \mathbf{A} = \beta \left[ \begin{array}{cc}
a_{11} & a_{12} \\
a_{21} & a_{32}\\
a_{31} & a_{32}
\end{array}
\right]
=
\left[ \begin{array}{cc}
\beta a_{11} & \beta a_{12} \\
\beta a_{21} & \beta a_{22}\\
\beta a_{31} & \beta a_{32}
\end{array}
\right],
\end{equation}	 
where $\beta$ is a scalar. 
Thus, every element is multiplied by the scalar. 

\section{Dot Product}
\index{Vector!product|(}
\index{Dot product|(}

The \emph{scalar product} (or \emph{dot product} or 
\emph{inner product}) is the product of two vectors of the same size, e.g.,
\begin{equation}
\mathbf{a}\cdot \mathbf{b} = \beta,
\end{equation}	 
where $\mathbf a$ is a row vector (or the transpose of a column vector) of length $n, \mathbf{b}$ is a column vector (or 
the transpose of a row vector), also of length $n$, and $\beta$ is the scalar product of $\mathbf a \cdot \mathbf b$.
Given the two 3-D vectors
\begin{equation}
\mathbf{a} = [a_1 \ a_2 \ a_3 ], \quad \mathbf{b} = \left[ \begin{array}{c}
b_1\\
b_2\\
b_3
\end{array}
\right],
\end{equation}	 
we sum the products of corresponding elements in the two vectors, obtaining
\begin{equation}
\beta = a_1 b_1 + a_2b_2 + a_3b_3.
\end{equation}
We may visualize this multiplication, as illustrated in Figure~\ref{fig:Fig1_dotproduct} for two 4-D vectors.
\PSfig[H]{Fig1_dotproduct}{The dot product of the two 4-D vectors $\mathbf{a} = [2 \ 1 \ 4 \ 5]$
and $\mathbf{b} = [1 \ 3 \ 4 \ 2]$
is obtained by multiplying the component pairs and calculating the sum of these products.}

Geometrically, this product can be thought of as multiplying the length of one vector by the 
component of the other vector that is parallel to the first, as shown in Figure~\ref{fig:Fig1_dotvector}:
\PSfig[H]{Fig1_dotvector}{Geometrical meaning of the dot product of two vectors.  Regardless of dimension, the
dot product is proportional to the cosine of the angle between the two vectors.}


As an example, think of $\mathbf b$ as a force and $|\mathbf a|$ as the magnitude of displacement, with their product equal to the work in the 
direction of $\mathbf a$. Thus:
\begin{equation}
\mathbf{a \cdot b = |a||b|} \cos (\theta),
\end{equation}	 
where the \emph{magnitude} of a vector $\mathbf x$ is given by
\begin{equation}
|\mathbf{x}| = \sqrt{x^2_1 + x ^2_2 + \ldots + x^2_n}. 
\end{equation}
The maximum principle says that the unit vector $\mathbf{(\hat{n})}$ making $\mathbf{ a \cdot \hat{n}}$ a maximum is that unit vector 
pointing in the same direction as $\mathbf{a}:$ If $\mathbf{\hat{n} \parallel a}$ then $\cos(\theta) = \cos(0^{\circ}) = 1$ and $\mathbf{a\cdot n = |a| |n|}\cos(\theta) =  
\mathbf{|a||\hat{n}| = |a|}$. This is equally true where $\mathbf{d}$ is any vector of a given magnitude --- that vector $\mathbf{\hat{n}}$ which 
parallels $\mathbf{d}$ will give the largest scalar product.
Parallel vectors thus have $\cos(\theta) = 1$, then $\mathbf{a \cdot b = |a||b|}$ and 
$\mathbf{a = \beta b}$ (i.e., two vectors are parallel if 
one is simply a scalar multiple of the other --- this property comes from equating direction cosines), 
where
\begin{equation}
\beta =  \mathbf{|a|/|b|}.
\end{equation}
In contrast, perpendicular vectors have $\cos \theta = \cos 90^\circ = 0$, so that $\mathbf{ a \cdot b} = 0$, where $\mathbf{a \bot   b}$.

Squaring vectors is simply the dot product of a vector with its own transpose, i.e.,
\begin{equation}
\mathbf{a}^2 = \mathbf{ a \cdot a}^T \mbox{ for row vectors}
\end{equation}
and
\begin{equation}
\mathbf{a}^2 = \mathbf{ a}^T \cdot \mathbf{a} \mbox{ for column vectors}.
\end{equation}
\index{Vector!product|)}
\index{Dot product|)}

\section{Matrix Multiplication}
\index{Matrix!multiplication|(}

Matrix multiplication requires ``conformable'' matrices. Matrices are conformable when 
there are as many columns in the first matrix as there are rows in the second matrix.  Consider
\index{Matrix!conformable}
\begin{equation}
\mathbf{C}_{(k,n)} = \mathbf{A}_{(k,p)} \cdot \mathbf{B}_{(p,n)}.
\end{equation}	 
The matrix product $\mathbf{C}$ is of order $k \times n$ and has elements $c_{ij}$, given by
\begin{equation}
c_{ij} = \sum^p _{q=1} a_{iq}b_{qj}.
\end{equation}	 
This is an extension of the scalar product --- in this case, each element of $\mathbf{C}$ is the scalar product of a row 
vector in $\mathbf{A}$ and a column vector in $\mathbf{B}$.  For instance, if
\begin{equation}
\left[
\begin{array}{cc}
c_{11} & c_{12} \\
c_{21} & c_{22} 
\end{array}
\right ]
=
\left[
\begin{array}{ccc}
a_{11}&  a_{12} &  a_{13}\\
a_{21}&  a_{22} &  a_{23}
\end{array}
\right ]
\left [
\begin{array}{cc}
b_{11} & b_{12}\\
b_{21} & b_{22}\\
b_{31} & b_{32}
\end{array}
\right],
\end{equation}
then
\begin{equation}
c_{12} = a_{11}b_{12} + a_{12}b_{22} + a_{13}b_{32}.
\end{equation}
We illustrate the situation in Figure~\ref{fig:Fig1_matprod1}.
\PSfig[h]{Fig1_matprod1}{The matrix product of two matrices $\mathbf{A}$ and $\mathbf{B}$.  The light blue row in $\mathbf{A}$ is ``dotted'' with
the light blue column in $\mathbf{B}$, resulting in the single, light blue element in $\mathbf{C}$.  Similarly, the dot
product of the light green vectors result in the single, light green element.  This process is repeated by letting
all rows in $\mathbf{A}$ be ``dotted'' with all the columns in $\mathbf{B}$.}
The order of multiplication is critical. Usually
\begin{equation}
\mathbf{A \cdot B \neq B \cdot A},
\end{equation}	 
and unless $\mathbf{A}$ and $\mathbf{B}$ are square matrices or the order of $\mathbf{A}^T$ is the same as the order of $\mathbf{B}$ (or vice 
versa), one of the two products cannot even be formed. The multiplication order is specified by stating

\begin{description}
\item [A] is \emph{pre}-multiplied by $\mathbf{B}$ (yielding $\mathbf{B \cdot  A}$)
\index{Matrix!premultiply}
\item [A] is \emph{post}-multiplied by $\mathbf{B}$ (yielding $\mathbf{ A \cdot B}$)
\index{Matrix!postmultiply}
\end{description}

The order in which the pairs are multiplied is not important \emph{mathematically}, i.e.,
\begin{equation}
\mathbf{ D = (A \cdot B) \cdot C = A \cdot (B \cdot C)},
\end{equation}
but we will see later that the order matter \emph{computationally}.
The \emph{transpose of a matrix product} is simply the multiplication of the transpose of each individual matrix in 
reverse order, i.e.,
\begin{equation}
\mathbf{D = A \cdot B \cdot C},
\end{equation}
\begin{equation}
\mathbf{D}^T = \mathbf{C}^T \cdot \mathbf{B}^T \cdot \mathbf{A}^T.
\label{eq:transposerule}
\end{equation}
\emph{Multiplication by $\mathbf{I}$} leaves the matrix unchanged, i.e.,
\begin{equation}
\mathbf{A \cdot I = I \cdot A = A}.
\end{equation}
For example, 
\begin{equation}
\left[ \begin{array}{ccc}
3 & 6 & 9 \\
2 & 8 & 7 
\end{array} \right]
\left[ \begin{array}{ccc}
1 & 0 & 0\\
0 & 1 & 0\\
0 & 0 & 1 \end{array}
\right]
 = 
\left[
\begin{array}{ccc}
3 & 6 & 9 \\
2 & 8 & 7
\end{array}
\right] .
\end{equation}
\emph{Premultiplication by a diagonal matrix} is written
$\mathbf{C = D \cdot A}$,
where $\mathbf{D}$  is  a diagonal matrix. Here, $\mathbf{C}$  is the $\mathbf{A}$ matrix with each \emph{row} scaled by the corresponding diagonal 
element of $\mathbf{D}$:
\begin{equation}
\mathbf{D}   =  \left[ \begin{array}{ccc}
d_{11}\\
& d_{22} \\
& & d_{33}
\end{array} \right] ,    \quad  \mathbf{A} = \left[ \begin{array}{ccc}
a_{11} & a_{12} & a_{13}\\
a_{21} & a_{22} & a_{23}\\
a_{31} & a_{32} & a_{33}
\end{array} \right ]
\end{equation}
\begin{equation}
\mathbf{C = D \cdot A}  =  \left[ \begin{array}{ccc}
a_{11}d_{11} & a_{12}d_{11} & a_{13}d_{11}\\
a_{21}d_{22} & a_{22}d_{22} & a_{23}d_{22}\\
a_{31}d_{33} & a_{32}d_{33} & a_{33}d_{33}
\end{array} \right]
\quad 
\begin{array}{l}
\leftarrow \mbox{ each element } \times d_{11}\\
\leftarrow \mbox{ each element } \times d_{22}\\
\leftarrow \mbox{ each element } \times d_{33}
\end{array} 
\end{equation}
\emph{Postmultiplication by a diagonal matrix} produces a matrix in which each \emph{column} has been 
scaled by the corresponding diagonal element $\mathbf{D}$.  Hence,
\begin{equation}
\mathbf{C = A \cdot D} = \left[\begin{array}{ccc} a_{11} d_{11} & a_{12} d_{22} & a_{13} d_{33}\\
a_{21} d_{11} & a_{22} d_{22} & a_{23} d_{33}\\
a_{31} d_{11} & a_{32}d_{22} & a_{33} d_{33}
\end{array} \right],
\end{equation}	 
where each column in $\mathbf{A}$ has been scaled by the corresponding diagonal matrix elements, $d_{ii}$.

\subsection{Computational considerations}
The matrix product
\begin{equation}
\mathbf{C}_{(k,n)} = \mathbf{A}_{(k,p)} \cdot \mathbf{B}_{(p,n)}
\end{equation}
involves $k \times n \times p$ multiplications and $k \times n  \times (p -1)$ additions.  hence,
\begin{equation}
\mathbf{E}_{(k,n)} = [ \mathbf{A}_{(k,p)} \cdot \mathbf{B}_{(p,q)}] \cdot \mathbf{C}_{(q,n)}
\end{equation}	 
gives $k \times p \times q$ multiplications, so
\begin{equation}
\mathbf{E} = [\mathbf{D}_{(k,q)}] \cdot \mathbf{C}_{(q,n)}
\end{equation}
gives $k\times q \times n$ multiplications, and
\begin{equation}
\mathbf{E} _{(k,n)} = \mathbf{A}_{(k,p)} \cdot [\mathbf{B}_{(p,q)} \cdot \mathbf{C}_{(q,n)}]
\end{equation}
gives $p\times q \times n$ multiplications, while
\begin{equation}
\mathbf{E}_{(k,n)} = \mathbf{A}_{(k,p)}\cdot [\mathbf{D}_{(p,n)} ]
\end{equation}	
gives $k \times p \times n$ multiplications. Therefore, the total number of operations depend on the order of multiplications:
\begin{enumerate}
\item $\mathbf{(A \cdot B) \cdot C} \Rightarrow kq(p+n)$ total multiplications
\item $\mathbf{A \cdot (B \cdot C)} \Rightarrow pn(k+q)$ total multiplications
\end{enumerate}
If both $\mathbf{A}$ and $\mathbf{B}$ are $100 \times 100$ matrices and $\mathbf{C}$ is $100 \times 1$, then $k = p = q = 100$, and $n 
= 1$. Multiplying using form (1) involves $\sim 1 \times 10^6$ multiplications, whereas form (2) involves $2 \times 
10^4$; so computing $\mathbf{B \cdot C}$ first, then premultiplying by $\mathbf{A}$ saves almost a million multiplications 
and almost an equal number of additions. Therefore, the order of operations is extremely important 
computationally for both speed and accuracy, as more operations lead to a greater accumulation of 
\emph{round-off errors}.
\index{Matrix!multiplication|)}

\section{Matrix Determinant}
\index{Matrix!determinant|(}
The \emph{determinant of a matrix} is a single number representing a property of a square matrix 
(and is dependent upon what the matrix represents). The main use here is for finding the inverse of a 
matrix or for solving simultaneous linear equations. Symbolically, the determinant is usually written as
det $(\mathbf{A})$, $\mathbf{|A|}$ or $\mathbf{||A||}$ (to differentiate from magnitude).
The calculation of a $2 \times 2$ determinant is carried out using the definition
\begin{equation}
|\mathbf{A}| = \left| \begin{array}{cc} a_{11} & a_{12} \\
a_{21} & a_{22}
\end{array} \right| = a_{11}a_{22} - a_{12}a_{21},
\end{equation}	 
which is the difference of the cross products. The calculation of an $n \times n$ determinant is given by
\begin{equation}
|\mathbf{A}| = a_{11} m_{11} - a_{12} m_{12} + a_{13} m_{13} - \cdots - (-1)^na_{1n} m_{1n},
\end{equation}	 
where $m_{11}$ is the determinant of $\mathbf{A}$ with the first row and column missing; $m_{12}$ is the determinant with 
the first row and second column missing, etc. For larger matrices the procedure is used recursively.
The determinant of a $1 \times 1$ matrix is just the 
particular element. An example of a $3 \times 3$ determinant follows:
\begin{equation}
|\mathbf{A} | = \left | \begin{array}{ccc}
a_{11} & a_{12} & a_{13}\\
a_{21} & a_{22} & a_{23}\\
a_{31} & a_{32} & a_{33}
\end{array}\right |
\end{equation}
\begin{equation}
m_{11} = \left| \begin{array}{ccc}
a_{11} & a_{12} & a_{13}\\
a_{21} & a_{22} & a_{23}\\
a_{31} & a_{32} & a_{33}
\end{array} \right |
= a_{22}a_{33} - a_{23}a_{32}
\end{equation}
\begin{equation}
m_{12} = \left| \begin{array}{ccc}
a_{11} & a_{12} & a_{13}\\
a_{21} & a_{22} & a_{23}\\
a_{31} & a_{32} & a_{33}
\end{array} \right |
= a_{21}a_{33} - a_{23}a_{31}
\end{equation}

\begin{equation}
m_{13} = \left| \begin{array}{ccc}
a_{11} & a_{12} & a_{13}\\
a_{21} & a_{22} & a_{23}\\
a_{31} & a_{32} & a_{33}
\end{array} \right |
= a_{21}a_{32} - a_{22}a_{31}
\end{equation}
So
\begin{equation}
\begin{array}{c}
|\mathbf{A}| = a_{11}m_{11} - a_{12}m_{12} + a_{13}m_{13}	 \\
= a_{11}(a_{22}a_{33} - a_{23}a_{32}) - a_{12}(a_{21}a_{33} - a_{23}a_{31}) + a_{13}(a_{21}a_{32} - a_{22}a_{31}).
\end{array}
\end{equation}
For a $4 \times 4$ determinant, each $m_{1i}$ would be an entire expansion given above for the 
$3 \times 3$ 
determinant --- one quickly needs a computer.

\index{Matrix!singular}
\index{Singular matrix}
A \emph{singular matrix} is a square matrix whose determinant is zero. A determinant is zero if:
\begin{enumerate}
\item Any row or column is zero.
\item Any row or column is equal to a linear combination of other rows or columns.
\end{enumerate}
As an example of a singular matrix, consider
\begin{equation}
|{\mathbf A}| = \left | \begin{array}{ccc} 1 & 6 & 4 \\
2 & 1 & 0 \\
5 & -3 & -4 
\end{array} \right |,
\end{equation}
where row $1 = 3\cdot$(row 2) $-$ row 3.  Then, the determinant becomes
\begin{equation}
\begin{array}{ccl}
|\mathbf{A}| & = & a_{11}(a_{22}a_{33}-a_{23}a_{32})-a_{12}(a_{21}a_{33}-a_{23}a_{31})+a_{13}(a_{21}a_{32}-a_{22}a_{31})     \\
& = & 1[1(-4)-0(-3)]-6[2(-4)-0(5)]+4[2(-3)-1(5)]=-4+48-44=0.  \\
\end{array}
\end{equation}

\index{Matrix!degree of clustering}
\index{Matrix!rank}
The \emph{degree of clustering} symmetrically about the principal diagonal is another (of many) 
properties of a determinant. The more the clustering, the higher the value of the determinant.
The \emph{rank} of a matrix is the number of linearly 
independent vectors that it contains (either row or column vectors).  Consider
\begin{equation}
\mathbf{A} = \left [ \begin{array}{cccc}
1 & 4 & 0 & 2 \\1 & 0 & 1 & -1\\
-3 & -4 & -2 & 0
\end{array} \right] .
\end{equation}
Since row $3 = -$(row 1)$ - 2\cdot$(row 2), or col $3 = $ col $1 - 1/4\cdot$(col 2) and col 4 $= -$(col 1)$ + 3/4\cdot$(col 2), 
the matrix $\mathbf{A}$ has rank 2 (i.e., it has only two linearly independent vectors, independent of whether 
viewed by rows or columns).

\index{Matrix!rank}
The \emph{rank} of a \emph{matrix product} must be less than or equal to the 
smallest rank of the matrices being multiplied:
\begin{equation}
\mathbf{A}_{\mbox{(rank 2)}}\cdot \mathbf{B}_{\mbox{(rank 1)}} = \mathbf{C}_{\mbox{(rank 1)}}.
\end{equation}	 
Therefore (and seen from another angle), if a matrix has rank $r$ then any matrix factor of it must have 
rank of at least $r$. Since the rank cannot be greater than the smallest of $k$ or $n$ in a $k \times n$ matrix, 
this definition also limits the size (order) of factor matrices. (That is, one cannot factor a matrix 
of rank 2, into two matrices of which either is of less than rank 2, so $k$ and $n$ of each factor must 
also be $\geq 2$).

\index{Matrix!determinant|)}

\section{Matrix Division (Matrix Inverse)}
\index{Matrix!division}
\index{Matrix!inverse}
%Was {Fig1_matrixalien}{Abducted by an alien circus company, Professor Wessel
%is forced to write Linear Algebra equations in center ring.}
\emph{Matrix division} can be thought of as multiplying by the \emph{inverse}.  Consider the scalar division
\begin{equation}
\frac{x}{b}=x\frac{1}{b}=xb^{-1},
\end{equation}	 
where we can write
\begin{equation}
bb^{-1}=1.
\end{equation}
Likewise, matrices can be effectively divided by multiplying by an inverse matrix. \emph{Nonsingular square 
matrices} may have an inverse symbolized as $\mathbf{A}^{-1}$ and satisfying $\mathbf{AA}^{-1} = \mathbf{A}^{-1}\mathbf{A} = \mathbf{I}$.
The calculation of a matrix inverse is usually done using elimination methods on the computer.
For a simple 2 x 2 matrix, its inverse is given by
\begin{equation}
\mathbf{A}^{-1} = \frac{1}{|\mathbf{A}|}   
\left [ \begin{array}{cc} a_{22} & -a_{12} \\
-a_{21} & a_{11} \\
\end{array}
\right ] .
\end{equation}
	 As an example, let
\begin{equation}
\mathbf{A}=
\left [\begin{array}{cc} 7 & 2\\
10 & 3\\
\end{array}
\right].
\end{equation}
We solve for the inverse as
\begin{equation}
\mathbf{A}^{-1}=\frac{1}{21-20}
\left [\begin{array}{cc}3 & -2\\
 -10 & 7\\
\end{array}
\right]=
\left [\begin{array}{cc}3 & -2\\
-10 & 7\\
\end{array}
\right],
\end{equation}
and as a check we note that
\begin{equation}
\mathbf{AA}^{-1}=
\left [\begin{array}{cc} 7 & 2\\ 10 & 3\\ 
\end{array} \right] \left [\begin{array}{cc}3 & -2\\ -10 & 7\\ \end{array} \right]=
\left [\begin{array}{ccr} 1 & 0\\ 0 & 1\\ \end{array} \right]=\mathbf{I}.
\end{equation}
Given the concept of a matrix inverse we may summarize a few useful matrix properties:
\begin{equation}
(\mathbf{A}^{-1})^{-1} = \mathbf{A},
\end{equation}
\begin{equation}
(\mathbf{A}^{-1})^T = (\mathbf{A}^T)^{-1} = \mathbf{A}^{-T},
\end{equation}
\begin{equation}
\mathbf{D} = \mathbf{ABC} \mbox{ then } \mathbf{D}^{-1} = \mathbf{C}^{-1} \mathbf{B}^{-1} \mathbf{A}^{-1}.
\end{equation}
	This ``reversal rule'' for inverse products may be useful for eliminating or minimizing the number 
of matrix inverses requiring calculation.

\section{Matrix Manipulation and Normal Scores}
\index{Normal scores|(}
We will look at a few examples of matrix manipulations. Consider the data matrix
\begin{equation}
\mathbf{A} = \left[ \begin{array}{c}
1 \ 2 \ 3 \\
4 \ 5 \ 6 \\7 \ 8 \ 9
\end{array} \right ]
\end{equation}
and unit row vector
\begin{equation}
\mathbf{j}^T_3 = [1 \ 1 \ 1].
\end{equation} 
To compute the mean of each column vector in $\mathbf{A}$ (here, each column has length $n = 3$), we note that
\begin{equation}
\mathbf{\bar{x}} _c = \frac{1}{3} \mathbf{j}^T_3 \mathbf{A}, \mbox{ and in general } \mathbf{\bar{x}} _c = \frac{1}{n} \mathbf{j}^T_n \mathbf{A}.
\end{equation}	
For our example, we find
\begin{equation}
\mathbf{\bar{x}}_c = \frac{1}{3} \left[ 1 \  1\  1 \right ] \cdot
\left[ \begin{array}{c}
1 \ 2 \ 3 \\ 4 \ 5 \ 6 \\ 7 \ 8 \ 9 \end{array} \right ] 
= \frac{1}{3} \left[ 12 \ 15 \ 18 \right] =  \left[ 4 \ 5 \ 6 \right].
\end{equation}
To compute the mean of each row vector in $\mathbf{A}$ (here, each row has length $k = 3$), let
\begin{equation}
\mathbf{\bar{x}}_r = \frac{1}{3} \mathbf{Aj}_3, \mbox{ and in general }  \mathbf{\bar{x}}_r = \frac{1}{k} \mathbf{Aj}_k .
\end{equation} 
Again, for our example, we find
\begin{equation}
\mathbf{\bar{x}}_r = \frac{1}{3}
\left [ \begin{array}{c}
1 \ 2 \ 3 \\
4 \ 5 \ 6 \\
7 \ 8 \ 9 
\end{array} \right ] \cdot 
\left [ \begin{array}{c}
1\\
1\\
1
\end{array} \right ] = \frac{1}{3}
\left[
\begin{array}{c}
6 \\
15\\
24 \end{array} \right ] 
=
\left[
\begin{array}{c}
2 \\
5\\
8 \end{array} \right ].
\end{equation}
Given these terms, how can we compute normal scores for a data table (or matrix)?  What we want in 
each cell are the elements
\begin{equation}
z_{ij} = \frac{a_{ij} - \bar{a}_j}{s_j}.
\end{equation} 
In matrix terminology we would first need to form the difference matrix, $\mathbf{D}$, given as
\begin{equation}
\mathbf{D = A} - \frac{1}{n}\mathbf{JA},
\end{equation}
where $\mathbf{J}$ is the $n \times n$ unit matrix (all entries equal 1).
Given the diagonal matrix $\mathbf{S}$ containing the standard deviation of each column defined as
\begin{equation}
\mathbf{S} = 
\left [ \begin{array}{cccc}
s_1 & 0 & \ldots & 0 \\
0 & s_2 & \ldots & 0 \\
\vdots & \vdots &  \ddots  & \vdots\\
0 & 0 & \ldots & s_n
\end{array} \right]
\end{equation}
we get the normal scores (here, $\mathbf{I}$ is of size $n \times n$) as
\begin{equation}
\mathbf{Z = DS}^{-1} = \left ( \mathbf{A} - \frac{1}{n} \mathbf{JA} \right) \mathbf{S}^{-1} = \left ( \mathbf{I} - \frac{1}{n} \mathbf{J} \right)
\mathbf{AS}^{-1}.
\end{equation}
\index{Normal scores|)}

\section{Solution of Simultaneous Linear Equations}
\index{Solution of simultaneous linear equations|(}

A system of four simultaneous linear equations in four unknowns $x_1, x_2, x_3, x_4$ can be written
\begin{equation}
\begin{array}{c}
a_{11}x_1+a_{12}x_2+a_{13}x_3+a_{14}x_4=b_1\\
a_{21}x_1+a_{22}x_2+a_{23}x_3+a_{24}x_4=b_2\\
a_{31}x_1+a_{32}x_2+a_{33}x_3+a_{34}x_4=b_3\\
a_{41}x_1+a_{42}x_2+a_{43}x_3+a_{44}x_4=b_4\\
\end{array}
\end{equation}
or, in matrix form, 
\begin{equation}
\mathbf {Ax=b},
\end{equation}	 
where
\begin{equation}
\mathbf{A}=
\left[\begin{array}{cccc}
a_{11} & a_{12} & a_{13} & a_{14}\\	 
a_{21} & a_{22} & a_{23} & a_{24}\\
a_{31} & a_{32} & a_{33} & a_{34}\\
a_{41} & a_{42} & a_{43} & a_{44}\\	 
\end{array}
\right]
\end{equation}
is called the coefficient matrix,
\begin{equation}
\mathbf{x}= \left[\begin{array}{c}x_1\\x_2\\x_3\\x_4\\
\end{array}\right]
\end{equation}
is the unknown vector, and
\begin{equation}
\mathbf{b}=\left[\begin{array}{c}b_1\\b_2\\b_3\\b_4\\
\end{array}
\right]
\end{equation}
is the right hand side (i.e., the observations).  Premultiplying both sides by $\mathbf{A}^{-1}$ yields
\begin{equation}
\mathbf{A}^{-1}\mathbf{Ax}=\mathbf{A}^{-1}\mathbf{b},
\end{equation}	 
hence
\begin{equation}
\mathbf{Ix=x=A}^{-1}\mathbf{b}
\end{equation}	 
gives the solution for values of $x_1, x_2, x_3, x_4$ which solve the system. For simplicity, the following example 
solves for two simultaneous equations only. Consider two equations in two unknowns (e.g., equations of lines 
in the $x_1-x_2$ plane):
\begin{equation}
\begin{array}{c}
5x_1 + 7x_{2} = 19\\
3x_1 - 2 x_2 = -1
\end{array} .
\end{equation}
In matrix form this system translates to
\begin{equation}
\left[ \begin{array}{cc}
5 & 7\\
3 & -2 
\end{array}
\right ] 
\left[
\begin{array}{c}
x_1\\
x_2
\end{array}
\right ] =
\left[ \begin{array}{c}
19\\
-1
\end{array}
\right ]
\end{equation}
or
\begin{equation}
\mathbf{A \cdot x = b}.
\end{equation}
To solve this matrix equation we need the inverse of $\mathbf{A}$, which is simply
\begin{equation}
\mathbf{A}^{-1} = \frac{1}{-10 - 21} \quad
\left[ \begin{array}{cc}
-2 & -7\\
-3 & 5
\end{array}\right ] = \left[ \begin{array}{cc}
\frac{2}{31} & \frac{7}{31}\\[4pt]
\frac{3}{31} & \frac{-5}{31}
\end{array} \right ] .
\end{equation}	 
Then, $\mathbf{x = A}^{-1}\cdot \mathbf{b}$, where
\begin{equation}
\mathbf{x = A}^{-1}\mathbf{b} = \left[ \begin{array}{cc}
\frac{2}{31} & \frac{7}{31}\\[4pt]
\frac{3}{31} & \frac{-5}{31}
\end{array}
\right] 
\left[ \begin{array}{c}
19\\
-1
\end{array} \right ] = \left [ \begin{array}{c}
\frac{38}{31} - \frac{7}{31} \\[4pt]
\frac{57}{31} + \frac{5}{31}
\end{array} \right ]
= \left[ \begin{array}{c}
1 \\
2 \end{array} \right ] .
\end{equation}
So, the values $x_1 = 1$ and $x_2 = 2$ solve the above system, or
\begin{equation}
\mathbf{x} = \left[ \begin{array}{c}
x_1\\ x_2 \end{array} \right]
=
\left[ \begin{array}{c}
1\\ 2 \end{array} 
\right ] .
\end{equation}	 

While this approach may seem burdensome, it is good because it is extremely general and 
allows for a straightforward solution to very large systems. However, it is 
true that direct (elimination) methods to the solution are in fact quicker for fully populated 
matrices:

\begin{enumerate}
\item 	A solution using the inverse matrix approach involves $n^3$ multiplications for the inversion and $n^2k$ 
more multiplications to finish the solution, where $n$ is the number of equations per set, and $k$ 
is the number of sets of equations (each of the same form but different $\mathbf{b}$ vector). The total 
number of multiplications is $n^3 + n^2k$.
\item	A solution by directly solving the linear equations involves $n^3/3 + n^2k$ multiplications.
\end{enumerate}

Hence, while the matrix form is easy to handle, one should not necessarily always use it blindly. We 
will consider many situations for which matrix solutions are ideal. For sparse or symmetrical 
matrices, the above relationships may not hold.
\index{Solution of simultaneous linear equations|)}

\subsection{Simple regression and curve fitting}
\index{Simple regression|(}
\index{Regression!simple|(}
\index{Curve fitting|(}

\PSfig[h]{Fig1_L2_error}{Graphical representation of the regression errors used in least-squares procedures.
We measure misfit vertically in the $y$-direction from data point to regression curve.}
	Whereas an interpolant fits each data point exactly, it is frequently advantageous to produce a 
smoothed fit to the data --- not exactly fitting each point, but producing a ``best'' fit. A popular (and 
convenient) method for producing such fits is known as the \emph{method of least squares}.
\index{Method of least squares}
\index{Least squares method}

	The method of least squares produces a fit of a specified (usually continuous) basis to a set of 
data points which minimizes the sum of the squared misfit (error) between the fitted curve 
and the data. The misfit can be measured vertically, as in Figure~\ref{fig:Fig1_L2_error}.
\index{Regression}
This \emph{regression} of $y$ on $x$ is the most commonly used method. Less common methods (i.e., more work 
involved) is the regression of $x$ on $y$ and even orthogonal regression (which we will return to later;
see Figure~\ref{fig:Fig1_y_and_ortho_error}).

\PSfig[h]{Fig1_y_and_ortho_error}{Two other regression methods: regressing $x$ on $y$ and orthogonal regression.
Here we measure misfits horizontally from data point to regression line or orthogonally onto the regression line,
respectively.}

	Consider fitting a single ``best'' linear curve to $n$ data points. This can be a scatter plot of $x(t)$, $d(t)$ 
plotted at similar values of $t$, or a simple $d = f(x)$ relationship. At any rate, $d$ (our data) are considered a 
function of $x$ (which may be a spatial coordinate or time). We wish to fit a line of the form
\begin{equation}
d(x) = m_1 + m_2 (x-x_0)
\end{equation}
and must therefore determine values for the model coefficients $m_1$ and $m_2$ that produce a line that minimizes the sum 
of the squared misfits (here, $x_0$ is a constant specified beforehand). In other words,
\begin{equation}
\mbox{minimize } \sum ^n _{i=1} \left [(d_{\mbox{computed}}(x_i) - d_{\mbox{observed}}(x_i) \right ]^2.
\end{equation}	 
Ideally, for each observation $d_i$ at location $x_i$ we should have
\begin{equation}
\begin{array}{c}
m_1 + m_2(x_1 - x_0) = d_1\\
m_1 + m_2(x_2 - x_0) = d_2\\
m_1 + m_2(x_3 - x_0) = d_3\\
\vdots\\
m_1 + m_2 (x_n - x_0) = d_n
\end{array}
\end{equation}
There are many more equations ($n$ --- one for each observed value of $d$) than unknowns (2 --- $m_1$ and 
$m_2$). Such a system is \emph{overdetermined} and there exists no unique solution (unless all the $d_i$'s do 
lie exactly on a single line, in which case any two equations will uniquely determine $m_1$ and $m_2$).  
In matrix form,
\index{Overdetermined system of equations}
\begin{equation}
\left[
\begin{array}{cc}
1 & (x_1 - x_0) \\
1 & (x_2 - x_0) \\
\vdots & \vdots \\
1 & (x_n - x_0) 
\end{array} \right]
\ \left [ \begin{array}{c}
m_1\\
m_2
\end{array} \right ] =
\left [ \begin{array}{c}
d_1\\
d_2\\
\vdots\\
d_n
\end{array} \right ],
\end{equation}
i.e., $\mathbf{G \cdot m = d}$.  Here, $\mathbf{G}$ represents how predictions of $\mathbf{d}$ are related to the model $\mathbf{m}$ and is often
called the \emph{design matrix}.
However, since $\mathbf{G}$ is a not square it has no inverse, hence the equation cannot be inverted and solved as is. 
Consider instead the \emph{misfit}, $e_i$, at each point, between prediction and observation:
\begin{equation}
\begin{array}{c}
m_1 + m_2 (x_1 - x_0) - d_1 = e_1\\
m_1 + m_2 (x_2 - x_0) - d_2 = e_2\\
\vdots\\
m_1 + m_2(x_n - x_0) - d_n = e_n
\end{array}
\end{equation}
We wish to determine the values for $m_1$ and $m_2$ that minimize
\begin{equation}
	\index{Misfit function}
E(m_1, m_2) = \sum^n_{i=1} e^2_i = {\mathbf e}^T{\mathbf e},
\end{equation}
where $\mathbf{e}^T = (e_1, e_2, ..., e_n)$ is the \emph{misfit vector}.       
This condition will minimize the variance of the residuals about the regression line and give the desired least-squares fit.
Thus, $E(m_1 ,m_2)$ and the minimum of this function (with respect to the two unknown coefficients) 
can be determined using simple differential calculus where, at the desired minimum, we require
\PSfig[h]{Fig1_3D_misfit}{(left) The solution we seek minimizes the misfit function $E(\mathbf{m}) = E(m_1, m_2)$,
which portrays a surface in 3-D.  Because of the functional (quadratic) form of $E$ we are guaranteed a unique global
minimum. (right) Two orthogonal cross-sections of $E$ along the axes $m_1$ and $m_2$.  We seek the solutions for these
two parameters so that the respective slopes in $E$ are zero simultaneously.}
\begin{equation}
\frac{\partial E(m_1,m_2)}{\partial m_1} = \frac{\partial E (m_1, m_2)}{\partial m_2} = 0.
\end{equation}
Thus, the \emph{slopes} of the misfit function with respect to each parameter must be zero (see Figure~\ref{fig:Fig1_3D_misfit}).  We find
\begin{equation}
\begin{array}{rcl}
\displaystyle
\frac{\partial E}{\partial m_1} & = & \displaystyle \frac{\partial}{\partial m_1} \left ( \sum^n_{i=1} \ e^2_i \right ) = \frac{\partial}{\partial m_1} \left \{ \sum^n_{i=1} \left [ m_1 + m_2 (x_i - x_0) - d_i \right ] ^2 \right \}\\*[4ex]
 &  = & \displaystyle 2 \sum^n_{i=1} \left [ m_1 + m_2 (x_i - x_0) - d_i \right ] = 0
\end{array}
\end{equation}
\begin{equation}
\begin{array}{rcl}
\displaystyle
\frac{\partial E}{\partial m_2} & = & \displaystyle \frac{\partial}{\partial m_2} \left ( \sum^n_{i=1} \ e^2_i \right ) = \frac{\partial}{\partial m_2} \left \{ \sum^n_{i=1} \left [ m_1 + m_2 (x_i - x_0) - d_i \right ] ^2 \right \}\\*[4ex]
 & = & \displaystyle 2 \sum^n_{i=1} \left [ m_1 + m_2 (x_i - x_0) - d_i \right ](x_i - x_0) = 0.
\end{array}
\end{equation}
These two equations can now be expanded into their individual terms, forming what are known as 
the \emph{normal equations}.  This system of two equations with two unknowns can be uniquely 
solved. Rearranging, we find
\index{Normal equations}
\begin{equation}
nm_1 + m_2 \sum^n_{i=1} (x_i - x_0) = \sum^n_{i=1} d _i,
\label{eq:normeq1}
\end{equation}
\begin{equation}
m_1 \sum^n_{i=1} (x_i - x_0) + m_2 \sum^n_{i=1} (x_i - x_0)^2 = \sum^n_{i=1} d_i (x_i - x_0).
\label{eq:normeq2}
\end{equation}
Notice that all sums involve known values that add to simple constants. 
Specifically, we must compute the sums
\begin{equation}
S_y = \sum^n_{i=1} d_i, \ S_{xy} = \sum^n_{i=1}d_i (x_i - x_0), S_x = \sum^n_{i=1}(x_i - x_0), \mbox{ and} \ 
S_{xx} = \sum^n_{i=1}(x_i - x_0)^2.
\end{equation}
Substituting these symbols into (\ref{eq:normeq1}) and (\ref{eq:normeq2}), we obtain
\begin{equation}
nm_1 + m_2 S_x = S_y
\end{equation}
\begin{equation}
m_1 S_x + m_2 S_{xx} = S_{xy}
\label{eq:slope_equation}
\end{equation}
Solving for the intercept yields
\begin{equation}
m_1 = \frac{1}{n} S_y - \frac{m_2}{n} S_x.
\label{eq:intercept_equation}
\end{equation}
We substitute $m_1$ into (\ref{eq:slope_equation}) and find
\begin{equation}
\left [ \frac{1}{n} S_y - \frac{m_2}{n} S_x \right ] S_x + m_2 S_{xx} = S_{xy}.
\end{equation}
Now solve for $m_2$:
\begin{equation}
\frac{1}{n} S_y S_x - \frac{m_2}{n} S_x^2 + m_2 S_{xx} = S_{xy},
\end{equation}
\begin{equation}
m_2 \left ( S_{xx} - \frac{1}{n} S_x^2 \right ) = S_{xy} - \frac{1}{n} S_y S_x.
\end{equation}	 
Finally,
\begin{equation}
m_2 = \left ( S_{xy} - \frac{1}{n} S_y  S_x \right ) / \left ( S_{xx} - \frac{1}{n} S_x^2 \right )  =\frac{n S_{xy} - S_x  S_y}{n S_{xx} - S_x^2},
\label{eq:slope_solution}
\end{equation}	 
and we substitute $m_2$ into (\ref{eq:intercept_equation}) to find
\begin{equation}
m_1 = \frac{S_{xx} S_y - S_x  S_{xy}}{n S_{xx} - S_x^2}.
\label{eq:intercept_solution}
\end{equation}	 
In matrix form the normal equations are
\begin{equation}
\left [
\begin{array}{cc}
n & \displaystyle \sum^n_{i=1}(x_i - x_0)\\
\displaystyle \sum^n_{i=1}(x_i - x_0) & \displaystyle \sum^n_{i=1}(x_i - x_0)^2
\end{array} \right ]
\left [
\begin{array}{c}
m_1\\ m_2
\end{array} \right ] =
\left[
\begin{array}{c}
\displaystyle \sum^n_{i=1}d_i\\
\displaystyle \sum^n_{i=1}d_i(x_i - x_0)
\end{array}
\right ],
\end{equation}
which may be simplified to
\begin{equation}
\left [
\begin{array}{cc}
n & S_x \\
S_x & S_{xx}
\end{array} \right ]
\left [
\begin{array}{c}
m_1\\ m_2
\end{array} \right ] =
\left[
\begin{array}{c}
S_y\\
S_{xy}
\end{array}
\right ] .
\end{equation}
Therefore, $\mathbf{Nm = v}$, and since $\mathbf{N}$ is square, symmetric and of full rank, this equation is solved in 
the standard manner:
\begin{equation}
\mathbf{N}^{-1} \mathbf{Nm = m = N}^{-1} \mathbf{v}.
\end{equation}	 
This problem was simple enough $(2 \times 2)$ to solve for $m_1$ and $m_2$ by brute force. For larger systems,
this approach becomes impractical and instead a matrix solution to the rectangular $\mathbf{G\cdot m = d}$ equation must be 
sought. We will next look at the general linear least-squares problem and find a solution in 
matrix notation.
\index{Simple regression|)}
\index{Regression!simple|)}
\index{Curve fitting|)}

\subsection{General linear least squares method, version 1}
\index{General linear least squares method|(}

	We have looked at a few special cases where we have sought to fit a model to data in a 
least-squares sense.  Fitting a straight line to the $x-d$ points was a very simple example of this 
technique. We will now look at the more general problem of finding the coefficients for \emph{any} 
linear combination of a chosen set of basis functions that fits a data set in a least squares sense. There are 
numerous situations where this is needed; some are listed in Table~\ref{tbl:LLS_examples}.

\begin{table}[h]
\centering
\begin{tabular}{|l|p{1.5in}|l|}
\hline
\bf{Situation}  & \multicolumn{1}{c|}{\bf{Model Parameters}} & \multicolumn{1}{c|}{\bf{Data}} \\ \hline
Curve Fitting & 
Coefficients of polynomials, Fourier series, etc. &
Points in $x-y$ plane \\ \hline
Gravity modeling &
Densities of subsurface polygons & 
Gravity observations\\ \hline
Hypocenter location & 
Small perturbations to hypocenter location & Seismic arrival times \\ \hline
\end{tabular}
\caption{Examples of situations where linear least squares solutions are used.}
\label{tbl:LLS_examples}
\end{table}

While the basis functions in these cases are all vastly different, they are all used in linear 
combinations to fit the observed data. We will therefore take time to investigate how such a 
problem is set up, and how the setup can be simplified with matrix algebra.
Some typical basis functions are given in Table~\ref{tbl:basis_funcs}.

\begin{table}[h]
\centering
\begin{tabular}{|c|c|}
\hline
\bf{Polynomial basis} & \bf{Fourier sine basis}\\ \hline
$g_1=x^0$  & $g_1=\sin(2\pi x/T)$\\ \hline
$g_2=x^1$  & $g_2=\sin(4\pi x/T)$\\ \hline
$g_3=x^2$  & $g_3=\sin(6\pi x/T)$\\ \hline
$\vdots$ & $\vdots$\\ \hline
$g_k=x^{k-1}$  & $g_k=\sin(2k\pi x/T)$ \\ \hline
\end{tabular}
\caption{Examples of basis functions used for modeling of data.}
\label{tbl:basis_funcs}
\end{table}

Consider the least squares fitting of any continuous basis of the form
\begin{equation}
g_1(x), g_2(x), g_3(x), \cdots , g_k(x).
\end{equation}
For example, we desire to fit a model with $k$ terms
\begin{equation}
d(x) = m_1g_1(x) + m_2g_2(x) + \cdots + m_k g_k(x)
\label{eq:lsgenmodel}
\end{equation}
to a data set of $n$  data points, where $n > k$, by minimizing $E(\mathbf{m})$ given by
\begin{equation}
E(\mathbf{m}) = E(m_1, m_2, \cdots, m_k) = \sum ^n _{i=1}  (e_i)^2  = \sum ^n _{i=1} (m_1 g_1(x_i) + m_2g_2(x_i) + \cdots + m_k g_k(x_i) - d_i)^2,
\end{equation}
or simply
\begin{equation}
E(\mathbf{m}) = \sum ^n _{i=1} (m_1g_{i1} + m_2g_{i2} + \cdots + m_k g_{ik} - d_i)^2,
\label{eq:L2_misfit}
\end{equation}
where $d_i$ is the observed value and $g_{ij}$ is the $j'$th basis function, evaluated at the location (or time) $x_i$. In other words,
$g_{ij} = g_j(x_i)$.

There are \emph{four} concepts of vital importance in a general linear least squares modeling problem:
\begin{enumerate}
	\item The \emph{observed data}, $(x_i, d_i), i = 1, n$, where $n$ is the number of observations.  These are all known quantities.
	\item The \emph{general linear model} (linear in $\mathbf{m} = m_j, j = 1,k$, with $k$ unknown parameters), given by (\ref{eq:lsgenmodel}).
	\item The $m$ chosen \emph{basis functions}, $g_j(x), j = 1, k$.  We can evaluate these for any $x$.
	\item The \emph{least squares misfit criteria}, given by (\ref{eq:L2_misfit}).
\end{enumerate}
We can write a linear system of equations for the misfit at each data point:
\begin{equation}
\begin{array}{c}
m_1 g_{11} + m_2 g_{12} + \cdots + m_k g_{1k} - d_1 = e_1\\
m_1 g_{21} + m_2 g_{22} + \cdots + m_k g_{2k} - d_2 = e_2\\
\vdots\\
m_1 g_{n1} + m_2 g_{n2} + \cdots + m_k g_{nk} - d_n = e_n\\
\end{array}.
\end{equation}     
To minimize $E$, we require
\begin{equation}
\displaystyle
\frac{\partial E(\mathbf{m})}{\partial m_{j}} = 0, \quad j = 1,k.
\label{eq:L2_criteria}
\end{equation}
Considering the first term (case $j = 1$), we see
\begin{equation}
\begin{array}{rcl}
\displaystyle \frac{\partial E(\mathbf{m})}{\partial m_1} & = & \displaystyle \frac{\partial}{\partial m_1}\sum ^n _{i=1}(m_1g_{i1} + m_2g_{i2} + \cdots m_k g_{ik} -d_i)^2 \\
 & = & \displaystyle 2 \sum ^n _{i=1}(m_1g_{i1} + m_2g_{i2} + \cdots m_k g_{ik} -d_i)g_{i1}= 0, \\
\end{array}
\end{equation}
while for the second term (case $j = 2$), we find
\begin{equation}
\begin{array}{rcl} 
\displaystyle \frac{\partial E(\mathbf{m})}{\partial m_2} & = & \displaystyle \frac{\partial}{\partial m_2}\sum ^n _{i=1}(m_1g_{i1} + m_2g_{i2} + \cdots m_k g_{ik} -d_i)^2 \\
& = & 2 \displaystyle \sum ^n _{i=1}(m_1g_{i1} + m_2g_{i2} + \cdots m_k g_{ik} -d_i)g_{i2}= 0. \\
\end{array}
\end{equation}
Consequently, for the $j$'th parameter,
\begin{equation}
\frac{\partial E(\mathbf{m})}{\partial m_j} = 2 \displaystyle \sum ^n _{i=1}(m_1g_{i1} + m_2g_{i2} + \cdots m_k g_{ik} -d_i)g_{ij}= 0.
\end{equation}
Rearranging these normal equations gives the square $k \times k$ system
\begin{equation}
\begin{array}{c}
m_1 \displaystyle \sum ^n _{i=1} g^2_{i1} + m_2 \displaystyle \sum ^n _{i=1} g_{i2} g_{i1} + \cdots + m_k 
 \displaystyle \sum ^n _{i=1} g_{ik}g_{i1} =  \displaystyle \sum^n _{i=1} d_i g_{i1}\\
m_1 \displaystyle \sum ^n _{i=1} g_{i1}g_{i2} + m_2 \displaystyle \sum ^n _{i=1} g^2_{i2} + \cdots + m_k  \displaystyle \sum ^n _{i=1} g_{ik}g_{i2} =  \displaystyle \sum^n _{i=1} d_i g_{i2}\\
\vdots \\
m_1 \displaystyle \sum ^n _{i=1} g_{i1}g_{ik} + m_2 \displaystyle \sum ^n _{i=1} g_{i2} g_{ik} + \cdots + m_k 
 \displaystyle \sum ^n _{i=1} g^2_{ik} =  \displaystyle \sum^n _{i=1} d_i g_{ik}
\end{array}
\end{equation}
or equivalently,
\begin{equation}
m_1 \displaystyle \sum ^n _{i=1} g_{i1}g_{ij} + m_2 \displaystyle \sum ^n _{i=1} g_{i2} g_{ij} + \cdots + m_k 
 \displaystyle \sum ^n _{i=1} g_{ik}g_{ij} =  \displaystyle \sum^n _{i=1} d_i g_{ij}, \quad j=1,k.
\end{equation}
This setup provides a \emph{closed system} of $k$ normal equations.  In matrix form,
\begin{equation}
\left [  \begin{array}{cccc}
\displaystyle \sum ^n _{i=1} g^2_{i1} & \displaystyle \sum ^n _{i=1} g_{i2}g_{i1} & \cdots &
\displaystyle \sum ^n _{i=1} g_{ik}g_{i1} \\
\displaystyle \sum ^n _{i=1} g_{i1}g_{i2} &  \displaystyle \sum ^n _{i=1} g^2_{i2} & \cdots &
\displaystyle \sum^n _{i=1} g_{ik}g_{i2} \\
\vdots & \vdots & \ddots & \vdots\\
\displaystyle \sum ^n _{i=1} g_{i1}g_{ik} &  \displaystyle \sum ^n _{i=1} g_{i2}g_{ik} & \cdots & \displaystyle \sum ^n _{i=1} g^2_{ik} 
\end{array}   \right ]	\left[ \begin{array}{c} m_1 \\ m_2 \\ \vdots \\ m_k     \end{array}  \right ]  = \left[  \begin{array}{c}\displaystyle \sum ^n _{i=1} d_{i}g_{i1}\\
\displaystyle \sum ^n _{i=1} d_{i}g_{i2}\\
\vdots \\
\displaystyle \sum ^n _{i=1} d_{i}g_{ik}\\
   \end{array} \right].
\label{eq:normalwsums}
\end{equation}
Hence, we simply have
\begin{equation}
\mathbf{N \cdot m = v},
\label{eq:Nxvsolution}
\end{equation}
where $\mathbf{N}$ is the (known) coefficient matrix, $\mathbf{m}$ the vector with the unknowns $m_j$, and $\mathbf{v}$ contains 
weighted sums of known (observed or computable) quantities. Solving for the $\mathbf{m}$ vector (since $\mathbf{N}$ is square, symmetric and 
of full rank) yields
\begin{equation}
\mathbf{N}^{-1} \cdot \mathbf{N \cdot m} = \mathbf{m} = \mathbf{N}^{-1} \cdot \mathbf{v}.
\end{equation}	 
The resulting $m_j$ values are the ones which satisfy (\ref{eq:L2_criteria}) and 
therefore the same ones, when combined with the chosen basis, that produce the ``best'' fit to the 
$n$ data points such that (\ref{eq:L2_misfit}) is minimized.

\subsection{General linear least squares method, version 2}

We will now look at a simpler approach to the same problem using matrix algebra. We have $\mathbf{e = G \cdot m - d}$, or
\begin{equation}
\left [ \begin{array}{c} e_1\\ e_2 \\ \vdots \\ e_n  \end{array} \right ] =
\left [
\begin{array}{cccc}
g_{11} & g_{12} & \cdots & g_{1k} \\	 
g_{21} & g_{22} & \cdots & g_{2k} \\
\vdots & \vdots & \ddots & \vdots\\
g_{n1} & g_{n2} & \cdots & g_{nk} \\
\end{array} \right ]
\cdot
\left [ \begin{array}{c} 
m_1\\ 
m_2\\
 \vdots\\ 
m_k   \end{array}   \right]
-
\left [ \begin{array}{c} d_1\\ d_2 \\ \vdots \\ d_n  \end{array} \right ].
\end{equation}
We note that each column vector of $\mathbf{G}$ is simply a single basis function evaluated at all our
observation points.  In fact, we could write $\mathbf{G}$ as
\begin{equation}
\mathbf{G} = \left [
\begin{array}{cccc}
\mathbf{g}_1 & \mathbf{g}_2 & \cdots & \mathbf{g}_k \\	 
\end{array} \right ],
\label{eq:Aasvectors}
\end{equation}
where
\begin{equation}
\mathbf{g}_j = \left [
\begin{array}{cccc}
g_j(x_1) & g_j(x_2) & \cdots & g_j(x_n) \\	 
\end{array} \right ]^T.
\end{equation}
We wish to find the $m_j$ values that minimize $E = \mathbf{e}^T\mathbf{e}$.
Minimizing the misfit with respect to the unknown model parameters $\mathbf{m}$ means we must solve
the $k$ linear equations that result from setting all partial derivatives of $E$ to zero (i.e., \ref{eq:L2_criteria}).
Using matrix algebra, we express the \emph{predicted} solution as $\hat{\mathbf{d}} = \mathbf{G \cdot m}$.
We may now express the misfit
between model and observations as $\mathbf{e} = \hat{\mathbf{d}} - \mathbf{d} = \mathbf{G \cdot m} - \mathbf{d}$
and use this expression to evaluate the misfit as
\begin{equation}
E(\mathbf{m}) = \sum ^n _{i=1}  (e_i)^2  = \mathbf{e}^T \cdot \mathbf{e} = \left(\hat{\mathbf{d}} - \mathbf{d}\right)^T\cdot \left(\hat{\mathbf{d}} - \mathbf{d}\right) = \left(\mathbf{G \cdot m} - \mathbf{d}\right)^T \cdot \left(\mathbf{G \cdot m} - \mathbf{d}\right).
\end{equation}
Expanding terms, we find
\begin{equation}
E(\mathbf{m}) =  \left(\mathbf{m}^T\mathbf{G}^T - \mathbf{d}^T\right) \cdot \left(\mathbf{G \cdot m} - \mathbf{d}\right) = \
\mathbf{m}^T\mathbf{G}^T\mathbf{Gm} - \mathbf{m}^T\mathbf{G}^T\mathbf{d} - \mathbf{d}^T\mathbf{Gm} + \mathbf{d}^T\mathbf{d},
\end{equation}
where we have used (\ref{eq:transposerule}) to handle the transpose of a matrix product.
Note that as $E$ is a scalar then each of these terms must evaluate to scalars as well.  To find the solution, we set
\begin{equation}
\frac{\partial E(\mathbf{m})}{\partial m_j} = \mathbf{\dot{m}}^T\mathbf{G}^T\mathbf{Gm} + \mathbf{m}^T\mathbf{G}^T\mathbf{G\dot{m}} - \mathbf{\dot{m}}^T\mathbf{G}^T\mathbf{d} - \mathbf{d}^T\mathbf{G\dot{m}} = 0, \quad j = 1,m,
\end{equation}
where the ``dot'' over a vector represents the derivative of that vector with respect to $m_j$.
We note the first and second terms are transposes of each other, as are the third and fourth terms.
However, since they all evaluate to scalars the two transposes must be identical and hence this
repetition simply constitutes a factor of two, which we delete by retaining only the first and third term:
\begin{equation}
\frac{\partial E(\mathbf{m})}{\partial m_j} = \mathbf{\dot{m}}^T\mathbf{G}^T\mathbf{Gm} - \mathbf{\dot{m}}^T\mathbf{G}^T\mathbf{d} = 0, \quad j = 1,k.
\end{equation}
What does the mysterious ``dot''-derivative, written as
\begin{equation}
\mathbf{\dot{m}}^T = \frac{\partial}{\partial m_j} \left (\mathbf{m}^T \right), j = 1,k,
\end{equation}
mean? We illuminate this term by trying some values of $j$, remembering $\mathbf{m}^T = [m_1\ m_2 \ \cdots \ m_k ]$:
\[\begin{array}{*{20}{c}}
{{\rm{Case }}j = 1:{\frac{\partial}{\partial m_1}\mathbf{m}} = {{\left[ {\begin{array}{*{20}{c}}
1&0& \cdots &0
\end{array}} \right]}^T}}\\[4pt]
{{\rm{Case }}j = 2:{\frac{\partial}{\partial m_2}\mathbf{m}} = {{\left[ {\begin{array}{*{20}{c}}
0&1& \cdots &0
\end{array}} \right]}^T}}\\[4pt]
 \vdots \\[4pt]
{{\rm{Case }}j = k:{\frac{\partial}{\partial m_k}\mathbf{m}} = {{\left[ {\begin{array}{*{20}{c}}
0&0& \cdots &1
\end{array}} \right]}^T}}
\end{array}\]
Thus, the $k$ linear equations can be combined into a single matrix equation, noting that all these derivatives (each producing a row vector)
combine to form the identity matrix, $\mathbf{I}$:
\begin{equation}
\frac{\partial }{{\partial {m_j}}}\left( {{{{\mathbf{m}}}^T}} \right), j = 1,k \to \left[ {\begin{array}{*{20}{c}}
1&0& \cdots &0\\
0&1& \cdots &0\\
 \vdots & \vdots & \ddots & \vdots \\
0&0& \cdots &1
\end{array}} \right] = {\mathbf{I}}.
\end{equation}
Hence, we may write
\begin{equation}
\mathbf{I}\mathbf{G}^T\mathbf{Gm} - \mathbf{I}\mathbf{G}^T\mathbf{d} = \mathbf{0},
\end{equation}
or by rearranging,
\begin{equation}
\mathbf{G}^T\mathbf{Gm} = \mathbf{G}^T\mathbf{d}.
\end{equation}
Because the $\mathbf{G}^T\mathbf{G}$ matrix is square and symmetric and thus can be inverted, we simply multiply by its inverse
and obtain the general least squares solution as
\begin{equation}
\mathbf{m} = \left [\mathbf{G}^T\mathbf{G}\right ]^{-1}\mathbf{G}^T\mathbf{d}.
\label{eq:lsgensolution}
\end{equation}
Comparing (\ref{eq:lsgensolution}) with (\ref{eq:Nxvsolution}) we see clearly that $\mathbf{N} = \mathbf{G}^T\mathbf{G}$
and $\mathbf{v} = \mathbf{G}^T\mathbf{d}$.  Furthermore, given (\ref{eq:Aasvectors}) we may write
$\mathbf{G}^T\mathbf{G}$ using the product
\begin{equation}
\mathbf{N} = \mathbf{G}^T\mathbf{G} = \left [
	\begin{array}{c}
	\mathbf{g}_1^T \\[6pt]
	\mathbf{g}_2^T \\[6pt]
	\vdots \\[6pt]
	\mathbf{g}_k^T \\	 
	\end{array} \right ] \cdot
	\left [
		\begin{array}{cccc}
		\mathbf{g}_1 & \mathbf{g}_2 & \cdots & \mathbf{g}_k \\	 
		\end{array}
	\right ] =
	 \left [
	\begin{array}{cccc}
	\mathbf{g}_1^T\mathbf{g}_1 & \mathbf{g}_1^T\mathbf{g}_2 & \cdots & \mathbf{g}_1^T\mathbf{g}_k \\[6pt]	 
	\mathbf{g}_2^T\mathbf{g}_1 & \mathbf{g}_2^T\mathbf{g}_2 & \cdots & \mathbf{g}_2^T\mathbf{g}_k \\[6pt]	 
	\vdots & \vdots & \ddots & \vdots \\[6pt]
	\mathbf{g}_k^T\mathbf{g}_1 & \mathbf{g}_k^T\mathbf{g}_2 & \cdots & \mathbf{g}_k^T\mathbf{g}_k \\	 
	\end{array} \right ],
	\label{eq:gdotg}
\end{equation}
which makes it clear that each element of $\mathbf{N}$, such as $n_{pq}$, is the dot product between two basis vectors $\mathbf{g}_p^T$ and $\mathbf{g}_q$, and
\begin{equation}
\mathbf{v} = \mathbf{G}^T\mathbf{d} = \left [
	\begin{array}{c}
	\mathbf{g}_1^T \\[6pt]
	\mathbf{g}_2^T \\[6pt]
	\vdots \\[6pt]
	\mathbf{g}_k^T \\	 
	\end{array} \right ] \cdot \mathbf{d} =
	 \left [
	\begin{array}{c}
	\mathbf{g}_1^T\mathbf{d} \\[6pt]	 
	\mathbf{g}_2^T\mathbf{d}  \\[6pt]	 
	\vdots \\[6pt]
	\mathbf{g}_k^T\mathbf{d} \\	 
	\end{array} \right ],
	\label{eq:gdotd}
\end{equation}
which shows each element of $\mathbf{v}$ is the dot product between each basis function $\mathbf{g}_j^T$ and the data vector $\mathbf{d}$.
This is simply what we found the hard way earlier (i.e., \ref{eq:normalwsums}).
Thus, to solve a general linear least squares problem, all we have to do is to evaluate $\mathbf{G}$ via (\ref{eq:Aasvectors}) and
the rest is taken care of by (\ref{eq:lsgensolution}).
\begin{example}
We are given a data set with four data pairs (2,1), (4,4), (6,3) and (8,4) ($n = 4$) and asked to determine the
coefficients for a \emph{quadratic} curve that best describes the data.  Except for special situations, we know that the
three-parameter curve will not pass through all the four points, so we decide to seek a least squares solution.

We write down the functional form for our quadratic curve as $d = m_1 + m_2 x + m_3 x^2$ and use it to
form the matrix equation
\begin{equation} \begin{array}{c}
m_1 + m_2 x_1 + m_3 x_1^2 = d_1\\[6pt]
m_1 + m_2 x_2 + m_3 x_2^2 = d_2\\[6pt]
m_1 + m_2 x_3 + m_3 x_3^2 = d_3\\[6pt]
m_1 + m_2 x_4 + m_3 x_4^2 = d_4\\[6pt]
\end{array},
\end{equation}
which for our data yields the linear system
\begin{equation}
 \left [
\begin{array}{ccccc}
1 & 2 & 4 \\	 
1 & 4 & 16 \\
1 & 6 & 36 \\
1 & 8 & 64 \\
\end{array} \right ]
\cdot
\left [ \begin{array}{c} 
m_1\\ m_2\\ m_3  \end{array}   \right]
=
\left [ \begin{array}{c} 1\\ 4 \\ 3 \\ 4  \end{array} \right ] .
\end{equation}
Using (\ref{eq:lsgensolution}) we find the solution to be
$d = -1.5 + 1.65 x -0.125x^2$.  Figure~\ref{fig:Fig1_GenLS} shows our data as well as the best quadratic curve.
\PSfig[h]{Fig1_GenLS}{Fitting a three-parameter quadratic curve to four data points by minimizing the least squares misfit.}
\end{example}
\index{General linear least squares method|)}

\subsection{Weighted least squares solution}
\index{Weighted least squares solution|(}

	What if some data constraints are more reliable than others? We may simply give those residuals more 
weight than the others, i.e.,
\begin{equation}
\mathbf{e} = \left[ \begin{array}{c} 0.8 e_1 \\ 2.1e_2\\ \vdots \\ 0.7 e_n \end{array} \right ] .
\end{equation}	 
In general, we can assign weights $w_i$ to each misfit so that the new weighted misfits become $e_i' = e_i\cdot w_i$.
Very often, the weights will simply be $s_{ii} = 1/s_i$, where $s_i$ is the one-sigma uncertainty in the $i$'th measurement $d_i$.
We implement such weighting by introducing a diagonal weight matrix,
\begin{equation}
{\mathbf S} = \left[\begin{array}{cccc}
s_{11} \\ & s_{22}\\ & & \ddots\\ & & & s_{nn} \end{array}
  \right ],
\end{equation}
which means the weighted residuals are $\mathbf{S}\cdot\mathbf{e}$ and the sum of the squared errors, $E$, becomes
\begin{equation}
E = \left (\mathbf{S}\cdot\mathbf{e}\right )^T\left (\mathbf{S}\cdot\mathbf{e}\right ) = \mathbf{e}^T \cdot \mathbf{S}^T \cdot \mathbf{S \cdot e = e}^T \cdot \mathbf{W \cdot e},
\end{equation}	 
where we have introduced $\mathbf{W = S}^T\mathbf{S}$. Since $\mathbf{S} \cdot \mathbf{e = S\cdot(G \cdot m - d)}$ we obtain
\begin{equation}
\begin{array}{rcl}
E(\mathbf{m}) & = & \mathbf{(S \cdot G \cdot m - S\cdot d)}^T \cdot \mathbf{(S \cdot G \cdot m - S \cdot d)} =
(\mathbf{m}^T \cdot \mathbf{G}^T \cdot \mathbf{S}^T - \mathbf{d}^T \cdot S\mathbf{^T) \cdot (S \cdot G \cdot m - S \cdot d)}\\
& = & \mathbf{m}^T \cdot \mathbf{G} ^T \cdot \mathbf{S}^T \cdot \mathbf{S \cdot G \cdot  m - m}^T \cdot \mathbf{G}^T \cdot \mathbf{S}^T \cdot \mathbf{S \cdot d - d}^T \cdot \mathbf{S}^T \cdot \mathbf{S \cdot G  \cdot m + d}^T \cdot \mathbf{S}^T \cdot \mathbf{S \cdot d}.
\end{array}
\end{equation}
We substitute $\mathbf{W = S}^T\mathbf{S}$, take the partial derivatives, and obtain
\begin{equation}
\frac{\partial E(\mathbf{m})}{\partial m_j }
= \mathbf{  0 =  \dot{m}}^T \cdot \mathbf{G}^T \cdot \mathbf{W  \cdot G \cdot  m + m}^T \cdot \mathbf{G}^T  \cdot \mathbf{W  \cdot G  \cdot \dot{m} - \dot{m}}^T \cdot \mathbf{G}^T \cdot \mathbf{W  \cdot  d - d}^T \cdot \mathbf{W  \cdot G  \cdot \dot{m}}, \quad j = 1,k.
\end{equation}	 
Since $\mathbf{m}$ only contains the $m_j$, we know $\mathbf{\dot{m}}^T = \mathbf{\dot{m} = I}$. We again find the $k$ normal equations can be written more compactly as the single matrix equation
\begin{equation}
\mathbf{G}^T \cdot \mathbf{W  \cdot G \cdot  m + m}^T \cdot \mathbf{G}^T  \cdot \mathbf{W  \cdot G  -  G}^T \cdot \mathbf{W  \cdot  d - d}^T \cdot \mathbf{W  \cdot G = 0}.
\end{equation}
As before, the second and fourth terms are the transposes of the first and third terms, and as they all represent the same terms our equation reduces to
\begin{equation}
\mathbf{G}^T \cdot \mathbf{W \cdot G \cdot m - G}^T \cdot \mathbf{W \cdot d = 0}.
\end{equation}	 
Thus, the weighted linear least squares solution is
\begin{equation}
\mathbf{m} = \left [ \mathbf{G}^T \cdot \mathbf{W \cdot G} \right ] ^{-1} \mathbf{G}^T \cdot \mathbf{W \cdot d}.
\label{eq:weightedLSgeneral}
\end{equation}
This solution is universal and applies to \emph{any} linear least squares problem one can imagine.
In the particular case when all $s_{ii} = 1$ the solution reduces to (\ref{eq:lsgensolution}).
\begin{example}
\PSfig[h]{Fig1_grav_model}{Observed and modeled gravity anomalies over a dense ore body.
While the model is nonlinear in $x$ and $z$, it is \emph{linear} in the coefficients $p_i$.}
We will try the least squares machinery on an example taken from exploration geophysics.
Figure~\ref{fig:Fig1_grav_model} shows how observed gravity anomalies ($d_i$, solid circles) vary
over a buried dense ore body as a function of location $x_i$.
Based on the inferred geometry of the ore (from subsurface geology seen in mine shafts, etc.) we
expect that a first-order approximation to the ore body could be a sphere buried at a depth of 5 km, with
a radius of 2.5 km, and located at 15 km to the right of the origin.  We would like to determine
the density of that ore body.  Exploration geophysics textbooks tell us that the gravity anomaly over
a buried sphere of radius $r$ and \emph{unit} density is
\begin{equation}
g_{sp}(x, z, r) = \gamma \frac{\frac{4}{3}\pi r^3 z}{(x^2 + z^2)^{3/2}},
\end{equation}	 
where $z$ is the depth to the center of the sphere, $x = 0$ is where the sphere is located, and $\gamma$
is the universal gravitational constant ($6.674\cdot10^{-11}$ m$^3$kg$^{-1}$s$^{-2}$).  However, 
inspection of the data suggests that the anomaly due to the ore body is superimposed on a regional field
with some curvature to it (i.e., dashed trend in Figure~\ref{fig:Fig1_grav_model}).  Therefore, we decide to model
these anomalies as a sum of a quadratic background (regional) field and the attraction of the sphere;
this is accomplished with the four-parameter linear model
\begin{equation}
g(x) = m_1 + m_2 x + m_3 x^2 + m_4 g_{sp}(x-15, 5, 2.5),
\end{equation}	 
where $m_4 = \Delta \rho$, the density contrast between the ore and the host rock.  In the parlance of
the previous sections, our basis functions $g_j(x)$ are \{1, $x$, $x^2$, and $g_{sp}(x)$\}.  To solve the
problem we need to evaluate the matrix equation $\mathbf{G \cdot m = d}$, i.e.,
\begin{equation}
\left[ \begin{array}{cccc}
1 & x_1 & x_1^2 & g_{sp}(x_1-15,5, 2.5) \\[5pt]
1 & x_2 & x_2^2 & g_{sp}(x_2-15,5, 2.5) \\[5pt]
1 & x_3 & x_3^2 & g_{sp}(x_3-15,5, 2.5) \\[5pt]
\vdots	&	\vdots & \vdots & \vdots \\
1 & x_n & x_n^2 & g_{sp}(x_n-15,5, 2.5)
\end{array}   \right]	 
\cdot
\left [ \begin{array}{c} 
m_1\\ m_2\\ m_3 \\ m_4  \end{array}   \right]
=
\left [ \begin{array}{c} d_1\\ d_2 \\ d_3 \\ \vdots \\ d_n  \end{array} \right ],
\end{equation}
whose solution becomes
\begin{equation}
\mathbf{m} = [ \mathbf{G}^T \cdot \mathbf{G} ]^{-1} \mathbf{G}^T \cdot \mathbf{d}.
\label{eq:LS_grav}
\end{equation}

Thus, we have solved a fairly complicated least squares modeling problem (solution is
the solid line in Figure~\ref{fig:Fig1_grav_model}).
\end{example}
\index{Weighted least squares solution|)}

\clearpage
\section{Problems for Chapter \thechapter}

\begin{problem}
If the major product moment $\mathbf{A}^T\mathbf{A} = \mathbf{0}$, show that all elements in
$\mathbf{A}$ must be zero, i.e. $a_{ij} = 0$. (Hint: what does an element in $\mathbf{A}^T\mathbf{A}$ represent?)
\end{problem}

\begin{problem}
An analytical experiment yields the following data pairs $t$ and $d(t)$:
\begin{table}[H]
\centering
\begin{tabular}{|c|c|} \hline
$t$ & $d$ \\ \hline
-0.82 & -0.86 \\ \hline
0.23 & -0.58 \\ \hline
1.35 & 0.54 \\ \hline
2.25 & 1.30 \\ \hline
3.33 & 2.20 \\ \hline
\end{tabular}
\end{table}
\begin{enumerate}[label=\alph*)]
\item Using the solutions for least-squares regression lines, i.e., (\ref{eq:intercept_solution}) and (\ref{eq:slope_solution})
above, determine the slope and intercept of the best-fitting line.
\item Plot this line and the data points.
\item Redo the problem using matrix algebra.  Write down the various components in the matrix equation
$\mathbf{G \cdot m = d}$.  What does $\mathbf{G \cdot m}$ represent?
\item Evaluate the matrix solution (i.e., $\mathbf{m} = [ \mathbf{G}^T\mathbf{G} ]^{-1}\mathbf{G}^T\mathbf{d}$)
and make sure it matches your answer in (a).
\item What is the total misfit, $E$?
\end{enumerate}
\end{problem}

\begin{problem}
A detailed study of seafloor heat flow was conducted along a transect perpendicular
to a mid-ocean ridge.  The results are reported as distance (km), heat flow (mWm$^{-2}$) pairs in
the file \emph{heatflow.txt}.  Because of hydrothermal circulation, we choose to model the
decay of heat flow with distance from the ridge as
\[
q(x) = q_0 + q_1x + \frac{q_2}{x}.
\]
\begin{enumerate}[label=\alph*)]
\item Using matrix algebra, what are the least-squares values for the model coefficients?
\item Plot the data as points and the model as a solid line evaluated every km from 25 km to 250 km
on the same graph.
\item What is the r.m.s (root-mean-square) misfit (r.m.s = $\sqrt{E/n}$)? Be sure to give units.
\item Using your error analysis skills, what is the estimated
heat flow at a distance of $x = 25 \pm 1$ km from the ridge axis? (Hint: Consider the uncertainty
in the function $q(x)$.)
\end{enumerate}
\end{problem}

\begin{problem}
The file \emph{faultstep.txt} contains distance (m) and relief (m) for a topographic profile across
an small normal fault.  We want to estimate the total vertical offset across the fault.  There is also a
linear trend in the data because the the fault sits on a gently dipping monocline.
\begin{enumerate}[label=\alph*)]
\item  Make a linear model that includes the trend and the step.  You may want to use the Heaviside 
step function (\texttt{heaviside} in MATLAB), defined as
$$
H(x) = \left \{ \begin{array}{rr} 
0, & x < 0 \\
\frac{1}{2}, & x = 0 \\
1, & x > 0
\end{array} \right.
$$
What is your model equation for the topographic profile?
\item  Let your best guess for the fault location be $x_0 = 145$ m and use matrix algebra
to solve for the model parameters.  Write the first few terms of the matrix equation
$\mathbf{G \cdot m = d}$ so you can see the pattern.  
Plot your model prediction on top of the data and indicate the values of the parameters.  What is the
rms misfit? What is the size of the fault offset?
\item  Guessing $x_0$ is not very robust.  Determine the best choice for $x_0$ that minimizes the total 
misfit $E$ by trying a range of $x_0$ (hint: evaluate $E(x_0)$ for a dense set of $x_0$ values
covering the likely range, plot $E$ versus $x_0$, and find the value of $x_0$ where $E$ has its minimum).
What is your final model parameters for the fault profile? Plot it on top of the data.
What is your minimum r.m.s. value?  What is the final size of the fault offset?
\end{enumerate}
\end{problem}

\begin{problem}
As sedimentary layers become buried they compact, reducing the porosity of the unit.
The empirical relation between depth ($z$) and porosity ($\theta$) in well-compacted sediments is called \emph{Athy's law}:
$$
\theta = \theta_0 e^{-\alpha z},
$$
where $\theta_0$ and $\alpha$ are constants that vary with sediment type and location, and $z$ is depth of burial.  
\begin{enumerate}[label=\alph*)]
\item	Given the data set below, find the weighted least-squares estimates of $\theta_0$ and $\alpha$ using matrix algebra,
taking the uncertainties in $\theta$ into account (note: Your first idea on how to do that is likely to be wrong!).
Plot the data and your Athy's law.  (Hint:  Athy's law as given
is not linear!  You must transform it first.)
\item	What is the predicted porosity ($\pm$ uncertainty) at a depth of 6 km ($\pm 200$m)?
\end{enumerate}
\begin{table}[H]
\centering
\begin{tabular}{|c|c|} \hline
\bf{Depth} (m) &	\bf{Porosity (\%)} \\ \hline
650	& $38 \pm 5.0$ \\ \hline
1000	& $35 \pm 4.0$ \\ \hline
2050	& $24 \pm 2.5$ \\ \hline
2950	& $18 \pm 2.0$ \\ \hline
4075	& $14 \pm 1.5$ \\ \hline
5030	& $9.8 \pm 1.0$ \\ \hline
\end{tabular}
\end{table}
\end{problem}

\begin{problem}
The rim of Halemaumau crater within Kilauea caldera has been digitized and its UTM coordinates ($x_i, d_i$) are
listed in \emph{halemaumau.txt}.  We wish to approximate this shape by a perfect circle with parameters
$(x_0, d_0, r)$.
\begin{enumerate}[label=\alph*)]
	\item What is the misfit function, $E$, that we want to minimize using the least-squares criterion?
	\item Determine the three parameters for the circular model by minimizing $E$.
	\item Estimate the area of the crater and use the standard deviation of the radial residual misfits to assign
	  one-sigma confidence bounds on the area.
\end{enumerate}
\end{problem}

\begin{problem}
	\newcounter{noisyc}
	\setcounter{noisyc}{\thechapter}
	\newcounter{noisyp}
	\setcounter{noisyp}{\theproblem}
The file \emph{noisy.txt} contains observations of a phenomenon known to oscillate at a single frequency, $\omega$.
This signal is superimposed on a linear trend in the presence of some random noise.
\begin{enumerate}[label=\alph*)]
	\item Assume you know the frequency $\omega$. What is the form of the model you could use to fit the data?
	Hint: $A\cos(\omega t - \phi) = a \cos(\omega t) + b \sin(\omega t)$.
	\item Plot the data and eye-ball the period and use it to determine the frequency.  Use least-squares to fit your
	model and plot it.  What is the misfit, $E$, for this trial model?
	\item Try a range of frequencies centered on your best guess and determine the frequency $\omega_0$ that minimizes
	the misfit.  Plot the optimal model, and in a separate diagram plot the misfits you obtained versus the frequencies
	you explored.
\end{enumerate}
\end{problem}

\begin{problem}
Having crash-landed on an alien planet you worry you may not have enough fuel to reach escape velocity.  To find out you
need to know the planet's gravitational attraction, $g_p$ (which on Earth is 9.81 m s$^{-2}$).  To get an estimate you decide
to drop rocks off cliffs of various heights and clock the time it takes for them to reach the valley floor.  Your data
are listed in the file \emph{drops.txt}.  You vaguely remember your high-school physics exploits where you learned that the vertical drop in vacuum
is related to drop time given by $h = \frac{1}{2}g_p t^2$.  Plotting the data you realize that the height measurements obtained with your damaged laser
rangefinder is biased by some unknown but constant offset.  Aargh, there is always something....
\begin{enumerate}[label=\alph*)]
	\item What is a suitable linear model that will relate your observations and unknowns?
	\item Determine the planet's gravity and the bias in your instrument using the linear least squares method.
\end{enumerate}
\end{problem}

\begin{problem}
An elongated sedimentary basin is estimated to have a width of 20 km and a depth of 4 km.  The gravity anomalies measured across the
basin are given in \emph{basingrav.txt} and show a broad negative anomaly due to the lower density sediments relative to
the surrounding higher density igneous bedrock, but some lateral regional trend is also apparent.  You decide to model the anomalies as
a linear regional trend plus the attraction of a
prism-shaped two-dimensional basin of given dimensions.  You may use the MATLAB function \texttt{g\_basin.m} to compute the gravity of the prism.
\begin{enumerate}[label=\alph*)]
	\item Determine the least-squares solution for the density contrast.  If the bedrock density is 2670 kg m$^{-3}$,
	what is your estimate of the sediment density?
	\item A seismic crew collects data that throw some doubt on your depth estimate of 4 km.  You decide to repeat your
	modeling for a range of depths (from 3 to 6 km in steps of 100 m) and keep track of the misfit $E$ as a function
	of the chosen depth.  Find the optimal depth that minimizes the misfit and report this depth and the corresponding density contrast.
\end{enumerate}
\end{problem}

%  $Id: DA1_Chap6.tex 688 2019-06-09 09:24:42Z pwessel $
%
\chapter{REGRESSION}
\label{ch:regression}
\epigraph{``The invalid assumption that correlation implies cause is probably among the two or three most serious and common errors of human reasoning.''}{\textit{Stephen Jay Gould, Paleontologist}}
Regression refers to a subset of data modeling where we fit a simple model with a linear trend in one (or more) dimensions.  Usually we also include a constant (intercept) term.  Entire books have been written about regression
and all the various methods, norms, and misfit-minimizations possible.  A summary of some of these developments will be given below.
\section{Line-Fitting Revisited}
\index{Regression!weighted least squares|(}

We will again consider the best-fitting line problem $y = a + bx$, this time with errors $s_i$ in the observed $y$-values.  We 
want to measure how well the model agrees with the data, and for this purpose we will use the $\chi^2$ 
function, i.e.,
\begin{equation}
\chi^2(a,b) = \sum^n_{i=1} \left ( \frac{y_i - a - bx_i}{s_i} \right ) ^2.
\label{eq:L2_line_chi2}
\end{equation}
Minimizing $\chi^2$  will give the best weighted least squares solution.  Again, we set the partial 
derivatives to zero and obtain
\begin{equation}
\begin{array}{c}
\displaystyle \frac{\partial \chi^2}{\partial a} = 0 = -2 \sum^n_{i=1} \left( \frac{y_i -a -bx_i}{s^2_i}\right),  \\
\ \\
\displaystyle \frac{\partial \chi^2}{\partial b} = 0 = -2 \sum^n_{i=1} \left( \frac{y_i -a -bx_i}{s^2_i}\right) x_i.
\end{array}
\label{eq:L2_line_ddx}
\end{equation}
Let us define the following terms (unless noted, all sums go from $i = 1$ to $n$):
\begin{equation}
S = \sum \frac{1}{s^2_i}, \quad S_x = \sum \frac{x_i}{s^2_i}, \quad S_y = 
\sum \frac{y_i}{s^2_i}, \quad S_{xx} = \sum \frac{x^2_i}{s^2_i}, \quad
S_{xy} = \sum \frac{x_i y_i}{s^2_i}.
\end{equation}
Then, (\ref{eq:L2_line_ddx}) reduces to
\begin{equation}
\begin{array}{rcl}
\displaystyle
aS  + bS_x & = & S_y, \\*[2ex]
aS_x + bS_{xx} & = & S_{xy}.
\end{array}
\end{equation} 	 
Introducing 
\begin{equation}
\Delta = SS_{xx} - S ^2 _x
\end{equation}
we find
\begin{equation}
\begin{array}{rcl}
a & = & \displaystyle \frac{S_{xx}S_y - S_x S_{xy}}{\Delta},\\*[2ex]
b & = & \displaystyle \frac{S S_{xy}- S_x S_{y}}{\Delta}.
\end{array}
\label{eq:L2_sol_ab}
\end{equation}
\PSfig[h]{Fig1_linefit_x}{The uncertainty in the line fit depends to a large extent on the
distribution of the $x$-positions as well as the uncertainties in the $y$-values.}
   All this is swell but we must also estimate the uncertainties in $a$ and $b$.  For the same $s_i$ we 
may get large differences in the uncertainties in $a$ and $b$ (e.g., Figure~\ref{fig:Fig1_linefit_x}).
As shown in Chapter~\ref{ch:error}, consideration of the propagation of errors (e.g., \ref{eq:uncert_func}) shows
that the variance $\sigma_f^2$ in the value of any function is
\begin{equation}
\sigma^2_f = \sum \left ( \frac{\partial f}{\partial y_i} \sigma_i \right )^2,
\label{eq:L2_line_error}
\end{equation}
where we now consider $f$ a function of all the $n$ independent parameters $y_i$. For our model,
$f$ is either $a$ or $b$ so the partial derivatives become
\begin{equation}
\begin{array}{rcl}
\displaystyle \frac{\partial a}{\partial y_i} & = & \displaystyle \frac{S_{xx} - S_x x_i}{s ^2_i \Delta},\\*[2ex]
\displaystyle \frac{\partial b}{\partial y_i} & = & \displaystyle \frac{Sx_{i} - S_x}{s ^2_i \Delta}.
\end{array}
\end{equation}
Inserting in turn these terms into (\ref{eq:L2_line_error}) now gives
\begin{equation}
\begin{array}{rcl} 
\displaystyle
\displaystyle s^2_a & = & \displaystyle \sum s^2_i \left [ \frac{S_{xx} - S_{x} x_i}{s ^2_i \Delta} \right ] ^2 = \sum \frac{S^2_{xx} - 2S_{xx}S_{x}x_i + S^2_x x^2_i}{s^2_i \Delta ^2}\\*[2ex]
& = & \displaystyle \frac{S^2_{xx}}{\Delta^2} \sum \frac{1}{s^2_i} - \frac{2S_{xx}S_x}{\Delta^2}
\sum \frac{x_i}{s^2_i} + \frac{S^2_x}{\Delta^2} \sum \frac{x^2_i}{s^2_i}= \frac{S^2_{xx}S}{\Delta^2} - \frac{2S_{xx}S^2_x}{\Delta ^2} + \frac{S_{xx}S^2_x}{\Delta^2} \\*[2ex]
& = & \displaystyle \frac{S_{xx} (S_{xx}S-S^2_x)}{\Delta^2} = \frac{S_{xx}}{\Delta}
\end{array}
\end{equation}
and	
\begin{equation}
\begin{array}{rcl}
\displaystyle s^2_b & = & \displaystyle \sum s^2_i \left [ \frac{Sx_i - S_x}{s^2_i \Delta} \right ] ^2 =
\sum \frac{S^2x^2_i - 2S \ S_x x_i + S^2_x}{s^2_i \Delta^2}\\*[2ex]
 &  = & \displaystyle \frac{S^2}{\Delta^2}\sum \frac{x^2_i}{s^2_i} - 
\frac{2S \ S_x}{\Delta^2} \sum \frac{x_i}{s^2_i} + \frac{S^2_x}{\Delta^2} \sum \frac{1}{s^2_i} = \frac{S^2S_{xx}}{\Delta^2} - \frac{2S \ S^2_x}{\Delta^2} + \frac{S S^2_x}{\Delta ^2}\\*[2ex]
& = & \displaystyle \frac{S(S_{xx}S - S^2_x)}{\Delta^2} = \frac{S}{\Delta}.
\label{eq:err_in_slope}
\end{array}
\end{equation}
Similarly, we can find the covariance $s_{ab}$ from
\begin{equation}
s^2_{ab} = \sum s ^2_i \left( \frac{\partial a}{\partial y_i} \right)
\left( \frac{\partial b}{\partial y_i}\right) = -\frac{S_x}{\Delta}.
\end{equation}
Thus, the correlation between $a$ and $b$ becomes
\begin{equation}
r_{ab} = \frac{-S_x}{\sqrt{SS_{xx}}}.
\label{eq:uncorrelated_a_b}
\end{equation}
It is therefore useful to shift the origin to $\bar{x}$ so that $r_{ab} = 0$, leaving our estimates for slope and intercept uncorrelated.  

	Finally, we must check if the fit is significant.  We determine 
critical $\chi ^2_\alpha$ for $n - 2$ degrees of freedom and test if our computed $\chi^2$ exceeds the critical limit.  If it 
does not, then we may say the fit is \emph{significant} at the $\alpha$ level of confidence.

\subsection{Confidence interval on regression}
\index{Regression!confidence interval}
\PSfig[h]{Fig1_Draper}{Solid line shows the least squares regression fit to the data points (blue circles), with
color bands reflecting different confidence levels.  Short vertical lines are the residual errors which are squared and summed
in (\ref{eq:reg_errstd}).}
The formalism in the previous section allowed us to derive
solutions for slope and intercept given via (\ref{eq:L2_sol_ab}). Let us consider the confidence interval on the prediction.  We write the least squares
fit as $\hat{y} = a + bx$, and by substituting $a = \bar{y} - b\bar{x}$ we obtain
\begin{equation}
	\hat{y} = \bar{y} + b(x-\bar{x}).
\end{equation}
Here, both the (weighted) mean $y$-value ($\bar{y}$) and slope ($b$) are subject to error and these in turn affect $\hat{y}$.  For some
chosen location $x_0$ the prediction for the regression would be
\begin{equation}
	\hat{y}_0 = \bar{y} + b(x_0-\bar{x}).
\end{equation}
In this formulation, with a local origin at $(\bar{x}, \bar{y})$, the correlation between $\bar{y}$ and $b$ is zero (e.g., \ref{eq:uncorrelated_a_b})
and $s^2_a = 1/S$ and $s^2_b = 1/S_{xx}$.
These facts allow us to compute the expected variance $V$ of the \emph{mean predicted value} by summing the separate variance terms:
\begin{equation}
	V(\hat{y}_0) = V(\bar{y}) + V(b)(x_0-\bar{x})^2 = \frac{s^2}{S} + \frac{s^2(x_0 - \bar{x})^2}{S_{xx}}.
\end{equation}
Here, $s$ is our sample estimate of the (weighted) \emph{standard deviation of the regression residuals}, $e_i = y_i - \hat{y}$, given by
\begin{equation}
	s^2 = \frac{\sum (e_i/s_i)^2}{S(n-2)/n},
	\label{eq:reg_errstd}
\end{equation}
where $n-2$ are the remaining degrees of freedom after computing $a$ and $b$.
To obtain confidence intervals on the linear regression we simply scale $s$ from (\ref{eq:reg_errstd}) by a critical
Student's $t$-value for the degrees of freedom $\nu$ and chosen confidence level (Table~\ref{tbl:Critical_t}), i.e.,
\begin{equation}
	\hat{y}_0 \pm t(\nu,\alpha/2) s \sqrt{\frac{1}{S} +  \frac{(x_0 - \bar{x})^2}{S_{xx}}}.
\end{equation}
For larger data sets the Student's $t$ values approach the normal distribution critical values $|z_{\alpha/2}|$.
Figure~\ref{fig:Fig1_Draper} illustrates how confidence bands on a least-squares regression fit takes
on a parabolic shape around the best-fit line.
\index{Regression!weighted least squares|)}

\section{Orthogonal Regression}
\index{Regression!orthogonal|(}
\subsection{Major axis}
\index{Regression!major axis|(}

	It is often the case that the uncertainties in our $(x,y)$ data affect both coordinates.  
Examples where this is the case include situations where both $x$ and $y$ are observed quantities 
(and hence are known to have errors).  It is also  applicable when $y$ is a function of $x$, but $x$ (e.g., distance 
or time) itself has uncertainties.  In these cases, orthogonal regression is the correct way to 
determine linear relationships between $x$ and $y$ (Figure~\ref{fig:Fig1_MA_misfit}).
\PSfig[h]{Fig1_MA_misfit}{The misfit is measured in the direction perpendicular to the line.  Note that
$(x_i, y_i)$ denote our data points (black circles) and $(X_i, Y_i)$ are the coordinates of their orthogonal
projections (white circles; only one is shown here) onto the regression line.}
We will use the least squares principle and minimize the sum of the squared \emph{perpendicular} 
distances $d_i^2$ from the data points $(x_i,y_i)$ to the regression line\footnote{This approach does not take into account
the actual uncertainties in $x$ and $y$ (which becomes very tedious algebraically) but instead focuses on the effect
of the orthogonal misfit criterion.}.  The function we want to minimize is
\begin{equation}
E = \sum_{i=1}^n \left [ (X_i - x_i)^2 + (Y_i - y_i)^2 \right ],
\label{eq:E_major_axis}
\end{equation}	 	
where lowercase $(x_i, y_i)$ again are our observations and uppercase $(X_i, Y_i)$ are the ``adjusted'' 
coordinates we want to find.  These are, of course, required to  lie on a straight line described by
\begin{equation}
Y_i = a + bX_i,
\label{eq:line_constraint}
\end{equation}	
or equivalently
\begin{equation}
f_i = a + bX_i - Y_i = 0.
\label{eq:Lagrange_major_axis}
\end{equation}
Thus, we cannot simply find \emph{any} set of $(X_i, Y_i)$ as they also have to lie on a straight line.  The problem 
of minimizing the function (\ref{eq:E_major_axis}) under the specified constraints (\ref{eq:Lagrange_major_axis}) can be solved by a method 
known as \emph{Lagrange's multipliers}.  This method says we should form a new function $F$ by adding the 
\index{Lagrange's multipliers}
original function (\ref{eq:E_major_axis}) and all the constraints (\ref{eq:Lagrange_major_axis}), with each constraint scaled by an unknown 
(Lagrange) multiplier $\lambda _i$.  Since (\ref{eq:Lagrange_major_axis}) is actually $n$ constraints, we find
\begin{equation}
F = E + \lambda_1 f_1 + \lambda_2 f_2 + \ldots + \lambda_n f_n = E + \sum ^n _{i=1} \lambda_i f_i.
\label{eq:MA_partials}
\end{equation}	 
We may now set the partial derivatives of $F$ to zero and solve the resulting set of equations:
\begin{equation}
\frac{\partial F}{\partial X_i} = \frac{\partial F}{\partial Y_i} = \frac{\partial F}{\partial a} = \frac{\partial F}{\partial b} = 0,
\end{equation}
or, when viewed separately (with all sums over $i = 1$ to $n$ unless explicitly stated), 
\begin{equation}
\frac{\partial F}{\partial X_i} = \sum_j^n \frac{\partial}{\partial X_i} ( X_j - x_j)^2  + \sum_j^n \frac{\partial}{\partial X_i}
(\lambda_j bX_j) = 0 \Rightarrow 2(X_i - x_i) + b \lambda_i = 0,
\end{equation}
\begin{equation}
\frac{\partial F}{\partial Y_i} = \sum_j^n \frac{\partial}{\partial Y_i} ( Y_j - y_j)^2  - \sum_j^n \frac{\partial}{\partial Y_i}
(\lambda_j Y_j) = 0  \Rightarrow 2(Y_i - y_i) -  \lambda_i = 0,
\end{equation}
\begin{equation}
\frac{\partial F}{\partial a} = \sum  \frac{\partial}{\partial a} (\lambda_i a) = 0  \Rightarrow \sum \lambda_i  = 0,
\label{eq:lambda_major}
\end{equation}
\begin{equation}
\frac{\partial F}{\partial b} = \sum \frac{\partial}{\partial b}(\lambda_i bX_i) = 0  \Rightarrow \sum \lambda_i X_i = 0.
\label{eq:lambda_x_major}
\end{equation}
Since each $i$ represents a separate equation, we find
\begin{equation}
2(X_i - x_i) = -b \lambda_i \Rightarrow X_i = x_i - b \lambda_i/2,
\label{eq:Xisolution}
\end{equation}
\begin{equation}
2(Y_i - y_i) = \lambda_i \Rightarrow Y_i = y_i + \lambda_i/2.
\end{equation}	 
Substituting these expressions for $X_i$ and $Y_i$ into (\ref{eq:line_constraint}), we obtain
\begin{equation}
y_i + \lambda_i/2 = a + b \left (x_i - b \lambda_i/2 \right ) = a + bx_i - b^2 \lambda_i/2
\end{equation}	 
or
\begin{equation}
\lambda_i = \frac{2}{1 + b^2}\left (a + bx_i - y_i\right ).
\label{eq:lambda_i_major}
\end{equation}
Now, (\ref{eq:lambda_major}), (\ref{eq:lambda_x_major}), and (\ref{eq:lambda_i_major}) gives us
$n+2$ equations in $n+2$ unknowns (all the $\lambda_i$ plus $a$ and $b$).  Combining 
(\ref{eq:lambda_i_major}) and (\ref{eq:lambda_major}) gives
\begin{equation}
\sum	  \frac{1}{1+b^2} \left ( a + b x_i - y_i\right ) =  0
\label{eq:step1_major}
\end{equation}	
and (\ref{eq:lambda_x_major}) using (\ref{eq:Xisolution}) gives
\begin{equation}
\sum \lambda_i x_i - b\lambda ^2_i/2 = 0,
\end{equation}	 
into which we substitute (\ref{eq:lambda_i_major}) and find
\begin{equation}
\sum \frac{1}{1 + b^2}\left (ax_i + bx^2_i - y_i x_i\right ) - \sum \frac{b}{(1+b^2)^2}\left (a+bx_i - y_i\right ) ^2 = 0.
\label{eq:step2_major} 
\end{equation}
These two equations (\ref{eq:step1_major} and \ref{eq:step2_major}) relate the parameters $a$ and $b$ to the given data values $x_i$ and $y_i$.  We find 
the solution by solving the equations simultaneously.  Noting that the denominator in (\ref{eq:step1_major}) cannot be zero, we find
\begin{equation}
\sum (a+ b x_i - y_i) = 0 \Rightarrow na + b \sum x_i = \sum y_i
\end{equation}	 
or
\begin{equation}
a = \bar{y} - b \bar{x},
\end{equation}
%	(4.16)
where $\bar{x}$  and $\bar{y}$ are the mean data values.  This expression for the intercept can now be substituted into 
(\ref{eq:step2_major}) so we may solve for the slope.  We multiply through by $(1 + b^2)^2$ and obtain
\begin{equation}
\sum (1 + b^2) \left ( \bar{y}x_i - b \bar{x} x_i + b x^2_i - x_i y_i \right ) -b \sum  \left ( \bar{y} - b \bar{x} + b x_i - y_i\right ) ^2 = 0
\end{equation}
which simplify to
\begin{equation}
\displaystyle (1 + b^2) \sum x_i \left ( \bar{y} - y_i + b (x_i - \bar{x}) \right ) -b \sum \left ( b ( x_i - \bar{x}) - (y_i - \bar{y})\right ) ^2 = 0,\\*[2ex]
\end{equation}
We now introduce the residuals $u_i = x_i - \bar{x}$ and $v_i = y_i - \bar{y}$.  Then
$$
\begin{array}{c}
\displaystyle (1 + b^2) \sum (u_i + \bar{x}) \left ( bu_i - v_i\right ) - b \sum \left (bu_i - v_i\right )^2 = 0, \\*[2ex]
\displaystyle (1 + b^2) \sum \left ( bu^2_i - u_i v_i  + \bar{x} bu_i - \bar{x} v_i \right ) - b\sum \left (b^2 u^2_i - 2b u_iv_i + v^2_i\right ) = 0, \\*[2ex]
\displaystyle \sum \left ( bu^2_i + b^3 u^2_i - u_iv_i - b^2 u_iv_i - b^3 u^2_i + 2b^2 u_iv_i - bv^2_i \right ) = 0, \\*[2ex]
\displaystyle \sum \left ( b^2u_iv_i + b (u^2_i - v^2_i)-u_i v_i \right ) = 0.
\end{array}
$$
where we have used the properties $\sum u_i = \sum v_i = 0$.  These steps finally give the solution for the slope as
\begin{equation}
b = \frac{\sum v^2_i - \sum u^2_i \pm \sqrt{(\sum u^2_i - \sum v^2_i)^2 + 4(\sum u_iv_i)^2}}{2 \sum u_iv_i}.
\end{equation}
This equation gives two solutions for the slope and we choose the one that minimizes $E$.  The other solution maximizes $E$ and
makes an angle of 90 degrees with the optimal solution.
\index{Regression!major axis|)}

\subsection{Reduced major axis (RMA) regression}
\index{Regression!reduced major axis (RMA)|(}

\PSfig[h]{Fig1_RMA_misfit}{RMA regression minimizes the sum of the \emph{areas} of the (white) rectangles
defined by the data points (black circles) and their orthogonal 
projection points (white circles; only one is shown here) on the regression line.}
In this alternative formulation of misfit we minimize the sum of the \emph{areas} of the rectangles defined by $\Delta x$ and $\Delta y$ (Figure~\ref{fig:Fig1_RMA_misfit}).
Hence, the function to minimize is
\begin{equation}
E = \sum (X_i - x_i)(Y_i - y_i).
\label{eq:E_RMA_axis}
\end{equation}
The constraints remain the same, i.e. $Y_i = a + bX_i$.  The Lagrange's multiplier method leads to a 
system of equations similar to those discussed in the previous section (\ref{eq:MA_partials}), but now we find
the $2n + 2$ equations
\begin{equation}
\frac{\partial F}{\partial X_i} = \frac{\partial}{\partial X_i} \sum_j^n (X_j - x_j)(Y_j - y_j) + \frac{\partial}{\partial X_i} \sum_j^n \lambda_j b X_j = 0 \Rightarrow Y_i - y_i + b \lambda_i = 0  \quad i = 1,n,
\end{equation}	 	
\begin{equation}
\frac{\partial F}{\partial Y_i} = \frac{\partial}{\partial Y_i} \sum_j^n (X_j - x_j)(Y_j - y_j) - \frac{\partial}{\partial Y_i} \sum_j^n \lambda_j Y_j = 0 \Rightarrow X_i - x_i - \lambda_i = 0 \quad i = 1,n,
\end{equation}
\begin{equation}
\frac{\partial F}{\partial a} = \frac{\partial}{\partial a} \sum \lambda_i a = 0 \Rightarrow \sum \lambda_i = 0,
\end{equation}
\begin{equation}
\frac{\partial F}{\partial b} = \frac{\partial}{\partial b} \sum \lambda_i b X_i = 0 \Rightarrow \sum \lambda_i X_i = 0.
\label{eq:RMA_lxi}
\end{equation}
We find     
\begin{equation}
X_i - x_i - \lambda_i = 0 \Rightarrow X_i = x_i + \lambda_i,
\end{equation}
\begin{equation}	           
Y_i - y_i + b \lambda_i = 0 \Rightarrow Y_i = y_i - b\lambda_i.
\end{equation}
Substituting these values into the equation for the line (i.e., $Y_i = a + bX_i$) gives
\begin{equation}
y_i - b \lambda_i = a + b (x_i + \lambda_i)
\end{equation}
or			       
\begin{equation}
\lambda_i = \frac{y_i - a - bx_i}{2b}.
\end{equation}	 
Since $\sum \lambda_i = 0$, we find again
\begin{equation}
a = \bar{y} - b \bar{x}.
\end{equation}	 
Substituting $\lambda_i, X_i$ and $a$ into (\ref{eq:RMA_lxi}) gives
$$
\sum \left ( x_i + \frac{y_i - \bar{y} + b \bar{x} - b x_i}{2b} \right ) \left ( 
\frac{y_i - \bar{y} + b \bar{x} - b x_i}{2b} \right) = 0.
$$
We let $u_i = x_i - \bar{x}$ and $v_i = y_i - \bar{y}$ as before and obtain 
$$
\sum \left (b (u_i + \bar{x} ) + v_i - bu_i \right ) (v_i - bu_i) = 0,
$$

$$
\sum (2b \bar{x} + bu_i + v_i) (v_i - bu_i) = 0,
$$

$$
\sum (2b \bar{x} v_i - 2b ^2  \bar{x} u_i + b u_i v_i - b^2 u^2_i + v^2_i - b u_i v_i ) = 0,
$$

$$
\sum v^2_i - b^2 \sum u^2_i = 0.
$$
where we again have used the property that sums of $u_i$ and $v_i$ are zero.  Finally, we obtain
\begin{equation}
b = \pm \sqrt{ \sum v^2_i / \sum u^2_i} = \pm \frac{s_y}{s_x},
\end{equation}
i.e., the best slope equals the ratio of the $y$ and $x$ observations' separate standard deviations.
The sign is indeterminate but is inferred from the sign of the correlation coefficient (a negative
correlation means a negative slope)\footnote{We cheated a bit as the individual terms in \ref{eq:E_RMA_axis}
might be negative; however, a more advanced derivation still yields the same solution.}.

\index{Regression!orthogonal|)}

\section{Robust Regression}
\index{Regression!robust|(}
\index{Robust!regression|(}

	In simple regression one assumes a relation of the type
\index{Explanatory variable}
\index{Response variable}
\begin{equation}
y_i = a + b x_i + \epsilon_i,
\label{eq:robregress}
\end{equation}
in which $x_i$ is called the \emph{explanatory variable} or \emph{regressor}, and $y_i$ is the \emph{response variable}.
Again, we seek to estimate $a$ and $b$ (intercept and slope) from the data 
$(x_i , y_i)$.  It is commonly assumed 
that the deviations $\epsilon_i$ are normally distributed.
Fortunately, in simple regression the observations $(x_i, y_i)$ are 2-D so they can be plotted.  It 
is always a good idea to do that first to see if any unusual features are present and to make sure 
the data are roughly linear.
Applying a regression estimator to the data $(x_i, y_i)$ will result in the two regression 
coefficients $\hat{a}$  and $\hat{b}$.  They are not the \emph{true} parameters $a$ and $b$, but our ``best'' \emph{estimates} of 
them.  We can insert those into (\ref{eq:robregress}) and find the predicted estimate as
\begin{equation}
\hat{y}_i = \hat{a} + \hat{b} x_i,\quad i = 1,n.
\end{equation}	 
The residual is then the difference between the observed and estimated values, yielding
\begin{equation}
e_i = y_i - \hat{y}_i,\quad i = 1,n.
\end{equation} 	
Note that there is a difference between $e_i$ (the misfit) and 
$\epsilon_i$ (the deviation), because $\epsilon_i = y_i - a - b x_i$ are
evaluated with the true unknown $a,b$.  We can compute $e_i$, but not $\epsilon_i$.

	The most popular regression estimator dates back to the early 1800's (to our old friends Gauss and Legendre) 
and is called the ``least-squares'' (LS) method since it seeks to minimize
\begin{equation}
E = \sum^n_{i=1} e^2_i.
\end{equation}	 
The rationale was to make the residuals very small.  Gauss preferred the least-squares 
criterion to other objective functions because in this way the regression coefficients could be 
computed explicitly from the data (no computers back in the day, at least not mechanical or electronic ones).
Later, Gauss introduced the  normal, or Gaussian, distribution for which least-squares is optimal.
	
\index{Regression!outlier|(}
More recently, people have realized that real data often do not satisfy the Gaussian 
assumption and this ``failure to comply'' may have a dramatic effect on the LS results.  In Figure~\ref{fig:Fig1_LS_pitfalls1} we have five points that lie almost 
on a straight line.  Here, the $LS$ line fits the data very well.  However, let us see what happens if 
we get a wrong value for $y_4$ because of a recording or copying error:
	 
\PSfig[h]{Fig1_LS_pitfalls1}{Pitfalls of least-squares regression, part I.  An outlying point in
the $y$-direction will effect the regression line considerably.}

\index{Regression!outlier in the y-direction}
The bad point $y_4$  is called an \emph{outlier in the y-direction}, and it has a dramatic effect on the $LS$ line 
which now is tilted away from the trend of the remaining data.  Such outliers have received the most 
attention because most experiments are set up to expect errors in $y$ only.  However, in 
observational studies it often happens that outliers occur in the $x_i$ data as well.

\PSfig[h]{Fig1_LS_pitfalls2}{Pitfalls of least-squares regression, part II.  Here, an outlying point in
the $x$-direction can have a huge effect on the regression line.}

Figure~\ref{fig:Fig1_LS_pitfalls2} illustrates the effect of an outlier in the $x$-direction.
\index{Regression!outlier in the x-direction}
It has an even more dramatic 
effect on $LS$ since it now is almost perpendicular to the actual trend.  Because this single point 
has such a large influence we denote it as a \emph{leverage} point.
\index{Regression!leverage point}
\index{Leverage point}
This is because the residual $e_i$ 
(measured in the $y$-direction) is enormous with regard to the original $LS$ fit.  The second $LS$ fit 
reduces this enormous error at the expense of increasing the errors at all other points.  In 
general, we call the $k$'th  point a leverage point if $x_k$ lies far from the bulk of the $x_i$.  Note that this 
definition does not take $y_i$ into account.  For instance, Figure~\ref{fig:Fig1_leverage} shows a ``good'' leverage 
point since it lies on the linear trend set by the majority of the data.  Thus, a leverage point only 
refers to its \emph{potential} for influencing the coefficients $\hat{a}, \hat{b}$.
When a point $(x_i, y_i)$ deviates from the linear relation of the majority it is called a \emph{regression outlier},
taking into account both $x_i$ and $y_i$ simultaneously.  As such, it is a vertical outlier or a bad 
leverage point.
	
It is often stated that regression outliers can be detected by looking at the $LS$ residuals. This is 
not always true.  The bad leverage point 1 has tilted the line so much  that its residual is very 
small.  Consequently, a rule that says ``delete points with highest $LS$ residual'' would first find points 
number 2 and 5, thereby deleting the good points first.  Of course, in simple $x-y$ regression we have the 
benefit of being able to plot the data so this is not often a problem, except when the number of 
data sets and points are large.
\index{Regression!outlier|)}

\PSfig[H]{Fig1_leverage}{The effect of leverage points in regression can be enormous, whether the data
point is a valid observation or a bad outlier.}

From the simple examples we have just seen, we find that the breakdown point for $LS$ 
regression is merely $1/n$ since one point is enough to ruin the day --- analogous to the breakdown 
point for the mean, which was also based on $LS$.
	
A first step toward a more robust regression was taken more than 100 years ago when 
Edgeworth suggested that one could instead minimize
\begin{equation}
E = \sum^n_{i=1} |e_i|,
\end{equation}	 
which we will call $L_1$ regression.  Unfortunately, while $L_1$ regression is robust with respect to 
outliers in $y$, it offers no protection toward bad leverage points.  Thus, the breakdown point is 
still only $1/n$.
	
While there are many methods that offer a higher breakdown point than $L_1$ and $L_2$, we will 
concentrate our presentation on one particular method.  Again, let us look at the $LS$ formulation:
$$
\begin{array}{cc}
\mbox{minimize} & \displaystyle E = \sum^n_{i=1} e^2_i. \\*[-2ex]
\hat{a}, \hat{b} \end{array}
$$
	 
\PSfig[H]{Fig1_LMS_regress}{Robust regression, such as LMS, is very tolerant of outlying points in 
both the $x$ and $y$ directions.}

At first glance, you would think that a better name for $LS$ would be least \emph{sum} of squares.  
Apparently, few people objected to removing the word ``sum'' as if the only sensible thing to do 
with $n$ numbers would be to add them.  Adding the $e_i^2$ terms is the same as using their mean (dividing 
by $n$ does not affect the minimization).  Why not replace the mean (i.e., the sum) by a median, which we know is 
very robust?  This yields the ``least median of squares'' (LMS) criterion:
\begin{equation}
\begin{array}{cc}
\mbox{minimize} & \mbox{median } \ e^2_i. \\
    \hat{a}, \hat{b} \end{array}
\end{equation}	 
It turns out the LMS fit is very robust with respect to outliers in $y$ as well as in $x$.  Its breakdown 
point is 50\%, which is the most we can ask for.  If more than half your data are bad then you cannot tell
which part is good unless you have additional information.
Figure~\ref{fig:Fig1_LMS_regress} shows what we get
using this method on the data that made the LS technique fail so badly.
	
The LMS line also has an intuitive geometric interpretation: it lies at the center of the narrowest 
strip that covers half of the data points.  By half of the points we mean $n/2 + 1$, and the thickness of 
the strip is measured in the vertical direction (i.e., Figure~\ref{fig:Fig1_LMS_geometry}).
	 
\PSfig[H]{Fig1_LMS_geometry}{Geometrical meaning of the LMS regression: the narrowest strip that
covers half the data points.}

An example of LMS regression comes from astronomy.  Astronomers often look for a linear 
relationship between the logarithm of the light intensity and the logarithm of the surface temperature of stars.  A 
scatter plot of observed quantities may look like Figure~\ref{fig:Fig1_astronomy}.  Here, the LMS line defines what is known 
as the \emph{main sequence}; the four outlying stars turned out to be red giants that do not follow the general 
trend.  The $LS$ fit produces a rather worthless compromise solution.  Here, the outliers are not so 
much errors as \emph{contamination from a different population}.

\PSfig[H]{Fig1_astronomy}{Example of robust regression in astronomy.  We see a Hertzsprung-Russell diagram of the star cluster
CYG OB1 with the least squares (dashed line) and LMS (solid line) fit.  The red giants are distorting the LS fit.  Data
taken from Rousseeuw, P. J., and A. M. Leroy (1987), \emph{Robust regression and outlier detection}, 329 pp., John Wiley and Sons, New York.}
\index{Regression!reduced major axis (RMA)|)}

\subsection{How to estimate LMS regression}
\index{LMS regression|(}
\index{Regression!LMS|(}

We can rewrite the minimization criterion as follows:
\begin{equation}
\begin{array}{cccc}
\mbox{minimize} \{ \mbox{median } \ e^2_i \}& = & \mbox{minimize} \{  \mbox{median} & ((y_i - \hat{b} x_i) - \hat{a})^2\} \\
\hat{a}, \hat{b} & & \hat{a}, \hat{b} & \end{array}
\end{equation}	 	
in the form
\begin{equation}
\begin{array}{ccc}
\mbox{minimize} \! \!  & \! \! \! \left \{ \! \! \! \phantom{\displaystyle \frac{\sum }{\ }} \! \!\! \mbox{minimize median} \right. & \! \! \! \left. (( y_i - \hat{b}x_i) - \hat{a})^2 \! \! \!\phantom{\displaystyle \frac{\sum }{\ }}\! \! \! \right \}\\*[-1ex]
\hat{b} & \hat{a} & \end{array} .
\end{equation}	 	
We will treat the two minimizations here separately.  The innermost minimization is the easy part, 
because for any given $\hat{b}$ it becomes essentially a 1-D problem, i.e., we want to find the value for $\hat{a}$ that 
minimizes the median
\begin{equation}
\begin{array}{cc}
\mbox{minimize} \! \! & \! \! \!  \left \{ \phantom{\displaystyle \frac{\sum }{\ }} \! \! \! \! \! \mbox{median } (u_i - \hat{a})^2 \right \}, \\*[-1ex]
\hat{a} & \end{array}
\end{equation}
where $u_i$ is calculated as $u_i = y_i - \hat{b} x_i$ (remember, we assumed that $\hat{b}$ was given).  This minimization problem is 
the same one we found earlier to give a good estimate of the mode.  Thus, this operation finds the mode 
of the $u_i$ data set.  We therefore need to find the $\hat{b}$ for which
\begin{equation}
E (\hat{b}) = \mbox{median} \ [(y_i - \hat{b}x_i) - \hat{a}]^2
\end{equation}	 
is minimal.  This is simply the minimization of a 1-D function $E(\hat{b})$ which is continuous but 
not everywhere differentiable.

To find this minimum we make the observation that the slope in 
the $x-y$ plane must be in the $\pm 90^{\circ}$ range (when expressed as an angle $\beta$, with $b = \tan \beta$).
We then simply perform a search for the optimal angle.  Starting 
with $\beta = -90 ^{\circ}$, we form the resulting $u_i$ and solve the 1-D minimization problem for $\hat{a}$, i.e., 
finding the LMS mode estimate $\hat{a}$.  We now increment the angle $\beta$ by $d\beta$ to, say, $-89 ^{\circ}$, and 
repeat the process.  At each step we keep track of what $E(\beta)$ is, and repeat these steps for all angles 
through $\beta = 90^{\circ}$.

\PSfig[H]{Fig1_LMS_bestslope}{Determining the best regression slope, $\hat{b} = \tan \hat{\beta}$.  The misfit function is not smooth but usually has
a minimum for the optimal slope. It is also likely to reveal several local minima that could trick a simpler search.}

Having found the slope $\hat{b}$ that gave the smallest misfit we may improve on this estimate by using a smaller step size $d\beta$ in this region 
to pinpoint the best choice for $\hat{b}$.  A plot of $E(\beta)$, shown in Figure~\ref{fig:Fig1_LMS_bestslope},
is very useful since it may tell us how 
unique the LMS regressions are: if more than one minimum are found they may indicate a possible 
ambiguity as there may be two or more lines that fit the data equally well.

	We will elaborate on the breakdown point for simple regression and illustrate it with 
a simple experiment.  Consider a data set that contains 100 good data points that exhibit a 
strong linear relation, computed from
\begin{equation}
y_i = 1.0 x_i + 2.0 + \epsilon_i \quad 1 \leq x_i \leq 4,
\end{equation}
with $\epsilon_i$ normally distributed, with $\mu = 0$ and $\sigma =0.2$.

\PSfig[h]{Fig1_LS_breakdown}{(a) Best LS fit to synthetic data set computed from a linear model with Gaussian noise.  
(b) Synthetic data set computed from a linear model with Gaussian noise, but now contaminated by points 
from another (bivariate) distribution centered on (7,2).  The LS line is pulled way off by these bad leverage points.
We can then plot the slope value as a function of the percentage of contamination.}

	Any regression technique, including $L_2$, will of course recover estimates of the slope and 
intercept that are very close to the true values 1 and 2 (Figure~\ref{fig:Fig1_LS_breakdown}a).  Then, we will start to contaminate the data 
by replacing ``good'' points with bad ones, the latter coming from a bivariate normal distribution 
with $\mu = (7, 2)$ and $\sigma_r = 0.5$.  We systematically substitute one bad point for a good point and 
recompute the regression parameters after each step.  What we find is that the LS estimate goes 
bad right away (Figure~\ref{fig:Fig1_LS_breakdown}b).  The bad points are basically bad leverage points, which we know the LS process 
cannot handle.  We keep track of the slope estimate after each substitution and graph the results
in Figure~\ref{fig:Fig1_breakdown}.
\index{Regression!breakdown point}

\PSfig[H]{Fig1_breakdown}{Breakdown plot for several regressors.  The LMS only breaks down when 50\% of the data are outlying.  
LS breaks down immediately because every point matters.}

	We call the percentage at which the slope starts to deviate significantly from the true value the \emph{breakdown point}.
The clear winner of this test is the LMS regression which 
keeps finding the correct trend until half the points have been replaced.  We should add that 
while LMS always have a breakdown point of 50\%, it is found that the effect of the outliers often
depends on the quality of the good data.  In cases where 
the good data exhibit a strong correlation the outliers do less damage than in a case where there 
is little or no correlation (Figure~\ref{fig:Fig1_contamination}).  Of course, when the correlation among the good data is minimal it is 
probably not very useful to insist that the data really exhibit a linear trend in the first place.

\PSfig[H]{Fig1_contamination}{Two data sets with the same degree of contamination.  However, one exhibit a much stronger correlation 
between the ``good'' points than the other.}
\index{LMS regression|)}
\index{Regression!LMS|)}

\subsection{How to find LMS 1-D Location (single mode)}
\index{Mode|(}

When we discussed estimates of central location we briefly mentioned that the value $\hat{x}$
that minimized
\begin{equation}
\begin{array}{cc}
\mbox{minimize} & \! \! \left \{ \mbox{median} \ (x_i - \hat{x})^2 \right \} \\
\hat{x} & \end{array}
\end{equation}
was called the LMS location and that it was a good approximation to the \emph{mode}, but how do we 
determine the LMS estimate?  It turns out that it is rather simple.  The following recipe 
will do:
\begin{enumerate}
\item	Sort the data into ascending order.
\item	Determine the shortest half of the sample, i.e., find the value for $i$ that yields the smallest of the differences $(x_{h+i} 
- x_i)$ with $h = n/2 + 1$.
\item	The LMS estimate is the midpoint of the shortest half:
\end{enumerate}
\begin{equation}
\mbox{LMS} \ = \frac{1}{2} (x_{h +i} + x_i).
\end{equation}	 	
You can empirically try this out by letting $\hat{x}$ take on all values within the data range, compute the median
of all $(x_i - \hat{x})^2$ values, and graph the curve and find its minimum (Figure~\ref{fig:Fig1_LMS}).

\PSfig[h]{Fig1_LMS}{LMS defines the mode while L2 defines the mean; both are the locations where their respective objective 
functions are minimized.}
\index{Mode|)}

\subsection{Making LMS ``analytical'' --- finding outliers}
\index{Regression!reweighted least squares|(}

	There is one problem with using the robust LMS parameters: the method is not \emph{analytical} so it 
does not lend itself easily to standardized statistical tests.  We will look into how we can 
overcome this obstacle.
	
The main problem comes from the fact that the outliers cause L$_2$ estimates to become unreliable.  We 
will avoid this problem altogether by using the best from both worlds (L$_2$ \emph{and} LMS):  We will use robust 
LMS techniques to find the best parameters in the regression model and then use the robust residuals to 
detect outliers.  Finally, we recompute weighted L$_2$ parameters with outliers given 
zero weights and other points given unit weights.  These L$_2$ estimates now represent only the 
``good" portion of the data and confidence limits and statistical tests may be based on the 
behavior of these good values.  We call this technique ``Re-weighted Least Squares'' (RLS).

\index{Outlier!identifying}
	First, we need a robust scale estimate for the residuals that will make them nondimensional.
It is customary to choose the preliminary scale estimate
\begin{equation}
s^0 =1.4826 \left( 1 + \frac{5}{n-2} \right ) \sqrt{\mbox{median } e^2_i},
\end{equation}	 	
where $e_i = y_i - \hat{y}_i$ are the residuals (i.e., misfits) at each point.
With this scale estimate we can evaluate the normalized residuals as
\begin{equation}
z_i = e_i/s^0.
\end{equation}	 
Now use these numbers to design the weights:
\begin{equation}
w_i = \left \{ \begin{array}{cl}
1, & |z_i| \leq 2.5\\
0, & \mbox{otherwise}
\end{array} \right. .
\end{equation}	 
The final LMS regression scale estimate is then given by
\begin{equation}
s^* = \sqrt{\displaystyle \sum w_i e^2_i / (\sum w_i -2)}.
\end{equation}	 	
The RLS regression parameters are therefore obtained by minimizing the weighted, squared residuals:
\begin{equation}
\min E = \sum^n_{i=1} w_i e^2_i.
\end{equation}	 	
This is simply the L$_2$ solution when only the good data are used.  As shown when we discussed 
the weighted L$_2$ regression problem, this technique will provide confidence intervals on both the 
slope and intercept and it also allows us to use the $\chi^2$-test to check whether the RLS fit is significant or not.
	
The strength of the linear relationship can again be measured by the (Pearson) correlation 
coefficient.  The LMS estimate of correlation is now given by
\begin{equation}
r = \sqrt{ 1 - \left (\frac{\mbox{median } |e_i|}{\mbox{MAD } y_i} \right )^2}
\end{equation}	 
with
\begin{equation}
\mbox{MAD } y_i = \mbox{median } | y_i - \tilde{y} |.
\end{equation}	 	
Compare this to the L$_2$ case, where
\begin{equation}
r =  \frac{s_{xy}}{s_x s_y} =
     \sqrt{ 1 - \frac{\sum (y_i - \hat{y}_i)^2}{\sum (y_i - \bar{y})^2}} = \sqrt{ 1 - \left (\frac{\bar{e}}{s_y} \right )^2}.
\end{equation}
This comparison shows that the robust estimates for ``average'' and ``scale'' have replaced the traditional
least-squares estimates in the expression for correlation.	
\index{Regression!robust|)}
\index{Robust!regression|)}
\index{Regression!reweighted least squares|)}

\section{Multiple Regression}
\index{Multiple regression|(}
\index{Regression!multiple|(}

\subsection{Preliminaries}

	Multiple regression is an extension of the simple regression problem for including more than 
one explanatory variable.  The most straightforward extension from 1-D to 2-D results in the 
determination of the constants $m_0, m_1,$ and $m_2$ in
\begin{equation}
d = m_0 + m_1 x + m_2 y,
\label{eq:plane}
\end{equation}
which you will recognize as the equation for a \emph{plane}.  A very common application of finding the 
best-fitting plane is to define and remove a regional (planar) trend from 2-D data sets so that 
local variations can be inspected and compared.  Let us say we have $n$ data points $d_i(x_i, y_i)$ in the 
plane and we want to determine the regional trend using standard $L_2$ techniques.  Using (\ref{eq:plane}) 
gives
\begin{equation}
m_0 + m_1 x_i + m_2 y_i = d_i,\quad i = 1,n
\end{equation}
or $\mathbf{G\cdot m=d}$.  We know how to solve these $L_2$ problems now and quickly state that the solution is 
given by the linear system:
\begin{equation}
\mathbf{G}^T \mathbf{Gm = G}^T \mathbf{d}
\end{equation}
or 
\begin{equation}
\mathbf{	N\cdot m = v}.
\end{equation}
We can find the $\mathbf{N}$ and $\mathbf{v}$ by performing the matrix operations (below, all sums are implicitly from $i = 1$ to $n$)
\begin{equation}
\left [ \begin{array}{ccc}
n & \displaystyle \sum x_i & \displaystyle \sum y_i \\*[2ex]
\displaystyle \sum x_i & \displaystyle \sum x^2_i & \displaystyle \sum y_i x_i \\*[2ex]
\displaystyle \sum y_i & \displaystyle \sum x_i y_i & \displaystyle \sum y^2_i \\*[2ex]
\end{array} \right ]
\left[ \begin{array}{c}
m_0 \\*[2ex]
m_1 \\*[2ex]
m_2 
\end{array} \right]  = 
\left[ \begin{array}{c}
\displaystyle \sum d_i \\*[2ex] \displaystyle \sum x_i d_i\\*[2ex] 
\displaystyle \sum y_i d_i
\end{array} \right ].
\end{equation}	 
To verify this system, note that
\begin{equation}
\mathbf{N} = \mathbf{G}^T \mathbf{G} = \left [ \begin{array}{ccccc}
1 & 1 & 1 & \cdots & 1 \\*[2ex]
x_1 & x_2 & x_3 & \cdots & x_n \\*[2ex]
y_1 & y_2 & y_3 & \cdots & y_n \end{array}  \right ] \cdot 
\left [ \begin{array}{ccc}
1 & x_1 & y_1 \\*[2ex]
1 & x_2 & y_2 \\*[2ex]
\vdots & \vdots & \vdots \\*[2ex]
1 & x_n & y_n
\end{array} \right ]
= \left [ \begin{array}{ccc}
n & \displaystyle  \sum x_i & \displaystyle \sum y_i \\*[2ex]
\displaystyle \sum x_i & \displaystyle \sum x^2_i & \displaystyle \sum x_i y_i \\*[2ex]
\displaystyle \sum y_i & \displaystyle \sum x_i y_i & \displaystyle \sum y^2_i
\end{array} \right ]
\end{equation}
and
\begin{equation}
\mathbf{v} = \mathbf{G}^T\mathbf{d} = \left [ \begin{array}{ccccc}
1 & 1 & 1 & \cdots & 1 \\*[2ex]
x_1 & x_2 & x_3 & \cdots & x_n \\*[2ex]
y_1 & y_2 & y_3 & \cdots  & y_n \end{array}  \right ] \cdot
\left [ \begin{array}{c}
d_1 \\
d_2 \\
\vdots \\
d_n
\end{array} \right ] = 
\left [ \begin{array}{c}
\displaystyle \sum d_i \\*[2ex]
\displaystyle \sum x_i d_i \\*[2ex]
\displaystyle \sum y_i d_i 
\end{array}
\right ] .
\end{equation}
In many situations we have a gridded data set where we have observations on an equidistant 
lattice.  We can then find the $\mathbf{N}$ matrix analytically since the geometry is so simple.
\begin{example} 
We will examine data set given in Table~\ref{tbl:xy_grid}.
\begin{table}[H]
\center
\begin{tabular}{|c||r|r|r|r|r|} \hline
\bf{X/Y} & -2 & -1 & 0 & 1 & 2 \\ \hline \hline
2 & -1 & 0 & 3 & 2 & 4 \\ \hline
1 & 1 & -1 & 2 & 2 & 3 \\ \hline
0 & 0 & 0 & 1 & 2 & 2 \\ \hline
-1 & -2 & 0 & 1 & 1 & 2  \\ \hline
-2 & -1 & -1 & 0 & -1 & 1 \\ \hline
\end{tabular}
\caption{Simple data set of measured values of $d(x,y)$ on an equidistant grid.}
\label{tbl:xy_grid}
\end{table}
In this case, we find $x_i$ and $y_i$  to be centered on 0.  This means both the terms $\sum x_i = \sum y_i = 0$ by 
definition.  Furthermore, since $\sum {x_i y_i}$ can be written
\begin{equation}
\sum^n_{i=1} x_i y_i = \sum^{x=2}_{x=-2} \ \sum^{y=2}_{y=-2} x y = \sum ^2_{x = -2} x \  \sum^{y=2}_{y=-2} y = 0 
\end{equation}	 
we are left with
\begin{equation}
\left [ \begin{array}{rrr}
25 & 0 & 0 \\
0 & 50 & 0 \\
0 & 0 & 50 \end{array} \right ]
\left [ \begin{array}{c}
m_0\\ m_1\\ m_2 \end{array} \right ] = \left [ \begin{array}{c}
20 \\ 38 \\ 25 \end{array} \right ],
\end{equation}	 
or directly
\begin{equation}
d = \frac{4}{5} + \frac{19}{25}x + \frac{1}{2} y.
\label{eq:example_solution}
\end{equation}
The data and the best-fitting plane are shown in Figure~\ref{fig:Fig1_multregress}.
\PSfig[h]{Fig1_multregress}{Data points (red cubes) used in Example~\thechapter.\theexample, with the least-squares solution
shown as a transparent green plane.  Small black dots indicate prediction from the solution in (\ref{eq:example_solution}),
while vertical lines connect data points with model predictions.}
\end{example}
It should surprise no one that in the general case (i.e., no organized grid structure) it is advantageous to translate 
the data to a new origin $\bar{x},\bar{y}$   and operate on the adjusted coordinates
\begin{equation}
u_i = x_i - \bar{x}, v_i = y_i - \bar{y}	 .
\end{equation}
The system to solve is then
\begin{equation}
\left [ \begin{array}{ccc}
n & 0 & 0 \\*[2ex]
0 & \displaystyle \sum u^2_i & \displaystyle \sum u_iv_i \\*[2ex]
0 & \displaystyle \sum u_i v_i & \displaystyle \sum v^2_i
\end{array} \right ] 
\left [ \begin{array}{c}
m_0\\ m_1\\ m_2 \end{array} \right ] = 
\left[ \begin{array}{c} \displaystyle \sum d_i \\*[2ex] \displaystyle \sum u_i d_i \\*[2ex] \displaystyle \sum v_i d_i \end{array} \right ].
\end{equation}	 
Since the first equation directly gives us $m_0 = \bar{d}$, we are left with the simple $2\times 2$ system
\begin{equation}
\left[ \begin{array}{cc}
\displaystyle \sum u^2_i & \displaystyle \sum u_i v_i \\*[2ex]
\displaystyle \sum u_i  v_i & \displaystyle \sum v^2_i
\end{array} \right ]
\left [ \begin{array}{c}
m_1\\ m_2 \end{array} \right ] = 
\left [ \begin{array}{c}
\displaystyle \sum u_i d_i \\*[2ex]
\displaystyle \sum v_i d_i 
\end{array}
\right ].
\end{equation}
	Going up one more dimension means we want to find a hyper plane in a 4-D coordinate 
system.  That situation become difficult to envision but easy to construct.  As an example, consider 
measurements of temperature of points described by  $(x,y,z)$ triplets.  One can easily find the 
equation of the hyper-plane
\begin{equation}
	T = m_0 + m_1 x + m_2 y + m_3 z
\end{equation}
that best fits the data in a least square sense.

\subsection{Multiple regression}

In general, we will be considering a linear regression 
with $k+1$ independent variables, and our regression model will be of the form
\begin{equation}
m_0 + m_1 x_1 +  m_2 x_2 + \cdots + m_k x_k + \epsilon = d.
 \end{equation}	 	
Our model states that the observations $d_i$ can be explained by a constant term $m_0$ plus a linear 
combination of $k$ variables, with each variable having its own ``slope''.  Thus, $m_4$ describes how 
fast $d$ changes when $x_4$ changes, holding all other $x_i$ fixed.  As usual, we explain the misfit as 
being due to deviations $\epsilon_i$, which we hope are normally distributed with zero mean.
	
In many situations it is obvious which values are the observations and which are the 
variables.  In our example with temperatures in space we wanted to  find how $T$ varies as a 
function of position.  However, in other cases it may be less clear-cut.  E.g., if we measure (depth, 
porosity, permeability, water content) in a core, we will probably let depth be a variable but which 
of the others do we pick as our $d$ observation?  In the general case this becomes arbitrary.  It also becomes 
arbitrary to measure the misfit in the $d$-direction only.  The extension of orthogonal regression to 
multiple regression is then a solution. In that case we would want to minimize the shortest distance 
between the data points and the hyper plane.  To prevent unseemly bleeding from the ears, we will not attempt this approach here.
	
Once we have selected our $d$-values (and used the symbol $x_{ij}$ to represent the $i$'th observation
of the $j$'th variable), we set up the problem in matrix form as
\begin{equation}
\left [ \begin{array}{ccccc}
1 & x_{11} & x_{12} &  \cdots & x_{1k} \\
1 & x_{21} & x_{22} &  \cdots & x_{2k} \\
\vdots & \vdots & \vdots & \cdots & \vdots \\
1 & x_{n1} & x_{n2} & \cdots & x_{nk} \end{array}\right ]
\left [ \begin{array}{c}
m_0\\ m_1\\ \vdots \\ m_k \end{array} \right ]  = 
\left [ \begin{array}{c}
d_1 \\ d_2\\ \vdots \\  d_n
\end{array} \right ],
\end{equation}	 
or $\mathbf{G\cdot m = d}$.  This is nothing more than our general least squares problem, whose solution is known to be 
\begin{equation}
\mathbf{m} = [ \mathbf{G}^T\mathbf{G} ]^{-1} \mathbf{G}^T\mathbf{d}.
\end{equation}	 	
However, in multiple regression we want to determine the \emph{relative importance} of the independent variables 
as predictors of the dependent variable $d$.  The values of the regression coefficients tells us little, 
since they depend on the units chosen.  Also, if the $\bar{x}_j$'s are very different in magnitude we may lose 
precision due to round-off errors.  Consequently, we choose to transform our data into normal scores
\begin{equation}
x'_{ij} = \frac{x_{ij} - \bar{x}_j}{s_j} \mbox{ and } d'_i = \frac{d_i - \bar{d}}{s_d}.
\end{equation}
where
\begin{equation}
s_j = \sqrt{ \frac{1}{n-1} \sum ^n _{i=1} (x_{ij} - \bar{x}_j)^2} \mbox{ and } s_d = \sqrt{ \frac{1}{n-1} \sum ^n _{i=1} (d_i - \bar{d})^2}
\end{equation}	 
are the sample standard deviations of the $j'$th variable and $d$, respectively.  To populate the normal equation matrix we must form 
$\mathbf{G}^T\mathbf{G}$ and carry out the summations.  
You will remember that the form of the normal matrix equations is
\begin{equation}
\left [ \begin{array}{ccccc}
n &  \sum x_1 &  \sum x_2 & \cdots &  \sum x_k \\[5pt]
 \sum x_1 &  \sum x^2_1 & \sum x_1 x_2 & \cdots &  \sum x_1 x_k \\[5pt]
 \sum x_2 &  \sum x_2 x_1 & \sum x^2_2 & \cdots &  \sum x_2 x_k \\[5pt]
\vdots & \vdots & \vdots & \cdots & \vdots \\[5pt]
 \sum x_k &  \sum x_k x_1  & \sum x_k x_2 & \cdots &  \sum x^2_k 
\end{array} \right ] 
\left [ \begin{array}{c}
m_0\\ m_1\\ \vdots \\ m_k \end{array} \right ]  = 
\left [
\begin{array}{c}
 \sum d \\[5pt]  \sum x_1 d \\[5pt]  \sum x_2 d \\[5pt] \vdots \\[5pt] \sum x_k d
\end{array} \right ] ,
\label{eq:mregsystem}
\end{equation}
with the sums implied to include all the data points (i.e., over $i = 1,n$).
The effect of normalizing is two-fold:
\begin{enumerate}
\item	Because the linear sums over $x_i$ and $d_i$ now are zero, $m_0' = 0$, and we must find only $k$ coefficients $m_1'$, ..., $m_k'$ instead of $k + 1$, as originally 
intended.  Consequently, the first row and first column of the system (\ref{eq:mregsystem}) are removed.
\item	The rest of the matrix becomes proportional to the correlation matrix, $\mathbf{C}$.
\end{enumerate}
For instance, examining the terms in the first equation in the normal system, i.e.,
\begin{equation}
\sum \frac{(x_{i1} - \bar{x}_1) ^2}{s^2_1} \quad \sum \frac{(x_{i1} - \bar{x}_1) (x_{i2} - \bar{x}_2)}{s_1s_2} \quad \cdots \quad \sum 
\frac{(x_{i1} - \bar{x}_1)(d_i - \bar{d})}{s_1s_d}
\end{equation}	 
we note they are simply
\begin{equation}
\frac{(n-1)s^2_1}{s^2_1}  \quad \frac{(n-1)s_{12}}{s_1 s_2} \quad \cdots \quad \frac{(n-1)s_{1d}}{s_1 s_d}.
\end{equation}	 
Since the constant $(n-1)$ appears in all terms we can delete it and find
\begin{equation}
\left [ \begin{array}{ccccc}
1 & r_{12} & r_{13} & \cdots & r_{1k} \\
r_{21} & 1 & r_{23} &  \cdots & r_{2k} \\
r_{31} & r_{32} & 1 &  \cdots & r_{3k} \\
\vdots & \vdots & \vdots & \ddots & \vdots \\
r_{k1} & r_{k2} & r_{k3} & \cdots & 1 \end{array} \right ]
\left [
\begin{array}{c}
m'_1 \\ m'_2 \\ \vdots \\ m'_k \end{array} \right ] =
\left [ \begin{array}{c}
r_{1d} \\ r_{2d} \\ r_{3d} \\ \vdots \\ r_{kd}
\end{array} \right ]
\end{equation}	 	
or $\mathbf{C} \cdot \mathbf{m'} = \mathbf{r}$.  Solving $\mathbf{m' = C}^{-1}\mathbf{r}$ we recover the unscaled regression parameters
\begin{equation}
m_j = m'_j \frac{s_d}{s_j},\quad j = 1,k,
\end{equation}	 
and we can reconstruct the missing intercept via
\begin{equation}
m_0 = \bar{d} - m_1 \bar{x}_1 - m_2 \bar{x}_2 - \cdots - m_k \bar{x}_k.
\end{equation}	 
As an indicator of the goodness-of-fit we use the \emph{coefficient of multiple regression},
\begin{equation}
R^2 = SS_R/SS_T = \sum^n_{i=1} (\hat{d}_i - \bar{d})^2 / \sum^n_{i=1} (d_i - \bar{d})^2,
\end{equation}	 
where $\hat{d}_i = d(x_i)$ is the predicted value.  For simple 2-D regression,
\begin{equation}
R^2 = r^2 = \frac{s_{xd}}{s^2_x s^2_d}.
\end{equation}	 
	While a simple inspection of the $m_j'$ can tell you which parameters seem to be most important, 
it is generally not enough to determine the best regression equation.  Obviously, if any of the $m_j'$ 
are close to or equal to zero we can say that the corresponding variable $x_j$ is largely \emph{irrelevant} to the 
regression and remove it from the system.  At other times, some of the variables may be linearly 
related to each other, leading to \emph{redundant} variables that should be removed.  One approach is to 
evaluate all possible regressions using all combinations of $x_j$'s and determine how many variables are 
necessary.
\index{Redundant variable}
	We can use an ANOVA test to resolve whether a coefficient is significant or not.  Suppose we have 
carried out a multiple regression for $p$ independent variables.  Next, we add $q$ additional variables and 
repeat the regression.  We would like to test whether these new variables are significant.  We construct
the ANOVA table given as Table~\ref{tbl:F_ANOVA}.
\begin{table}[H]
\center
\begin{tabular}{|l|c|c|c|c|} \hline
\bf{Source of Variation} & \bf{Degrees of Freedom} & \bf{Sums of Squares} & \bf{Mean Square} & $F$ \\ \hline
{1st regression} & $p$ & $SS_{R1}$ & &\\[3pt] \hline
{2nd regression} & $p + q$ & $SS_{R2}$ & & \\[3pt] \hline
{Difference between} & $q$ & $\Delta SS_R = $ & $MSR = \Delta SS_{R}/q$ & $\frac{MSR}{MSE}$ \\[3pt]
{1st and 2nd regression} & & $SS_{R2} - SS_{R1}$ & & \\[3pt] \hline
{Residuals} & $n - p - q - 1$ & $SSE$ & $MSE = SSE/(n - p - q - 1)$ & \\[3pt] \hline
{Total} & $n - 1$ & $SS_T$ & & \\ \hline
\end{tabular}
\caption{ANOVA table used to determine whether or not the addition of $q$ extra parameters results in
a significant reduction in misfit (i.e., a significant improvement in variance explained).}
\label{tbl:F_ANOVA}
\end{table}
\index{\emph{F}-test}
\index{Test!\emph{F}}
\noindent
The observed $F$ value is
\begin{equation}
F = \frac{\Delta SS_{R}/q}{SSE/(n-p-q-1)}.
\end{equation}	 
With explained $SS$ in \% defined as $ESS = 100 R^2$ we find
\begin{equation}
F = \frac{(ESS_{p+q} - ESS_p)/q}{(100 - ESS_{p+q})/(n-p-q-1)}.
\end{equation}	 	
Comparing the observed $F$ statistic to a table of critical values (see Appendix~\ref{sec:Ftables})
will tell us if the new $q$ parameters produce a significant improvement or not.
\begin{example}	
\PSfig[h]{Fig1_CanadaGold}{Geological map of part of Superior province, Canadian shield (After G.S.C.-Map no 1250A by R.J.W. Douglass).
Area of interest is indicated by the heavy outline.}
We will use multiple regression to look for a linear relation between the occurrences of gold 
and various lithological units in the area west of Quebec (Agterberg, 1974).  By overlying a grid on the geological 
map, we can estimate the density of gold occurrence per grid cell and find the area percentage of 
the rock units in the same cell.  The gold densities are our ``observed'' data $d_i, i=1,n$ and the rock types 
our ``variables'' $x_j, j = 1,k$.  We chose six types of rocks ($k=6$) for the mapping, as listed in Table~\ref{tbl:rocks}.
\begin{table}[H]
\center
\begin{tabular}{|c|l|} \hline
\bf{Variable} & 	\bf{Lithology} \\ \hline
$x_1$ & 	Granitic rocks, gneisses, acidic intrusive rocks \\ \hline
$x_2$ & 	Mafic intrusive rocks (gabbros and diorites)\\ \hline
$x_3$ &		Ultramafics\\ \hline
$x_4$ & 	Early Precambrian sedimentary rocks\\ \hline
$x_5$ &  	Acidic volcanics (rhyolites and pyroclastics)\\ \hline
$x_6$ & 	Mafic volcanics (basalts and andesites)\\ \hline
\end{tabular}
\caption{List of variable names and the corresponding rock types used in the regression for gold densities.}
\label{tbl:rocks}
\end{table}
The counting resulted in $n = 113$ cells, so $\mathbf{G}$ has dimensions $113 \cdot 7$ (i.e., six factors plus the intercept).  However, 
because of the constraint that
\begin{equation}
\sum^6_{j=1} x_{ij} = 100
\end{equation}	 
for all blocks (required to equal a total of 100\%), the rock type percentages are linearly dependent 
and we will not be able to invert $\mathbf{G}^T\mathbf{G}$.  We will instead obtain all multiple regression 
solutions where we regress $d$ on $x_1$, $d$ on $x_2$, etc., for all six variables.  Next, we try all possible 
combinations of two variables, then combinations of three, four, and finally five (using all six is not possible
since they are dependent.)  We find that we need to perform $N_r$ individual regression analyses, given by
\begin{equation}
N_r = \left ( \begin{array}{c}
6 \\ 1 \end{array} \right ) + \left ( \begin{array}{c}
6 \\ 2 \end{array} \right ) + \left ( \begin{array}{c}
6 \\ 3 \end{array} \right ) + 
\left ( \begin{array}{c}
6 \\ 4 \end{array} \right ) + 
\left ( \begin{array}{c}
6 \\ 5 \end{array} \right )  = \sum ^5 _{j=1} 
\left ( \begin{array}{c}
6 \\ j \end{array} \right ) = 62.
\end{equation}	 
This number $N_r$ goes up rapidly with the number of coefficients.  We make a table of the 
results where we tabulate the parameter numbers and the associated ESS (Table~\ref{tbl:mregress62}).
\begin{table}[H]
\center
\begin{tabular}{|cr|cr|cr|cr|} \hline
$x_j$	& 	\bf{ESS} (\%) &	$x_j$	& 	\bf{ESS} (\%) &	$x_j$	& 	\bf{ESS} (\%) &	$x_j$		& 	\bf{ESS} (\%) \\ \hline
1	&	8.64	&	3,5	&	16.29	&	2,3,5	&	16.43	&	1,3,4,6		&	25.61 \\ \hline
2	&	1.45	&	3,6	&	0.33	&	2,3,6	&	1.79	&	1,3,5,6		&	28.37 \\ \hline
3	&	0.19	&	\bf{4,5}	&	\bf{28.15}	&	2,4,5	&	28.18	&	1,4,5,6		&	28.61 \\ \hline
4	&	10.48	&	4,6	&	10.56	&	2,4,6	&	12.46	&	\bf{2,3,4,5}		&	\bf{29.40} \\ \hline
\bf{5}	&	\bf{15.56}	&	5,6	&	16.07	&	2,5,6	&	16.14	&	2,3,4,6		&	12.78 \\ \hline
6	&	0.17	&	1,2,3	&	8.95	&	\bf{3,4,5}	&	\bf{29.32}	&	2,3,5,6		&	16.84 \\ \hline
1,2	&	8.85	&	1,2,4	&	14.33	&	3,4,6	&	10.96	&	2,4,5,6		&	28.18 \\ \hline
1,3	&	8,76	&	1,2,5	&	19.21	&	3,5,6	&	16.72	&	3,4,5,6		&	29.33 \\ \hline
1,4	&	13.54	&	1,2,6	&	22.60	&	4,5,6	&	28.15	&	\bf{1,2,3,4,5}	&	\bf{29.41} \\ \hline
1,5	&	18.86	&	1,3,4	&	13.79	&	1,2,3,4	&	14.56	&	1,2,3,4,6	&	29.41 \\ \hline
1,6	&	22.52	&	1,3,5	&	14.40	&	1,2,3,5	&	19.83	&	1,2,3,5,6	&	29.41 \\ \hline
2,3	&	1.58	&	1,3,6	&	22.58	&	1,2,3,6	&	22.65	&	1,2,4,5,6	&	29.41 \\ \hline
2,4	&	12.42	&	1,4,5	&	28.19	&	1,2,4,5	&	28.24	&	1,2,4,5,6	&	29.41 \\ \hline
2,5	&	15.65	&	1,4,6	&	24.87	&	1,2,4,6	&	27.45	&	2,3,4,5,6	&	29.41 \\ \hline
2,6	&	1.68	&	1,5,6	&	28.34	&	1,2,5,6	&	29.37	&			&	      \\ \hline
3,4	&	10.85	&	2,3,4	&	12.72	&	1,3,4,5	&	29.32	&			&	      \\ \hline
\end{tabular}
\caption{Percentage explained sum of squares (\emph{ESS}) for 62 possible regressions, with density of gold
occurrences regressed on up to five lithological variables. Bold entries are the best combinations (from \emph{Agterberg}, 1974).}
\label{tbl:mregress62}
\end{table}
Examining all the regression results we will focus on each single combination that gave the highest ESS for that number of
parameters (bold entries in Table~\ref{tbl:mregress62}); these are isolated in the separate Table~\ref{tbl:xj_fits}.
\begin{table}[H]
\center
\begin{tabular}{|l|c|} \hline
$x_j$ combination & \bf{ESS} (\%) \\ \hline
5 & 15.56\\ \hline
4, 5 & 28.15 \\ \hline
3, 4, 5 & 29.32 \\ \hline
2, 3, 4, 5 & 29.40 \\ \hline
1, 2, 3, 4, 5 & 29.41 \\ \hline
\end{tabular}
\caption{Each row shows the best parameter combination of 1--5 variables and the corresponding percentage of
explained variation (ESS).}
\label{tbl:xj_fits}
\end{table}
We are now in the position to test the solutions for significance.  First, we will take a look at the best
one-term $(x_5)$ solution.  With $p = 0, q = 1, n = 113$, we find
\index{\emph{F}-test}
\index{Test!\emph{F}}
\begin{equation}
F = \frac{(15.56 - 0)/1}{(100 - 15.56)/(113 -1 -1)} = 20.46.
\end{equation}	 
From the table, $F_{0.95,1,111} = 3.93$, so it is clear that $x_5$ (acidic volcanics) is a significant component.  Will the 
addition of $x_4$ lead to a significant improvement?  Now, $p = 1, q = 1$, so 
\begin{equation}
F = \frac{(28.15 - 15.56)/1}{(100 - 28.15)/(113 - 1 -1 -1 )} = 19.72.
\end{equation}
This also far exceeds $F_{0.95,1,110} = 3.93$.  Going one step further, we check if $x_3$ should be 
included.  Here, $p = 2$ and $q = 1$, so
\begin{equation}
F = \frac{(29.32 - 28.15)/1}{(100 - 29.32)/(113 - 2 - 1 -1 )} = 1.80.
\end{equation}	 
The critical $F_{0.95,1,109}$ is still 3.93, so we must conclude that $x_3$ is an irrelevant variable, and 
obviously there is no need to consider any other of the remaining $x_j$'s that gave poorer fits still.  The result of this analysis 
is that the density of gold occurrences may be explained by a linear model that depends on the 
presence of acidic volcanics and Precambrian sediments in the area.  
However, the correlation is not all that impressive $(R^2 = 28.15\%)$, suggesting that rock type 
alone is but one possible indicator of prospective regions.  Applying this technical concept to a
unexplored area may produce some disappointment.
\end{example}
\index{Multiple regression|)}
\index{Regression!multiple|)}

\clearpage
\section{Problems for Chapter \thechapter}

\begin{problem}
The table \emph{hawaii.txt} contains triplets of (distance, age, $\Delta$age) values
obtained along the Hawaiian seamount chain.  Distance is small-circle distance from Kilauea (in km),
age is radiometric seamount/island dates (in My) with associated one-sigma uncertainties (in My).
You will also need the \texttt{regress\_ls.m} function for this problem.
\begin{enumerate}[label=\alph*)]
\item  Plot the data points with error bars (hint: see help \texttt{errorbar}).
\item  Find the simple least squares regression line, state the parameter values, and overlay the predictions on your plot.
\item  Find the weighted least squares regression line, state the parameter values, and overlay the predictions on your plot.
\item What is the geological meaning of the slopes and intercepts?  Comment on the values.
\end{enumerate}
\end{problem}

\begin{problem}
During a calibration, nine pairs of readings were obtained with a UV spectro-photometer used to measure chemical oxygen demand
in waste water (in mg/l; see file \emph{COD.txt}) versus UV absorbance.  Remove the mean from each coordinate set and
determine the least-squares regression fit.
Plot the data and the least-squares regression line.  Compute the
uncertainty in the slope using (\ref{eq:err_in_slope}) and plot the two regression lines going through the origin that
correspond to this range in slopes.  Finally, determine the orthogonal regression solutions (both Major Axis and Reduced Major Axis)
and plot them as well.  Do these fall within the
range of acceptable least-squares regression trends?
\end{problem}

\begin{problem}
	\newcounter{Conrad}
	\setcounter{Conrad}{\theproblem}
	\newcounter{Conradchap}
	\setcounter{Conradchap}{\thechapter}
The file \emph{c2407.txt} contains distance $x$ (in km), bathymetry $z$ (in m), and crustal age $t$ (in Myr) along a
segment across the Mid-Atlantic Ridge.  
\begin{enumerate}[label=\alph*)]
\item	Find the deterministic component given by the theoretical prediction
$$
z = d_r + c \sqrt{t}.
$$ 
	and report the parameters.  Note $z$ is a function of age, not distance.
\item	Plot the bathymetry and your best-fitting deterministic component on the same plot as functions of distance (not age).
\end{enumerate}
\end{problem}

\begin{problem}
\newcounter{RSL}
\setcounter{RSL}{\theproblem}
The files \emph{hilo.txt} and \emph{honolulu.txt} contain tide gauge 
readings (the first column is time (in years), the second is monthly average values in mm relative to some
arbitrary origin) from Hilo and Honolulu harbors, respectively.  It is 
believed that the observations represent the sum of two phenomena: (1) eustatic (global) sea-level 
variations, and (2) tectonic subsidence of the Big Island.
\begin{enumerate}[label=\alph*)]
\item Plot the data.  Use \texttt{regress\_ls.m} to determine the regression lines for the two series.  Superimpose 
these lines on the graphs and label the plots with the values of the regression slopes.  What do 
these slopes represent?
\item Assuming Oahu is tectonically stable, what is the tectonic subsidence rate at Hilo?
\item Alternatively, we only use the time period that the two data series have in common and directly 
subtract the Honolulu series from the Hilo series.  What is the tectonic subsidence rate based on 
these differences, and how does it compare to your answer in (2)?
\end{enumerate}
\end{problem}

\begin{problem}
	\index{Reweighted least squares (RLS)}
Data sets showing the separate weights of an animal and its brain present a strong correlation in log-log space (\emph{bb\_weights.txt}).
Use the methodology of ``reweighted least squares'' (RLS) to determine this regression and calculate residual $z$-scores
relative to the RLS regression. With a threshold of 2.5, which animals depart from the simple trend? How would you
explain these outliers?
\end{problem}

\begin{problem}
	The file \emph{sherwood.txt} contains a series of petrophysical measurements on samples
	of the Triassic Sherwood sandstone [from Lourenfemi, M.O., 1985, \emph{Math Geol., 17}, 845--452.]
	The five columns contain permeability (mm/s), porosity ($\phi$), matrix
	conductivity, true formation factor, and induced polarization.  Using multiple regression,
	how many of the last four parameters are significant in a regression to explain the permeability,
	assuming a 95\% of confidence?
\end{problem}

%  $Id: DA1_Chap7.tex 670 2018-12-20 20:37:20Z pwessel $
%
\chapter{SEQUENCES AND SERIES ANALYSIS}
\label{ch:sequences}
\epigraph{``If you can't explain it simply, you don't understand it well enough.''}{\textit{Albert Einstein, Physicist}}
\index{Sequence}
\index{Data!sequence}
	The geosciences are replete with observational data that can be viewed as ordered sequences.  
Their single most important property is that they form a \emph{sequence}, and the \emph{positions} where 
data points occur within the sequence are paramount.  Contrast that arrangement to a set of repeated 
measurements of some quantity, say ten determinations of sandstone densities.  Our goal of 
determining the average density is not affected by how we order our data --- the order in the 
sequence is not important.  Sequential data therefore often consist of series with \emph{pairs} of 
variables:  The first indicates the position in the sequence, the other gives the observation.  A 
special family of such sequences consists of those in which an observation is given as a function of \emph{time}.  
The analysis of such data is traditionally called \emph{time series analysis}.  In this book we will 
extend these techniques to include ``space series'' as well, i.e., time and distance will be 
considered interchangeable, and we will discuss such data and the methods used to analyze them at
length in Chapter~\ref{ch:spectralanallysis}.
 
      We can think of many examples of natural data that are time --- or space ---
sequences, e.g., a series of temperatures as a function of time, tide gauge readings taken over  a 
long period, topography measured along a transect, and much more.  While such data sets will be our 
main concern, we should not forget that much sequential data do not have the format 
of a ``time''-series.  We might be considering a stratigraphic sequence consisting of the lithologic 
states encountered in a sedimentary succession.  The stratigraphy might show a cyclothem of 
shale--sandstone--shale--sandstone--shale--coal, etc., going from top to bottom.  We would like to 
investigate the significance of the succession, but cannot put a meaningful scale on the sequence.  
It is clear that the succession of lithologies represents changes over time, but we have no way of 
estimating the time scale.  Could we use thickness as a proxy for time?  Thickness is certainly related 
to time through sedimentation rates, but these are known to vary.  Additional complications include 
hard-to-estimate effects like compaction and erosion.  Furthermore, the thicknesses are likely to change 
significantly from location to location.  Thus, if we use thickness or any other measure of down-hole
position it may obscure the examination of the \emph{successions}, which is the primary objective of interest. 
For example, consider an observation that sandstone is the second state and coal the sixth state
in a sequence.  Clearly, this relationship has no meaning that can be 
expressed numerically, e.g., ``sixth'' is not $3 \times$ ``second''.  Obviously, we have here a problem of a 
different nature than the usual time-series analysis mentioned above.  

	Yet another type of sequence is a series of \emph{events}.  Such data may be historical records of 
earthquakes in California, volcanic eruptions of Mauna Loa, or reversals of the Earth's magnetic field.
In these cases, the data simply consist of the time interval \emph{between} events or a cumulative length of time over
which the events occurred, and special analysis techniques are required.

Thus, the nature of the sequential data and the type of sequence determine what questions we 
may hope to answer by subjecting our data to analysis.  The purpose of any of these methods is to 
facilitate answers to questions such as these:
\begin{enumerate}
\item	Are the data random, or do they exhibit a trend or pattern?
	\item	If there is a trend, what form does it have?
	\item	Are there any periodicities in the data?
	\item	What can be estimated or predicted from the data?
	\item	Are there other questions specific to the situation?
\end{enumerate}

	We shall see that while we often are concerned with the analysis of a single sequence of data, there 
are many instances in which we want to compare two or more sequences.  One obvious example 
to geologists is \emph{stratigraphic correlation}, based on either lithologic sections or well log data.  Sequence 
correlation may speed up routine correlations and detect subtle correlations which may be hard to 
detect by eye.

The methods for comparing two or more sequences can be grouped into two 
broad classes.
In the first, the exact position in a sequence matters, and a correlation is only 
significant if it takes place at the correct location.  One example is the comparison of an X-ray 
diffraction chart with standard charts in attempts to recognize minerals.  The comparison can 
only take place at certain angles.  If the shape of a spectral peak centered on 20$^\circ$ in the data looks 
exactly like the bump at 30$^\circ$ in the standard chart it is of no significance: both peaks would have 
to occur at the same angle.

     In the other class of methods the absolute position is not important, only relative position 
matters.  These processes, like \emph{cross-correlation}, are very similar to the mental process of 
geologic correlation.  However, these methods are limited because they cannot take stretching 
and compression of the scale into account.  Nevertheless, in many problems there is no distortion and we may 
use such techniques with some success.

\section{Markov Chains}
\index{Markov chains|(}

As mentioned above, many geological experiments result in data sequences consisting of 
ordered successions of \emph{mutually exclusive} states.  We already mentioned the lithology variations 
in stratigraphic sections.  Other examples include:
\begin{itemize}
\item	The changes in minerals across a line in a thin-section.
\item	Drill holes through zoned ore bodies.
\end{itemize}
The observations may be obtained at evenly spaced intervals, or we may simply register the 
position when a change of state occurs.  In the first case we would expect repeated states; the 
latter will obviously not contain such runs since we only record changes of state. 

     Such data may be subjected to \emph{cross-association} and/or \emph{auto-association} techniques, but right 
now we are primarily concerned with the nature of transitions rather than the relative positions of 
states in the sequence.  Therefore we will, for the moment, pay less attention to the positions of observations 
within the succession and instead concentrate on acquiring information about the \emph{tendency} 
of one state to follow another.
\begin{example}
\PSfig[H]{Fig1_lithology}{Example of a stratigraphic section with four separate lithologies.
Assessing the lithology at every 1-foot interval down the section resulted in 62 states and thus we recorded 61 transitions of state.}

     Let us look at an example of a stratigraphic section.  Here, we have determined the lithology 
at one foot intervals down a stratigraphic section.  This exercise resulted in a sequence of states.
	We find four mutually exclusive states: A) sandstone, B) limestone, C) shale, and 
D) coal.  There are 62 observed states, hence 61 transitions.  We tabulate the transitions in a 
4 x 4 matrix, since we have four possible transitions for each of the four states.  E.g., from sandstone 
(A) we may change to A, B, C, or D, since repeated states may occur.  The \emph{transition frequency 
matrix} $\mathbf{A}$ given in Table~\ref{tbl:markov1} expresses all the observed possibilities.
\index{Transition frequency matrix}
\index{Matrix!transition frequency}
\begin{table}[h]
\center
\begin{tabular}{|c|r|c|r|c|c|}
\hline
 & \bf{A} & \bf{B} & \bf{C} & \bf{D} & \bf{Row Total}\\ \hline
\bf{A} & 17 & 0 & 5 & 0 & 22 \\ \hline
\bf{B} & 0 & 5 & 2 & 0 & \, 7 \\ \hline
\bf{C} & 5 & 2 & 17 & 3 & 27 \\ \hline
\bf{D} & 0 & 0 & 3 & 2 & \, 5 \\ \hline
\bf{Column total} & 22 & 7 & 27 & 5 & 61 \\ \hline
\end{tabular} 
\caption{Transition frequency matrix for the transitions seen in Figure~\ref{fig:Fig1_lithology}.}
\label{tbl:markov1}
\end{table}
We can now see that element $a_{ij}$ reads ``number of transitions from state $i$ to state $j$''.
For our section, the matrix is symmetric.  However, in general this will not be the case, so $a_{ij} \neq   a_{ji}$.

	The tendency for one state to succeed another can be made clearer by converting the 
frequencies to percentages or fractions.  We do this by dividing each row by its row total.  These 
percentages may be considered conditional probabilities in that they measure the probability 
that state $j$ will follow \emph{given} that the present state is $i$, which we write as
$P(j|i)$ or $P(i \rightarrow  j)$.  The resulting \emph{transition probability matrix} $\mathbf{P}$
is given by Table~\ref{tbl:markov2} and graphically
illustrated in Figure~\ref{fig:Fig1_Markov}.
\index{Transition probability matrix}
\index{Matrix!transition probability}
\begin{table}[h]
\center
\begin{tabular}{|c|l|l|l|l|}
\hline
\bf{From/To} & \ \ \bf{A}  & \  \bf{B}  & \ \ \bf{C}  & \  \bf{D} \\ \hline
\bf{A} & 0.77 & 0.00 & 0.23 & 0.00 \\ \hline
\bf{B} & 0.00 & 0.71 & 0.29 & 0.00 \\ \hline
\bf{C} & 0.19 & 0.07 & 0.63 & 0.11 \\ \hline
\bf{D} & 0.00 & 0.00 & 0.60 & 0.40 \\ \hline
\end{tabular}
\caption{The transition probability matrix for the transitions in Table~\ref{tbl:markov1}.}
\label{tbl:markov2}
\end{table}

\PSfig[h]{Fig1_Markov}{A cyclic diagram may be used to represent the frequencies of transition between
the various lithologies.}

This operation will usually result in an asymmetrical matrix.  If we divide the row totals (the 
counts) by the grand total number of counts we find the relative \emph{proportions} of the four lithologies.  
This is called the marginal or \emph{fixed probability vector}:
\index{Marginal probability vector}
\index{Fixed probability vector}
\index{Vector!fixed probability}
\index{Vector!marginal probability}
\begin{equation}
\mathbf{f} = 
\left [ \begin{array}{cccc}
0.36 & 0.12 & 0.44 & 0.08
\end{array}
\right ].
\end{equation}
You may remember (if not, brush up on the probability theory in Chapter~\ref{ch:basics}) that the joint probability of two events 
$A$ and $B$ is
\begin{equation}
P(A \cap B) = P(B|A)\cdot P(A),
\end{equation}
which we can rearrange to give
\begin{equation}
P(B|A) = \frac{P(A \cap B)}{P(A)}.
\end{equation}
The probability that state $B$ will follow $A$ is the probability that both states $A$ and $B$ will occur,
divided by the probability that $A$ occurs.  Now, if $A$ and $B$ are independent states, then (e.g., \ref{eq:jointindependent})
\begin{equation}
P(A \cap B) = P(A) \cdot P(B).
\end{equation}
Therefore, if there are no dependencies then the probability that $B$ will follow $A$ is simply the 
probability that $B$ occurs.  This must hold for all independent states, so
\begin{equation}
P(B|A) = P(B|B)  = P(B|C) = P(B|D) = P(B).
\end{equation}
This result provides us with an opportunity to predict what the transition probability matrix should look 
like if the occurrence of a lithologic state at one point were completely independent of the 
lithology at the underlying point.  Naturally, that matrix will have rows matching the fixed 
probability vector.  So, for our stratigraphic example, we find the expected matrix to be
$$
\begin{array}
{|c|c|c|c|c|} \hline
 & \bf{A} & \bf{B} & \bf{C} & \bf{D} \\ \hline
\bf{A} & 0.36 & 0.12 & 0.44 & 0.08  \\ \hline
\bf{B} & 0.36 & 0.12 & 0.44 & 0.08 \\ \hline
\bf{C} & 0.36 & 0.12 & 0.44 &  0.08 \\ \hline
\bf{D} &  0.36 & 0.12 & 0.44 &  0.08\\ \hline
\end{array}
$$
Finally, we are now in the position to compare the observed transition frequencies to the predicted frequencies 
and test the null hypothesis that all lithologic states are independent of the state immediately below it.  
To do so we use a $\chi^2$-test after first converting the expected percentages back to counts or frequencies.  
We find
\begin{equation}
\left [\begin{array}{cccc}
22 &  &  & \\
 & 7 &  &  \\
 &  & 27 &  \\ 
 &  &  & 5 \end{array}
\right ] \cdot
\left [\begin{array}{cccc}
0.36 & 0.12 & 0.44 & 0.08\\
0.36 & 0.12 & 0.44 & 0.08 \\
0.36 & 0.12 & 0.44 & 0.08 \\ 
0.36 & 0.12 & 0.44 & 0.08 \end{array}
\right ] 
= 
\left [ \begin{array}{rcrc} 
7.9 & 2.6 & 9.7 & 1.8 \\
2.5 & 0.8 & 3.1 & 0.6 \\
9.7 & 3.2 & 11.9 & 2.2 \\
1.8 & 0.6 & 2.2 & 0.4 \end{array}
\right ].
\label{eq:markov_E}
\end{equation}
The test statistic is, as usual, given by 
\index{Test!$\chi^2$ (``chi-squared'')}
\index{Test!chi-squared ($\chi^2$)}
\index{$\chi^2$ test (``chi-squared'')}
\index{Chi-squared test ($\chi^2$)}
\begin{equation}
\chi^2 = \sum^n_{i=1} \frac{(O_i - E_i)^2}{E_i},
\end{equation}
where $O_i$ is the observed number of transitions and $E_i$ is the expectation for each transition, as given by (\ref{eq:markov_E}).  The 
degrees of freedom, $\nu$, is $(k-1) \cdot (k-1)$, with $k = 4$.  One degree of freedom is lost from each row and column 
because all rows of $\mathbf{P}$ must sum to 1 and we computed $\mathbf{f}$ from the row sums.  

     For the $\chi^2$-test to be valid, each category should have an expected value of at least 5.  Several 
of our categories do not fulfill that criteria.  Because we only are testing whether the transition 
frequencies are independent (random) or not, we may combine some categories to raise the 
expected values above 5.  Hence, we use the four largest categories $A\rightarrow A, A\rightarrow C, C\rightarrow A, C\rightarrow C$ and
the combinations $B\rightarrow$ any, $D\rightarrow$ any, and $[ A\rightarrow B, A\rightarrow D, C\rightarrow B, C\rightarrow D ]$.  We find $\chi^2$ to be 
\begin{equation}
\chi^2 = \begin{array}{c}
\displaystyle \frac{(17 - 7.9)^2}{7.9} +  \frac{(5 - 9.7)^2}{9.7} + \frac{(5-9.7)^2}{9.7} +   \frac{(17 - 11.9)^2}{11.9} + \\*[2ex]
\displaystyle \frac{(7-7.0)^2}{7.0} +  \frac{(5-5.0)^2}{5.0} +    \frac{(5-9.8)^2}{9.8} = 19.57.
\end{array}
\end{equation}
From Table~\ref{tbl:Critical_chi2} we find the critical value of $\chi^2$ with $\nu = 9$ and a 5\% level of significance to be 16.92.  
Since our computed value exceeds the critical value, we must reject the hypothesis that 
successive states are independent.  It appears there is a significant tendency for certain states to 
be followed by certain other states.  
\end{example}

     Sequences in which the state at one point is \emph{partially} dependent, in a probabilistic sense, on 
the previous state is called a \emph{Markov chain}.  It is intermediate between deterministic (fully 
predictable) and completely random sequences.  The section we examined has first-order 
Markov properties.  This means there is statistical dependency between points and their 
immediate predecessor. 

Higher-order Markov chains can exist as well. We can use the transition probability matrix to predict what the lithology might be \emph{two} feet 
above a point.  For example, we might want to fill in the missing part of a section in the 
statistically most reasonable way.  Let us say we start in state B (limestone).  The probabilities of 
reaching the next state is then given as 
\begin{equation}
\begin{array}{clr}
B \rightarrow A & \mbox{(sandstone)}	& 	0\%  \\
B \rightarrow B & \mbox{(limestone)}	&	71\% \\
B \rightarrow C & \mbox{(shale)}	&	29\% \\
B \rightarrow D & \mbox{(coal)}		& 	0\%
\end{array}
\end{equation}
Let us pretend that the next state is shale (C). Then, reaching the following state would be associated with the probabilities
\begin{equation}
\begin{array}{llr}
C \rightarrow A & \mbox{(sandstone)}	& 	19\% \\
C \rightarrow B & \mbox{(limestone)}	& 	7\%  \\
C \rightarrow C & \mbox{(shale)}	& 	63\% \\
C \rightarrow D & \mbox{(coal)}		& 	11\%
\end{array}
\end{equation}
Therefore, the probability that the sequence will be limestone $\rightarrow$ shale $\rightarrow$ limestone is 
\begin{equation}
P(B \rightarrow C) \cdot P (C \rightarrow B) = 29\% \cdot 7\% = 2\%.
\end{equation}
However, we can also reach limestone in two steps by way of the limestone $\rightarrow$ limestone $\rightarrow$
limestone path.  Now
\begin{equation}
P (B \rightarrow B) \cdot P (B \rightarrow B) = 71\% \cdot 71\% = 50\%.
\end{equation}
Since $B \rightarrow
A$  and $B\rightarrow D$ have zero probability, we can state that the probability of finding 
limestone two steps up above limestone, regardless of intervening lithology, is the sum 
\begin{equation}
P(B \rightarrow ? \rightarrow B) = P(B \rightarrow B \rightarrow B) + P(B \rightarrow C \rightarrow B) = 50\% + 2\% = 52\%.
\end{equation}
One can use the same approach to calculate the probability of any lithology two steps up, but 
fortunately there is a more efficient way: These multiplications and additions are exactly those that 
define a matrix multiplication.  Multiplying the transition matrix by itself (i.e., squaring it) yields
$\mathbf{P}^2$, which describes the \emph{second-order} Markov properties of the stratigraphic section:
\begin{equation}
\left [ \begin{array}{rrrr}
0.77 & 0 & 0.23 & 0 \\
0 & 0.71 & 0.29 & 0 \\
0.19 & 0.07 & 0.63 & 0.11 \\
0 & 0 & 0.60 & 0.40 \end{array} \right ]^2 = \left [ \begin{array}{rrrr}
0.64 & 0.02 & 0.32 & 0.02 \\
0.06 & 0.52 & 0.39 & 0.03\\
0.27 & 0.09 & 0.53 & 0.11 \\
0.11 & 0.04 & 0.62 & 0.23
\end{array} \right ]
\end{equation}
Note that again the rows have unit sums.  If we wanted to know whether the second-order Markov 
properties are significant we convert the frequency percentages to counts by multiplying $\mathbf{P}^2$
by the observed row totals again.  However, this time the product will approximate the second-order
transition we would have \emph{likely} observed had we measured them directly from the data.  We find
\begin{equation}
\left[ \begin{array}{rrrr}
14.1 & 0.4 & 7.0 & 0.6\\
0.4 & 3.7 & 2.7 & 0.2 \\
7.1 & 2.5 & 14.2 & 3.1 \\
0.6 & 0.2 & 3.1 & 1.1
\end{array} \right ]
\end{equation}
and carry out another 
$\chi^2$-test.  This time we obtain $\chi^2 = 7.73$ (critical value is still 16.92 since the expectations
are independent of the step length).  Hence, we must conclude that there are no 
significant second-order Markov properties present and that the lithology two steps away appears to 
be a random selection given the volume distribution of the rock types.
\index{Markov chains|)}

\section{Embedded Markov Chains}
\index{Embedded Markov chains|(}
\index{Markov chains!embedded|(}
\label{sec:embmarkov}

	The choice of a sampling interval introduces an arbitrary element into our sequence analysis.  
This can be avoided if we only record the transitions of state whenever they occur.  It follows 
that the transition frequency matrix will have zeros along the diagonal since no state can follow 
itself.  Sequences which cannot contain repeated states are called \emph{embedded Markov chains}.
\begin{example}
	Let us look at a particular example from a borehole through a sedimentary delta plain in
Scotland.  We have five lithologies: A (mudstone), B (shale), C (siltstone), D 
(sandstone), and E (coal).  The analysis yields
\begin{equation}
% MathType!MTEF!2!1!+-
% faaagCart1ev2aaaKnaaaaWenf2ys9wBH5garuavP1wzZbqedmvETj
% 2BSbqefm0B1jxALjharqqtubsr4rNCHbGeaGqiVu0Je9sqqrpepC0x
% bbL8FesqqrFfpeea0xe9Lq-Jc9vqaqpepm0xbba9pwe9Q8fs0-yqaq
% pepae9pg0FirpepeKkFr0xfr-xfr-xb9Gqpi0dc9adbaqaaeGaciGa
% aiaabeqaamaabaabaaGcbaqbamqabmGaaaqaaaqaauaadeqabuaaaa
% qaaiaadgeaaeaacaWGcbaabaGaam4qaaqaaiaadseaaeaacaWGfbaa
% aaqaauaadeqafeaaaaqaaiaadgeaaeaacaWGcbaabaGaam4qaaqaai
% aadseaaeaacaWGfbaaaaqaamaadmaabaqbamqabuqbaaaaaeaacaaI
% WaaabaGaaGymaiaaigdaaeaacaaIZaGaaGOnaaqaaiaaikdacaaIXa
% aabaGaaGynaiaaikdaaeaacaaIYaGaaGioaaqaaiaaicdaaeaacaaI
% 0aaabaGaaGinaaqaaiaaicdaaeaacaaIZaGaaGinaaqaaiaaikdaae
% aacaaIWaaabaGaaGinaiaaiwdaaeaacaaIXaGaaG4maaqaaiaaikda
% caaI5aaabaGaaGymaaqaaiaaisdacaaI1aaabaGaaGimaaqaaiaaio
% daaeaacaaIYaGaaGioaaqaaiaaikdacaaIZaaabaGaaGyoaaqaaiaa
% iIdaaeaacaaIWaaaaaGaay5waiaaw2faaaqaaaqaauaadeqabuaaaa
% qaaaqaaaqaaaqaaaqaaaaaaaqbamqabmqaaaqaaaqaauaadeqafeaa
% aaqaaiabg2da9aqaaiabg2da9aqaaiabg2da9aqaaiabg2da9aqaai
% abg2da9aaaaeaacqGH9aqpaaqbamqabmqaaaqaaiabfo6atbqaauaa
% deqafeaaaaqaaiaaigdacaaIYaGaaGimaaqaaiaaiodacaaI2aaaba
% GaaGyoaiaaisdaaeaacaaI3aGaaGioaaqaaiaaiAdacaaI4aaaaaqa
% amaanaaabaGaaG4maiaaiMdacaaI2aaaaaaaaaa!662C!
\begin{array}{*{20}{c}}
{}&{\begin{array}{*{20}{c}}
A&B&C&D&E
\end{array}}\\
{\begin{array}{*{20}{c}}
A\\
B\\
C\\
D\\
E
\end{array}}&{\left[ {\begin{array}{*{20}{c}}
0&{11}&{36}&{21}&{52}\\
{28}&0&4&4&0\\
{34}&2&0&{45}&{13}\\
{29}&1&{45}&0&3\\
{28}&{23}&9&8&0
\end{array}} \right]}\\
{}&{\begin{array}{*{20}{c}}
{}&{}&{}&{}&{}
\end{array}}
\end{array}\begin{array}{*{20}{c}}
{}\\
{\begin{array}{*{20}{c}}
 = \\
 = \\
 = \\
 = \\
 = 
\end{array}}\\
 = 
\end{array}\begin{array}{*{20}{c}}
\Sigma \\
{\begin{array}{*{20}{c}}
{120}\\
{36}\\
{94}\\
{78}\\
{68}
\end{array}}\\
{\overline {396} }
\end{array}
\end{equation}
The fixed probability vector is found by dividing the row totals by the grand total:
\begin{equation}
\mathbf{\mathbf{f}} = [0.30 \quad    0.09 \quad    0.24 \quad   0.20 \quad   0.17].
\end{equation}
To test whether the observed sequence has Markovian properties or independent states we 
may use the $\chi^2$-test in a similar way to what we did above.  The problem is that we cannot use 
the fixed vector to estimate the independent transition frequency matrix since that results in 
nonzero diagonal terms, which is forbidden.  We must therefore use a different method to 
estimate the necessary matrix.

	Imagine our sequence is a \emph{censored} sample from a sequence where repeats \emph{may} occur.  Its 
transition matrix would be identical to the one we observed except it would have nonzero 
diagonal terms.  If we converted this matrix to probabilities and raised it to a high power, we 
would find the transition probability matrix for a sequence with independent states.  We could 
then discard the diagonal terms, adjust the off-diagonal terms (to ensure they sum to 1), and end up with $P$ 
for an embedded sequence of independent states.  This result is achieved by trial-and-error 
since we do not know the number of repeated states in the envisioned sequence.  We want to find 
diagonal entries which do not change when the matrix is raised to higher powers.  An iterative scheme is used:
\begin{enumerate}
\item	Place arbitrary large estimates (1--2 magnitudes larger that your observations) into the diagonal positions in the observed 
matrix.
\item	Divide row totals by the grand total to get diagonal probabilities.
\item	Calculate new diagonal estimates by multiplying the diagonal probabilities from step 2 by the 
latest row sums.
\item	Repeat process steps (2) and (3) until the diagonal terms remain unchanged, typically after 
10--20 iterations.
\end{enumerate}
We will try this procedure on the Scottish data.  For our matrix, we try inserting 1000 first:
\begin{equation}
% MathType!MTEF!2!1!+-
% faaagCart1ev2aaaKnaaaaWenf2ys9wBH5garuavP1wzZbqedmvETj
% 2BSbqefm0B1jxALjharqqtubsr4rNCHbGeaGqiVu0Je9sqqrpepC0x
% bbL8FesqqrFfpeea0xe9Lq-Jc9vqaqpepm0xbba9pwe9Q8fs0-yqaq
% pepae9pg0FirpepeKkFr0xfr-xfr-xb9Gqpi0dc9adbaqaaeGaciGa
% aiaabeqaamaabaabaaGcbaqbamqabiqaaaqaamaadmaabaqbamqabu
% qbaaaaaeaacaaIXaGaaGimaiaaicdacaaIWaaabaGaaGymaiaaigda
% aeaacaaIZaGaaGOnaaqaaiaaikdacaaIXaaabaGaaGynaiaaikdaae
% aacaaIYaGaaGioaaqaaiaaigdacaaIWaGaaGimaiaaicdaaeaacaaI
% 0aaabaGaaGinaaqaaiaaicdaaeaacaaIZaGaaGinaaqaaiaaikdaae
% aacaaIXaGaaGimaiaaicdacaaIWaaabaGaaGinaiaaiwdaaeaacaaI
% XaGaaG4maaqaaiaaikdacaaI5aaabaGaaGymaaqaaiaaisdacaaI1a
% aabaGaaGymaiaaicdacaaIWaGaaGimaaqaaiaaiodaaeaacaaIYaGa
% aGioaaqaaiaaikdacaaIZaaabaGaaGyoaaqaaiaaiIdaaeaacaaIXa
% GaaGimaiaaicdacaaIWaaaaaGaay5waiaaw2faaaqaauaadeqabuaa
% aaqaaaqaaaqaaaqaaaqaaaaaaaqbamqabiqaaaqaauaadeqafeaaaa
% qaaiabg2da9aqaaiabg2da9aqaaiabg2da9aqaaiabg2da9aqaaiab
% g2da9aaaaeaafaWabeqabaaabaGaeyypa0daaaaafaWabeGabaaaba
% qbamqabuqaaaaabaGaaGymaiaaigdacaaIYaGaaGimaaqaaiaaigda
% caaIWaGaaG4maiaaiAdaaeaacaaIXaGaaGimaiaaiMdacaaI0aaaba
% GaaGymaiaaicdacaaI3aGaaGioaaqaaiaaigdacaaIWaGaaGOnaiaa
% iIdaaaaabaWaa0aaaeaafaWabeqabaaabaGaaGynaiaaiodacaaI5a
% GaaGOnaaaaaaaaaaaa!6EF7!
\begin{array}{*{20}{c}}
{\left[ {\begin{array}{*{20}{c}}
{1000}&{11}&{36}&{21}&{52}\\
{28}&{1000}&4&4&0\\
{34}&2&{1000}&{45}&{13}\\
{29}&1&{45}&{1000}&3\\
{28}&{23}&9&8&{1000}
\end{array}} \right]}\\
{\begin{array}{*{20}{c}}
{}&{}&{}&{}&{}
\end{array}}
\end{array}\begin{array}{*{20}{c}}
{\begin{array}{*{20}{c}}
 = \\
 = \\
 = \\
 = \\
 = 
\end{array}}\\
{\begin{array}{*{20}{c}}
 = 
\end{array}}
\end{array}\begin{array}{*{20}{c}}
{\begin{array}{*{20}{c}}
{1120}\\
{1036}\\
{1094}\\
{1078}\\
{1068}
\end{array}}\\
{\overline {\begin{array}{*{20}{c}}
{5396}
\end{array}} }
\end{array}
\end{equation}
We next obtain the new diagonal probabilities to be
\begin{equation}
\left [ \begin{array}{ccccc}
0.208 & & & & \\
& 0.192 & & &\\
& & 0.203 & & \\
& & & 0.200 & \\
& & & & 0.198
\end{array}
\right ] .
\end{equation}
Step (3) is to update the diagonal elements via the new row sums, hence we find
\begin{equation}
% MathType!MTEF!2!1!+-
% faaagCart1ev2aaaKnaaaaWenf2ys9wBH5garuavP1wzZbqedmvETj
% 2BSbqefm0B1jxALjharqqtubsr4rNCHbGeaGqiVu0Je9sqqrpepC0x
% bbL8FesqqrFfpeea0xe9Lq-Jc9vqaqpepm0xbba9pwe9Q8fs0-yqaq
% pepae9pg0FirpepeKkFr0xfr-xfr-xb9Gqpi0dc9adbaqaaeGaciGa
% aiaabeqaamaabaabaaGcbaqbamqabiqaaaqaamaadmaabaqbamqabu
% qbaaaaaeaacaaIYaGaaG4maiaaiodaaeaacaaIXaGaaGymaaqaaiaa
% iodacaaI2aaabaGaaGOmaiaaigdaaeaacaaI1aGaaGOmaaqaaiaaik
% dacaaI4aaabaGaaGymaiaaiMdacaaI5aaabaGaaGinaaqaaiaaisda
% aeaacaaIWaaabaGaaG4maiaaisdaaeaacaaIYaaabaGaaGOmaiaaik
% dacaaIYaaabaGaaGinaiaaiwdaaeaacaaIXaGaaG4maaqaaiaaikda
% caaI5aaabaGaaGymaaqaaiaaisdacaaI1aaabaGaaGOmaiaaigdaca
% aI1aaabaGaaG4maaqaaiaaikdacaaI4aaabaGaaGOmaiaaiodaaeaa
% caaI5aaabaGaaGioaaqaaiaaikdacaaIXaGaaGOmaaaaaiaawUfaca
% GLDbaaaeaafaWabeqafaaaaeaaaeaaaeaaaeaaaeaaaaaaauaadeqa
% ceaaaeaafaWabeqbbaaaaeaacqGH9aqpaeaacqGH9aqpaeaacqGH9a
% qpaeaacqGH9aqpaeaacqGH9aqpaaaabaqbamqabeqaaaqaaiabg2da
% 9aaaaaqbamqabiqaaaqaauaadeqafeaaaaqaaiaaiodacaaI1aGaaG
% 4maaqaaiaaikdacaaIZaGaaGynaaqaaiaaiodacaaIXaGaaGOnaaqa
% aiaaikdacaaI5aGaaGinaaqaaiaaikdacaaI4aGaaGimaaaaaeaada
% qdaaqaauaadeqabeaaaeaacaaIXaGaaGinaiaaiEdacaaI4aaaaaaa
% aaaaaa!67D6!
\begin{array}{*{20}{c}}
{\left[ {\begin{array}{*{20}{c}}
{233}&{11}&{36}&{21}&{52}\\
{28}&{199}&4&4&0\\
{34}&2&{222}&{45}&{13}\\
{29}&1&{45}&{215}&3\\
{28}&{23}&9&8&{212}
\end{array}} \right]}\\
{\begin{array}{*{20}{c}}
{}&{}&{}&{}&{}
\end{array}}
\end{array}\begin{array}{*{20}{c}}
{\begin{array}{*{20}{c}}
 = \\
 = \\
 = \\
 = \\
 = 
\end{array}}\\
{\begin{array}{*{20}{c}}
 = 
\end{array}}
\end{array}\begin{array}{*{20}{c}}
{\begin{array}{*{20}{c}}
{353}\\
{235}\\
{316}\\
{294}\\
{280}
\end{array}}\\
{\overline {\begin{array}{*{20}{c}}
{1478}
\end{array}} }
\end{array}
\end{equation}
Repeating step (2) with the new matrix gives
\begin{equation}
\left [ \begin{array}{ccccc}
0.239 & & & & \\
& 0.159 & & &\\
& & 0.214 & & \\
& & & 0.199 & \\
& & & & 0.189
\end{array}
\right ] ,
\end{equation}
and we keep repeating this process until it stabilizes.  In the end we find the marginal probability 
vector:
\begin{equation}
\left [ \begin{array}{ccccc}
0.335 \\
& 0.074 \\
& & 0.235 \\
& & & 0.181\\
& & & & 0.155
\end{array}
\right ]
\end{equation}
with a corresponding grand total of 524.

	Now, because all states are independent in the random case we will test for, the probability 
that state $j$ will follow state $i$ is simply $P(i \rightarrow j) = P(i) \cdot P(j)$.  This allows us to construct the 
\emph{expected} transition probability matrix
\begin{equation}
\mathbf{P}_e = \left [ \begin{array}{ccccc}
0.125 & 0.026 & 0.083 & 0.064 & 0.055 \\
0.026 & 0.006 & 0.017 & 0.013 & 0.012\\
0.083 & 0.017 & 0.055 & 0.043 & 0.03 \\
0.064 & 0.013 & 0.043 & 0.033 & 0.028 \\
0.055 & 0.012 & 0.036 & 0.028 & 0.024
\end{array} \right ].
\end{equation}
Scaling these probabilities by the grand total gives the expected frequencies
\begin{equation}
\mathbf{E} = \left [ \begin{array}{ccccc}
65.6  & 13.6 & 43.5 & 33.5  & 28.8  \\
13.6  & 3.1   & 8.9   & 6.8   & 6.3  \\
43.5  & 8.9   & 28.8 & 22.5  & 18.9 \\
33.5 & 6.8   & 22.5  &17.3   & 14.7  \\
28.8  & 6.3  & 18.9  & 14.7 &  12.6 
\end{array} \right ].
\end{equation}
\index{Test!$\chi^2$ (``chi-squared'')}
\index{Test!chi-squared ($\chi^2$)}
\index{$\chi^2$ test (``chi-squared'')}
\index{Chi-squared test ($\chi^2$)}
We strip off the diagonal elements and use the off-diagonal counts to evaluate the $\chi^2$ statistic.  
In this particular case, we find $\chi^2 = 172$ which greatly exceeds the critical value of 19.68 for $v = (k-1)^2 - 
k = 11$ degrees of freedom; the test indicates a strong first order Markov sequence.
\end{example}
\index{Embedded Markov chains|)}
\index{Markov chains!embedded|)}

\section{Series of Events}
\index{Series of events|(}
\label{sec:seriestest}
	One of many types of time series that occur in the natural sciences is the \emph{series of events}.  
Examples of such sequences include the historical records of earthquake occurrences,  volcanic 
eruptions, floods, storms and hurricanes, geomagnetic reversals (Figure~\ref{fig:Fig1_series}), landslides, and tsunamis.  These series
share some common characteristics:
\begin{itemize}
\item	Events are distinguished based on \emph{when} they occur in time.
\item	Events are essentially \emph{instantaneous} in the context of your range.
\item	Events are so \emph{infrequent} that they do not overlap in time.
\end{itemize}
In some cases, spatial data sequences may be considered a series of events.  Consider a traverse 
across a thin-section.  We may be interested in the occurrence of some rare mineral.  Another 
possibility may be the occurrence of bentonite (volcanic ash layers) in a sedimentary sequence.  
However, when a spatial scale acts as a proxy for the actual time scale, we know that the analysis 
will be susceptible to errors caused by varying sedimentation rates, compaction, and erosion.
\PSfig[H]{Fig1_series}{Times of reversals in the Earth's magnetic field during a 40 million year
interval.  While each reversal may take a few thousand years to complete, these reversals are
essentially instantaneous when seen as part of the much longer geological record.}

	With most studies of series of events we hope to find out what the basic features of the series 
are and how we can relate the distribution of length intervals to a physical mechanism.  We must 
first consider the possibility of a trend in the data.   Thus, we will use a test designed to detect 
trends in the rate of occurrence.  It works by simply comparing the mean (or centroid) of a series 
to its midpoint and test their separation for significance.
The centroid is given by
\begin{equation}
\bar{t} = \frac{1}{n} \sum^n_{i=1} t_i.
\end{equation}
We find
\index{Test!series of events}
\begin{equation}
z = \frac{\bar{t}-t_{1/2}}{T/ \sqrt{12 n}},
\end{equation}
where $t_{1/2}$ is the half-point and $T$ is the length of series.  The denominator is obtained by taking
the standard deviation of a uniform distribution over the range $\{0,T\}$ (which equals $T/\sqrt{12}$) and
using the central limits theorem to give the expected standard deviation of the centroid.
The test is very sensitive to changes in the rate 
of occurrence since the centroid is a $L_2$ estimate of location.   If no trend is detected, then we may 
conclude that the series is \emph{stationary}.
\index{Stationary}
	Many geological processes produce events that should be uniformly distributed in time.  For 
instance, the steady motion of the tectonic plates produce a steady increase in stress on a fault, 
which will slip to relieve the stress.  To test for uniformity we note that the cumulative 
distribution of a uniform series is a straight line from 0 to 1 over the time interval.  We can then 
compare this line to the stair-step cumulative function of our observed series of events and apply 
the Kolmogorov-Smirnov test on the largest discrepancy, as discussed in Chapter~\ref{ch:testing}.
\index{Series of events|)}

\section{Run Test}
\index{Run test|(}

	The simplest sequence imaginable is a succession of observations that can take on only two 
mutually exclusive states or values.  Consider a rock collector looking for fossils:  Each time he 
or she opens a concretion there may or may not be a fossil present.  We find TRUE or FALSE, 
yes or no, or 1 or 0.  Similar sequences are generated by coin tosses.  Twenty tosses may give the 
series
\begin{equation}
HTHHTHTTTHTHTHHTTHHH
\label{eq:heads}
\end{equation}
with 11 heads and 9 tails, close to the expected 10/10.  The probability of finding a given number 
of heads ($x$) in a series of $n$ tries is given by the binomial distribution (\ref{eq:binomial_dist}).
However, there is nothing in that expression that takes the \emph{order} in which the heads appear into account.  We would find 
a sequence of 10H followed by 10T to be highly unlikely; the same goes for alternate HTH...  Yet, the 
probability of the \emph{number} of heads is unchanged.  We test such binary sequences for randomness 
of occurrence by examining the number of \emph{runs}, defined as \emph{uninterrupted sequences of the same 
state}.
\index{Runs}
\begin{example}
In our sequence above (\ref{eq:heads}) we have the following 13 runs:
\begin{equation}
\begin{array}{rrcrrcrrrcccc}
                 H & T & HH &  T &  H &  TTT & H & T &  H &   T &   HH &   TT &   HHH\\
                  1 & 2 &  3 &     4 &   5 &     6 &    7 &   8 &   9 &    10 &  11 &    12 &     13\end{array} 
\end{equation}
This is a job for the $U$-test.
We test the significance of the runs by finding all possible ways of arranging $n_1$ items of state 1 
and $n_2$ items of state 2.  The total number of runs is called $U$, and we can consult tables for critical 
values of $U$ given $n_1, n_2$, and $\alpha$ (our confidence level).  However, for large $n_1, n_2 > 10$  the 
distribution is approximated by a normal distribution with a mean of 
\index{Test!run}
\index{Test!U}
\index{U test}
\begin{equation}
\bar{U} = \frac{2n_1 n_2}{n_1 + n_2} + 1
\end{equation}
and variance
\begin{equation}
s^2_U = \frac{2n_1 n_2 (2n_1 n_2- n_1 - n_2)}{(n_1 + n_2)^2 (n_1 + n_2 -1)}.
\end{equation}
We may then simply use the $z$-statistic
\begin{equation}
z = (U - \bar{U})/ s_U
\end{equation}
and see if our calculated $z$ value exceeds the $\pm z_{\alpha/2}$ interval.  For our H/T series
$n_1 = 11$ and $n_2 = 9$, so we find
\begin{equation}
\bar{U} = \frac{2 \cdot 11 \cdot 9}{11 + 9} + 1 = 10.9 \mbox{ and}
\end{equation}
\begin{equation}
s_U = \sqrt{\frac{(2 \cdot 11 \cdot 9)(2 \cdot 11 \cdot 9 - 11 - 9)}{(11 + 9)^2(11+9 -1)}} = \sqrt{4.6} = 2.1,
\end{equation}
which gives
\begin{equation}
z = \frac{13 - 10.9}{2.1} = 1.0.
\end{equation}
With $z_{\alpha/2} = z_{0.025} = \pm 1.96$ (i.e., Table~\ref{tbl:Critical_t}) we cannot reject the null hypothesis that the sequence appears random.
\end{example}

	The geological application of run tests may seem somewhat obscure, since most data 
consist of more than two mutually exclusive states.  A related procedure is a statistical method 
for examining runs up and down.  Here we are again considering two distinct ``states'', i.e., 
whether an observation is larger or smaller than the preceding observation.  Let us examine the 
data set illustrated in Figure~\ref{fig:Fig1_Run}.

\PSfig[H]{Fig1_Run}{Example of how a run test can be used on continuous data. Any multipoint segment with
the same sign of slope is defined as a ``run''.}

The segment ABC is a ``run up'' since the slopes AB and BC are both positive.  Likewise, GHI is a 
``run down''.  CDEF is also down because all slopes are negative except DE, which is zero.  IJ can 
be part of GHIJ or IJK.  For most floating point data there will be few points that exactly equal 
their neighbors.  If we only consider the sign of the slope we get the sequence
$$
+\ +\ -\ 0\ -\ +\ -\ -\ 0\ +
$$
Regarding the first 0 as a $`-$' and the second 0 as a $`+$', we find five runs: three of $`+$' and two of $`-$'.  Then, the $U$-test is directly 
applicable.  Again, we need more than 10 occurrences of each type to use the normal distribution 
approximation introduced earlier.  It is clear that one can apply runs test by converting data to a binary series
by almost any method, provided the hypothesis tested reflects the \emph{dichotomizing} method.  A 
common technique is to dichotomize a series by removing the median or mean value and look 
for randomness of runs about the central location. 
\begin{example}
Consider density measurements of ore samples across a magnetite body.  We 
want to know if the densities vary randomly about the median or if a trend is present.  The data 
are given in Table~\ref{tbl:runtest}.
\begin{table}[h]
\center
\begin{tabular}{|c|c|c|c|c|c|c|c|c|c|} \hline
3.57 & 3.63 & 2.86 & 2.94 & 3.42 & 2.85 & 3.67 & 3.78 & 3.86 & 4.02 \\ \hline
 -   &  -   &  -   &  -   &  -   &  -   &  -   &  -   &  -   &  +   \\ \hline
4.56 & 4.62 & 4.31 & 4.58 & 5.02 & 4.68 & 4.37 & 4.88 & 4.52 & 4.80 \\ \hline
 +   &  +   &  +   &  +   &  +   &  +   &  +   &  +   &  +   &  +   \\ \hline
4.55 & 4.61 & 4.93 & 4.60 & 4.51 & 3.98 & 4.22 & 3.52 & 2.91 & 3.87 \\ \hline
 +   &  +   &  +   &  +   &  +   &  +   &  +   &  -   &  -   &  -   \\ \hline
3.52 & 3.77 & 3.84 & 3.92 & 4.09 & 3.86 & 4.13 & 3.92 & 3.54 &      \\ \hline
 -   &  -   &  -   &  -   &  +   &  -   &  +   &  -   &  -   &      \\ \hline
\end{tabular}
\caption{Density measurements and their signs indicating if larger or smaller than the median density.}
\label{tbl:runtest}
\end{table}
The median density is found to be 3.98.  We subtract this value and store the signs of the 
deviations below the corresponding density.  We observe $U = 7$, with $n_1 = 19$, $n_2 = 20$.  For these numbers, 
\index{Test!U}
\index{U test}
\begin{equation}
\bar{U} = 20.5, \ \ s_U = 3.1,
\end{equation}
and we obtain $z = -4.4$.  Because our observed $U$ is far outside both the 95\% and 99\% confidence intervals we conclude 
that the variations about the median are not random but systematic.
\end{example}
There are of course many 
more variants of the run test shown here.  In general, such tests are \emph{nonparametric} in that they 
do not require the underlying distribution to be known to us.
\index{Run test|)}

\section{Autocorrelation}
\index{Autocorrelation|(}

	The main purpose of time series analysis is to take the order of the observations into account 
and try to learn the properties of the data set, such as discovering any periodicities, trends, or repeating patterns, and 
then use such characteristics to infer something about the process being observed.  Repetitions and 
other patterns in a sequence can be found by computing a measure of the ``self-similarity'' of the 
sequence, that is, to what extent a piece of the sequence looks like another piece of the same sequence.  One such 
measure is known as the \emph{autocorrelation}.
     
In Section~\ref{sc:cc} we discussed the correlation between two variables $x_i$ and $y_i$ and found it 
to be given by 
\begin{equation}
r = \displaystyle \frac{s_{xy}}{s_x s_y},
\end{equation}
where $s_x, s_y$ are the sample standard deviations and $s_{xy}$ the sample covariance, given by
\begin{equation}
s_{xy} = \frac{\displaystyle \sum^n_{i=1} (x_i - \bar{x})(y_i - \bar{y})}{n-1} = \frac{\displaystyle  \sum^n_{i=1} x_i y_i - n \bar{x} \bar{y}}{n-1}.
\end{equation}
The concept of the autocorrelation is to let both $x_i$ and $y_i$ be the same signal $y_i$ and then compare 
these two, identical, time-series.  Of course, with 
$x_i = y_i$ we find
\begin{equation}
s_{yy} = \frac{\displaystyle \sum^n_{i=1} y^2_i -  n\bar{y}^2}{n-1} = s^2 _y
\end{equation}
and
\begin{equation}
r =\frac{ \displaystyle  s^2_y}{s_y s_y} = 1,
\end{equation}
meaning the signal is perfectly correlated with itself.

\PSfig[H]{Fig1_Autocorrelation}{Autocorrelation is the correlation between a time-series and its
identical clone at different lags, $\tau$.}

To get more useful information from the autocorrelation we will shift all values in the second 
copy one step to the left (Figure~\ref{fig:Fig1_Autocorrelation}).  So, instead of having the covariance
be composed of the terms
\begin{equation}
s_{yy}(0) \propto  y_1 \cdot y_1 + y_2 \cdot y_2 + \ldots + y_n \cdot y_n
\end{equation}
we will instead have
\begin{equation}
s_{yy}(1) \propto y_2 \cdot y_1 + y_3 \cdot y_2 + \ldots + y_n \cdot y_{n-1}.
\end{equation}
We then get
\begin{equation}
s_{yy}(1) = \frac{\displaystyle  \sum^n_{i=2} y_i y_{i-1} - \frac{1}{n -1}\displaystyle  \sum^n_{i=2} y_i \sum ^n _{i=2} y_{i- 1}}{n-2}.
\end{equation}
Shifting one step further will give a different result.  In general, if we shift the second sequence 
$\tau$ steps relative to the first sequence we find the series' \emph{autocovariance}:
\begin{equation}
\index{Lag in autocorrelation}
\index{Autocorrelation!lag}
\index{Autocovariance}
\begin{array}{c}
s_{yy}(\tau) = \frac{\displaystyle \sum^n_{i=1 + \tau} y_i y_{i-\tau} -
\frac{1}{n-\tau}
\displaystyle \sum ^n_{i=1 + \tau} y_i \displaystyle \sum^n_{i = 1 + \tau} y_{i-\tau}}{n - \tau - 1} = \\[9pt]
\frac{(n-\tau) \displaystyle \sum ^n_{i=1+\tau} y_i y_{i - \tau} -
\displaystyle \sum^n_{i=1 + \tau} y_i \sum^n_{i = 1 + \tau} y_{i-\tau}}{(n - \tau)(n-\tau - 1)},
\end{array}
\end{equation}
where we call the number of shifts, $\tau$, the \emph{lag}.  It is assumed throughout our time-series 
discussion that the sequences are evenly spaced with spacing $\Delta t$ and contain $n$ points, so that the 
length of a sequence is $T = \Delta t(n-1)$.  Figure~\ref{fig:Fig1_Lag} illustrates the situation for a certain lag.

Computing the autocovariance for all lags from $\tau = 0$ to about $\tau = n/4$ results in
the autocovariance function $s_{yy}(\tau)$.  This function will tell us if the
sequence exhibits self-similarity 
and how much we must shift it (i.e., what is the lag) to reach a maximum in the autocovariance.
 
\PSfig[H]{Fig1_Lag}{Autocorrelations can be high for large lags $\tau$ if the data have ``repeating'' features.}

Plotting the autocovariance function for our sequence gives an \emph{autocovariogram},
illustrated in Figure~\ref{fig:Fig1_AC}.
\index{Autocovariogram}

\PSfig[H]{Fig1_AC}{The autocovariogram for a typical time-series.  At zero lag we obtain the variance
of the time-series.}

As was the case with the covariance for a set of paired values, the autocovariance depends on the 
units of the data, which makes it less useful for comparison purposes.  Again, the solution is to 
normalize it by the variance of the sequence. We find the variance to be given by
\begin{equation}
s^2_{y} = \frac{\displaystyle \sum^n_{i=1+\tau} (y_i - \bar{y})^2}{n - \tau - 1}.
\end{equation}
If we assume the mean and variance remain unchanged by the lag $\tau$, we obtain
\begin{equation}
r_\tau = \frac{(n-\tau ) \displaystyle \sum^n_{i=1+\tau} y_i y_{i-\tau} - \sum^n_{i=1+\tau} y_i \cdot \displaystyle \sum^n_{i=1+\tau} 
y_{i -\tau} }
{(n-\tau )(n-\tau - 1) \left[ \frac{1}{(n - \tau - 1)} \left ( \displaystyle \sum^n_{i=1+\tau} y^2_i - (n - \tau ) \bar{y}^2 \right) \right ]}
\end{equation}
which reduces to
\begin{equation}
r_\tau = \frac{ \displaystyle \sum^n_{i=1+\tau}y_i y_{i-\tau} - (n-\tau ) \bar{y}^2}
{ \displaystyle\sum^n_{i=1+\tau } y^2 _i - (n - \tau )\bar{y}^2}.
\end{equation}
The effect of normalizing by the variance is to obtain the \emph{autocorrelation} which only takes on 
values in the $[-1,+1]$ range.  This \emph{autocorrelogram} for our sequence remains unchanged in shape 
but now has a maximum of $+1$ for zero lag, i.e., the series is in perfect correlation with itself.

     Since it is arbitrary if we consider one copy of the sequence shifted by $-\tau$ or the other by $+\tau$, 
the autocorrelation function is symmetric about zero lag, i.e.
\begin{equation}
r_{-\tau} = r_\tau.
\end{equation}
The autocorrelogram can be used to reveal characteristics of a time series.  Commonly, one would 
like to compare the observed correlogram to predicted autocorrelograms for simple models or 
processes.  The simplest of all models is the one in which successive observations are (1) 
independent and (2) normally distributed.  Since each observation $y_i$ is independent of any other 
observation $y_{i-\tau}$ we expect 
\begin{equation}
r_\tau = \left \{ \begin{array}{cc}
1, & \tau = 0\\
0, & \mbox{elsewhere}
\end{array} \right . 
\end{equation}
\index{Autocorrelation!white noise}
\index{White noise}
\noindent
The expected autocorrelation for a totally random process (also called 
\emph{white noise}) is zero, with a variance of $\sigma^2 = 1/n$, when $n > 30$ (Figure~\ref{fig:Fig1_whitenoise}).
\PSfig[h]{Fig1_whitenoise}{White noise and its autocorellogram.  White noise is completely uncorrelated,
thus the expected value of correlation is zero for nonzero lags.  The finite length of a time-series
leads to departures from this theoretical prediction, and the gray band indicates expected variation
from zero correlation for a 95\% confidence level.}

	Stationary time-series with short term correlations will typically have an autocorrelogram where the first few 
coefficients are significantly nonzero whereas the remainder are close to zero.  If the time series 
has a tendency to alternate direction at every step, then the autocorrelogram will alternate too, with $r_1 < 0$.  If our time-series
is nonstationary (i.e., it includes a trend), then $r_\tau$ will not approach zero except for very 
large values of the lag.  Other correlations tend to be completely masked by the tendency for an 
observation to be systematically larger (or smaller) than its predecessor.  It is therefore always 
necessary to remove linear trends from time-series prior to analysis of $r_\tau$.  Strict periodicities in 
the data will be mimicked in the correlogram: A signal $y_t = A \cos \omega t$ will have an autocorrelation 
of $r_\tau \sim \cos \omega \tau$ (for large $n$).
	
Outliers can wreak havoc on the estimation of autocorrelation coefficients.  This is not surprising since estimating the
correlation is an L$_2$ process.  It is therefore important to suppress any outliers, using robust 
statistical methods, to insure good and stable coefficients.
\index{Autocorrelation|)}

\section{Cross-Correlation}
\index{Cross-correlation|(}

	Rather than using two identical series, an obvious extension of the autocorrelation method is to
compare two \emph{different} time-series at various lags.  From such an undertaking we would expect to learn 
two things: 1) The strength of the relationship between the two series, and 2) the lag that 
maximizes the correlation.  This process is called \emph{cross-correlation}, and it differs from 
autocorrelation in several ways:
\begin{enumerate}
\item It may not be possible to specify the zero lag position, unless the two series share a common 
origin and scale.
\item The cross-correlation will in general be asymmetric.
\item The two series may be of different length.
\end{enumerate}
The correlation coefficient for the match position $\tau$ (relative to an arbitrary origin, unless the series have a 
common origin) is simply
\begin{equation}
r_{\tau} = \frac{(n-1) \displaystyle \sum ^n_{i=1} x_i y_i - \sum^n_{i=1} x_i \sum ^n _{i-1} y_i}
{\sqrt{ \left ( (n-1) \displaystyle \sum ^n_{i=1} x_i ^2 - \left (\displaystyle \sum ^n_{i=1} x_i \right )^2 \right ) \left ( (n-1)
\displaystyle \sum ^n_{i=1} y_i^2 - \left ( \displaystyle \sum ^n_{i=1} y_i \right )^2 \right )  }},
\end{equation}
where $x_i$ and $y_i$ are the two series and the sum over $n$ represents the point pairs that overlap
for this particular lag position $\tau$ (Figure~\ref{fig:Fig1_CCLag}). One difference with 
the expression for the autocorrelation is that the denominator will depend (and thus varies) with $n$, while for 
the autocorrelation we used the variance for the entire chain.  
For this reason the cross-correlation is somewhat less stable.  A simple $t$-test for the correlation $r_{\tau}$
to determine significance can be obtained by calculating
\begin{equation}
	t = r_{\tau} \sqrt{\frac{n-2}{1-r_{\tau}^2}}
\end{equation}
and determine if the observed $t$ exceeds critical $t_{\alpha/2,n-2}$ as in a standard, two-sided $t$-test.

\PSfig[h]{Fig1_CCLag}{Cross-correlation between two separate time-series for an arbitrary lag.}

Cross-correlation as defined here is most useful when the two signals have a common origin and 
time scale so that zero lag can be identified.  While $\tau = 0$ always gives $r_\tau = 1$ for autocorrelation, 
$r_0$  may be zero for cross-correlation if one signal is delayed with respect to the other.   After 
compiling the cross-correlation for all lags (positive and negative) we may find that the 
correlation is maximized for a particular lag $\tau _m$. If the correlation is significantly  nonzero we may 
draw the conclusion that there is a direct correlation between the two series, but this effect is 
\emph{delayed} by some time $\Delta t \cdot \tau_m$.   Examples of data pairs that may exhibit 
such cross-correlation include
\begin{enumerate}
\item	Time series of the amount of water injected into a well and the intensity of seismicity.  Such 
a cross-correlation demonstrates the importance of pore-pressure on the friction on faults.  
The increased pore-pressure reduces the effective normal stress on the fault and allows 
for more earthquakes to occur after accounting for the delaying effect of groundwater flow.

\item Glacial loading histories and land emergence since the end of the ice age.  The viscous properties
of the mantle retard the isostatic response and produce a delay between ice melt and land emergence.

\item	Any process in which the input signal $x(t)$ is delayed and gives output $y(t)$.  The optimal 
lag $\tau$ in the cross-correlation between $x$ and $y$ will tell us something about  the process that 
caused the delay.  This could be flow through a permeable medium, viscous response of 
the mantle, the inelastic response of the solid Earth to tidal forces, etc.
\end{enumerate}
\index{Cross-correlation|)}

\section{Geologic Correlation}
\index{Geologic correlation|(}
\index{Correlation!geologic|(}

	The automated correlation of geological quantities quickly runs into trouble because
cross-correlation techniques require a constant and common time scale for the two time series, which is 
often not the case.  Depending on the particular nature of the data sets, systematic 
distortions such as variable sedimentation rates for sedimentary sequences in drill holes and variable 
seafloor spreading rates for magnetic anomalies will render cross-correlation problematic.  Also, at other times the 
two series may have nominal values such as lithologic states and therefore cannot be assigned a numerical 
value.  The extensions of autocorrelation and cross-correlation techniques to deal with nominal 
data are named \emph{autoassociation} and \emph{cross-association}, respectively.
Correlation of such data can be enumerated by \index{Auto-association}\index{Cross-association}sliding
the two data sets by one another and counting the number of matching states, e.g., 
sandstone at the same position as sandstone, and divide the result by the number of comparisons.  
Plotting this ratio $r_\tau$ as a function of match position $\tau$ may reveal a preferred location where the 
match is optimal.  
\begin{table}[h]
\center
\begin{tabular}{cccccccccccccccccc}
S & S & C & C   & L &  L  & C & L &  S  &  S  & S & C & S &  C &  C &  L  & L & \\
  &   &   & $|$ &   & $|$ &   &   & $|$ & $|$ &   &   &   & $|$ &   & $|$ &   &  \\   
  &   &   & C   & C &  L  & L & S &  S  &  S  & C & S & C &  C &  L &  L  & S & S  \\
\end{tabular}
\label{tbl:cross_assoc}
\caption{Two lithologic sequences of sandstone (S), clay (C), and limestone (L) are compared using the cross-association 
technique.  For the match position shown, there are six exact matches out of 14 possible, so $r_\tau = 6/14 = 0.43$.}
\end{table}

Simple binomial probability theory may then be used to test if this ratio is significant or if it is what one can 
expect from two random sequences of the same composition.
\index{Geologic correlation|)}
\index{Correlation!geologic|)}

\clearpage
\section{Problems for Chapter \thechapter}

\begin{problem}
The stratigraphic column below represents the lithologic successions in a sequence taken 
from a delta plain where we find the lithologies sandstone (light blue), siltstone (light gray), clay (light green), 
and coal (black).  Examine and record the transitions between lithologies every meter.
\PSfig[H]{Fig1_MarkovProblemSet1}{Observed stratigraphic section for Problem~\thechapter.\theproblem.  Distances are in meters.}

\begin{enumerate}[label=\alph*)]
\item What is the transition frequency matrix, $\mathbf{A}$, for this sequence?

\item Determine the fixed probability vector, $\mathbf{f}$.

\item Evaluate the transition probability matrix, $\mathbf{P}$.

\item At the 95\% level of confidence, are the transitions random?

\item $\mathbf{S = P \cdot P}$ gives the second order Markov transition matrix.  At the 95\% level, are there significant 
second-order properties in the sequence?
\end{enumerate}
\end{problem}

\begin{problem}
	Same questions as above, but this time using the transitions between just three lithologies:
	mudstone (light blue), siltstone (beige), and coal (black).
	\PSfig[H]{Fig1_MarkovProblemSet2}{Observed stratigraphic section for Problem~\thechapter.\theproblem.  Distances are in meters.}
\end{problem}

\begin{problem}
	The file \emph{embedded.txt} contains the embedded Markov chain transitions between four lithologies:
	A = shale with fossils, B = siltstone, C = sandstone, and D = coal.  Determine if there is evidence
	for a first-order cyclicity at the 95\% level of confidence using the test for embedded Markov chains.
\end{problem}

\begin{problem}
The file \emph{aso.txt} lists the years the Japanese volcano Aso has erupted during the period 1229--1962.  
Use a Kolmogorov-Smirnov test to determine (at the 95\% level of confidence) whether the 
events are uniformly distributed over the time period. Plot the two cumulative distributions involved in the test.
\end{problem}

\begin{problem}
The file \emph{GK2007.txt} lists all magnetic reversals and the duration of each chron from the last 155 million year.  
Use a series of events test (at the 95\% level of confidence) to determine whether the 
reversals are uniformly distributed over the time period.
\end{problem}

\begin{problem}
The data set \emph{limestone.txt} contains the thickness of successive limestone beds in the Lower Jurassic
from a formation in Wales.   Using the run test, is there a pattern in this sequence of thicknesses?
\end{problem}

\begin{problem}
\newcounter{Vostok}
\setcounter{Vostok}{\theproblem}
The 3-km long Vostok ice core from Antarctica resolves temperature variations relative to the present via
oxygen isotopes. These data are given in table \emph{vostok.txt}, which contains equidistant depths (in meter), the
corresponding times (in year), and the relative change in temperature (in $^{\circ}$C).
\begin{enumerate}[label=\alph*)]
\item Since the autocorrelation calculation requires an equidistant interval we must compute the autocorrelation
of the temperature changes as a function of depth.
At what lag $> 0$ is the autocorrelation maximized?  What does this lag represent and how is it related to
time?
\item Because of compaction, depth is not a good proxy for time, especially for the deeper (older) sections.
To analyze the temporal periodicities we thus need an equidistant time-series. Use MATLAB's \texttt{spline}
or other software to interpolate the data onto an
equidistant interval in time ($\Delta t = 25$) and compute the autocorrelation
of the resampled time-series.  How do the two autocorrelations differ? What period would you now select for the
dominant periodicity?
\end{enumerate}
\end{problem}

\begin{problem}
We will be revisiting Problem~\theConradchap.\theConrad\, so make sure you solve that problem first.
\begin{enumerate}[label=\alph*)]
\item Determine the residual bathymetry $r(x)$ and calculate their first differences, $n(x)$, using
MATLAB's \texttt{diff} operator, and plot these values versus distance.
\item Compute the autocorrelation of $n(x)$ for lags $\tau = 0$ through 5 (using MATLAB's \texttt{xcorr} function).
The 99\% confidence interval for white noise is known to be $\pm 3/\sqrt{n}$.  For these lags, are the data
compatible with a null hypothesis that states $n(x)$ may be considered white noise?
\end{enumerate}
\end{problem}

\begin{problem}
	An engineer generates a two-pulse, noisy signal that he sends through a ``black box'' filtering operator.
The experimental setup records the time (in seconds) and both the input and output magnitudes, reproduced in file
\emph{blackbox.txt}.  The black box filter seems to both smooth the output and delay it in time.
Use the cross-correlation technique to determine the lag induced by the filter.
\end{problem}

\begin{problem}
Two stratigraphic sections (A and B) separated by a few hundred meters have been obtained (Figure~\ref{fig:Fig1_crossassoc}).
Use the cross-association technique to determine for what shift the second section B best fits the first section A.
Plot your calculated match ratios $r_\tau$ versus the vertical offset of B relative to A and find the best fit.
Draw your new tie-lines between the various layers.
\PSfig[H]{Fig1_crossassoc}{Observed stratigraphic sections for Problem~\thechapter.\theproblem.}
\end{problem}

%  $Id: DA1_Chap5.tex 135 2015-05-07 02:09:16Z pwessel $
%
\chapter{SPECTRAL ANALYSIS}
\label{ch:spectralanallysis}
\epigraph{``Science is spectral analysis. Art is light synthesis.''}{\textit{Karl Kraus, Writer}}

	In Chapter~\ref{ch:sequences} we were preoccupied with the topic of time-series analysis in the time-domain 
and learned a few things about the autocorrelation and cross-correlation techniques.  
In this chapter, we will take the different perspective of studying the \emph{periodicities} present in a time 
series.
\index{Frequency content}
	At the heart of spectral analysis lies the notion of a signal's \emph{frequency content}.  This concept is 
utilized to decompose an observed signal into simpler components of known shape.  Because many real 
observations in fact contain periodic components that fluctuate in a predictable way (e.g., yearly, monthly, 
daily), it is desirable to use periodic functions as the basic building blocks of the time-series.  
The most obvious choices are the trigonometric functions \emph{sine} and \emph{cosine}.
\index{Fourier!series}
	The use of sines and cosines to approximate data and functions goes back to the early 1700s 
but was given mathematical rigor and extensive treatment by Joseph Fourier\index{Fourier, J.} late in the 18th century.  
Fourier proved that any continuous, single-valued function could be represented by a series of 
sinusoids --- today we know such series by the name \emph{Fourier series}.  Thus, spectral analysis 
involves finding the components of the Fourier series and interpreting the frequency content 
represented by the series.  Spectral analysis also goes by other names, such as frequency analysis and 
harmonic analysis.  Before we get into the details we must review some terminology and basic 
trigonometry.

\section{Basic Terminology}
\PSfig[h]{Fig1_sincos}{The periodic functions cosine and sine are defined as the $x$ and $y$-components of
the counter-clockwise spinning unit vector $r$ as a function of the rotation angle $\phi$.  a)
Spinning vector at a specific time $t_0$, yielding an angle $\phi_0$ and the corresponding $x$ and $y$
values indicated by the dashed lines. b) Over time, these components trace the sine and cosine functions.}

Consider a unit vector that rotates counterclockwise (Figure~\ref{fig:Fig1_sincos}).  The time it takes to
complete one cycle is 
called the \emph{period}, $T$.  The $y$ and $x$ coordinates are then periodic functions of $t$ and are given by the 
sine and the cosine, respectively.  The \emph{radial frequency}, $f$, of the signal is the number of complete 
revolutions per second.  Hence,
\index{Period}
\index{Sinusoid!period}
\index{Sinusoid!frequency}
\index{Sinusoid!wavelength}
\index{Wavelength}
\index{Frequency!radial}
\begin{equation}
x(t) = \cos (2 \pi ft) = \cos (\omega t) \quad y(t) = \sin (2\pi ft) = \sin (\omega t)
\end{equation}
The period $T = 1/f$ has units of seconds per cycle.  Instead of radial frequency we may use the \emph{angular 
frequency}, $\omega = 2 \pi f$, which has units of radians/sec.  For spatial data, the period $T$ corresponds to 
the \emph{wavelength}, $\lambda$, and the angular frequency is referred to as the \emph{wavenumber}, $k = 2\pi/\lambda$.
The \emph{amplitude}, \index{Wavenumber}\index{Frequency!angular}\index{Sinusoid!amplitude}$A$,
of the signal is the length of the radial vector, $r$.

Instead of requiring that the sine curve go 
through zero at an even number of $\pi$, we can shift it horizontally by subtracting a constant $\phi$  
from the argument.
\PSfig[H]{Fig1_phase}{A phase-shifted sine curve (dotted line) is shifted along the $t$-axis.}
The constant $\phi$ is called the \emph{phase} of the signal (Figure~\ref{fig:Fig1_phase}).
It is clear that the cosine and sine are out of phase 
by 90$^\circ$.  Let us assume that a particular time-series has one single periodic component with angular 
frequency $\omega$.  An example of such a series is shown in Figure~\ref{fig:Fig1_onecos},
where we would need to find both $A$ and $\phi$.  Unfortunately, while this model is linear in $A$ it is \emph{nonlinear} 
in $\phi$.  However, using the trigonometric identity for the cosine of a difference between two angles we find
\index{Phase}
\index{Sinusoid!phase}
\PSfig[H]{Fig1_onecos}{Sinusoid with arbitrary phase can be considered a sum of a sine and a cosine, both with zero phase.}
\noindent
\begin{equation}
d(t) = A\cos (\omega t -\phi ) = A [ \cos \phi \cos \omega t + \sin \phi \sin \omega t ] = a \cos \omega t + b \sin \omega t,
\label{eq:sinusoid}
\end{equation}
which \emph{is} linear in both $a$ and $b$.  Thus, instead of finding one amplitude and a phase we instead find the two 
amplitudes $a$ and $b$ for a cosine and sine pair, respectively, each with \emph{no} phase shift.  We may then easily recover the original
parameters $A$ and  $\phi$ using
\begin{equation}
	\index{Sinusoid!amplitude}
	\index{Amplitude!sinusoid}
	\index{Sinusoid!phase}
	\index{Phase!sinusoid}
A = \sqrt{a^2 + b^2 }, \quad \phi = \tan^{-1} b/a.
\end{equation}
	More often than not, the observed signal will contain many different sinusoids of different 
periods, phases, and amplitudes.  We can use the subscript $j$ to indicate the $j$'th component of the 
series.  Perhaps a complete Fourier series for a signal $y(t)$ could therefore be written
\begin{equation}
d(t) =  \sum^\infty_{j=0} a_{j} \cos \omega_{j}t + b_{j} \sin \omega_{j}t,
\end{equation}
where $\omega_j$ represent the various angular frequency components?  The $a_j$ and $b_j$ coefficients could then be found 
by standard least squares techniques.  However, if we try to solve this system for many 
components we quickly run into computational problems.  To avoid this problem we must look 
at \emph{harmonics}.
\subsection{Harmonics}
\index{Harmonics}
\index{Fundamental frequency}
\index{Frequency!fundamental}

\PSfig[h]{Fig1_fundamental}{The fundamental frequency (solid line) has period $T$ and typically represents the length of our data.  Also shown
is the second harmonic (dashed line).}
The \emph{fundamental} frequency of a signal has period $T$ (scaled to $2\pi$; see Figure~\ref{fig:Fig1_fundamental})
and reproduces a full cycle corresponding to the length of the data signal.  Consequently, 
\index{First harmonic}
\index{Second harmonic}
\begin{equation}
f_{F} = 1/T, \quad \omega_{F} = 2 \pi f_{F} = 2 \pi/T.
\end{equation}
The first harmonic is the sinusoid which makes two complete oscillations over the period $T$.  However, 
because $f_1 = 2f_F,$ this first harmonic sinusoid is usually called the \emph{second harmonic} (though 
some people still refer to it as the first harmonic).  Using that notation,
\begin{equation}
\begin{array}{ll}
f_1 = f_F = 1/T & \omega_{1} = 2\pi f_F = 2\pi/T\\
f_2 = 2f_F = 2/T & \omega_{2} = 4\pi f_F = 4\pi/T\\
\vdots & \vdots \\
f_n = nf_F = n/T & \omega_{n} = 2n\pi f_F= 2n\pi/T\\
\end{array}
\end{equation} 
Superposition of harmonics will always produce a new periodic function with period $T$.

\subsection{Beats}
Nearby frequency components can interact in interesting ways.  Consider 
\begin{equation}
d(t) = \cos \omega_1t + \cos \omega_2 t,
\end{equation}	 
where
\begin{equation}
\omega_1 = \omega + \delta \omega \quad \omega_2 = \omega - \delta \omega,
\end{equation}
and $\delta \omega$ is small.  Using trigonometric identities,
\begin{equation}
d(t) = 2 \cos (\delta \omega t)\cos(\omega t).
\end{equation}
\index{Amplitude!modulation (AM)}
\index{Beat (amplitude modulation)}
\index{Modulation!amplitude (AM)}
Thus, the component $\cos (\omega t)$ (with $T = 2 \pi/\omega$) has a slowly varying amplitude according to $\cos 
( \delta \omega t)$, which is the \emph{modulation} function.  This phenomenon is referred to as a \emph{beat} (Figure~\ref{fig:Fig1_AM}).
\PSfig[h]{Fig1_AM}{Graphical representation of a ``beat'' or amplitude modulation.  Modulating the amplitude
of a constant frequency carrier wave is the basis for AM radio, while FM broadcasts use a constant amplitude
carrier wave and modulate the frequency instead.}
As $\delta \omega$ gets smaller, the period of the beat curve gets longer and the phenomenon becomes more 
noticeable. In acoustics, the two frequencies $\omega _1$ and $\omega_ 2$ are often too high to hear but the beat is 
within an audible range.  \emph{Modulation} is the general phenomenon of a sinusoid with a varying 
amplitude (of any functional form).
\index{Modulation!frequency (FM)}
\index{Frequency!modulation (FM)}
\index{AM radio signals}
\index{FM radio signals}

\section{Fitting the Fourier Series}
\label{sec:FFS}
Consider fitting a set of data, $d_i$, using a set of sine and cosines as the basis functions.
At each observation time, $t_i$, our model predictions would be
\begin{equation}
\hat{d_i} = m_0 + m_1 \sin 2 \pi t_i/T + m_2 \cos 2 \pi t_i/T + m_3 \sin 4 \pi t_i/T + m_4 \cos 4 \pi t_i/T + \dots,
\end{equation}
where $T = n \Delta t$ if   $t_i$ are evenly spaced.  We want to interpolate the values exactly, hence at our observations,
\begin{equation}
\begin{array}{c}
m_0 + m_1 \sin 2 \pi t_1/T + m_2 \cos 2 \pi t_1/T + m_3 \sin 4 \pi t_1/T + \dots = d_1\\
m_0 + m_1 \sin 2 \pi t_2/T + m_2 \cos 2 \pi t_2/T + m_3 \sin 4 \pi t_2/T + \dots = d_2\\
\vdots\\
m_0 + m_1 \sin 2 \pi t_n/T + m_2 \cos 2 \pi t_n/T + m_3 \sin 4 \pi t_n/T + \dots = d_n\\
\end{array},
\end{equation}
and this gives us $n$ equations in $n$ unknowns.  Written as a matrix equation,
\begin{equation}
\left [ \begin{array}{ccccc}
1 & \sin 2 \pi t_1/T &  \cos 2 \pi t_1/T & \sin 4 \pi t_1/T & \dots \\
\vdots & \vdots & \vdots & \vdots & \dots \\        
1 & \sin 2 \pi t_n/T & \cos 2 \pi t_n/T & \sin 4 \pi t_n/T   & \dots \\
\end{array} \right ]
\times 
\left[ \begin{array}{c}
m_0 \\
\vdots \\
m_{n-1} \\
\end{array} \right]
=
\left[ \begin{array}{c}
d_1\\
\vdots\\
d_n \\
\end{array} \right] .
\end{equation}
This standard matrix equation, $\mathbf{G\cdot m = d}$, can now be solved for the $n$ coefficients of the sine and cosine 
series (i.e., the Fourier series) in the usual (matrix) way (i.e., $\mathbf{m} = (\mathbf{G}^T\mathbf{G})^{-1}\mathbf{G}^T\mathbf{d}$).
However, because of orthogonality 
relationships between harmonics, the coefficients can be found analytically (this was fist shown 
by Lagrange in the 1800's using just the sine components over a range of $x$ from 0 to $\pi$).
	
First, we rewrite the Fourier series as
\begin{equation}
\hat{d_i} = a_0 + \sum^{\leq n/2}_{j=1} \left[ a_j \cos \frac{2 \pi jt_i}{n \Delta t}
 + b_j \sin \frac{2 \pi jt_i}{n \Delta t} \right],\quad i = 1,n,
\label{eq:thesummation}
\end{equation}
where $\leq n/2$ means the largest whole integer resulting from the division $n/2$.
Thus, the unknown vector $\mathbf{m}$ would now contain the renamed components
\begin{equation}
\mathbf{m}^T = \left[ a_0 \quad a_1 \quad \dots a_ {\leq n/2} \quad b_1 \quad b_2 \quad \dots \quad b_{\leq n/2} \right] .
\end{equation}
The number of data points $n$ may be any integer value, but we will see it makes a minor difference if $n$ is or is not divisible by two.
If $n$ is an even number, then for $j = n/2$ we find the two last terms in (\ref{eq:thesummation}) to be
\begin{equation}
a_{n/2} \cos \frac{2 \pi n t_i}{2n \Delta t} + b_{n/2} \sin \frac{2 \pi n t_i}{2n \Delta t} = a_{n/2} \cos \frac { \pi t_i}{ \Delta t} + b_{n/2} \sin \frac{ \pi t_i}{ \Delta t}.
\end{equation}
Since $t_i = (i-1) \Delta t$, where $i = 1, \cdots , n$, we have
\begin{equation}
a_{n/2} \cos (i-1) \pi + b_{n/2} \sin (i-1) \pi = \left[-1 \right]^{(i-1)} a_{n/2}.
\end{equation}
This is true, since $\sin (i-1) \pi$ = 0 for all $i$, hence the $b_{n/2}$ term drops out and we find
\begin{equation}
\mathbf{m}^T = \left[ a_0 \quad a_1 \quad \dots \quad a_{n/2} \quad b_1 \quad b_2 \quad \dots \quad  b_{(n/2)-1} \right],
\end{equation}
which gives us $1 + (n/2) + (n/2) - 1 = n$ coefficients for $n$ unknowns, and thus a solvable system results.

On the other hand, if $n$ is \emph{odd} then for $j < (n/2) = (n-1)/2$, both the sine and cosine terms remain and we have
\begin{equation}
\mathbf{m}^T = \left[ a_0 \quad a_1 \quad \dots \quad  a_{(n-1)/2} \quad b_1 \quad b_2 \quad  \dots \quad b_{(n-1)/2} \right],
\end{equation}
which again yields $1 + (n-1)/2 + (n-1)/2 = n$ coefficients for $n$ unknowns.  Since, for $j = 0$,
\begin{equation}
a_0 \cos 0 + b_0 \sin 0 = a_0,
\end{equation}
and thus there is never a $b_0$ term.  Using the convention $a_0 = \frac{a_0}{2}$ and $a_{n/2} = \frac{a_{n/2}}{2}$,
the Fourier series may be written as
\begin{equation}
\hat{d_i} = \sum^{ \leq n/2}_{j=0} \left[a_j \cos \frac{2 \pi jt_i}{n \Delta t} + b_j \sin \frac {2 \pi j t_i}{n \Delta t} \right],\quad i = 1,n,\mbox{ with } a_0 = \frac{a_0}{2},\quad a_{n/2} = \frac{a_{n/2}}{2}
\end{equation}	 	
(the $a_{0}/2$ and $a_{n/2}/2$ convention is only a convenience that will become obvious later).  The frequency
\index{Fourier frequency}
\index{Frequency!Fourier}
\index{Fourier!frequency}
\begin{equation}
\omega_j = \frac{2 \pi j}{n \Delta t} = \frac{2 \pi j}{T} 
\end{equation}
is called the $j$'th \emph{Fourier frequency}.  Notice if $n$ is odd, then $j < n/2$, so $\omega_j < \pi/\Delta t$ and hence $\pi/\Delta t$ is \emph{not}
a Fourier frequency, otherwise it \emph{is} a 
Fourier frequency and the sine term is zero.  Also, the highest frequency $f_{n/2} = (n/2)/ (n \Delta t) = 1/(2 \Delta t)$ is called the \emph{Nyquist} 
frequency, which we will return to later.
\index{Nyquist frequency}
\index{Frequency!Nyquist}
\index{Fourier!series orthogonality|(}
\index{Orthogonality!Fourier series|(}
	
Lagrange took advantage  (in a brute force and laborious manner) of the following five
relationships of harmonic components (which are easily shown using the corresponding integral 
relationships):
\begin{equation}
\sum^{n}_{i=1} \cos \omega_{j}t_{i} = \left\{ \begin{array}{cc} 0, & j \neq 0 \\ n, & j = 0\\
\end{array} \right.,
\label{eq:cosonly}
\end{equation}
\begin{equation}
\sum^{n}_{i=1}\sin \omega_{j}t_{i} = 0,
\end{equation}
\begin{equation}
\sum^{n}_{i=1} \cos \omega_{j}t_{i} \cos \omega_{k}t_{i} = 
\left\{ \begin{array}{ll} n/2, & j=k\neq 0,n/2\\n, & j=k=0, n/2\\ 0& j \neq k\\ \end{array} \right.,
\label{eq:coscos}
\end{equation}
\begin{equation}
\sum^{n}_{i=1} \sin \omega_{j}t_{i} \cos \omega_{k}t_{i} = 0,
\label{eq:sincos}
\end{equation}
and
\begin{equation}
\sum^{n}_{i=1} \sin \omega_{j}t_{i} \sin \omega_{k}t_{i} = 
\left\{ \begin{array}{ll} n/2, & j=k\\ 0& j \neq k\\ \end{array} \right. .
\label{eq:sinsin}
\end{equation}
For a proof, consider the integral corresponding to (\ref{eq:cosonly}):
\begin{equation}
\int^{T}_0 \cos \omega_{j} tdt = \frac{1}{\omega_{j}} \int^{T\omega_{j}}_{0} \cos udu = \frac{1}{\omega_{j}} \sin u \left |^{T \omega {j}}_0 = \frac{T}{2 \pi j} \right. \left( \sin 2 \pi j - \sin 0\right) = 0 \left( j \neq 0 \right).
\end{equation}	 
For $j = 0$,
\begin{equation}
\int^{T}_{0} \cos 0tdt = \int^{T}_{0}dt = T.	 
\end{equation}
It is also easy to visualize these relationships.  For instance, see Figure~\ref{fig:Fig1_ortho} for the a graphical proof of (\ref{eq:sincos}).
\PSfig[h]{Fig1_ortho}{Orthogonality of two cosine harmonics drawn as solid ($\omega_1$) and dashed ($\omega_3$) lines,
with green circles and squares as the hypothetical discrete samples.  The products in (\ref{eq:coscos}) are represented by the
product of the red (negative) and blue (positive) signed lengths.  For this pair, you can see that for each red--blue
line there is another of opposite orientation, thus canceling each other and yielding a final sum of zero.}
\index{Discrete Fourier transform|(}
\index{Fourier!discrete transform|(}

Returning to the Fourier series, we have
\begin{equation}
\hat{d_i} = \sum^{ \leq n/2}_{j=0}\left[ a_{j} \cos \omega_{j}t_{i} + b_{j} \sin \omega_{j}t_{i}\right],\quad i = 1,n
\label{eq:fourierseries}
\end{equation}
and once again we want to minimize the misfit between data and model:
 \begin{equation}
E=\sum^{n}_{i=1}e^2_i = \sum^{n}_{i=1}\left[ d_i - \hat{d_i}\right]^2 = \sum^{n}_{i=1}\left[ d_{i}-\sum^{\leq \frac{n}{2}}_{j=0}\left( a_{j} \cos \omega_{j}t_{i} + b_{j} \sin \omega_{j}t_{i}\right)\right]^2.
\end{equation}
Our unknowns are the $n$ coefficients $a_j$ and $b_j$.  Taking the partial derivatives of $E$ with respect to those parameters gives
\begin{equation}
\frac{\partial E}{\partial a_{k}}=2 \sum^{n}_{i=1}\left[d_{i}-\sum^{ \leq n/2}_{j=0} \left(a_{j} \cos \omega_{j} t_{i} + b_{j} \sin \omega_{j} t_{i}\right) \right] \cos \omega_{k} t_{i} = 0, \quad k =0, \dots, \leq \frac{n}{2}
\end{equation}
\begin{equation}
\frac{\partial E}{\partial b_{k}} = 2 \sum^{n}_{i=1}\left[ d_{i}-\sum^{< n/2}_{j=1} \left( a_{j} \cos \omega_{j} t_{i} + b_{j} \sin \omega_{j} t_{i}\right) \right] \sin \omega_{k} t_{i} =0, \quad k = 1, \dots, < \frac{n}{2}.
\end{equation}
Eliminating the factor of 2 and rearranging, we obtain
\begin{equation}
\sum^{n}_{i=1}d_{i} \cos \omega_{k} t_{i} = \sum^{\leq n/2}_{j=0}\left[ a_{j}\sum^{n}_{i=1} \cos \omega_{j} t_{i} \cos \omega_{k} t_i + b_{j} \sum^{n}_{i=1} \sin \omega_{j} t_{i}\cos \omega_{k} t_i \right ],
\label{eq.thecosterm}
\end{equation}
\begin{equation}
\sum^{n}_{i=1}d_{i} \sin \omega_{k}t_{i} = \sum^{< n/2}_{j=1}\left[ a_{j}\sum^{n}_{i=1} \cos \omega_{j}t_{i} \sin \omega_{k}t_{i} + b_{j}\sum^{n}_{i=1} \sin \omega_{j}t_{i} \sin \omega_{k}t_{i} \right] .
\label{eq:dft_n2}
\end{equation}
The  five orthogonality relationships can now be employed.  For $k = 0$, (\ref{eq.thecosterm}) becomes
\begin{equation}
\sum^{n}_{i=1}d_{i}\cos \omega_{0}t_{i} = \sum^{\leq n/2}_{j=0}\left[a_{j} \sum^{n}_{i=1} \cos \omega_{j}t_{i} \cos \omega_{0} t_{i} +b_{j} \sum^{n}_{i=1} \sin \omega_{j}t_{i} \cos \omega_{0}t_{i} \right],
\end{equation}
and since $\omega_{0} = 0$, $\cos \omega _0 t_i$ is unity. Furthermore, we already determined that $b_0 = 0$, thus
\begin{equation}
\begin{array}{c}
\sum^{n}_{i=1}d_{i} = a_{0} \sum^{n}_{i=1} \cos \omega_{0}t_{i} + a_{1}\sum^{n}_{i=1} \cos \omega_{1} t_{i} + a_2 \sum^{n}_{i=1} \cos \omega _2 t_i \\[4pt]
+ \cdots + b_1 \sum^n_{i=1} \sin \omega _1 t_i + b_2 \sum ^n_{i=1} \sin \omega_2 t_i + \cdots .
\end{array}
\end{equation}
Using the relationships (\ref{eq:coscos}) and (\ref{eq:sincos}), we find
\begin{equation}
\sum^n_{i=1} d_i = a_0n + a_1 0+ a_2 0 +  \cdots b_1 0 + b_2 0 + \cdots,
\end{equation}
and with our convention $a_0 = a_0/2$ we obtain
\begin{equation}
\sum^n_{i=1} d_i = \frac{n}{2}a_0 \quad \Rightarrow \quad a_0 = \frac{2}{n} 
\sum^n_{i=1} d_i = 2 \bar{d},
\end{equation}
i.e., the first coefficient $a_0$ is simply twice the mean of the data (a consequence of our convention for $a_0$ 
that includes the factor of 1/2).	
Using the same approach for $k = 1$, we first obtain
\begin{equation}
\begin{array}{c}
\displaystyle \sum^n_{i=1} d_i \cos \omega_1 t_i = a_0 \sum^n_{i=1} \cos \omega _0 t_i \cos \omega _1 t_i +  a_1 \sum^n_{i=1} \cos \omega_1 t_i \cos \omega _1 t_i + a_2 \sum^n_{i=1} \cos \omega_2 t_i \cos 
\omega _1 t_i + \cdots \\*[2ex]
+ b_1 \displaystyle \sum^n_{i=1} \sin \omega_1 t_i \cos \omega_1 t_i + b_2 \displaystyle \sum^n_{i=1} \sin \omega_2 t_i \cos \omega_1 t_i + \cdots.
\end{array}
\end{equation}
Using the orthogonality relationships, we find
\begin{equation}
\displaystyle \sum^n_{i=1}  d_i \cos \omega_1 t_i = a_0 0 + a_1 \frac{n}{2} + a_2 0 + \cdots + b_1 0 + b_2 0 + \cdots .
\end{equation}
Therefore,
\begin{equation}
a_1 = \frac{2}{n} \displaystyle \sum^n_{i=1} d_i \cos \omega_1 t_i.
\label{eq:a1coeff}
\end{equation}
Since the orthogonality relationships are the same for all $k = 1, 2, \cdots < n/2$ as they were for $k = 1$  in 
(\ref{eq:a1coeff}), we find
\begin{equation}
a_k = \frac{2}{n} \displaystyle \sum^n_{i=1} d_i \cos \omega_{k} t_i, \quad k=0, 1, \cdots, < \frac{n}{2}.
\end{equation}
\PSfig[h]{Fig1_FourierFit}{Example of a data set (thin line with connected dots) and two fitted Fourier components.  Here,
we show the least-squares solution for the two harmonics $\omega_3$ (solid line) and $\omega_9$ (dashed line).  When
these and all other harmonics are evaluated they will sum to equal the original data set.}
Finally, we need to look at the last case, $k = n/2$ (for even $n$).  For $k = n/2$,
\begin{equation}
\begin{array}{c}
\displaystyle \sum^n_{i=1} d_i \cos \omega_{n/2} t_i = a_0 \sum^n_{i=1} \cos \omega_{0} t_i
\cos \omega_{n/2} t_i + a_1 \sum^n_{i=1}
\cos \omega_{1} t_i      \cos \omega_{n/2} t_i + a_2 
\sum^n_{i=1} \cos \omega_{2} t_i
 \cos \omega_{n/2} t_i + \cdots \\*[2ex]
+ b_1 \displaystyle \sum^n_{i=1} \sin \omega_1 t_i \cos \omega _{n/2}t_i + b_2 
\displaystyle \sum^n_{i=1} \sin \omega_2 t_i \cos \omega_{n/2} t_i + \cdots.
\end{array}
\end{equation}
Again, using the orthogonality relationships,
\begin{equation}
\displaystyle \sum^n_{i=1} d_i \cos \omega _{n/2} t_i = a_0 0 + a_1 0 + \cdots + na_{n/2} + \cdots + b_1 0 + b_2 0 + \cdots .
\end{equation}
Since we required $a_{n/2} = a_{n/2}/2$, we find
\begin{equation}
a_{n/2} = \frac{2}{n} \displaystyle \sum^n_{i=1} d_i \cos \omega_{n/2} t_i.
\end{equation}
Therefore, with the convention that $a_0 = a_0/2$ and $a_{n/2} = a_{n/2}/2$, and interchanging the dummy subscripts
$k$ and $j$, the $a_j$ can be defined as
\begin{equation}
	\index{Discrete cosine transform}
	\label{eq:DCT}
\boxed{a_j = \frac{2}{n} \displaystyle \sum^n_{i=1} d_i \cos \omega_j t_i, \quad 0 \leq j \leq \frac{n}{2}.}
\end{equation}
We now turn our attention to the other half of the normal equations (\ref{eq:dft_n2}) involving the $\sin \omega _j t_i$  
terms.  Previously, we have shown that $b_0 = b_{n/2} = 0$ so we will only need to look at one case $(k = 
1)$ and generalize the result for all $k$.  We find
\begin{equation}
\begin{array}{c}
\displaystyle \sum^n_{i=1} d_i \sin \omega_{1} t_i = a_0 \sum^n_{i=1} \cos \omega_{0} t_i
\sin \omega_{1} t_i + a_1 \sum^n_{i=1}
\cos \omega_{1} t_i \sin \omega_{1} t_i + a_2 \sum^n_{i=1} \cos \omega_{2} t_i
  \sin \omega_{1} t_i  + \cdots  \\*[2ex]
+ b_1 \displaystyle \sum^n_{i=1} \sin \omega_1 t_i \sin \omega _{1} t_1 +  b_2 
\displaystyle \sum^n_{i=1} \sin \omega_2 t_i \sin \omega_{1} t_i + \cdots.
\end{array}
\end{equation}
Thus,
\begin{equation}
\displaystyle \sum^n_{i=1} d_i \sin \omega_1 t_i = a_0 0 + a_1 0  + \cdots + b_1 \frac{n}{2} + b_2 0 + \cdots,
\end{equation}
and
\begin{equation}
b_1 = \frac{2}{n} \sum^n_{i=1} d_i \sin \omega_1 t_i.
\end{equation}
Since the orthogonality relationships hold for all $k$, we again interchange $k$ and $j$ and find
\begin{equation}
	\index{Discrete sine transform}
	\label{eq:DST}
\boxed{b_j = \frac{2}{n} \sum^n_{i=1} d_i \sin \omega_j t_i, \quad 0 < j < \frac{n}{2}.}
\end{equation}

The formulae for $a_j$ and $b_j$ are called the \emph{Discrete Cosine and Sine Transforms} and combined they
define the \emph{Discrete Fourier Transform}. Figure~\ref{fig:Fig1_FourierFit} shows an
example of a data set and the determination of two Fourier components.
\index{Discrete sine transform}
\index{Discrete cosine transform}
\index{Discrete Fourier transform|)}
\index{Fourier!discrete transform|)}
\begin{example}
Consider the time-series $d = [ 1 \ 0 \ \ \mbox{-2} \  \ \ \mbox{-1} \ \ 1 \ 2 \ 1 \ 0.5 ]^T$   with 
$\Delta t = 1, n = 8, T = 8$.  The Fourier frequencies are therefore
\begin{equation}
\begin{array}{lll}
\omega_j = 2 \pi j /T, & \omega_0 = 0, & \omega_1 = \pi/4, \\
\omega_2 = \pi/2,  & \omega_3 = 3 \pi/4, & \omega_4 = \pi.
\end{array}
\end{equation}
Solving for the coefficients, we find
\begin{equation} \begin{array}{lllll}
a_0 = 0.3125, & a_1 = -0.0884, & a_2 = 0.7508, & a_3 = 0.0804, & a_4 = -0.0625, \\
b_1 = -1.3687, & b_2 = 0.625, & b_3 = 0.1313.
\end{array}
\end{equation}
Thus, our Fourier series is
\begin{equation}
\begin{array}{lll}
d(t)&  = &  \displaystyle 0.3125 - 0.0884 \cos \frac{\pi}{4} t -1.3687 \sin \frac{\pi}{4} t + 0.7508 \cos  \frac{\pi}{2} t\\\\[2pt]
&  & + \displaystyle  0.625 \sin \frac{\pi}{2}t  + 0.0804 \cos \frac{3 \pi}{4}t + 0.1313 \sin \frac{3 \pi}{4} t - 0.0625 \cos \pi t \end{array}.
\end{equation}
\end{example}
Note that all the terms have periods that are multiples of the fundamental frequency; hence the 
Fourier representation must itself be periodic with the same fundamental period $T$.  We will later 
see that the assumption of periodic data may have grave consequences for our coefficients.

\subsection{The power of orthogonality}
Let us pause and lament the demise of our dear friend from Chapter~\ref{ch:matrix}, the design matrix $\mathbf{G}$.
What just happen to it in our analysis?
Well, recall equations (\ref{eq:gdotg}) and (\ref{eq:gdotd}), both overflowing with dot-products of the basis vectors $\mathbf{g}_j$.
Yet, with our choice of harmonics we found that these basis vectors were in fact orthogonal and thus their dot-products yielded
zero except along the matrix diagonal (where we got the simple constant $n/2$).  With $\mathbf{G}$ being diagonal, the formidable $[\mathbf{G}^T\mathbf{G}]^{-1}$
collapsed to the identity matrix $\mathbf{I}$ scaled by $2/n$, as we just witnessed. Now \emph{that} is the power of
orthogonal functions (of which sine and cosine are just two possibilities) and why they are so widely used in data analysis as well as for modeling in the physical sciences.


\index{Fourier!series orthogonality|)}
\index{Orthogonality!Fourier series|)}

\section{The Periodogram}
\label{sec:periodogram}
\index{Periodogram|(}
\index{Spectrum|(}

	We determined that the Fourier series expansion of our observed time-series $d_i$ could be 
written
\begin{equation}
\hat{d_i} = \sum^{\leq n/2}_{j=0} [ a_j \cos \omega_j t_i + b_j \sin \omega_j t_i ].
\end{equation}
Remember that (\ref{eq:sinusoid}) started out by trying to fit a cosine of arbitrary amplitude $A_j$ and phase $\phi_j$, 
but that we could rewrite this single term as a sum of a cosine and sine components with different amplitudes 
and zero phases.  We found
\begin{equation}
a_j = A_j \cos \phi_j, \quad b_j = A_j \sin \ \phi_j.
\end{equation}
From these expressions we readily find a component's full amplitude and phase.  Dividing the 
$b_j$ by $a_j$ gives 
\begin{equation}
\tan \phi_j = b_j / a_j \quad \Rightarrow \quad \phi _j = \tan ^{-1} (b_j / a_j).
\end{equation}
Squaring $a_j$ and $b_j$ and adding them gives
\index{Power spectrum}
\begin{equation}
A^2_j = a^2_j + b^2_j.
\end{equation}
The \emph{periodogram} is constructed by plotting $A_j^2$ versus $j$, $f_j$, $\omega_j$, or $P_j$.  While often called the 
\emph{power spectrum}, it is strictly speaking a raw, discrete periodogram.  The true spectrum is a 
smoothed periodogram showing frequency components of statistical regularity.  However, the 
periodogram is the most common form of output of a Fourier transform.  Figure~\ref{fig:Fig1_periodogram}
shows the periodogram for the function
\begin{equation}
d(t) = \frac{1}{2} \cos \omega_1 t + \frac{3}{4} \cos \omega_2 t + \frac{1}{2} \sin \omega_3 t
	+ \frac{1}{4} \cos \omega_3 t + \frac{1}{3} \cos \omega_4 t + \frac{1}{5} \sin \omega_4 t + \frac{1}{3} \sin \omega_6 t - \frac{3}{5}.
\label{eq:periodogram}
\end{equation}
\PSfig[h]{Fig1_periodogram}{Raw periodogram of the function given in (\ref{eq:periodogram}).  The peak corresponds to the
$A_0^2 = a_0^2$ term defined to be twice the mean (-0.6) squared.}
Let us look, for a moment, at the variance of the time series expansion.  Recall, the variance is 
given by
\begin{equation}
s^2 = \frac{1}{n-1} \sum^n _{i=1} (\hat{d_i} - \bar{d}) ^2.
\end{equation}
We shall write the Fourier series as
\begin{equation}
\hat{d_i} = \bar{d} + \sum_{j=1} ^{\leq \frac{n}{2}} \left (a_j \cos \omega _j t_i + b_j \sin \omega_j t_i \right ),
\end{equation}
by pulling the constant (mean) term out separately.  Since the two means cancel, we find
\begin{equation}
s^2 = \frac{1}{n-1}  \sum^n_{i=1} \left \{ \left [ \sum_{j=1} ^{\leq \frac{n}{2}} \left (a_j \cos \omega_{j} t_i + b_j \sin \omega_{j} t_i \right ) \right ] \left [ \sum_{q=1} ^{\leq \frac{n}{2}} \left (a_q \cos \omega_{q} t_i + b_q \sin \omega_{q} t_i \right ) \right ] \right \}.
\end{equation}
Also recall that, because of orthogonality, all the cross terms $(q \neq j)$ resulting from the full expansion 
of the two squared expressions will be zero when summed over $i$, while the remaining terms will sum to $n/2$ (since $j,q > 0$).
Hence, we are left with
\begin{equation}
s^2 = \frac{n}{2(n-1)} \sum_{j=1} ^{\leq \frac{n}{2}} (a^2_j + b^2_j) \sim \frac{1}{2}\sum_{j=1} ^{\leq \frac{n}{2}} A^2_j.
\end{equation}
Therefore, the power spectrum (periodogram) of $(a_j^2 + b_j^2)$ versus $\omega_j$ is a plot showing the 
contribution of individual frequency components to the total variance of the signal.  For this reason, 
the power spectrum is often called the variance spectrum.  However, most of the time it is simply called ``the 
spectrum.''  Hence, the Fourier transform converts a signal from the time domain to the frequency 
domain (or wavenumber domain), where the signal can be viewed in terms of the contribution of 
the different frequency components of which it is made.  The phase spectrum ($\phi_j$ versus $\omega_j$)
shows the relative phase of each frequency component.  In general, phase spectra are more difficult
to interpret than amplitude (or power) spectra.

\subsection{Aliasing of higher frequencies}
\index{Aliasing}
\index{Nyquist frequency}
\index{Frequency!Nyquist}
	We mentioned before that the highest frequency (or shortest period, or wavelength) that can be 
estimated from the data is called the Nyquist frequency (or period, or wavelength), given by
\begin{equation}
f_N = f_{n/2} = \frac{1}{2\Delta t}, \quad \omega_N = 2\pi f_N = \frac{\pi}{\Delta t}\quad P_{n/2} = 2 \Delta t.
\end{equation}
Higher frequencies, whose wavelengths are less than twice the spacing between sample points 
\emph{cannot be detected}.  However, when we sample a signal every $\Delta t$ and the original signal has 
higher frequencies than $f_{n/2}$, we introduce \emph{aliasing}.  Aliasing means that some frequencies will 
leak power into other frequencies.  This concept is readily seen by sampling a high-frequency 
signal at a spacing larger than the Nyquist interval.

\PSfig[h]{Fig1_aliasing}{Aliasing: A short-wavelength signal that is not sampled at the Nyquist frequency
or higher will instead appear as a longer-wavelength component that does not exist in the actual data.}
Sampling of the high-frequency signal actually results in a longer-period signal (Figure~\ref{fig:Fig1_aliasing}).
When Clint Eastwood's wagon wheels seem to spin backwards in an old Western movie --- that's aliasing:  The 24 
pictures/sec rate is simply too slow to capture the faster rotation of the wheels.
	
\subsection{Significance of a spectral peak}
\index{Test!spectral peak}
In some applications we may be interested in testing whether a particular 
component is dominant or if its larger amplitude is due to chance.  The statistician R. A. Fisher\index{Fisher, R. A.} devised a test that 
calculates the probability that a spectral peak $s_j^2$ will exceed the value $\sigma_j^2$ of a hypothetical time series 
composed of independent random points.  We must evaluate the ratio of the variance contributed by the
maximum peak to the entire data variance:
\begin{equation}
g = \frac{s^2 _j}{2s^2},
\label{eq:computed_g}
\end{equation}
where $s^2_j$ is the largest peak in the periodogram (we divide by two to get its variance contribution)
and $s^2$ is the variance of the entire series.  For 
a prescribed confidence level, $\alpha$, the critical value that we wish to compare to our observed $g$ is
\begin{equation}
g_{\alpha,k} \approx 1 - \exp \left( \frac{\ln \alpha - \ln k}{k-1} \right ),
\label{eq:critical_g}
\end{equation}
with $k = n/2$ (for even $n$) or $k = (n-1)/2$ (for odd $n$).  Should our observed $g$ (obtained via \ref{eq:computed_g}) exceed this 
critical value we decide that the dominant component is real and reflects a true 
characteristic of the phenomenon we are observing.  Otherwise, $s^2_j$ may be large simply by 
chance.

\subsection{Estimating the continuous spectrum}

	The power spectrum or periodogram obtained from the Fourier coefficients is discrete, yet 
we do not expect the power at frequency $\omega_j$ to equal the underlying continuous $P(\omega)$ at exactly 
$\omega_j$, since the discrete spectrum must necessarily represent some average value of power at all frequencies between $\omega_{j-1}$ and 
$\omega_{j+1}$.  In other words, the computed power at $\omega_j$ also represents the power from nearby frequencies 
not among the chosen harmonic frequencies $\omega_j$.  Furthermore, the uncertainty in any individual 
estimate $p^2_j$ is very large; in fact, it is equal to $\pm p^2_j$ itself.

 	Can we improve (i.e., reduce) the uncertainties in $p^2_j$ by using more data points or sample the 
data more frequently?  The unpleasant answer is that the periodogram estimates do not become 
more accurate at all!  The reason for this is that adding more points simply produces power 
estimates at a greater number of frequencies $\omega_j$.  The only way to reduce the uncertainty in the 
power estimates is to smooth the periodogram over nearby discrete frequencies.  This can be 
achieved in one of two ways:

\begin{enumerate}
\item	Use a time-series that is $M$ times longer (so $f_1' = f_1/M$) and \emph{sum} the $M$ power estimates $p^2_k$
straddling each original $\omega_j$ frequency to obtain a smooth estimate $p^2_j = \sum p^2_k$.
\item	Split the original data into $M$ smaller series, find the $p^2_j$ for each series, and take the \emph{mean} of 
the $M$ estimates for the same $j$ (i.e., the same frequency).
\end{enumerate}
\index{Windowing}
In both cases the variance of the power spectrum estimates drop by a factor of $M$, i.e., $s^2_j = p^2_j/M$.
The exact way the smoothing is achieved may vary among analysts.  Several different 
types of weights or spectral \emph{windows} have been proposed, but they are all relatively similar.  These windows 
arose because, historically, the power spectrum was estimated by taking the Fourier transform of 
the \emph{autocorrelation} of the data; hence many windows operated in the lag-domain.  The 
introduction of the Fast Fourier Transform made the FFT the fastest way to obtain the spectrum, 
which then is simply smoothed over nearby frequencies.  The FFT is a very rapid algorithm for 
doing a discrete Fourier transform, especially if $n$ is a power of 2.  It can be shown that one can 
always split the discrete transform into the sum of two discrete, scaled transforms of subsets of the data.  
Applying this result recursively, we eventually end up with a sum of transforms of data sets 
with one entry, whose transform equals itself.  While mathematically equivalent, there is a huge 
difference computationally:  While the discrete Fourier transform's execution time is proportional to $n^2$, 
the FFT only takes $n\cdot \log(n)$.  For a data set of $10^6$ points, the speed-up is a factor of $> 75,000$.

	By doing a Fourier Analysis, we have transformed our data from one domain (time or space) 
to another (frequency or wavenumber).  A physical analogy is the transformation of light sent 
through a triangular prism.  White light is composed of many frequencies, and the prism acts as a 
frequency analyzer that separates the various frequency components, here represented by colors.  Each color band 
is separated from its neighbor by an amount proportional to their difference in wavelength, and 
the intensity of each band reflects the amplitude of that component in the white light.  We know 
that by examining the spectrum we can learn much about the composition and temperature of the 
source and the material the light passed through.  Similarly, examining the power spectra of 
other processes may tell us something about them that may not be apparent in the time domain.
Consequently, spectral analysis remains one of the most powerful techniques we have for examining
temporal or spatial sequences.

\subsection{First-Order Spectrum Interpretation}
\label{sec:firstorderspectrum}
\PSfig[h]{Fig1_spectratypes}{Simplified representations of typical spectra that are called ``white'' (left; equal power at all frequencies),
``red'' (middle; power falling off with increasing frequency), and ``blue'' (right; power increasing with frequency).}
Per Section~\ref{sec:periodogram}, the raw power spectrum, or \emph{periodogram}, is obtained by plotting the squared amplitude $A_j^2$ versus
frequency.  Often, a spectrum will fall into one of three categories (see Figure~\ref{fig:Fig1_spectratypes}):
\begin{description}
	\item [white:] This is a spectrum that shows little or no amplitude variation with frequency.  Random values
	such as independent samples drawn from a normal distribution will have a white spectrum.\index{White spectrum}\index{Spectrum!white}
	\item [red:] This spectrum is dominated by long-wavelength (low-frequency) signals, with the spectrum tapering
	off for higher frequencies.  This is very common behavior in observed data, such as topography and potential fields (gravity, magnetics).
	It may also be indicative of data that represent an integrated phenomenon.\index{Red spectrum}\index{Spectrum!red}
	\item [blue:] This spectrum is dominated by short-wavelength (high-frequency) signal, with the spectrum tapering
	off for lower frequencies.  Data that depend on derivatives, such as slopes and curvatures, might behave this way, being higher-order
	derivatives of a red-spectrum topography signal.\index{Blue spectrum}\index{Spectrum!blue}
\end{description}
One reason for the prevalence of red or blue spectra for natural phenomena can be understood if we consider what effect a temporal derivative (e.g., $d/dt$) has in the frequency domain.  Given that the Fourier series representation of data can be written
\begin{equation}
\hat{d}(t) = \sum_{j = 0}^{\leq n/2} A_j \cos \left (\omega_j t - \phi_j \right ),
\end{equation}
taking the derivative yields
\begin{equation}
\frac{d}{dt}\hat{d}(t) = \sum_{j = 0}^{\leq n/2} -\omega_j \cdot A_j \sin \left (\omega_j t - \phi_j \right ),
\end{equation}
In effect, we multiply each Fourier amplitude by its corresponding frequency, hence amplitudes at higher frequencies are preferentially enhanced while those at lower frequencies are attenuated.  This scaling
will make the spectrum more ``blue''.  By analogy, integration in the temporal domain has the effect of \emph{dividing} the spectrum by the frequency, conversely
``reddening'' the spectrum.  This frequency effect is what we allude to when we say that taking a derivative typically make data noisier (it amplifies
the short-wavelength or high-frequency components in the data) while integration tends to make data smoother by attenuating the same
components.  Of course, these statements assume that the uncertainties in the data are mostly at high frequencies, but some data have more
uncertainty at low frequencies, in which case the situation is reversed.
These simple considerations may be useful when interpreting your observed spectra.  Finally, note that the derivative
also introduces a phase change of $\pi/2$ (90\DS) since the cosine and the negative sine are shifted by 90\DS.  A second multiplication (i.e., for a second-derivative result)
leads to a 180\DS\ change in phase since we are essentially multiplying by $-1$ (this does not affect power, which is proportional to amplitude squared).

\index{Periodogram|)}
\index{Spectrum|)}

\section{Convolution}
\index{Convolution|(}

	\emph{Convolution} represents one of the most fundamental operations of time series analysis and 
is one of the most physically meaningful.  Consider the passage of a signal through a linear filter, 
where the filter (a ``black box'') will modify a signal passing through it
(Figure~\ref{fig:Fig1_blackbox}).  For instance, it may
\begin{enumerate}
\item Amplify, attenuate or delay the signal.
\item Modify or eliminate specific frequency components.
\end{enumerate}
 
\PSfig[H]{Fig1_blackbox}{Example of convolution between an input signal and a filter.}

Consider the propagation of a seismic pulse through the upper layers of the Earth's crust, as illustrated
in Figure~\ref{fig:Fig1_earthfilter}.  The generated pulse may be sharp and thus have high
frequencies, yet the recorded signal that traveled through the crust may be much smoother and
include repeating signals that reflect internal boundaries.
 
\PSfig[H]{Fig1_earthfilter}{Convolving a seismic pulse with the Earth gives a seismic trace that may
reflect changing properties of the Earth with depth.}

Convolution is this process of linearly modifying one signal using another signal.  In Figure~\ref{fig:Fig1_earthfilter} we 
convolved the seismic pulse with the ``Earth filter'' to produce the observed returned seismogram.  
Symbolically, we write the convolution of a signal $d(t)$ by a filter $p(t)$ as the integral
\index{Deconvolution}
\index{Inverse filtering}
\index{Filtering!inverse}
\begin{equation}
h(t) = d(t) * p(t) = \int_{-\infty}^{+\infty} d(u) \cdot p(t-u) du,
\label{eq:convolution}
\end{equation}
where $*$ represents the convolution operator.
\emph{Deconvolution}, or \emph{inverse filtering}, is the process of unscrambling the convolved signal to 
determine the nature of the filter \emph{or} the nature of the input signal.  Consider these two cases:
\begin{enumerate}
\item If we knew the exact shape of our seismic pulse $d(t)$ and seismic signal received, $h(t)$, we could 
deconvolve the data with the pulse to determine the (filtering) properties of the upper layers of the Earth through 
which the pulse passed (i.e., $p(t) = d^{-1}(t) * h(t)$).
\item If we wanted to determine the exact shape of our pulse $d(t)$, we could pass it through a known 
filter $p(t)$ and deconvolve the output with the shape of the filter (i.e., $d(t) = p^{-1}(t) * h(t)$).
\end{enumerate}
The hard work here is to determine the inverse functions $d^{-1}(t)$ or $p^{-1}(t)$, which is akin to matrix inversion.
Other examples of convolution include:
\begin{enumerate}
\item Filtering data --- using running means, weighted means, removing specific frequency components, 
etc.
 \item Recording a phenomenon with an instrument that responds slower than the rate at which the 
phenomenon changes, or which produces a weighted mean over a narrow interval of time,
or which has lower resolving power than the phenomenon requires.
\item Conduction and convection of heat.
\item Deformation and the resulting gravity anomalies caused by the flexural response of the lithosphere
to a volcano.
\end{enumerate}

	Convolution is most easily understood by examining its effect on discrete functions.  
First, consider the discrete impulse $d(t)$ sent through the filter $p(t)$, as illustrated in Figure~\ref{fig:Fig1_conv1}:
\PSfig[H]{Fig1_conv1}{A filter's impulse response is obtained by sending an impulse $d(t)$ through the filter $p(t)$.}
\noindent
The output $h(t)$ from the filter is 
known as the \emph{impulse response function} since it represents the response of the filter to an 
impulse, $d(t)$.  It represents a fundamental property of the filter $p(t)$.
\index{Impulse response function}
Next, consider a more complicated input signal convolved with the filter, as shown in Figure~\ref{fig:Fig1_conv2}:
\PSfig[H]{Fig1_conv2}{Filtering seen as a convolution.}
\noindent
Since the filter is linear, we may think of the input as a series of individual impulses. The output 
is thus the sum of several impulse responses scaled by their amplitudes and shifted in 
time.  Calculating convolutions is a lot like calculating cross-correlations, except 
that the second time-series must be reversed.  Consider the two signals as finite sequences on separate strips of 
paper (Figure~\ref{fig:Fig1_conv3}).
\PSfig[H]{Fig1_conv3}{Graphical representation of a convolution.  We write the discrete values of $d(t)$ and $p(t)$ on two separate strips of paper.}
\noindent
We obtain the zero lag output by aligning the paper strips as shown in Figure~\ref{fig:Fig1_conv4},
after reversing the red strip.
\PSfig[H]{Fig1_conv4}{Convolution, zero lag.  Reverse one strip and arrange them to yield a single overlap.}
\noindent
The zero lag result $h_0$ is thus simply $d_0 \cdot p_0$.  Moving on, the first lag results from the alignment shown in  Figure~\ref{fig:Fig1_conv5}.
\PSfig[H]{Fig1_conv5}{Convolution, first lag.  We shift one strip by one to increase the overlap.}
\noindent
This simple process is repeated, and for each lag $k$ we evaluate $h_k$ as the sum of the products of the overlapping 
signal values.  This is a graphic (or mechanical) representation of the discrete convolution 
equation (compare this operation to the integral in \ref{eq:convolution}).
Consider the convolution of the two functions shown in Figure~\ref{fig:Fig1_conv6}.
 
\PSfig[H]{Fig1_conv6}{Moving averages is obtained by the convolution of data with a rectangular function of unit area.}
\noindent
If we look at this convolution with the moving strips of paper approach, we get the setup illustrated in Figure~\ref{fig:Fig1_conv7}:
 
\PSfig[h]{Fig1_conv7}{The mechanics of convolutions, this time without the paper strips.}
\noindent
Given the simple nature of $p(t)$, we can estimate the values of $h_k$ directly:
\begin{equation}
\begin{array}{rcl}
h_0 & = & d_{0}/5 \\[4pt]
h_1 & = &  \frac{1}{5} (d_0 + d_1)\\
 & & \vdots \\
h_4 & = &  \frac{1}{5} ( d_0 + d_1 + d_2 + d_3 + d_4)\\[4pt]
h_5 & = & \frac{1}{5} ( d_1 + d_2 + d_3 + d_4 + d_5)\\
& & \vdots \\
h_{18}&  = & d_{14}/5 \end{array}
\end{equation}	 

\PSfig[h]{Fig1_conv8}{The final result of the convolution is a smoothed data set since any short-wavelength signal
will be greatly attenuated.}
\noindent
This is simply a five-point running (or moving) average of $d(t)$, and the result is shown in Figure~\ref{fig:Fig1_conv8}.
An $n$-point average would be the 
result if $p(t)$ consisted of $n$ points, each with a value of $1/n$.

\subsection{Convolution theorem}
\index{Convolution theorem}

Although not shown here, it can be proven that a convolution of two functions $p(t)$ and $d(t)$ 
in the time-domain is equivalent to the product of $P(f)$ and $D(f)$ in the frequency domain 
(here, uppercase letters indicate the Fourier transforms of the lowercase, time-domain functions).
The converse is also true, thus
\begin{equation}
\begin{array}{rcl}
p(t) * d(t) & = & h(t) \quad \leftrightarrow \quad P(f) \cdot D (f) = H(f),\\[4pt]
p(t) \cdot d(t) & = & z(t) \quad \leftrightarrow \quad P(f) * D (f) = Z(f).\\
\end{array}
\end{equation}
Because convolution is a slow calculation it is often advantageous to transform our data from one
domain to the other, perform the simpler multiplication, and transform the data back to the original
domain.  The availability of \emph{fast Fourier transforms} (FFTs) makes this approach practical.
\index{Fast Fourier transform (FFT)}
\index{FFT (Fast Fourier transform)}
\section{Sampling Theory}
\index{Sampling!theory|(}
\index{Sampling!theorem}

\index{Sampling!theorem}
\index{Band-limited}
The \emph{sampling theorem} states that if a function is \emph{band-limited} (i.e., the transform is zero for all 
radial frequencies $f > f_N$), then the continuous function $d(t)$ can be uniquely determined from knowledge 
of its sampled values given a sampling interval $\Delta t \leq  1/(2 f_N)$.  From distribution theory, we have
\begin{equation}
d_t = \sum^{+\infty} _{j= - \infty} d(t) \delta (t-j \Delta t) = \sum^\infty _{j= - \infty} 
d(j \Delta t) \delta ( t - j \Delta t) = d(t) \cdot \Delta (t),
\end{equation}
where
\index{Sampling!function}\index{Comb function}
\begin{equation}
\Delta (t) = \sum^{+\infty }_{j= - \infty} \delta ( t - j \Delta t)
\end{equation}
is the sampling or ``comb'' function in the time domain (Figure~\ref{fig:Fig1_sampl1}).
Thus, $d_t$ is the continuous function $d(t)$ sampled at the discrete times $j\Delta t$.
Consequently, it is true that the original signal $d(t)$ can be reconstructed exactly from
its sampled values $d_t$ via the \emph{Whittaker-Shannon} interpolation formula\index{Whittaker-Shannon interpolation}\index{Interpolation!Whittaker-Shannon}
\begin{equation}
	d(t) = \sum_{j=-\infty}^{+\infty} d_j \sinc \left( \frac{t - j\Delta t}{\Delta t} \right),
	\label{eq:WhittakerShannon}
\end{equation}
where $d_j = d(j\Delta t)$ are the sampled data values and the $\sinc$ function\index{$\sinc$ (sinc function)} is defined as
\begin{equation}
	\sinc(x) =  \frac{\sin \pi x}{\pi x}.
	\label{eq:sincfunction}
\end{equation}

Recall that the multiplication of two functions in 
the time domain is equivalent to the convolution of the their Fourier transforms in the frequency domain, 
hence
\PSfig[h]{Fig1_sampl1}{The sampling or ``comb'' function, $\Delta (t)$, represents mathematically what we
do when we sample a continuous phenomenon $d(t)$ at discrete, equidistantly spaced times.}
\noindent
\begin{equation}
d(t) \cdot \Delta (t) \leftrightarrow D(f) * \Delta (f).
\end{equation}	 
The time-domain expression is visualized in Figure~\ref{fig:Fig1_sampl2}.
\PSfig[H]{Fig1_sampl2}{Sampling equals multiplication of a continuous signal $d(t)$ with a comb function $\Delta (t)$ in the time-domain.}
\noindent
The transformed function $\Delta(f)$ can be shown to be a series of impulses as well (Figure~\ref{fig:Fig1_sampl3}).
\PSfig[H]{Fig1_sampl3}{The Fourier transform of the comb function, $\Delta (t)$, is another comb function, $\Delta (f)$, with a spacing of $1/\Delta t$ between impulses.}
\noindent
In the frequency domain, $d(t)$ is represented as $D(f)$ and illustrated in Figure~\ref{fig:Fig1_sampl4}.
We note that while the time-domain comb function $\Delta(t)$ is a series of impulses spaced every $\Delta t$,
the frequency-domain comb function $\Delta(f)$ is also a series of impulses, but spaced every $1/\Delta t$.
The time and frequency domain spacings of the comb
functions are thus reciprocal: A finer sampling interval leads to a larger distance between the impulses in the
frequency domain.
\PSfig[H]{Fig1_sampl4}{The Fourier transform of our continuous phenomenon, $d(t)$.  We assume it is band-limited so that the transform
goes to zero beyond the highest frequency, $f_N$.}
\noindent
Given $D(f)$ and $\Delta(f), D(f) * \Delta(f)$ is schematically shown in Figure~\ref{fig:Fig1_sampl5}.
\PSfig[H]{Fig1_sampl5}{Replication of the transform, $D(f)$, due to its convolution with the comb function, $\Delta (f)$.}
\noindent
If the impulses in $\Delta(f)$ are spaced  closer than $1/\Delta t$ then there will be some overlap between the $D(f)$ replicas
that are centered at the location of each impulse (see Figure~\ref{fig:Fig1_sampl6}).
\PSfig[H]{Fig1_sampl6}{Aliasing in the frequency domain occurs when the sampling interval $\Delta t$ is too large.}
\noindent
This overlap introduces \emph{aliasing} (which we shall discuss more later).  To prevent aliasing, we must ensure $\Delta t \leq 1/(2 f_N)$,
where $f_N$ is the highest (radial) frequency component present in the time series.  As mentioned earlier, we call $f_N$ the
\emph{Nyquist frequency} and the Nyquist sampling interval is $\Delta t  =  1/(2 f_N)$, hence $f_N = 1/(2 \Delta t)$.
\index{Aliasing}
\index{Frequency!Nyquist}
\index{Nyquist frequency}
As long as we follow the sampling theorem and select  $\Delta t \leq 1/(2 f_N)$, with $f_N$ being the highest 
frequency component, there will be no spectral overlap in $D(f)* \Delta (f)$ and we will be able to recover 
$D(f)$ completely.  Therefore (and to prove the sampling theorem) we recover $D(f)$ by truncating the signal:
\begin{equation}
D(f) = [ D (f) * \Delta (f) ] \cdot H(f),
\end{equation}
\index{Gate function}
\noindent
which is illustrated in Figure~\ref{fig:Fig1_sampl7} as a multiplication of the replicating spectrum with a \emph{gate} function, $H(f)$.
\PSfig[h]{Fig1_sampl7}{Truncation of the Fourier spectrum via multiplication with a rectangular gate function, $H(f)$.}

\subsection{Aliasing, again}
\PSfig[h]{Fig1_aliasing2}{Aliasing as seen in the time domain.  Thin line shows a phenomenon with period $P$.
The circles and heavy dashed line show
the signal obtained using a sampling rate of $1.25P$, while the squares and dashed
line show a constant signal ($f = 0$) obtained with a sampling rate of $2P$.}
Aliasing can be viewed from several angles.  Conceptually, if $\Delta t > 1/(2 f_N)$ (where $f_N$ is 
the highest frequency component in phenomenon of interest), then a high frequency component will \emph{masquerade} in the 
sampled series as a lower, artificial frequency component, as shown in Figure~\ref{fig:Fig1_aliasing2}.
\noindent
If $\Delta t$ is a multiple of $P$ (e.g., see the squares in Figure~\ref{fig:Fig1_aliasing2}), then this frequency component is indistinguishable from a 
horizontal line (i.e., a constant, with frequency $f = 0$).  If  $\Delta t = 5 P/4$ (see circles in Figure~\ref{fig:Fig1_aliasing2}) then this frequency 
component is indistinguishable from a component with frequency $1/5 P$ (i.e., period of $5P$).  Therefore, the 
under-sampled frequency components manifest themselves as lower frequency components 
(hence the word alias).  In fact, every frequency \emph{not} in the range
\index{Principal alias}
\begin{equation}
0 \leq f \leq 1/(2\Delta t)
\end{equation}
has an alias in that range --- this is its \emph{principal alias}.  Furthermore, any frequency $f_H > f_N$ 
will be indistinguishable from its principal alias.  That is, the actual frequency $f_H = f_N + \Delta f$ will appear as the aliased frequency $f_L = f_N - \Delta f$.
\index{Folding frequency}
\index{Frequency!folding}
	Because of this relationship, the Nyquist frequency $(f_N)$ is often called the \emph{folding frequency} since the
aliased frequencies ($f > f_N$) will appear at their principal aliases folded back into the range $\leq f_N$ (Figure~\ref{fig:Fig1_nyquist}).
Therefore, when computing the transform of a data set,
any frequency components in the phenomenon with true frequencies $f > f_N$ have been folded back into 
the resolved frequency range during sampling.  Consequently, we must carefully choose $\Delta t$ so that the powers at frequencies $f' > f_N$ are either small 
or nonexistent, or we must ensure that $f_N$ is high enough so that the aliased part of the spectrum only affects 
frequencies higher than those of interest ($f \leq f_I$, see Figure~\ref{fig:Fig1_nyquist}).

\PSfig[h]{Fig1_nyquist}{Aliasing and folding frequency.  Power at higher frequencies than the Nyquist ($f_N$) will
reappear as power at lower frequencies, ``folded'' around $f_N$.  This extra power (orange) is then added to the
actual power and the result is a distorted, total power spectrum (red). Selecting the Nyquist frequency so that aliasing only affects frequencies higher
than the frequencies of interest $(f \leq f_I)$.  In this case, the extra power (orange) that is folded around $f_N$
does not reach into the lower frequencies of interest, and consequently the total spectrum is unaffected for frequencies
lower than $f_I$.}
\index{Convolution|)}
\index{Sampling!theory|)}

\section{Aliasing and Leakage}
\index{Aliasing|(}
\index{Leakage|(}

\PSfig[H]{Fig1_AL}{The continuous and band-limited phenomenon of interest, represented both in the time and frequency domains.
Left column represents the time domain and the right column represents the frequency domain, separated by a vertical dashed gray line.
The multiply, convolve, and equal signs indicate the operations that are being performed. a) Continuous phenomenon, b) Sampling function,
c) Infinite discrete observations, d) Gate function, e) Truncated discrete observations, f) Assumed periodicity $T$, g) Aliasing and leakage of signal.}

We were exploring the relationship between the continuous and discrete Fourier transform 
and  found that we could illustrate the process graphically.  First, we found that we had to sample 
the time-series $d(t)$ (Figure~\ref{fig:Fig1_AL}a).  The sampling of the phenomenon by the sampling function $\Delta(t)$
(Figure~\ref{fig:Fig1_AL}b) is a multiplication in the time-domain, which implies a convolution in the frequency domain.
This sampling yields discrete observations in the time domain, but the multiplication in the time
domain equals a convolution in the frequency domain, enforcing periodicity of the spectrum (Figure~\ref{fig:Fig1_AL}c).
Depending on the chosen sampling interval we may or may not have spectral overlap (aliasing). 
This discrete infinite series must then be truncated to contain a finite number of observations.  
The truncation is conceptually performed by multiplying our infinite time series with a finite gate function.
\index{Gate function}
\index{Data!truncation}
This truncation of the infinite and periodic signal amounts to a multiplication in the time domain with a gate function, $h(t)$,
whose transform is
\begin{equation}
	H(f) = \sinc (fT) = \frac{\sin \pi fT}{\pi fT},
\end{equation}
with both functions displayed in Figure~\ref{fig:Fig1_AL}d.  This process results in the finite discrete observations
shown in Figure~\ref{fig:Fig1_AL}e.
It is this truncation that is responsible for introducing \emph{leakage}.

Leakage arises because the truncation implicitly 
assumes that the time-series is periodic with period $T$ (Figure~\ref{fig:Fig1_AL}f).  Consequently, the
discretization of frequencies is equivalent to enforcing a periodic signal (Figure~\ref{fig:Fig1_AL}g).
Because both the time and frequency domain functions have been convolved with a series of 
impulses (by $\Delta(t)$ in time and $\Delta(f)$ in frequency), both functions are periodic in $n$ discrete values, so 
the final discrete spectrum  (for a real series as shown here) between $0$ and $f_N$ represents the 
discrete transform of the series on the left (which is periodic over $T$).

	If the procedure in Figure~\ref{fig:Fig1_AL} is followed mathematically, it is seen that the continuous Fourier 
transform is related to the discrete Fourier transform by the steps outlined graphically above.  
These show that a discrete Fourier transform will \emph{differ} from the continuous transform by two effects:
\begin{enumerate}
\item Aliasing --- from discrete time domain sampling.
\item Leakage --- from finite time domain truncation.
\end{enumerate}
Aliasing can be prevented by choosing $\Delta t \leq 1/(2 f_N)$ or reduced as discussed previously.  Leakage is 
always a problem for most observed (and hence truncated) time series.
As discussed, leakage arises from truncation in the time domain, which corresponds to a 
convolution with a $\sinc$ function in the frequency domain.  Conceptually, consider the effect of time domain 
truncation (Figure~\ref{fig:Fig1_trunc1}).
Fourier analysis is essentially fitting a series of sines and cosines (using the harmonics of the 
fundamental frequency $1/T$) to the series $d(t)$.  Since the Fourier series is necessarily periodic, it 
follows that
\begin{equation}
d(T/2 + \Delta t) = d( - T/2).
\end{equation}
In other words, the transform is equivalent to that of a time series in which $d(t)$ is repeated every $T$ (Figure~\ref{fig:Fig1_trunc2}).
 
 \PSfig[h]{Fig1_trunc1}{Truncation of a continuous signal, the equivalent of multiplying the signal
with a gate function $h(t)$, determines the fundamental frequency, $f = 1/T$.}
\PSfig[h]{Fig1_trunc2}{Artificial high frequencies are introduced due to the forced periodicity of a truncated time-series, which
produces a discontinuous signal (highlighted by the gray regions).}
\noindent
The leakage (conceptually) thus results from the frequency components that must be present to 
allow the discontinuity, occurring every $T$, to be fit by the Fourier series.  If the series $d(t)$ 
is perfectly periodic over $T$ then there is no leakage because $d(T + \Delta t) = d(\Delta t)$ and the transition will be 
continuous and smooth across $T$. 

	To minimize leakage we attempt to minimize the discontinuity (between $d(0)$ and $d(T)$) or 
minimize the lobes of the $\sinc (fT)$ function convolving the spectrum.  This is accomplished by 
truncating the time series with a more gently sloping gate function (called a taper, fader, window, 
etc.). In other words, we use a smoother function that has fewer high frequency components (Figure~\ref{fig:Fig1_trunc3}).

\noindent
\index{Gate function}
\index{Bartlett window}
\index{Windowing!Bartlett}
\index{Hanning window}
\index{Windowing!Hanning}
\index{Parzen window}
\index{Windowing!Parzen}
\index{Hamming window}
\index{Windowing!Hamming}
\index{Bartlett-Priestley window}
\index{Windowing!Bartlett-Priestley}
The triangular function is the \emph{Bartlett} window, which is the rectangle function convolved with 
itself (hence its transform is $\sinc^2 (fT)$).  The dashed line is the split cosine-bell window.  Other 
windows include:  \emph{Hanning} (a cosine taper), \emph{Parzen} (similar to Hanning but decays sooner and  
more steeply, \emph{Hamming} (like Hanning), and \emph{Bartlett-Priestley} (which is quadratic and has ``optimal'' properties, 
satisfying specific error considerations.)  All of these tapers have transforms that are less 
oscillatory than the $\sinc$ function but they are also wider. Therefore, multiplication of the time series with one 
of these gate functions results in a convolution whose transform in the frequency domain will smear spectral peaks 
more than the $\sinc$ function did. In return, it will not introduce ripples far away from these spectral peaks.

\PSfig[h]{Fig1_trunc3}{Alternative gate functions and their spectral representations.  The less abrupt
a gate function is in the time domain the less ringing it will introduce in the frequency domain.}
	Note that multiplying by, say, a Hanning window will make $d(T/2+\Delta t) \sim d(-T/2)$, so 
the bothersome discontinuity is eliminated --- however damping of all $d(t)$ away from $d(T)$ acts like a modulation, 
which accounts for the smearing of spectral peaks. Hence, leakage is still not completely eliminated.
\index{Aliasing|)}
\index{Leakage|)}

\section{Complex Fourier Series}

As mentioned, Fourier series combine \emph{even} (cosine) and \emph{odd} (sine) components.
It is common to simplify these expressions by using \emph{complex notation},
in which a \emph{complex number} $z$ is written
\begin{equation}
	z = x + iy.
\end{equation}
Here, $x$ is considered the \emph{real} part, $y$ is the \emph{imaginary} part,
and $i = \sqrt{-1}$ is the \emph{imaginary number}\index{Imaginary number}.  The concept of considering a complex number as a
point $(x, y)$ in the complex plane was first presented by Caspar Wessel\footnote{No relation!} in 1799\index{Wessel, C.}, but like many mathematical inventions
others had independently dabbled with this both before and after 1800.  In the end, our old friend K. F. Gauss made
many contributions to complex number theory.

Simple rules govern elementary
operations on complex numbers.  For instance, addition and subtraction follow
\begin{equation}
	\begin{array}{c}
	z = (a + ib) + (c + id) = (a + c) + i(b + d),\\
	z = (a + ib) - (c + id) = (a - c) + i(b - d),
	\end{array}
\end{equation}
while multiplication becomes
\begin{equation}
	z = (a + ib)(c + id) = ac + iad + ibd + i^2bd = (ac - bd) + i(bc + ad).
\end{equation}
Division is converted to a multiplication by a complex number and a division by a real number
simply by multiplying both numerator and denominator by the denominator's \emph{complex conjugate}, for
which the imaginary part changes sign:
\begin{equation}
	z = \frac{(a + ib)}{(c + id)} = \frac{(a + ib)(c - id)}{(c + id)(c - id)} = \frac{(ac + bd) + i(bc + ad)}{(c^2 + d^2)}.
\end{equation}
The use of the complex notation simplifies much of the algebra associated with Fourier
analysis and is therefore mathematically more convenient to use.  This is especially true
in two and higher dimensions. We will employ
\emph{Euler's formula}\index{Euler's formula} for complex numbers in our expressions of the sine and cosine transform.
Thus, we digress to discuss Euler's formula.

\subsection{Euler's formula}
\index{Euler's formula}
\PSfig[H]{Fig1_Euler_stamps}{Leonard Euler (1707--1783)\index{Euler, L.} was one of the most productive mathematicians of his era.
You know you did something right when your mugshot ends up on stamps, even in former countries like DDR and the Soviet Union.
His famous equation $e^{i\pi} + 1 = 0$, relating the five most important numbers in mathematics, follows from his complex
relation given in (\ref{eq:eulerrelation}) for $\omega t = \pi$.}
Using a Taylor series expansion\index{Taylor series}, we can write
\begin{equation}
\sin x = x - \frac{x^3}{3!} + \frac{x^5}{5!} - \frac{x^7}{7!} + \cdots,
%(16.1a)
\end{equation}

\begin{equation}
\cos x = 1 - \frac{x^2}{2!} + \frac{x^4}{4!} - \frac{x^6}{6!} + \cdots,
%(16.1b)
\end{equation}
and
\begin{equation}
e^x = 1+ x + \frac{x^2}{2!} + \frac{x^3}{3!} + \frac{x^4}{4!} + \cdots.
%(16.2)
\label{eq:exseries}
\end{equation}
Now, let us introduce
$$
x = i \theta = \sqrt{-1} \theta.
$$
Inserting this expression into (\ref{eq:exseries}) yields
%		(16.3)
\begin{equation}
\begin{array}{rcl}
e^{i\theta} & = & \displaystyle 1 + i\theta-\frac{\theta^2}{2!} - \frac{i \theta^3}{3!}
+ \frac{\theta^4}{4!} + \frac{i \theta^5}{5!} - \cdots \\[14pt]
& = & \displaystyle \left ( 1 - \frac{\theta^2}{2!} + \frac{\theta^4}{4!} - \cdots\right )
+ i \left ( \theta - \frac{\theta^3}{3!} + \frac{\theta^5}{5!} - \cdots \right ) = \\[14pt]
& & \cos \theta + i \sin \theta.
\end{array}
\end{equation}
For $x = -i\theta$ we instead get
$$
\begin{array}{rcl}
e^{-i\theta} & = & \displaystyle 1 - i\theta - \frac{\theta^2}{2!} + \frac{i \theta^3}{3!} - \frac{\theta^4}{4!} - \frac{i \theta^5}{5!} - \cdots \\[14pt]
& = & \displaystyle \left ( 1 -\frac{\theta^2}{2!} + \frac{\theta^4}{4!} - \cdots\right ) - i\left ( \theta - \frac{\theta^3}{3!} + \frac{\theta^5}{5!} - \cdots\right ) = \cos \theta - i \sin \theta.
\end{array}
$$
We will associate $\cos \theta$ with the positive $x$-axis and $i \sin \theta$ with the positive $y$-axis.
Thus, $e^{i\theta}$ is a \emph{unit vector} rotated an angle $\theta$ counter-clockwise from the $x$-axis
(examine Figure~\ref{fig:Fig1_sincos} one more time). For the
angular frequencies in the Fourier series (where $\theta = \theta(t) = \omega t)$, the Euler relation is
\begin{equation}
\boxed{e^{\pm i\omega t} = \cos \omega t \pm i \sin \omega t,}
\label{eq:eulerrelation}
%(16.4)
\end{equation}
while the \emph{inverse} relationships are
\begin{equation}
\begin{array}{c}
\cos \omega t = \frac{1}{2} (e^{i\omega t} + e^{-i \omega t} ), \\[14pt]
\sin \omega t = \frac{1}{2 i } (e^{i\omega t} - e^{-i \omega t} ).
\end{array}
%(16.5)
\end{equation}
We will refresh (or introduce) a few concepts that apply to complex numbers.
The \emph{complex conjugate}\index{Complex!conjugate} (denoted by $^*$) of a function $f(x)$ is written $f^*(x)$, with components
\begin{equation}
%(16.6)
\begin{array}{lcr}
f(x) & = & R (x) + i I(x), \\[14pt]
f^{\ast}(x) & = & R (x) - i I(x).
\end{array}
\end{equation}
Here, $R(x)$ is the real part and $I(x)$ is the imaginary part of $f(x)$, respectively. The \emph{magnitude}\index{Complex!magnitude} or
amplitude of a complex number is given by
\begin{equation}
A(x) = | f(x) | = \sqrt{f(x) \cdot f^*(x)} = \sqrt{R^2 (x) + I^2 (x)},
%(16.7)
\end{equation}
while the \emph{phase} is obtained via
\begin{equation}
\phi(x) = \tan^{-1} \frac{I(x)}{R(x)}.
%(16.7)
\end{equation}

\subsection{Using the complex notation}
\index{Orthogonality!Fourier series|(}
By using Euler's formula the \emph{five} orthogonality relations discussed in Section~\ref{sec:FFS} become just \emph{one}:
\begin{equation}
%(16.8)
\sum^n_{\ell=1} e^{i \omega_j t_{\ell}} \cdot e^{-i \omega_k t_{\ell}} =
\left \{ \begin{array}{cc}
n, & j=k \\[12pt]
0, & \mbox{otherwise}
\end{array} \right.
\label{eq:DA2_16.8}
\end{equation}
where we now use $\ell$ to indicate the point number (since $i$ here represents the imaginary number).
The equivalent integral relationship is obviously
\index{Orthogonality!Fourier series|)}
\begin{equation}
\int^T_{0} e^{i \omega_j t} \cdot e^{-i \omega_k t} dt =
\left \{ \begin{array}{cc}
T, & j=k \\[12pt]
0, & \mbox{otherwise}
\end{array} \right.
\label{eq:orthocomplex}
\end{equation}
Any real-valued function can be represented as a complex-valued function with a zero
imaginary part.
Using the inverse Euler relations, the Fourier series (\ref{eq:fourierseries}) can now be rewritten in complex form
as follows, again assuming that $n$ is an even number:
\begin{equation}
\begin{array}{rcl}
d_{\ell} & = & \displaystyle \sum^{n/2}_{j=0} \left [ a_j \cos \omega_j t_{\ell} + b_j \sin \omega_j t_{\ell} \right] =
\displaystyle \sum^{n/2}_{j=0}
\left [ \frac{a_j}{2} \left (e^{i\omega_j t_{\ell}} + e^{-i\omega_j t_{\ell}} \right ) +
\frac{b_j}{2i} \left (e^{i\omega_j t_{\ell}} - e^{-i\omega_j t_{\ell}} \right ) \right] \\[14pt]
& = & \frac{1}{2}\displaystyle \sum^{n/2}_{j=0} \left [\left (a_j + \frac{b_j}{i} \right) e^{i\omega t_j}
+ \left (a_j - \frac{b_j}{i} \right ) e^{- i \omega_j t_{\ell}} \right ] \\[14pt]
& = & \displaystyle \frac{1}{2} \sum^{n/2}_{j=0} \left [ \left (a_j - ib_j \right ) e^{i \omega_j t_{\ell}} + \left (a_j + ib_j \right)
e^{-i \omega_j t_{\ell}} \right ].
\end{array}
\label{eq:DA2_16.10}
\end{equation}
%(16.10)
Since the second term contains $e^{-\omega_j t_{\ell}}$, we shall consider the effect of introducing \emph{negative}
frequencies (i.e., let $j$ take on negative values) in simplifying this expression further\index{Frequency!negative}. In general, for negative $j$ (i.e., $-j$),
\begin{equation}
\left (a_{-j} - ib_{-j}\right )e^{i\omega_{-j} t_{\ell}} = \left (a_j +ib_j \right ) e^{-i \omega_j t_{\ell}},
%(16.11)
\end{equation}
which we see are related by the complex conjugate definition
\begin{equation}
J_{-j} = {J^*}_j,
%(16.12)
\end{equation}
because
$$
\begin{array} {rcl}
a_{-j} &=& \displaystyle \frac{2}{n} \sum^n_{\ell=1} d_{\ell} \cos \omega_{-j} t_{\ell} =\frac {2}{n} \sum^n_{\ell=1} d_{\ell} \cos \frac{-2 \pi j}{T}t_{\ell} = \frac{2}{n} \sum^n_{\ell=1} d_{\ell} \cos \frac{2\pi j}{T}t_{\ell} = a_j,\\[14pt]
b_{-j} &=& \displaystyle \frac{2}{n} \sum^{n}_{\ell=1} d_{\ell} \sin \omega_{-j} t_{\ell} = \frac {2}{n} \sum^n_{\ell=1} d_{\ell} \sin \frac {-2 \pi j}{T} t_{\ell} = \frac{-2}{n} \sum^n_{\ell=1} d_{\ell} \sin \frac{2 \pi j}{T} t_{\ell} = -b_j.
\end{array}
$$
Consequently, $a_j$ is an \emph{even} function, $b_j$ is an \emph{odd} function, and obviously
$$
e^{i \omega_{-j} t_{\ell}} = e^ {-i\omega_j t_{\ell}}.
$$
Therefore, the second term of (\ref{eq:DA2_16.10}) can be dropped if we merely extend the sum over
$-n/2 \leq j < n/2$:
\begin{equation}
\boxed{d_{\ell} = \sum^{< n/2}_{j = -n/2} \frac{1}{2} (a_j - ib_j) e^{i\omega_j t_{\ell}}= \sum^{< n/2}_{j = -n/2} J_j e^{i \omega_j t_{\ell}},\quad \ell=1,n.}
\label{eq:DA2_16.13}
%(16.13)
\end{equation}
Notice that the complex form has \emph{twice} as many coefficients as the real form, reflecting the fact
that each value of $d_{\ell}$ contains a real and an imaginary component.  Thus, there are actually \emph{twice} as many $d_{\ell}$
values as before, even though for most observations the imaginary components will all be zero.
Equation (\ref{eq:DA2_16.13}) represents the general complex form of the Fourier series, with complex coefficients
\begin{equation}
J_j = \frac{1}{2} (a_j-ib_j).
%(16.14)
\end{equation}
By substituting the expressions for $a_j$ (\ref{eq:DCT}) and $b_j$ (\ref{eq:DST}) into the expression above
(for $J_j$) \emph{or} by multiplying Equation (\ref{eq:DA2_16.13}) by $e^{-i\omega_k t_{\ell}}$, summing over all $t_{\ell}$, and using the
orthogonality relation (\ref{eq:DA2_16.8}), we can derive the general complex form for $J_j$ via
$$
\begin{array}{rcl}
\displaystyle \sum^{n}_{\ell=1} d_{\ell} e^{-i \omega_k t_{\ell}}&=& \displaystyle \sum^{< n/2}_{j = -n/2} J_j \sum^{n}_{\ell=1} e^{i \omega_j t_{\ell}} e^{-i \omega_k t_{\ell}},\\
%(16.15a)
\end{array}
$$
yielding
\begin{equation}
\displaystyle \sum^{n}_{\ell=1} d_{\ell} e^{-i \omega_k t_{\ell}} = n J_k.
%(16.15)
\end{equation}
Hence, and swapping the dummy indices $j$ and $k$, we find
\index{Complex!discrete Fourier transform}
\begin{equation}
\boxed{J_j = \displaystyle \frac{1}{n} \sum^{n}_{\ell=1} d_{\ell} e^{-i \omega_j t_{\ell}} , \quad -\frac{n}{2} \leq j < \frac{n}{2}.}
\label{eq:DA2_16.15}
%(16.15)
\end{equation}
Equation (\ref{eq:DA2_16.15}) is the \emph{complex discrete Fourier transform} of the series $d_{\ell}$ and equation (\ref{eq:DA2_16.13}) is
the \emph{complex discrete inverse Fourier transform}. Since $J_{n/2} = J_{-n/2}$ (and both are real since $b_{n/2} = 0$) we need not compute it for each index,
hence $-n/2 \leq j < n/2$. This arrangement also eliminates the awkward need to define $a_0$ and $a_{n/2}$ as one-half their values.

When $d_{\ell}$ is a \emph{real-valued} series, then $d_{\ell}=d^*_{\ell}$~because all the imaginary parts are zero, so
$$
J_j=\frac{1}{n} \sum^{n}_{\ell=1} d_{\ell} e^{-i \omega_j t_{\ell}} = \frac{1}{n} \sum^{n}_{\ell=1} d_{\ell} (\cos \omega_j t_{\ell} - i \sin \omega_j t_{\ell})
= \frac{1}{n} \sum^{n}_{\ell=1} d_{\ell} \cos \omega_j t_{\ell} - \frac{i}{n} \sum^{n}_{\ell=1} d_{\ell} \sin \omega_j t_{\ell}
$$
and hence
$$
J_{-j} = J^\ast_j.
$$
Therefore, the transform (i.e., $J_j$) is completely determined by the positive values of $j$, as
developed earlier, i.e., they are completely determined by the $n$ values of $a_j$ and $b_j$.
Specifically, if we need to recover $a_j$ and $b_j$ from the $J_j$ coefficients we simply note that
\begin{equation}
	J_{-j} + J_j = J^\ast_j + J_j = \frac{1}{2} (a_j+ib_j) + \frac{1}{2} (a_j-ib_j) = a_j,
\end{equation}
and likewise
\begin{equation}
	J_{-j} - J_j = J^\ast_j - J_j = \frac{1}{2} (a_j+ib_j) - \frac{1}{2} (a_j-ib_j) = i b_j.
\end{equation}
The negative frequencies confuse many practitioners of spectral analysis.  What are they? What do they represent?\index{Frequency!negative}
Are they components going backwards in time or space? 
It is important to remember that we simply introduced these as a \emph{mathematical convenience} in order to arrive at a
simple and compact transform, i.e., (\ref{eq:DA2_16.15}).  Physically, there are only positive frequencies.

\subsection{The continuous Fourier transform in 1-D}

For a continuous function $g(t)$,  equivalent expressions make up the \emph{forward} and \emph{inverse Fourier transform} pair,
typically written in complex (and symmetric) form as
\begin{equation}
%(16.16a)
G(f) = \int ^\infty _{-\infty} g(t) e^{-i 2 \pi t f} dt
\label{eq:FT1D}
\end{equation}
%(16.16b)
and
\begin{equation}
g(t) = \int ^\infty _{-\infty} G(f) e^{+i 2 \pi t f} df,
\label{eq:IFT1D}
\end{equation}
respectively, for radial frequency $f$.  As noted, the main differences between the continuous transforms and the discrete transforms
(which we necessarily must use to operate on observed data) are:
\begin{enumerate}
	\item Our data have a finite, nonzero sampling interval, whereas the theory considers continuous distributions.
	This difference opens up the possibility of \emph{aliasing}.
	\item Our data have a finite length (i.e., a fundamental period), whereas the theoretical distributions exist over all time.
	This limitation may lead to issues involving \emph{leakage}.
\end{enumerate}

\section{Computing Fourier Transforms}
Most practitioners of spectral analysis will find themselves performing the Fourier transform via a special
algorithm known as the \emph{Fast Fourier Transform} (FFT).  This algorithm as well as practical information on how
to use it and how to prepare our data for it is the focus of this section.

\subsection{The Fast Fourier Transform (FFT)}
\index{Fast Fourier transform (FFT)|(}
\index{FFT (Fast Fourier transform)|(}

The discrete Fourier transform involves $n$ coefficients that each requires $n$ multiplications to be determined.
Hence, the entire calculations is $O(n^2)$ in computer time.  Can we do better than that?
Since $\omega_j = 2 \pi j / T$, $T = n \Delta t$, and $t_\ell = \ell \Delta t$ for $\ell =0, ..., n-1$ it follows that the exponential term
in (\ref{eq:DA2_16.15}) is
\begin{equation}
e^{ -i \omega_j t_{\ell}} = e^{-\frac{i 2 \pi j}{n \Delta t}\ell \Delta t} = \left [ e^{-\frac{2 \pi i}{n}} \right ]^{j\ell} = W^{j\ell}.
\end{equation}
Note what happened to time here; it cancels, leaving just a ratio of integers.
We may now write
\begin{equation}
J_j = \displaystyle \frac{1}{n} \sum^{n-1}_{\ell=0} y_{\ell} W^{j\ell}.
\end{equation}
It is common in FFT implementations to perform the division by $n$ separately (at the end), so let us just consider
\begin{equation}
J_j = \displaystyle \sum^{n-1}_{\ell=0} y_{\ell} W^{j\ell}.
\end{equation}
You may note that if $j = 0$ we obtain $J_0 = \bar{y}$ (apart from the $1/n$ term, that is).  Now, consider this rearrangement:
\begin{equation}
J_j = \displaystyle \sum^{n-1}_{\ell=0} y_{\ell} e^{-\frac{2 \pi i j\ell}{n}} =
	\sum^{n/2-1}_{\ell=0} y_{2\ell} e^{-\frac{2 \pi i j (2\ell)}{n}} +
	\sum^{n/2-1}_{\ell=0} y_{2\ell+1} e^{-\frac{2 \pi i j (2\ell+1)}{n}}.	
\end{equation}
Here, we have simply rewritten the sum as two separate series containing the even ($2 \ell$) and odd ($2 \ell + 1$) terms of
the observations, respectively.  This expression can be manipulated further to yield
\begin{equation}
J_j = \displaystyle \sum^{n/2-1}_{\ell=0} y_{2\ell} e^{-\frac{2 \pi i j \ell}{(n/2)}} +
	e^{-2 \pi i j / n} \sum^{n/2-1}_{\ell=0} y_{2\ell+1} e^{-\frac{2 \pi i j \ell}{(n/2)}},
\label{eq:FFT1}
\end{equation}
which we can write as
\begin{equation}
J_j = J_j^e + W^j J_j^o.
\end{equation}
\PSfig[h]{Fig1_FFT_split}{Each dot-product making up a Fourier transform for a single frequency can be split in two by considering the odd (gray)
and even (white) observations separately, and by recursively applying this partitioning
until we reach a single element we can speed of the Fourier transform enormously.}
The first term is the Fourier transform of the even-numbered observations (i.e., considering only the 0th, 2nd, 4th, etc., values), while the
second term is the Fourier transform of just the odd-numbered observations (i.e., the 1st, 3rd, 5th, etc. values), scaled by the constant $W^j$. Since
each transform only deals with $n/2$ points we find the calculation time to be proportional to $O(2(n/2)^2) = O(n^2/2)$.
So while (\ref{eq:FFT1}) is mathematically identical to (\ref{eq:DA2_16.15}), the partitioning into two sums leads to a 50\% reduction in computation time.
But wait, there is more! The partitioning idea can be continued recursively on the two separate $J_j^e$ and $J_j^o$
sums as well.  For instance, the two terms in the transform for $J_j^e$ requiring $n/2$ data values each can themselves be written as
\begin{equation}
\begin{array}{ccc}
	J_j^e & = & J_j^{ee} + W^j J_j^{eo}, \\[14pt]
	J_j^o & = & J_j^{oe} + W^j J_j^{oo},
\end{array}
\end{equation}
with each term being a transform requiring just $n/4$ data points.  If $n$ is initially some power of 2, then carrying this splitting
all the way to a single number (whose transform is itself) yields a computational workload that is $O(n \log n)$.
These are huge savings compared to our initial estimate of $O(n^2)$!  For instance, with $n = 10^6$, the difference is
a factor of 50,000, while for $n = 10^7$ the operations are 400,000 times faster.  For each magnitude in $n$, this ratio
increases by another factor of $\sim 8$.  This clever result is essentially the \emph{Cooley-Tukey}\index{Cooley, J. W.}
\index{Tukey, J. W.}\index{Cooley-Tukey FFT algorithm} FFT algorithm from the 1960s.  However, its
origin goes further back to the 1940s with \emph{Danielson}\index{Danielson, G. C.} and \emph{Lanczos}\index{Lanczos, C.} and their work
on x-ray scattering as well as further back to \emph{Gauss}\index{Gauss, K. F.}.  Apparently, every neat mathematical idea can eventually
be traced back to Gauss or Euler...

\subsection{FFT implementations}

Both MATLAB and Octave implement the fast Fourier transform in a similar fashion, as do most mathematical function libraries.  There are three
functions we will need to become familiar with:
\begin{description}
	\item [fft:] The forward fast Fourier transform.\index{MATLAB!fft}\index{fft (MATLAB)}
	\item [ifft:] The inverse fast Fourier transform.\index{MATLAB!ifft}\index{ifft (MATLAB)}
	\item [fftshift:] Rearranging the order of spectral coefficients.\index{MATLAB!fftshift}\index{fftshift (MATLAB)}
\end{description}
To see how these work and what they expect as input we will look at the arguments of data and frequencies.
Let our data be represented by the data array
\begin{equation}
\mathbf{d} = [ \quad d_1, \quad d_2, \quad \ldots, \quad d_n \quad ],
\end{equation}
where $n$ is even and represents the number of data points.  Then, taking the transform [i.e., \texttt{J = fft(d);} in MATLAB or Octave] yields the array
\begin{equation}
\mathbf{J} = [ \quad J_0, \quad J_1, \quad J_2, \quad \ldots, \quad J_{\frac{n}{2}-1}, \quad J_{-\frac{n}{2}}, \quad J_{-\frac{n}{2}+1}, \quad \ldots, \quad J_{-1} \quad ].
\label{eq:FFT2}
\end{equation}
Clearly, the amplitudes are split between the positive (first half) and negative (second half) sets of the coefficients, i.e.,
\begin{equation}
| J_k | = | J_{-k} |.
\end{equation}
Plotting $J(f)$ is awkward since the order is discontinuous with respect to frequency, $f$.  To place $J_0$ symmetrically
in the middle requires shifting of the coefficients.  This is done by \emph{fftshift} [i.e., \texttt{J = fftshift(J);}], which results in the array
\begin{equation}
J = [ \quad J_{-\frac{n}{2}}, \quad J_{-\frac{n}{2}+1}, \quad \ldots, \quad J_{-1}, \quad J_0, \quad J_1, \quad J_2, \quad \ldots, \quad J_{\frac{n}{2}-1} \quad ],
\end{equation}
where the very first item ($J_{-\frac{n}{2}}$) is the real coefficient associated with the Nyquist frequency (since the
sine term is identically zero), and $J_0$ reflects the mean (real) value of $d$.  These are the only two coefficients
that \emph{do not} appear twice in (\ref{eq:FFT2}).
Note that your particular time coordinates $t_{\ell}$ play \emph{no part} in the transform.  It is implicitly assumed that
$t_{\ell} = {\ell}\Delta t$ and that $t_0 = 0$, so make sure you are careful in selecting your origin time.

\subsection{Detrending and windowing}

\PSfig[h]{Fig1_taper}{(top) Original topographic profile.  If this data set were subjected to Fourier analysis,
then the jump discontinuity due to the forced periodicity would introduce leakage. (middle) We reduce the leakage
by using a smooth rather than rectangular window function.  Here, a \emph{Tukey} window is shown, which is a cosine taper
between zero at the ends and unity in the middle. (bottom) Multiplying our data with the window function yields a data
set that no longer has a jump discontinuity.}
\index{Tukey window}
Prior to computing spectral estimates with an FFT we should do our utmost to reduce the influence of leakage.  As we discussed,
leakage arises due to truncation of a hypothetically infinite data series to a finite length data set (i.e., when making our observation), and the range of
the observed data becomes our fundamental (and probably artificial) period.  Since spectral analysis uses
periodic functions (sines and cosines) to represent the data, we are in effect forcing our data to be periodic as well, with the presumably
arbitrary data range as the fundamental period.  Unless the data happen to start and end on the same value we may have a potentially large
offset between the two values, which essentially constitutes a step function (e.g., upper panel of Figure~\ref{fig:Fig1_taper}).  Decomposing the data into sines and cosines
means we will approximate this step using  Fourier building blocks, thus leaking energy over a wide range of frequencies.
The standard approach to minimizing this problem is to use a gentler gate function than the rectangle we implicitly used when we captured our
data.  Figure~\ref{fig:Fig1_taper} (middle panel) shows what is called a Tukey window, which essentially is a half cosine wavelength that
connects the area outside the window (zero weight) to the central portion of the window (unit weight).  By multiplying our
observed data by this window we eliminate the step mismatch between the start and end of the data (e.g., lower panel of Figure~\ref{fig:Fig1_taper}), and this data modification dramatically
reduces the effect of leakage.  Note, however, that leakage is not completely eliminated since our tapering still changes the observed signal
in some way.  For one thing, it effectively shortens the fundamental period by an amount proportional to the ramp margin.

\subsection{Zero-padding}
\label{sec:zeropad}
These are two common and seemingly related problems in spectral analysis:
\begin{enumerate}
	\item For many data sets, the frequency resolution $\Delta f = 1/T$ may not be small enough to resolve spectral
	components that are closely spaced (remember, all the Fourier frequencies we use are integer multiples of $\Delta f$).
	\item The discrete points in the raw periodogram may be too widely spaced to resolve the actual frequencies
	of the peaks in question.    
\end{enumerate}
\PSfig[H]{Fig1_zeropad}{Zero-padding our data means to extend the range of the windowed data by adding zeros until the length
(i.e., the number of data points) reaches the next power of 2.}
Unfortunately, the first issue can only be addressed by collecting a longer time series so as to decrease $\Delta f$ to
a point where the two peaks are clearly separated in the periodogram.  Without this added resolution such peaks will
be seen as one wider (blurred) peak.  The second problem, however, can be addressed by
extending the finite time series with zeros prior to taking the FFT (Figure~\ref{fig:Fig1_zeropad}).  By adding zeroes, the length of the data
series ($n$) increases, effectively adding additional frequency components in between those that would be obtained for the
original, non-padded series.  In essence, we obtain an \emph{interpolation} of the spectral density estimates.  Note that
zero-padding helps fill in the shape of the spectrum but of course there is no improvement in the fundamental frequency resolution.
Nevertheless, zero-padding is widely used for several reasons:
\begin{enumerate}
	\item It smooths the shape of the periodigram via spectral interpolation.
	\item It may resolve potential ambiguities where the frequency difference between line spectra is greater
	than the fundamental frequency resolution.
	\item It may help define the exact frequency of the peaks of interest by reducing the quantization of the power.
	\item It may increase $n$ to an integer power of 2, thus speeding up the analysis when the FFT is used.
\end{enumerate}
While these benefits are all good, we note that adding zeros does not help us to distinguish closely spaced frequency components
that could not be resolved in the original time series.

\index{FFT (Fast Fourier transform)|)}
\index{Fast Fourier transform (FFT)|)}


\section{Filtering}
\index{Filtering|(}

Filtering data is a major data processing procedure that is used throughout the natural sciences.  Such procedures are applied to
reduce ``undesired'' features in the observations or to enhance the ``desired'' features.  What the desired and
undesired features are may change completely from application to application and may even be interchangeable.
Filters may be expressed as convolutions (and thus may take advantage
of the speed-up provided by the convolution theorem) or they must be executed relatively slowly in the time (or space) domains.
Both types of filters play important roles in data processing and interpretation.  Before we discuss a range of well-known
filters we will digress to revisit convolution and examine various properties of the Fourier transform.

\subsection{Convolution and the Fourier Transform}

Before revisiting convolution, we need to take a closer look at the analytical
Fourier transform as given by (\ref{eq:FT1D}).
Note that this general (continuous) form differs from the discrete form (\ref{eq:DA2_16.15}) in that it is no
longer dependent upon discrete Fourier frequencies nor is it restricted to a finite period, $T$.
Of course, for an actual transform of a discrete time series the discrete Fourier transform is always
used, thus in practice we will consider the Fourier frequencies only. The general integral form is used to transform
\emph{analytic} expressions, such as may arise in the solution to differential equations, and is thus useful in other ways as well.

	Using the notation of (\ref{eq:FT1D}), Fourier transform pairs are usually designated by
$$
g(t) \Leftrightarrow G(f),
$$
i.e., the transform of a function $g(t)$, which is a function of time (or distance) and traditionally denoted by lower-case letters,
is represented by the corresponding uppercase version of
the function, $G(f)$, and varies as a function of frequency $f$ (or wavenumber $k$ for spatial data).
\begin{example}
We will look at an important analytic Fourier transform pair.  Consider the instantaneous impulse or delta function
$$
%		(19.2)
g(t)= \delta (t).
$$
Per the definition of a \emph{Dirac delta function}\index{Dirac delta function}, its transform will be
\begin{equation}
G(f)= \int_{-\infty}^\infty \delta (t) e^{-i 2 \pi f t} dt= e^{0} = 1.
\label{eq:spike}
\end{equation}
\PSfig[H]{Fig1_delta}{(a) A delta-function (or spike) in one domain (here, the time domain) will be
transformed (b) to a constant in the other domain.  Note that the phase spectrum is zero since there
are no imaginary part to the transform.}
Thus, the narrowest possible function in the time domain has the broadest possible spectral expression, reflecting
the reciprocal nature of time and frequency (Figure~\ref{fig:Fig1_delta}).  Another way to look at this result is to say that an impulse is simply
a sum of cosines of unit amplitude for all frequencies.  This equivalence is used in geophysical exploration for oil and gas on
land. Instead of setting off explosions (i.e., an impulse), \emph{Vibroseis}\index{Vibroseis} trucks generate sweeping harmonic signals of increasing frequency.
In post-processing, the Earth's response to all these harmonics can be combined into the equivalent response to an actual impulse.  Because we cannot
use frequencies approaching infinity the resulting impulse will have a finite width, such as indicated in Figure~\ref{fig:Fig1_spike}.
\PSfig[h]{Fig1_spike}{Per (\ref{eq:spike}), if we sum up all the harmonics of uniform amplitude we should obtain a delta-function.
Here we add the first 40 harmonics, yielding an approximate impulse, which illustrates the principle behind Vibroseis.}
\end{example}
Other uses of analytic Fourier transforms include obtaining insight into the nature of spectra obtained from real processes
by studying those of very simple models.

\subsection{The scale theorem}
\index{Scale theorem}
\index{Theorem!scale}
\PSfig[h]{Fig1_scaletheorem}{(a) A wide (heavy line) and narrow (dashed line) function in the time domain will transform to (b) a
narrow and wide function in the frequency domain, respectively.}
What happens if we contract or expand the time axis?  We find the rule
\begin{equation}
\mbox{If} \ g(t) \leftrightarrow G (f)\ \mbox{then} \ g(at) \leftrightarrow \displaystyle \frac{1}{|a|} G \left (\frac{f}{a}\right ).
%(19.6)
\label{eq:scaletheorem}
\end{equation}
Proof, using $u = at$:
$$
\int^\infty_{-\infty} g(at) e^{-i 2 \pi f t} dt = \frac{1}{a} \int^\infty_{-\infty} g(u) e^{-i 2 \pi u/a} du = \frac {1}{|a|} G\left (\frac{u}{a}\right ).
$$
If $a$ is negative, then the limits of integration are reversed, so the frequency domain
amplitudes are still scaled positively, hence the $|a|$. Therefore, if a time scale is
contracted by $a$, the frequency is expanded $(1/|a|)$; this is expected since time is inversely
proportional to frequency. The amplitude of the frequency is decreased, which maintains
a constant area relative to the unscaled function.

\subsection{The shift theorem}
\index{Shift theorem}
\index{Theorem!shift}
If we apply a shift to the time axis, we find another rule:
\begin{equation}
\mbox{If} \ g(t) \leftrightarrow G(f) \ \mbox{then} \ g(t-t_0) \leftrightarrow G(f) e^{-i 2\pi f t_0}.
\label{eq:shifttheorem}
\end{equation}
%(19.7)
In polar form,
$$
G(f) e^{-i 2 \pi f t_0}=[A(f)e^{i \theta (f)}] e^{-i 2 \pi f t_0} = A(f)e^{i (\theta(f)- 2 \pi f t_0)}.
$$
Therefore, a shift in the time axis (a \emph{translation}) does not affect the amplitude spectrum, it
only affects the \emph{phase} spectrum.  It linearly shifts the phase by $\theta_0 = 2 \pi f t_0$, thus the
slope of the phase shift is directly proportional to the time shift $t_0$ (i.e., if phase were plotted
against $f$, then the slope would equal $t_0$).  This means the transform of a shifted impulse $\delta(t-t_0)$ is
$e^{i(\theta-\theta_0)}$, showing that the transform of random noise $n(t)$ has a constant amplitude spectrum $N(f) = 1$
but a random phase spectrum.

\subsection{The gate function}
\index{Gate function}
Finally, we will look at the transform of a very useful construction known as a
\emph{gate} function:
\begin{equation}
g(t)=\left \{\begin{array} {cl} 1, & |t| \leq T/2\\
0, & \mbox{elsewhere}
\end{array} \right..
\end{equation}
		%(19.9)
It transforms as follows:
$$
G(f)=\int^\infty_{-\infty} g(t) e^{-i 2\pi f t} dt = \int^{T/2}_{-T/2} e^{-i 2 \pi f t} dt = \int^{T/2}_{-T/2} (\cos 2\pi f t - i\sin 2\pi f t) dt = \int^{T/2}_{-T/2} \cos 2\pi f t dt.
$$
Here we have used the fact that the sine term, being odd, will integrate to zero. Using the substitution $u = 2 \pi f t$ we obtain
$$
G(f)=\frac{1}{2\pi f} \int^{\pi f T}_{-\pi f T} \cos u du = \frac{1}{2 \pi f}\sin u \mid ^{\pi f T}_{-\pi f T}= \frac {1}{2 \pi f}\left (\sin(\pi f T)-\sin(-\pi f T)\right).
$$
We rearrange this expression to give
\begin{equation}
G(f)= T \frac{\sin(\pi f T)}{\pi f T}= T \sinc (f T),
\end{equation}
%(19.10)
which is the definition of the $\sinc$-function.
\PSfig[h]{Fig1_sinc}{Truncating an infinite and periodic signal amounts to multiplying it in time with a gate function, $g(t)$,
whose transform $G(f) = T\sinc (fT)$.}

\subsection{The convolution theorem}
\index{Theorem!convolution}
\index{Convolution theorem}
\label{sec:convolutiontheorem}
	The \emph{convolution theorem} simply states that a convolution in the time domain is equivalent to
a multiplication in the frequency domain, and vice-versa, written as
\begin{equation}
s(t) * g(t) \leftrightarrow S (\omega) \cdot G (\omega).
%		(19.11)
\end{equation}
If $h(t) = s(t) * g(t)$, then
$$
H (\omega) = \int^\infty_{-\infty} h(t) e^{-i \omega t} dt = \int^\infty_{-\infty} \left [ \int^\infty_{-\infty} s(\tau) g (t- \tau) d\tau \right ]
e^{-i \omega t} dt.
$$
Notice the lag is defined as $t$ in the innermost integral. Interchanging the order of integration (this assumes
continuity of the integrand over $\pm\infty$) gives
$$
H (\omega) = \int^\infty_{-\infty} s(\tau) \left[ \int^\infty_{-\infty} g( t - \tau) e^{-i \omega t} dt \right ] d\tau.
$$
Let $u = t - \tau$, so $dt = du$. Then,
$$
\begin{array}{rcl}
H (\omega) & = & \displaystyle \int^\infty_{-\infty} s(\tau) \left[ \int^\infty_{-\infty} g(u)
e^{-i \omega(u + \tau)} du \right ] d\tau = \int^\infty_{-\infty} s(\tau) e^{-i \omega \tau}
\left[ \int^\infty_{-\infty} g(u) e^{-i \omega u} du \right ] d\tau \\[10pt]
& = & \displaystyle G (\omega) \int^\infty_{-\infty} s(\tau) e^{-i \omega \tau} d\tau = S(\omega) \cdot G (\omega).
\end{array}
$$
The result of this theorem provides the basic tool from which (among other things) the
continuous Fourier transform is related to the discrete Fourier transform. Also, because
of symmetry,
\begin{equation}
s(t) \cdot g (t) \leftrightarrow S (\omega) * G (\omega).
		%(19.12)
\end{equation}

\subsection{Parseval's theorem}
\index{Theorem!Parseval}
\index{Parseval's theorem}
\label{sec:Parseval}

\emph{Parseval's theorem} is a statement relating the variance of a data set to its power spectrum.
For a continuous function $g(t)$, the statement becomes
\begin{equation}
	\int_{-\infty}^{+\infty} \mid g(t) \mid^2 dt = \int_{-\infty}^{+\infty} \mid G(f) \mid^2 df.
	\label{eq:parceval}
\end{equation}
In Section~\ref{sec:periodogram} we derived a discrete version of Parseval's theorem by using a Fourier series expansion
for $g(t)$, removing its mean, and computing the variance of the finite time series.  Using
our complex notation we may now simply write
\begin{equation}
y_{\ell} = \sum^{< n/2}_{j \geq -n/2} J_j e^{i \omega_j t_{\ell}}, \quad \ell = 1,n
\end{equation}
and form the discrete version of (\ref{eq:parceval}) via (\ref{eq:DA2_16.15}) to yield
\begin{equation}
\sum^{n}_{\ell=1} y_{\ell}^2 = \sum^{n}_{\ell=1} \left \{ \left [ \sum^{< n/2}_{j \geq -n/2} J_j e^{i \omega_j t_{\ell}} \right ] \left [ \sum^{< n/2}_{k \geq -n/2} J_k e^{i \omega_k t_{\ell}} \right ] \right \}.
\end{equation}
Because of orthogonality (\ref{eq:DA2_16.8}), only the $j = k$ cross-terms will be nonzero and we find
\begin{equation}
\sum^{n}_{\ell=1} y_{\ell}^2 = n\sum^{< n/2}_{j \geq -n/2} J_j^2.
\end{equation}
By normalizing by $(n-1)$ we see the partitioning of data variance into its frequency contributions:
\begin{equation}
s^2 = \frac{1}{n-1}\sum^{n}_{\ell=1} y_{\ell}^2 = \frac{n}{n-1}\sum^{< n/2}_{j \geq -n/2} J_j^2  \approx \sum^{< n/2}_{j \geq -n/2} J_j^2.
\end{equation}

\subsection{Convolution filters}

	Filtering of data is typically performed in order to either smooth the signal or suppress power at 
particular frequencies or wavenumbers.  So far, we have learned that filtering can be considered an 
example of a convolution between the data $d(t)$ and the filter $p(t)$, i.e.,
\begin{equation}
h(t) = d(t) * p(t).
\end{equation}
Thus, we can immediately take advantage of the convolution theorem and write
\index{Convolution theorem}
\begin{equation}
H(f) = D(f) \cdot P(f).
\end{equation}
Hence, we can simply take the Fourier transform of $d(t)$ and multiply its spectral components $D(f)$ by 
$P(f)$.  For example, we learned previously that a simple MA filter consists of convolving the signal with a 
rectangle function of width $W$ (Figure~\ref{fig:Fig1_filter1}).\index{Filtering!moving average}
\PSfig[h]{Fig1_filter1}{Moving average filter convolution seen in the time domain.}
\noindent
Figure~\ref{fig:Fig1_filter2} shows how this operation might look like in the frequency domain.

\PSfig[h]{Fig1_filter2}{Moving average filtering seen in the frequency domain.}
\noindent
Thus, the MA filter output is obtained by multiplying the signal's Fourier coefficients by a $\sinc$ function.  This 
means that some of the coefficients will have their signs reversed, which we know equates to a 
phase change of $180^{\circ}$.  Also, while the filter coefficients do fall off with increasing $f$, they do so in a very 
oscillatory way and take a long time to approach zero.  This fact would suggest that the $\sinc$ function is a 
poor choice for a filter if you really wanted to cut out, say, high frequency information.  Perhaps we 
should instead design $P(f)$ directly so that it truncates all power at frequencies higher than $f_{\mbox{cut}}$?
Surely, by simply removing all power from frequencies of no interest we should retain the part of the signal
that we are most interested in.
This alternative operation, which is simply carried out in the frequency domain, is illustrated in Figure~\ref{fig:Fig1_filter3}.
The filtered output is finally obtained by transforming $H(f)$ back to the time domain (Figure~\ref{fig:Fig1_filter4}).
\PSfig[h]{Fig1_filter3}{Truncation of high frequencies is straightforward in the frequency domain.}

We discover that the output has a lot of ``ripples'' (here somewhat exaggerated) at approximately the period corresponding to $1/f_{\mbox {cut}}$.  
This ``ringing'' is caused by our convolving of $d(t)$ with $p(t)$, which in this case is a $\sinc$ function.  Because all power 
at higher frequencies are completely eliminated, the power at the remaining highest frequency $f_{\mbox{cut}}$
stands out.  This effect is referred to as ``Gibbs' phenomenon''.
\index{Ringing}
\index{Gibbs' phenomenon}

\PSfig[H]{Fig1_filter4}{``Ringing'' in the time domain due to truncation in the frequency domain.}
	It is clear from these two examples of the use of a rectangular-shaped function that the price we 
pay for having a sharp truncation of filter coefficients in one domain is excessive ``ringing'' in the 
other domain.  This, of course, is simply the convolution theorem at work.  A rectangle function 
is necessarily a poor choice for a filter because of the slowly decaying oscillatory nature of the 
$\sinc$ function.

\subsubsection{Gaussian filter}
\index{Filter!Gaussian}
\index{Gaussian filter}
\label{sec:gaussianfilter}
\PSfig[H]{Fig1_FilterWidth}{Gaussian filter is typically specified by its ``full width'', defined to be
$W = 6\sigma$.  While a Gaussian curve never reaches zero weight, its amplitude will have fallen to
$\sim 1$ percent of peak value at $t = \pm 3\sigma$.}
While we do desire to find a filter that rapidly tapers off power, say, beyond a certain
frequency, we know that in the limit, when the gradual taper approximates a step-function,
our transform will approximate a $\sinc$-function. How do we find a good filter?
It will depend on the application. Because of the inverse relationship between time and frequency (i.e., $f =
1/T)$, we saw that the \emph{scale theorem}\index{Scale theorem}\index{Theorem!scale} (\ref{eq:scaletheorem}) stated
$$
g(at)\leftrightarrow \frac{1}{|a|} G \left (\frac{f}{a} \right ).
$$
Thus, making a filter broader in one domain narrows it in the other. In the limit $(T \rightarrow \infty )$
we recover the transform pair
$$
1 \leftrightarrow \delta(t).
$$
Clearly, somewhere in the middle there must be a function that looks similar in both
domains, i.e.,
$$
g(t) \leftrightarrow G(f) = g(f).
$$
There is such a function; here we will simply state that the Gaussian normal distribution
behaves this way. Let
$$
g(t)=e^{-\pi t^2}.
$$
To find its transform we must follow (\ref{eq:FT1D}) and integrate
$$
G(f) = \int^{+ \infty}_{-\infty}
e ^{- \pi t^2} e ^{-2 \pi ift} dt =
\int ^{+ \infty}_{-\infty}
e^{-(\pi t ^2 + 2 \pi ift)}
dt.
$$
Notice that the exponent is almost of the form $(a+b)^2$. We complete the square by adding and
subtracting the missing term:
$$
G(f) = \int^{+ \infty}_{-\infty} e ^{-( \pi t ^2 + 2 \pi ift + i^2 \pi f ^2 -i ^2 \pi f^2)}
dt = \displaystyle \int ^{+ \infty}_{-\infty} e^{-( \sqrt{\pi} t + i \sqrt{\pi} f ) ^2 } e^{i^2 \pi f^2} dt.
$$
With $u = \sqrt{ \pi} t + i \sqrt{\pi} f $ and $du = \sqrt{\pi} dt$ we get
\begin{equation}
G(f) = \frac{1}{\sqrt{\pi}} e ^{- \pi f^2} \int ^{+ \infty}_{-\infty} e^{-u^2} du = e ^{- \pi f^2} = g (f),
\label{eq:gaussfilt1d}
\end{equation}
since the definite integral equals $\sqrt{\pi}$.  Filters based on the Gaussian function are some of the most used filters in data analysis and
data processing. Because of their smooth transform properties we know that they will
minimize ringing in the other domain.
\PSfig[h]{Fig1_gfilt_time}{(a) Raw topography profile shows considerable short-wavelength noise.
(b) Topography after application of a 80-km full-width Gaussian filter.  Note we lose 40 km of data
at either end, corresponding to the filter's half-width.}

\subsubsection{Butterworth filter}
\index{Butterworth filter}
\index{Filter!Butterworth}
\PSfig[h]{Fig1_BWfilter}{Various Butterworth filters for different orders.  The higher
the order (a), the more ringing will be apparent in the time-domain (b).}
	Another well-known set of frequency-domain filters has the functional form
\begin{equation}
%		(22.3)
G(f) = \sqrt{\frac{1}{1 + (f/f_0) ^{2n}} }.
\end{equation}
These are the so-called \emph{Butterworth} filters and are often used for low-pass filtering purposes.
The frequency $f_0$ defines the halfway point of the filter: The power of $G(f_0)$ is always 1/2,
regardless of the exponent $(2n)$. The value of $n$ determines how fast the filter falls off.
The higher the value of $n$, the faster the filter will drop off beyond $f_0$.
We see the sharper the drop off, the more ringing in the time domain, again reflecting Gibbs'
phenomenon caused by truncating the spectrum too rapidly. As always, there will have
to be a trade-off between how sharply you want to reduce the spectrum and how much
ringing you are willing to tolerate.\\

\subsubsection{Wiener filter}
\index{Wiener filter}
\index{Filter!Wiener}
\label{sec:Wiener}
Most observed time series can be thought of as a sum of two components:

\begin{enumerate}
\item The signal, $u(t)$, that we want to analyze,
\item The noise, $n(t)$.
\end{enumerate}
The measured signal is therefore corrupted by the noise. We will call this the \emph{observed signal},
$c(t)$. In addition, the measuring process may not be able to record all frequencies present in the phenomenon being observed.
Hence, the true signal $u(t)$ is ``blurred'' or smeared, which we can describe as a convolution
of $u(t)$ with the known instrument response, $r(t)$. The smeared signal is then
\begin{equation}
s(t)=r(t)*u(t)=\int^\infty_{-\infty} r(t-\tau)u(\tau)d \tau \quad \leftrightarrow \quad S(f)= R(f) \cdot U(f).
		%(16.1)
\end{equation}
Add in the noise component, $n(t)$, and we have
\begin{equation}
c(t)=s(t)+n(t)  \quad \leftrightarrow  \quad C(f)=S(f)+N(f).
\label{eq:DA2_16.2}
		%(16.2)
\end{equation}
In the absence of noise we can easily invert (or deconvolve) this equation by solving $U (f) =
S(f)/R(f)$. However, with noise a different method must be applied. We want to find
the optimal filter, $\phi(t)$ [or $\Phi(f)$] which, when applied to the measured signal $c(t)$ [or
$C(f)$], and then deconvolved by $r(t)$ [or $R(f)$], produces a signal $\hat{u}(t)$ [or $\hat{U}(f)$] that is as
close as possible to the uncorrupted signal $u(t)$ [or $U (f)$]. In other words, we will
estimate the true signal by
\begin{equation}
\hat{U}(f) = \frac{C(f) \cdot \Phi (f)}{R (f)}.
\label{eq:DA2_16.3}
%(16.3)
\end{equation}
What do we mean by being ``close'' to the true signal? We mean in a least-square sense, i.e.,
\begin{equation}
%		(16.4)
\min \int^\infty _{- \infty} \left [ \hat{u} (t) - u(t) \right ]^2 dt \leftrightarrow \min \int^\infty _{- \infty}
\left [ \hat{U} (f) - U (f) \right ]^2 d f,
\label{eq:percivalth}
\end{equation}
where we have used Parseval's theorem to express the misfit in the frequency domain.
With the transform of the noise given by $N(f)$, we substitute (\ref{eq:DA2_16.2}) and (\ref{eq:DA2_16.3})
into (\ref{eq:percivalth}) to find
$$
\min \int^\infty _{- \infty} \left \{
\frac{\left [ S(f) + N (f) \right ] \cdot \Phi (f)}{R (f) }
- \frac{S (f)}{R(f)} \right \} ^2 d f.
$$
Carrying out the square we obtain
$$
\min \int^\infty _{- \infty}
\frac{1}{ R^2 (f)}
\left \{ [ S^2 (f) + 2S (f) N (f) + N^2 (f) ] \Phi ^2 (f) - 2S (f) [ S(f) + N (f)] \Phi (f) + S^2 (f) \right \} d f,
$$
which simplifies to
\begin{equation}
%		(16.5)
\min \int^\infty _{- \infty}
\frac{1}{ R^2 (f)} \left \{ S^2 (f) [ 1 - \Phi (f)]^2 + N^2 (f) \Phi (f) \right \} d f.
\label{eq:DA2_16.5}
\end{equation}
This simplification is valid because we assume $S$ and $N$ are uncorrelated, hence their product integrated over all
frequencies must equal zero.

Obviously, (\ref{eq:DA2_16.5}) is only minimized if the integrand is minimized for all frequencies. Thus, we
can determine the best choice for $\Phi (f)$ by taking the derivative of the integrand with
respect to $\Phi$ and setting it to zero:
$$
-2S ^2 (f) \left [ 1 - \Phi (f) \right ] + 2 N^2 (f) \Phi (f) = 0.
$$
Since this identity must hold for all frequencies we solve for the optimal filter as
\index{Wiener, N.}
\begin{equation}
\Phi (f) = \frac{S^2 (f)} {S^2 (f) + N^2 (f)} = \frac{S^2(f) } {C^2(f)}.
%(16.6)
\label{eq:DA2_16.6}
\end{equation}
\PSfig[h]{Fig1_Wienerfilter}{An optimal Wiener filter depends on the practitioner's ability
to separate the spectrum ($C^2$) into the contributions from the signal ($S^2$) and the noise ($N^2$).}
This is the optimal filter known as the \emph{Wiener} filter, named after \emph{Norbert Wiener}. Note that it only involves the power of
$S$, the observed smeared signal, and the power of $N$, the noise. In particular, equation (\ref{eq:DA2_16.6}) does not contain $U$, the true
signal. This simplifies life: We can determine $\Phi(f)$ independently of $R(f)$. However, we
need to isolate $S^2$ and $N^2$ from $C^2$. There is no unique way to do that unless we have some
extra information. Luckily, a way out is often presented by looking at the spectrum $C^2$.
Often, it will be clear what shape the signal and noise spectra must have. Consider the
situation sketched in Figure~\ref{fig:Fig1_Wienerfilter}.
It appears that the noise spectrum is slightly tilted. We simply extrapolate this trend for all
frequencies and subtract it from $C^2$ to get $S^2$. We can now numerically form the filter $\Phi(f)$ from $S^2$ and $N^2$. As
you can see, the filter will be unity where the noise is minimal and drop smoothly to zero
where the noise dominates. Simple, but very powerful, and not very sensitive to errors in
isolating $S^2$ and $N^2$. In fact, a crude separation by eye based on a power spectrum is
usually adequate.


\subsection{Time-domain filters}
Many useful filters cannot be specified in the form of a convolution and thus cannot be made
extremely efficient via the magic of the convolution theorem and the availability of FFTs.  These time-domain (or space-domain)
filters must be executed on the data directly and may take considerable computational power.
\subsubsection{Median filter}
\index{Median filter|(}
\index{Filter!median}
\PSfig[h]{Fig1_MA_filterspike}{Convolution filters have a hard time when data sets contain outliers.}
	We have assumed for most of the time that all filtering can be described by a
convolution, and in general this is the case. The advantage of being able to describe
filtering as a convolution is very important: Thanks to the convolution theorem we can
simply transform both data and filter into the frequency domain, perform a multiplication
and take the inverse transform. This approach can be vastly faster than computing the convolution
in the time domain.  Why would we give up that advantage? Consider again a discrete
MA filter of a finite width, say
$$
g = \left \{ \frac{1}{5} \frac{1}{5} \frac{1}{5} \frac{1}{5} \frac{1}{5} \right \}.
$$
When convolved with data, it simply returns the mean value of the points inside the filter
width. However, we know the mean is a least squares estimate of ``average'' value. What
happens if we have the occasional bad data point? Because the filter uses an $L_2$ norm it returns bad values for
the entire filter width (see Figure~\ref{fig:Fig1_MA_filterspike}).
Clearly, this is not a desirable result. However, we can design a more \emph{robust} filter by
using a more robust estimate of ``average'' value. A good estimator that is insensitive
to the occasional outlier is the \emph{median}\index{Median}, defined as
$$
\tilde{x} = \left\{ \begin{array}{cl}
x_{\frac{n+1}{2}}, & n \ \mbox{is odd}\\
\frac{1}{2} \left ( x_{\frac{n}{2}} + x_{\frac{n+1}{2}} \right), & n \ \mbox{is even}
\end{array} \right.,
$$
where it is assumed the $x_i$ have been sorted.
\PSfig[h]{Fig1_median_filterspike}{Median filters excel in eliminating single or narrow sets of outliers.}
Unfortunately, the median is not an analytic function and has no transform (ponder what
its ``impulse response'' would be!). Hence, we are forced to calculate the result of median
filtering in the time domain only, as convolution is not applicable. An $n$-point median
filter will, for each lag, return the median value of the current $n$ points. Thus, the filtering
of the previous data example would result in the removal of the spike (Figure~\ref{fig:Fig1_median_filterspike}).
In addition to being robust, i.e., insensitive to outliers, a median filter also preserves step-functions,
provided the step length exceeds half the filter width.
\PSfig[h]{Fig1_filter_steps1d}{Convolution filters (top) will blur sudden steps while median filters (bottom)
will pass the steps unchanged.}
For data that consist of a noisy signal superimposed on a step-like background, median
filters are very useful  (see Figure~\ref{fig:Fig1_filter_steps1d}). Consider the task of finding the ``regional'' depth from
bathymetry data. The regional depth is ``contaminated'' by seamounts and faults and
may also abruptly increase or decrease across fracture zones (which are approximately step functions due to jumps in
crustal age).
	Finding the median of all points inside the filter width at each output point seems like
an expensive operation since the data must be sorted for each output location. Fortunately, a simple
iterative scheme may be used to speed up operations. Earlier, we found that the median satisfies the equality
\begin{equation}
\sum^n _{i=1} \frac{x_i - \tilde{x}}{|x_i - \tilde{x}|} = 0,
\label{eq:median_equation}
\end{equation}
since this ratio can only take on the values $-1, 0, +1$. With $\tilde{x}$ being defined as having as many values
above as below it (and any number of values may be equal to it) the sum must necessarily equal zero.
We may rewrite (\ref{eq:median_equation}) as
$$
\sum^n _{i=1} \frac{x_i}{|x_i - \tilde{x}|} -
\sum^n _{i=1} \frac{\tilde{x}} {|x_i - \tilde{x}|} = 0,
$$
which can be used to determine the median via successive iterations, i.e.,
\begin{equation}
\tilde{x}_k = \frac{\displaystyle \sum^n _{i=1} \frac{x_i}{|x_i - \tilde{x}_{k-1}|} } {\displaystyle \sum^n _{i=1} \frac{1}{|x_i - \tilde{x}_{k-1}|} },
\label{eq:median_iteration}
\end{equation}
where we obtain the $k$'th estimate of the median based on the previous value $\tilde{x}_{k-1}$.  Initializing $\tilde{x}_0$ to the mid-point between the minimum and maximum
data value, this scheme converges fairly quickly (e.g., Figure~\ref{fig:Fig1_medsearch}).  This is especially true for filtering since we would expect the median
output from the previous filter position to be similar to the median at the next output location.  In fact, the previous filter output is usually a good starting point
for the next output position.
\PSfig[h]{Fig1_medsearch}{We iteratively apply (\ref{eq:median_iteration}), and after seven iterations find the median ($\tilde{x} = -4956$) for a bathymetric data set with 11,313 points.}
\index{Median filter|)}

\subsubsection{Mode filter}
\index{Mode filter}
\index{Filter!mode}
\index{Least median of squares (LMS)}
\index{LMS (Least median of squares)}
	Finally, one can design a filter based on the \emph{mode}\index{Mode} rather than the median. Such filters
are called \emph{maximum likelihood} filters and return the most frequently occurring value
within the filter width. These filters are used to track representative levels through very noisy
data. We can implement such a filter using the \emph{Least Median of Squares} (LMS) approximation to the mode, as
discussed in great detail in Chapter~\ref{ch:regression}, or one can use a histogram (binning) approach and select the center
of the bin with most points.  Mode filtering has been used extensively in bathymetric studies to recover regional depths in the
presence of seamounts and plateaus.

\index{Least median of squares (LMS)}
\index{LMS (Least median of squares)}

\subsubsection{High-pass, low-pass, and band-pass filters}

As their names imply, these are filters that seek to retain a limited part of the frequency
content in the signal. Sometimes high-pass is called low-cut and low-pass is called high-cut,
for obvious reasons. In the frequency domain, low-pass and high-pass filters are illustrated
in Figure~\ref{fig:Fig1_low_high_filter}.
\PSfig[h]{Fig1_low_high_filter}{(left) A typical low-pass filter will
leave the low-frequency components unchanged and attenuate high-frequency components. (right) A high-pass filter
is complimentary and reverses what is being attenuated.}
\index{Filter!band-pass}
\index{Filter!low-pass}
\index{Filter!high-pass}
It is straightforward to design a high-pass filter that is complementary to the low-pass filter using
\begin{equation}
H (f) = 1 - L (f).
\label{eq:highpassf}
\end{equation}
In the time-domain, we find the corresponding filter as the inverse transform of $H(f)$:
\begin{equation}
H (f) = 1 - L (f) \leftrightarrow h(t) = \delta (t) - l (t).
\label{eq:highpasst}
\end{equation}
Thus, high-pass filtering the data $d(t)$ in the time domain is described by
$$
y(t) = d(t) * h (t) = d(t)* [\delta (t) - l(t)]
= d(t)* \delta (t)- d(t)* l (t) = d(t) - d(t) * l(t).
$$
This operation simply says that we high-pass-filter a data set by first using a (complementary) low-pass
filter and then subtract the output from the original signal.
Band-pass filters are simply a linear combination of high-pass and low-pass filters. You can
either design $B(f)$ directly or multiply data by $H(f)$ then by $L(f)$ (or two convolutions in
the time domain) or let $B(f) = L(f)\cdot H(f)$.
Note that (\ref{eq:highpasst}) is not restricted to convolution filters only.  Any filter, such as
our spatial median and mode filters, that can
be applied to data may be used to compute the complementary high-pass filter simply by subtracting
the low-passed output from the original observations.
\PSfig[h]{Fig1_band_filter}{A band-pass filter is simply a combination of a low- and high-pass filter
with different cutoff frequencies, which results in a certain band of frequencies being relatively unaffected.}

\index{Filtering!Gaussian}
\index{Filtering|)}

\clearpage
\section{Problems for Chapter \thechapter}

\begin{problem}
	The monthly flow of water in Cave Creek, Kentucky are given in the file \emph{cavecreek.txt},
	with the line number indicating months beginning with October, 1952.  Plot the data and compute the
	power spectrum.  What is the Nyquist frequency? Determine the most significant peak.  What period does it correspond to? Is the peak significant
	at the 95\% level of confidence?
\end{problem}

\begin{problem}
	The daily average temperatures in Billings, Montana (in \DS F) are given in file \emph{Billings.txt},
	with the first record corresponding to January 1, 1995.  Plot the data and compute the
	power spectrum.  What is the Nyquist frequency for this data set? Find the most significant peak in the spectrum.
	What period does it reflect, and is the peak significant at the 95\% level of confidence?
\end{problem}

\begin{problem}
	The numbers of sunspots recorded per month for the period 1749--2015 are given in \emph{sspots.txt}.
	Plot the data and compute the power spectrum.  What is the Nyquist frequency for this data set? Find the most significant peak in the spectrum.
	What period does it reflect, and is the peak significant at the 95\% level of confidence?
\end{problem}

\begin{problem}
	The daily discharge of the Columbia River (in m$^3$/day) are given in the file \emph{columbia.txt}, covering
	most of the interval 2001--2008. Plot the data and compute the
	power spectrum.  What is the Nyquist frequency? Determine the most significant peak.  What period does it correspond to? Is the peak significant
	at the 95\% level of confidence?
\end{problem}
	
\begin{problem}
	Revisit problem~\ref{ch:regression}.\theEDA\ and 
	remove the linear trends from the data.  Use \texttt{dft.m} (type \texttt{help dft}) to get the amplitudes $a_j$ and $b_j$ for 
	each Fourier series.  Make a plot of the raw power spectrum for each time-series using periods rather 
	than frequency on the $x$-axis (skip the zero frequency which represents the infinite period).  What is the 
	Nyquist period?
	What are the dominant periods in the data?  How well do the two series agree on this issue (a 
	qualitative answer is fine).
\end{problem}

\begin{problem}
	Scientists from Scripps Institution of Oceanography have been measuring the CO$_2$ concentration on top of Mauna Loa since 1958 (\url{http://scrippsco2.ucsd.edu}). The
	table \emph{CO2.txt} shows the record up to 2015 with decimal year versus CO$_2$ in ppm.  The second column
	contains the raw data, while the third column contains a few interpolated values when raw data were missing (this
	makes the data continuous and suited to Fourier analysis).
	\begin{enumerate}[label=\alph*)]
		\item Plot the data and determine the least-squares quadratic trend.
		\item Remove this trend from the data and subject the residuals to spectral analysis.
		Determine the two strongest spectral peaks in the signal.  What periods do these represent?
	\end{enumerate}
\end{problem}

\begin{problem}
	Revisit the noisy single-period data set \emph{noisy.txt} discussed in Problem~\thenoisyc.\thenoisyp.
	Remove the least-squares regression line to detrend the data
	and compute the power spectrum of the residuals.  Do you recover the single period you
	found earlier? Why might you get a different result in this analysis?
\end{problem}

\begin{problem}
	We will revisit problem~\ref{ch:sequences}.\theVostok\ where we discussed
	the 3-km long Vostok ice core from Antarctica that resolves temperature variations relative to the present via
	oxygen isotopes. These data are given in table \emph{vostok.txt}, which contains equidistant depths (in meter), the
	corresponding times (in year) and the relative change in temperature (in $^{\circ}$C).
	Because of compaction, depth is not a good proxy for time, especially for the deeper (older) sections.
	To analyze the temporal periodicities we thus need an equidistant time-series. Use MATLAB's \texttt{spline}
	or other software to interpolate the data onto an equidistant interval in time ($\Delta t = 25$).  
	What is the Nyquist frequency for this data set? Determine the most significant peak.  What period does it correspond to?
	Is the peak significant at the 95\% level of confidence?
\end{problem}

%	DA1_Chap9.tex
%
\chapter{ANALYSIS OF DIRECTIONAL DATA}
\label{ch:directional}
\epigraph{``Essentially, all models are wrong, but some are useful.''}{\textit{George E. P. Box, Statistician}}

	Much environmental data contain information about orientations in the plane where the \emph{orientations} 
and not the lengths of the features are the important attributes.  Examples of such data types abound: Strikes and dips 
of bedding planes, fault surfaces, and joints are familiar from structural geology, augmented by glacial striations, sole marks, 
lineaments in satellite images, directions of winds and currents, and much more.  First, we must distinguish between 
\emph{directional} and \emph{oriented} data.  Directional data can take on unique values in the entire 0--360$^{\circ}$ 
range, like drumlins and wind directions, while oriented data consist of ``two-headed'' vectors: there is a 180$^{\circ}$ 
 ambiguity 
inherent in the data.  Examples of oriented data include fault traces and other lineaments on maps.  
Such data require special care in the analysis.
	However, environmental data involve not only directions in the plane but spatial directions as well, which  
introduces another degree of complexity.  Because 3-D vector data are very common, particularly in 
the Earth and environmental sciences, we need to examine such data in more detail.  We find that much of the 
measurements used in structural geology, such as strike and dip of fault planes, can be expressed
as a normal vector to the fault plane.  Other examples include vector measurements of the 
geomagnetic field, palaeomagnetic measurements, stress directions, wind and current directions, and determinations of 
crystallographic axes for petrofabric studies.

\index{Directional data}
\index{Data!directional}
\index{Oriented data}
\index{Data!oriented}

\section{Circular Data}
\subsection{Displaying directional distributions}
	Directional data can be displayed in circular diagrams.  One can either plot each direction as 
a unit vector, or we may count the number of vectors within a given angular sector and draw a polar histogram of 
the distribution (Figure~\ref{fig:Fig1_dir1}):
\index{Polar histogram}
\index{Histogram!polar}
\PSfig[H]{Fig1_dir1}{Two types of graphical presentations of the same directional data set.  (a) Windrose diagram shows
all individual directions, (b) Sector diagram shows distribution via a polar histogram (a so-called \emph{rose} diagram).}
\noindent
Unfortunately, the sector diagram as described here is biased in the way it presents the data. 
Nevertheless, it is still the most commonly used type of display for directional data.  Where is this bias
coming from? Consider the area of a sector of width $\Delta \alpha$, given by
\begin{equation}
A=\frac{\pi r^{2}\Delta\alpha}{360} \propto r^2.
\end{equation}
We see that the area of a sector is proportional to the radius squared, whereas for a conventional Cartesian histogram the 
column area is proportional to height, not height squared.  Consequently, this leads to a visual distortion of
sectors with high counts.  Therefore, we should let the sector 
radius be proportional to the square root of the frequency (or count) to ensure that the final rose diagram 
will have an area proportional to frequency.  If this is not done, the larger counts for some sectors will completely 
swamp smaller sectors due to the $r^2$ effect.  Thus, small but significant trends may not be 
detected or may simply be considered noise.

	Before we can statistically analyze orientation data, the 180$^{\circ}$ ambiguity must be accounted 
for.  The simplest way to do this is to double all angles and analyze these doubled angles instead.  For instance, if 
two fault traces are reported as having strikes of 45$^{\circ}$ and 225$^{\circ}$ (which means the two strikes have the 
same orientation), we double the angles and find $90^{\circ}$ and $450^{\circ} - 360^{\circ} = 90^{\circ}$.
Statistics derived from doubled angles can then be divided by two to recover their original meanings.

\index{Mean direction}	
\index{Direction!mean}	
\index{Resultant}	
The dominant or \emph{mean direction} can be found by computing the \emph{vector resultant} 
(i.e., the vector sum) of the unit vectors that represent the various directions in the data.  Since the 
coordinates of these vectors are given by
\begin{equation} 
x_{i}=\cos \theta_{i},\quad y_{i}=\sin \theta_{i},
\label{eq:xy_components}
\end{equation}	 	
we find the coordinates of the resultant $\mathbf{R} = (x_r, y_r)$ to be
\index{Vector!resultant}
\index{Vector!resultant length}
\begin{equation} 
x_{r}=\displaystyle \sum^{n}_{i=1} \cos \theta_{i}, \quad y_{r}=\displaystyle \sum^{n}_{i=1} \sin \theta_{i}.
\end{equation}	 	
The mean direction $\bar{\theta}$ is then simply given by
\begin{equation}
	\index{Mean direction}
	\index{Direction!mean}
\bar{\theta}= \tan^{-1}(y_{r}/x_{r})=\tan^{-1} \left(\sum^{n}_{i=1} \sin \theta_{i}/\sum^{n}_{i=1}\cos \theta_{i} \right).
\label{eq:meandir}
\end{equation}
Computer implementations of (\ref{eq:meandir}) should take care to use the \texttt{atan2} function rather than \texttt{atan}
so that the mean direction is found in the correct quadrant.
The magnitude $R$ of the resultant should be normalized by the number of vectors before it can be 
used further.  This gives us the \emph{mean resultant length}
\begin{equation}
	\index{Mean resultant length}
	\index{Vector!mean resultant length}
\bar{R} = \frac{R}{n} = \frac{\sqrt{x^2_r + y^2_r}}{n},
\end{equation}
which ranges from 0 to 1 and is a measure of (normalized) dispersion analogous 
to the variance.  Since it increases for focused distribution we often convert it to a
\emph{circular variance} via
\begin{equation}
	\index{Variance!circular}
s^{2}_{c}=1- \bar{R}= \left( n-R \right) / n.
\end{equation}
\PSfig[H]{Fig1_vonMises}{The circular von Mises distribution.  One population is centered on $\mu = 75$ (dashed line), with very low directionality
($\kappa = 1$), and shows the effect of wrapping the wide distribution around the full circle.  The other, at $\mu = -75$, is much more directional ($\kappa = 10)$.)}
To test various statistical hypotheses about circularly distributed data we need a 
probability density function that can be used to obtain critical values.  A traditional distribution that has 
been used extensively in studies of directional data is the \emph{von Mises} distribution, given by
\begin{equation}
	\index{Probability distribution!von Mises}
	\index{von Mises distribution}
p(\theta) = \frac{1}{2 \pi I_0 ( \kappa)} e^{\kappa \cos\left( \theta - \mu \right)},
\label{eg:von_Mises}
\end{equation}
where $\kappa$ is a measure of the \emph{concentration} of the distribution about the mean direction $\mu$, and $I_0$ is 
the modified Bessel function of the first kind and order zero (it normalizes the 
cumulative distribution of $p(\theta)$ over $360^{\circ}$ to unity.)  A large $\kappa$ means we have a strongly preferred direction.
The concentration parameter $\kappa$ is obviously related to
$R$ (or $s_c^2$) and this relationship can be 
derived if we assume that the data represent a random sample from a population having a von Mises 
distribution.  Statistical tables (e.g., Table~\ref{tbl:Critical_kappa2}) report solutions to (\ref{eq:kappa2d}) that relate $\kappa$ and $\bar{R}$.

\subsection{Test for a random direction}
\index{Test!Rayleigh}
\index{Test!random direction}

	The simplest test for circular data is to determine whether the directional observations are 
random or not, i.e., that there is no \emph{preferred direction}.  In that case we have a \emph{uniform} distribution.
In terms of the von Mises distribution (\ref{eg:von_Mises}), a uniform distribution means that $ \kappa$ must be zero.
Hence, the hypothesis test becomes
\begin{equation}
H_0: \kappa =0 \quad \mbox{versus} \quad H_1: \kappa>0.
\end{equation}	 
If the data came from a uniform distribution (i.e., $ \kappa$ = 0), we would expect a small but not necessarily zero value for $\bar{R}$.  Lord Rayleigh devised this
test, which gives critical values for $\bar{R}$ depending on the  $\nu = n$ degrees of freedom and chosen level of significance, $\alpha$.
For $\nu \geq 10$ we can approximate the critical mean resultant via
\begin{equation}
	\bar{R}_{\alpha,n} = \frac{1}{2n}\sqrt{1+4n(n+1)-\left (\log \alpha + 2n + 1 \right)^2},
\end{equation}
while for smaller samples we need to consult exact tables (e.g., Table~\ref{tbl:Critical_R2}).
If $\bar{R}$ exceeds the critical $\bar{R}_{\nu, \alpha}$, then we must reject the null hypothesis.
\begin{example}
We have measured the strikes of fault planes in an area 
of the sea floor (called Area 1) imaged by a side-scan sonar device and found the following orientations:
$$
 \theta_i : 110^{\circ}, 300^{\circ}, 310^{\circ}, 135^{\circ}, 138^{\circ}, 320^{\circ}, 141^{\circ}, 145^{\circ}, 330^{\circ}, 335^{\circ}, 280^{\circ}, 160^{\circ}, 170^{\circ} \quad n=13.
$$
Since these are oriented features, we double the angles to remove any ambiguity in the orientations:
$$
 2\theta_i : 220^{\circ}, 240^{\circ}, 260^{\circ}, 270^{\circ}, 276^{\circ}, 280^{\circ}, 282^{\circ}, 290^{\circ}, 300^{\circ}, 310^{\circ}, 200^{\circ}, 320^{\circ}, 340^{\circ}.
$$
Summing up the sines and cosines via (\ref{eq:xy_components}), we find
\begin{equation}
x_r = 1.2972, \quad y_r = -10.349,		 
\end{equation}
which gives us
\begin{equation}
2\bar{\theta}_1 = 277^{\circ}, \quad \bar{R}_1 = 0.802.
\end{equation}  
We convert back to original angles by dividing by 2 and find the mean orientation to be
\begin{equation}
\bar{\theta}_1 = 138.5^{\circ}.
\end{equation}	 
The null hypothesis is not affected by doubling the angles since we are simply testing if $\kappa = 0$.
At the $\alpha = 0.05$ level of significance we use Table~\ref{tbl:Critical_R2} to find critical $\bar{R}_{13,0.05} = 0.475$.  Clearly, this value is greatly
exceeded by the observed $\bar{R} = 0.802$ and we must conclude that the fault strikes are not randomly oriented.
\end{example}

\subsection{Test for a specific direction}
\index{Test!specific direction}

	There are instances when we would like to test whether an observed trend equals a 
specified trend.  The determination of critical values for such a scenario is more difficult and involves 
using complex equations or charts.  As an alternative approach we may find the \emph{confidence angle} $\Delta \theta$ around
the mean direction of the sample. We can then investigate whether this angular interval is large enough to accommodate the specified trend
we want to test against.  The approximate standard error in $\bar{\theta}$ is given (in radians) by
\index{Confidence interval!mean direction}
\begin{equation}
s_e \approx 1/\sqrt{n \bar{R}k} = 1/\sqrt{Rk},
\end{equation}
where $k$ is our sample estimate of $\kappa$, the population parameter.  If we assume that the estimation errors are normally distributed
then we may use critical $z$-values for a given confidence level to construct the angular interval as
\begin{equation}
\bar{\theta} \pm \left | z_{\frac{\alpha}{2}} \right | s_e,
\end{equation}	 
which will contain the true mean (double angle) orientation, $2\mu, 100\cdot\alpha$ \% of the time.
\begin{example}
In our fault strike data case above,
it is believed that the faults arose as tensional features in response to tectonic forces operating in the N$30^{\circ}$E
direction.  Under such a stress regime we would theoretically expect the faults to trend at 90$^{\circ}$ to the direction of these tensile
stresses, i.e., in the N120$^{\circ}$E direction.  From Table~\ref{tbl:Critical_kappa2} we find $\kappa =2.897$, which yields
\begin{equation}
s_{e_2} = \frac{1}{ \sqrt{13 \cdot 0.802 \cdot  2.897}} = 0.1820 = 10.4^{\circ}.
\end{equation}
Note that this is the standard deviation about the \emph{doubled} mean angle orientation of $277^{\circ}$.  We divide this deviation by
2 to find the standard deviation associated with the original orientations, $s_{e_1} = 5.2^{\circ}$.  The 
$95 \%$ confidence interval corresponds to $z_{0.95} = 1.96$, so the confidence interval around $\bar{\theta}_1$ becomes	 
\begin{equation}
\bar{\theta}_1 = 138.5^{\circ} \pm 10.2^{\circ}.
\end{equation}
Since the predicted direction of $120^{\circ}$ is outside this interval we conclude that the observed
mean direction deviates from that predicted based on the orientation of current tectonic forces.  It is possible that the
orientation of stresses may have changed since the formation of the faults.
\end{example}

\subsection{Test for equality of two mean directions}
\index{Test!equality of two mean directions}
\label{sec:twoDdir}
	Another common situation where we may want to apply a statistical test is to determine if two 
sample mean directions are equal at some prescribed level of confidence.  This situation, of course, is
reminiscent of the two-sample mean test for scalars that we described in Section~\ref{sec:twomeans}. For example, we may 
have obtained new side-scan sonar data for an area farther away and would like to test whether the 
mean direction of fault strikes at the second location is similar to what was found at the first location.
We can carry out this 
test by comparing the vector resultants of the two groups to that produced by pooling the
two data sets and recalculate a single grand resultant.  The key principle here
is that if the mean directions are different, then the pooled resultant should be \emph{shorter} than
the sum of the two individual resultants.  This expectation can be examined using an \emph{F}-test by
comparing observed $F$ to critical $F_{1,n-2,\alpha}$.  Unfortunately,
due to approximations needed to relate Cartesian and circular critical values, the procedure to evaluate
the $F$-statistic changes with the dispersion parameter, $\kappa$, represented by our sample-based value, $k$.
For large $k > 10$, we may compute
\index{\emph{F}-test}
\index{Test!\emph{F}}
\begin{equation}
F = \frac{(n-2)(R_1 + R_2 - R_p)}{n - R_1 -R_2},
\end{equation}
with $n$ being the combined number of observations, $R_1$ and $R_2$ are the full resultants (not \emph{mean} resultants)
for each data set, and
$R_p$ is the resultant from the combined data. Here, $k$ is estimated from $\bar{R}_p$, the mean resultant.
However, if $2 < k < 10$ then a more accurate $F$-statistic is obtained via the adjustment
\begin{equation}
F^{\ast} = \left(1 + \frac{3}{8k} \right) F,
\label{eq:betterFk}
\end{equation}
while for even smaller values of $k$ special tables must be used.
\begin{example}
We will use the equality test to determine whether the mean
strike of the faults in our second area (Area 2) is the same as what was found for the first area
($\bar{\theta}_1 = 138.5^{\circ}$, double angle = $277^{\circ}$).  The new data are: 
$$
\theta_i = 91^{\circ}, 280^{\circ}, 111^{\circ}, 115^{\circ}, 118^{\circ}, 300^{\circ}, 122^{\circ}, 126^{\circ}, 130^{\circ}, 80^{\circ}, 320^{\circ}, 149^{\circ} \quad (n = 12),
$$
which we double to get
$$
2\theta_i = 182^{\circ}, 200^{\circ}, 222^{\circ}, 230^{\circ}, 236^{\circ}, 240^{\circ}, 240^{\circ}, 244^{\circ}, 252^{\circ}, 260^{\circ}, 160^{\circ}, 280^{\circ}, 298^{\circ}.
$$
Carrying out a similar analysis on our new data
we now find $2\bar{\theta}_2$ = 234.5$^{\circ}$ and $\bar{R}_2$ = 0.805. The hypothesis becomes 
$H_0:\bar{\theta}_1 = \bar{\theta}_2$ versus $H_1:\bar{\theta}_1 \neq \bar{\theta}_2$, at 
$\alpha = 0.05$.  For the combined data set (with $n = 25$), we obtain
\begin{equation}
2\bar{\theta}_p =256.7^{\circ}, \quad \bar{R}_p =0.749.
\end{equation}
\index{\emph{F}-test}
\index{Test!\emph{F}}
We find observed $k$ from Table~\ref{tbl:Critical_kappa2} to be 2.36.  Hence, we use (\ref{eq:betterFk}) to obtain
\begin{equation}
F = \left(1 + \frac{3}{8 \cdot 2.36} \right) \frac{23 \left(13\cdot0.802 + 12\cdot0.805 - 25\cdot0.749 \right)}{ 25-13\cdot0.802-12\cdot0.805}=7.42.
\end{equation}	 
Table~\ref{tbl:Critical_F95} gives the critical $F_{1,n-2,\alpha}$ value for 1 and 23
degrees of freedom as 4.28.  Therefore, we must 
reject the null hypothesis and conclude that the fracture directions in the two areas appear to have
different orientations at the $95 \%$ confidence level.
\end{example}

\subsection{Robust directions}
\index{Robust!directions}
\index{Direction!robust}

\PSfig[h]{Fig1_dir2}{The effect of outliers on the mean direction is most severe when the outlier is orthogonal to the bulk of the data directions.
Headed vectors represent the mean resultants.}
	The concept of robustness is also applicable to directional data since such data may also contain outliers.
However, directional outliers cause less harm than their Cartesian counterparts discussed earlier since directional data are forced
to be periodic.  It is clear from Figure~\ref{fig:Fig1_dir2} that an outlier causes most damage to estimates of mean direction when it is oriented
90$^{\circ}$ away from the trend of the bulk of the data.
We may find the circular analog of the median by finding the direction $\tilde{\theta}$ that minimizes the sum
\begin{equation}
\mbox{minimize} \  \sum^n_{i=1} d \left( \theta_i, \tilde{ \theta} \right),
\end{equation}
where $d\left( \theta_i, \tilde{\theta}\right)$ is the arc distance between $\theta_i$ and 
$ \tilde{ \theta} $ measured along the perimeter of the unit circle.  Since this distance is proportional
to $\mid \theta_i - \tilde{ \theta} \mid$   it is the equivalent of minimizing the sum
\begin{equation}
\mbox{minimize}  \ \sum^n_{i=1} \mid \theta_i - \tilde{ \theta} \mid,
\end{equation}
which we know gives the median of the $ \theta_i$ data set.  Similarly, a LMS mode estimate can be found by
determining the midpoint of the shortest arc containing $n/2+1$ points.  Again, this would involve sorting
the directions first, possibly after doubling orientations.
\begin{example}
Consider the wind directions
$$
\theta_i: 75^{\circ}, 85^{\circ}, 90^{\circ}, 98^{\circ}, 170^{\circ} \quad 
\bar{ \theta}=100.7^{\circ}.
$$	 
The shortest arc over $5/2+1=3$ points is located between 85$^{\circ}$ and 98$^{\circ}$, giving the mode
estimate $\hat{\theta} = 91.5^{\circ}$. This estimate suggests that 170$^{\circ}$ is an outlier
with respect to the rest of the atmospheric data.
\end{example}

\subsection{Data with length and direction}
\index{Digitizing}
\index{Uncertainty!digitizing}

	In the analysis so far we have only considered the direction (or orientation) of a feature and not its length.
However in many cases, such as fracture data or fault traces, the features will have very different lengths.  
The analysis described above, when applied to such data, would give both a one km long and a 100 km long fault
the same weight, which does not seem to be appropriate.  We 
can account for this bias by weighing the directions by the respective \emph{lengths} of the features.  By keeping 
track of the length of the faults (and not their numbers) per sector we can obtain a rose diagram that 
reflects the proportions of the various fracture directions.  The rationale employed here is that large and/or 
long faults may be more representative of the tectonic stresses than a few short fractures.  The 
rose diagram may then be normalized by the total length of the fractures to give overall proportions in 
percent.

	Staying with fault strike data, it is clear that in many regions the faults are not entirely straight 
lines but may actually curve or bend.  From a directional analysis point of view, such fractures must first be approximated by shorter 
straight line segments.  It then becomes obvious that we must weigh the pieces by their lengths, 
otherwise the directional frequencies would depend on the number of pieces used.  Fault traces 
must therefore be digitized prior to analysis on a computer.  A typical fault trace is 
displayed in Figure~\ref{fig:Fig1_dir3}.
\PSfig[h]{Fig1_dir3}{A digitized fault trace.  Open circles indicate the digitized points.  Digitizing too finely may lead to short-wavelength
noise in the representation of the fault.  Points within the dashed box are examined in Figure~\ref{fig:Fig1_dir4}.}
\noindent
Depending on the angular width of the polar histogram, the digitized fault may end up in two 
bins corresponding to the orientations $\alpha_1$ and $\alpha_2$.  We must also be aware of the fact that the digitizing process will introduce uncertainties in the 
digitized points.  To see how this may affect the analysis, consider the line segment in Figure~\ref{fig:Fig1_dir4}.
\PSfig[h]{Fig1_dir4}{The uncertainty $r_s$ in the locations of two digitized points (A and B) from Figure~\ref{fig:Fig1_dir3} introduce uncertainties
$\Delta d$ and $\Delta \alpha$ in the length ($d_0$) and orientation ($\alpha_0$) of a line segment, respectively.}
\noindent
We may assume that the exact position of each point is uncertain, here represented by the one-sigma uncertainty estimate $s_r$ in radial position
(gray circles) for each point.  There are two points of interest here:  First, the length of the segment, $d$,
will have an uncertainty since it reflects a difference between two uncertain values.  In Chapter~\ref{ch:error}, we found this to be
\begin{equation}
\Delta d = \sqrt{2 s_r}, \quad d= d_0 \pm \Delta d,
\label{eq:dig_error}
\end{equation}
assuming the uncertainties are \emph{independent}.
Second, we see that the direction (or orientation) $\alpha$ may be in error by $ \pm \Delta \alpha$, given by
\begin{equation}
\Delta \alpha = \tan^{-1} \left( 2 s_r/d \right), \quad \alpha = \alpha_0 \pm \Delta \alpha.
\end{equation}
Consequently, one should not digitize lines so frequently that $ \Delta \alpha$ exceeds the desired polar histogram interval.
For instance, if this interval is 10$^{\circ}$ then you are best served by making the average digitizing interval 
$d > 10 s_r$.  Alternatively, one can \emph{filter} the digitized track so that short-wavelength noise is smoothed out before binning the segments.  

	The length of all segments that have a direction within the width of a single bin is simply computed by adding
up all the individual lengths.  Note that since internal nodes are shared by adjacent line segments the uncertainty
in the total length is independent of the number of line segments (e.g., Figure~\ref{fig:Fig1_digline}) and only depends on (\ref{eq:dig_error}).  However,
the \emph{sum} of the lengths of the individual segments will have an uncertainty that is cumulative since the segments are no longer connected.
This method will provide a frequency distribution with error bars in which directions with many small segments will have higher 
uncertainty than directions with fewer and longer faults.

\section{Spherical Data Distributions}
\index{Spherical data distributions|(}
\index{Data!spherical|(}

The analysis of 3-D directions and orientations is an extension of the methods used for circular data.
It is common to require that these 3-D vectors have unit lengths so that their endpoints all lie
on the surface of a sphere with unit radius --- hence the name spherical distributions.
Similar to the 2-D case for circular data, there will be spherical data that only reflect orientations (i.e., \emph{axes}) rather than directions.	 
\PSfig[h]{Fig1_3D}{The relation between the Cartesian ($x, y, z$) and spherical ($\phi, \theta, r$) coordinate systems.}
\noindent
We need to use a 3-D Cartesian coordinate system to describe the unit vectors (Figure~\ref{fig:Fig1_3D}).  Thus, any 
vector $\mathbf{v}$ is uniquely determined by the triplet ($x,y,z$).  We could also use spherical angles
$\theta$ (colatitude) \index{Colatitude} and $\phi$ (longitude) to specify the vector direction, assuming the length $r=1$.
We relate the Cartesian coordinates and the spherical angles as follows:
\begin{equation}\begin{array}{rll}
x & = & \sin \theta \cos \phi,\\
y & = & \sin \theta \sin \phi,\\
z & = & \cos \theta.\\
\end{array}
\end{equation}	 	
However, geological measurements like \emph{strike}\index{Strike} and \emph{dip}\index{Dip} are more commonly used than Cartesian 
coordinates and spherical angles and these follow their own convention.  We define a new local 
coordinate system in which $x$ points toward north, $y$ points east, and $z$ points vertically down 
(i.e., in order to maintain a right-handed coordinate system).  In such a system, fault plane dips are 
expressed as positive angles.  For the fault plane in Figure~\ref{fig:Fig1_AD} we find that the angle $A$ is the 
azimuth of the strike of the plane and $D$ is the dip, measured positive down.  The slip-vector
\emph{OP} is then given by its components
\begin{equation} \begin{array}{lll}
x = - \sin A  \cos D,\\
y = \cos A \cos D,\\
z = \sin D.\\
\end{array}
\end{equation}
\PSfig[h]{Fig1_AD}{Local, right-handed coordinate system shows the convention used in structural geology.
Here, $A$ is the strike (measured from north over east) of the dipping plane (light green) and $D$ is its dip, measured as the angle between
the dip vector \emph{OP} and its projection \emph{ON} onto the horizontal $x-y$ plane.  The red plane containing
the dip vector is orthogonal to the light green plane.}

Once we have converted our ($A, D$) data to ($x,y,z$) we can compute such quantities as mean 
direction and spherical variance, which are simple extensions of the 2-D or directional analogs.  
The length of the resultant vector is simply
\begin{equation}
	\index{Vector!mean resultant}
	\index{Mean resultant vector}
R = \sqrt{\left( \displaystyle \sum x_i \right)^2 + \left( \displaystyle \sum y_i \right)^2 + 
\left( \sum z_i \right)^2},
\label{eq:Rlength}
\end{equation}
with the sum taken over all the $n$ points. The resultant vector
is usually normalized to give $\bar{R} = R / n$.  The coordinates $\bar{x}, \bar{y}$ and $\bar{z}$
of the mean vector $\mathbf{m}$ are then obtained via
\begin{equation}
\bar{x} = \displaystyle \sum x_i/n,\quad
\bar{y} = \displaystyle \sum y_i/n,\quad
\bar{z} = \displaystyle \sum z_i/n,
\end{equation}
so that the mean slip-vector is given by its two components
\begin{equation} \begin{array}{ccc}
\bar{D} = \sin^{-1} \bar{z},\\
\bar{A} = \tan^{-1} (- \bar{x}/ \bar{y}).
\end{array}
\end{equation}
If all the vectors are closely clustered then the resultant $R$ will approach $n$, but if the vectors
are more randomly distributed then $R$ will approach zero.  As in 2-D, we can use $R$ (or $\bar{R}$) as the basis
for the \emph{spherical variance}, $s^2_s$, give by
\begin{equation}
s^2_s = (n-R) / n = 1 - \bar{R}.
\end{equation}
\subsection{Test for a random direction}
\index{Test!Rayleigh}
\index{Test!random direction}
\PSfig[h]{Fig1_Fisher}{Well-focused Fisher distribution on a sphere, centered on a point in northern India, with $\kappa = 40$.}

We can perform simple hypothesis tests in a manner analogous to those we carried out for 2-D directional 
data.  Unlike the case for 2-D we now require a \emph{spherical} probability density function from which we can derive critical values for
comparison with our observed statistics.  In response to the need for 3-D statistical analysis, in particular
for palaeomagnetic studies that started to explore ``continental drift'' in the 1950s, the famous British
statistician Ronald Fisher\index{Fisher, R. A.} developed a suitable theoretical distribution
for spherical data.  His probability density function has since been called the \emph{Fisher} distribution on a sphere and is
given by
\index{Continental drift}
\index{Palaeomagnetism}
\begin{equation}
	\index{Probability distribution!Fisher}
	\index{Fisher distribution}
P(\mathbf{x}) = \frac{ \kappa}{ 4 \pi \sinh \kappa} e^{\kappa \left(\mathbf{x} \cdot \boldsymbol{\mu} \right)},
\label{eq:fisher}
\end{equation}	 	
where the dot product between the mean direction $\boldsymbol{\mu}$ and
any other direction $\mathbf{x}$ equals the cosine of the angle $\psi$ between them, $\kappa$ is the precision parameter similar to the one we encountered
for the von Mises distribution on the circle, and $\sinh$ is the \emph{hyperbolic sine} function (which is needed to ensure the cumulative distribution
of $P(\mathbf{x})$ over the unit sphere equals one).

Fisher\index{Fisher, R. A.} showed one can estimate $\kappa$ from the sample, provided $n > 7$ and $\kappa > 3$.
The estimate is then given by
\begin{equation}
k \approx \kappa = \frac{n-1}{n-R},
\end{equation}
but more accurate tables also exist (e.g., Table~\ref{tbl:Critical_kappa3}) which solve (\ref{eq:kappa3d}).

Testing a spherical distribution for randomness follows the same approach used for directional data:  
We must first evaluate $\bar{R}$, the mean resultant.  Then, we state the null hypothesis to be
\begin{equation}
H_0:\kappa = 0 \quad H_1:\kappa >0.
\end{equation}	 
As before, this test is executed by establishing critical values of $\bar{R}$ given the prescribed level of 
confidence, $\alpha$.  Table~\ref{tbl:Critical_R3} shows such critical $\bar{R}$ values for selected values of $\alpha$ and $n$.
\begin{example}
Let us examine a set of palaeomagnetic measurements.  We 
have been given six measurements of \emph{declination} and \emph{inclination}, reported as
\index{Declination}
\index{Inclination}
$$
\begin{array}{|l||r|r|r|r|r|r|}
\hline
\mbox{Dec} & 105^{\circ} & 130^{\circ} & 115^{\circ}
& 120^{\circ} & 118^{\circ} & 145^{\circ} \\ \hline
\mbox{Inc} & 40^{\circ} & 49^{\circ} & 57^{\circ} & 32^{\circ} & 55^{\circ} & 45^{\circ}\\ \hline
\end{array}
$$
First, we note that a dip-vector specified by strike $A$ and dip $D$ is not using the same geometry as
a magnetic field vector given by its magnetic declination $D$ and inclination $I$.
Because the projection of the dip-vector onto the horizontal plane is 90\DS\ away from the strike,
we must use a slightly modified conversion suitable for magnetic vectors to obtain the Cartesian coordinates:
\begin{equation} \begin{array}{lll}
x = \cos D \cos I,\\
y = \sin D \cos I,\\
z = \sin I.\\
\end{array}
\end{equation}
We convert the $n = 6$ observed declinations and inclinations to $x,y,z$ and find
$$
\begin{array}{|l||r|r|r|r|r|r||l|r|}
\hline
x & -0.20 & -0.42 & -0.23 & -0.42 & -0.27 & -0.58 & \bar{x} & -0.354 \\ \hline
y &  0.74 &  0.50 &  0.49 &  0.73 &  0.51 &  0.41 & \bar{y} &  0.564 \\ \hline
z &  0.64 &  0.75 &  0.84 &  0.53 &  0.82 &  0.71 & \bar{z} &  0.715 \\ \hline
\end{array}
$$
Computing the mean direction and resultant gives
\begin{equation}
\bar{I}_1 = \sin ^{-1} \bar{z} = 45.7^{\circ},
\end{equation}
\begin{equation}
\bar{D}_1 = \tan^{-1} \bar{y} / \bar{x} = 122.1^{\circ},
\end{equation}
\begin{equation}
\bar{R}_1 = \frac{1}{6} \cdot 5.86 = 0.977,
\end{equation}
\begin{equation}
k_1 = \frac{6-1}{6-5.86} = 35.7.
\end{equation}	 
It is evident that our distribution has a clear preferred direction since $k_1$ is so large (the critical
$\bar{R}$ for $\alpha = 0.05$ is only 0.642).
\end{example}

\subsection{Test for a specific direction}
\index{Test!specific direction}

Often, we will be interested in testing whether the observed mean direction equals a 
prescribed direction, given the uncertainties due to random errors.  As for circular data, such tests
are best performed by constructing the $\alpha$ confidence region around the mean direction.  This
statistic is based on the Fisher\index{Fisher, R. A.} distribution and gives a spherical cap radius for a \emph{cone of confidence}
around the mean direction.  As usual, this radius is a function of both the confidence level $\alpha$ and $R$.  
We find
\index{Cone of confidence}
\begin{equation}
\delta_{1- \alpha} = \cos^{-1} \left \{1- \frac{n-R}{R} \left[ \left (\frac{1}{ \alpha}\right )^{ \frac{1}{n-1}}-1 \right] \right \}.
\label{eq:exactconer}
\end{equation}
This fairly complicated expression simplifies considerably if we assume (or actually know) that $k > 7$ and
standardize our tests for $\alpha = 0.05$.  Then, 
\begin{equation}
\delta_{95\%} \approx 140^\circ /\sqrt{kn}
\label{eq:cone95}
\end{equation}
given in spherical degrees.  We may now say that there is a 95\% probability that the true mean direction lies within the cone of 
confidence specified by the angular radius $\delta_{95\%}$.

The application of the cone of confidence is more involved than
for circular data since we cannot directly compare the two angular measures defining the mean direction (e.g., strike, dip) to the
comparable quantities of a specific direction direction to be tested, here called $\mathbf{\hat{v}}$.
Instead, we need to utilize their Cartesian vector representations.  The procedure is still relatively straightforward:
\begin{enumerate}
	\item State the null hypothesis that the unit vectors are the same, i.e., $H_0: \mathbf{\hat{m}} = \mathbf{\hat{v}}$.
	\item Take the dot-product of the mean unit vector with the test unit vector, i.e, $c = \mathbf{\hat{m}} \cdot \mathbf{\hat{v}}$.
	\item Since a dot product gives the cosine of the spherical angle between two vectors we find this distance
	to be given by $\psi = \cos^{-1} c$.
	\item If $\psi$ exceeds $\delta_{95\%}$ then $\mathbf{\hat{v}}$ is outside the cone of confidence and we can reject
	the null hypothesis; otherwise we must reserve judgment.
\end{enumerate}
For other levels of confidence we would substitute the general radius specified in (\ref{eq:exactconer}).
\begin{example}
Given the data we just analyzed we want to test if the mean vector we found is compatible with an hypothesis
that says the inclination should be 45\DS and the declination should be 120\DS.  First, we find the
unit vector pointing in the same direction as the mean resultant, $\mathbf{m}$.  This vector is
\begin{equation}
	\mathbf{\hat{m}} = \frac{\mathbf{m}}{|\mathbf{m}|} = \frac{(-0.354, 0.564, 0.715)}{\sqrt{0.354^2 + 0.564^2 + 0.715}} = (-0.3621, 0.5770, 0.7321).
\end{equation}
We evaluate the test vector to be
\begin{equation}
	\mathbf{\hat{v}} = (\cos 120^{\circ} \cdot \cos 45^{\circ}, \sin 120^{\circ} \cdot \cos 45^{\circ}, \sin 45^{\circ}) = (-0.3536, 0.6124, 0.7071).
\end{equation}
Hence, the angle between the mean and the test vector is given by their dot product:
\begin{equation}
	\psi = \cos^{-1} \left (0.3621 \cdot 0.3536 + 0.5770 \cdot 0.6124 + 0.7321 \cdot 0.7071 \right ) = \cos^{-1} (0.9990) = 2.53^{\circ}.
\end{equation}
To determine the  radius of the 95\% confidence cone we use (\ref{eq:cone95}) and find
\begin{equation}
\delta_{95\%} \approx 140^{\circ} / \sqrt{ 35.7 \cdot 6 } = 9.6^{\circ}.
\end{equation}	 
This means that the true population mean direction probably (i.e., with 95\% level confidence) lies within a spherical cone of 
radius $9.6^{\circ}$ centered on the observed mean direction.  Clearly, our test vector $\mathbf{v}$ also lies inside this cone.
Therefore, we cannot reject the null hypothesis that they point in the same direction.
\end{example}

\subsection{Test for equality of two mean directions}
\index{Test!equality of two mean directions}
This test is equivalent to the one administered for circular data, relying on the same $F$-statistics and
equations; see Section~\ref{sec:twoDdir} for details.
\begin{example}
We also obtained palaeomagnetic measurements from a nearby second site:
$$
\begin{array}{|l||r|r|r|r|r|r|} \hline
\mbox{Dec} & 65^{\circ} & 72^{\circ} & 51^{\circ} & 75^{\circ} & 50^{\circ} & 45^{\circ} \\ \hline
\mbox{Inc} & 25^{\circ} & 20^{\circ} & 30^{\circ} & 23^{\circ} & 18^{\circ} & 33^{\circ} \\ \hline
\end{array}
$$
We perform the conversion to Cartesian components and find
$$
\begin{array}{|l||r|r|r|r|r|r||l|r|}
\hline
x & 0.38 & 0.29 & 0.55 & 0.24 & 0.61 & 0.59 & \bar{x} & 0.443 \\ \hline
y & 0.82 & 0.89 & 0.67 & 0.89 & 0.73 & 0.59 & \bar{y} & 0.766 \\ \hline
z & 0.42 & 0.34 & 0.50 & 0.39 & 0.31 & 0.54 & \bar{z} & 0.418 \\ \hline
\end{array}
$$
which gives a resultant $R = 5.88$ and the mean resultant (and mean declination, inclination) as
\begin{equation}
\bar{R}_2 = 0.979, \quad \bar{I}_2 = 24.7^{\circ},  \quad \bar{D}_2 = 59.9^{\circ},  \quad n = 6.
\end{equation}
We would like to compare these two mean directions to determine if they are statistically equivalent at the 95\% 
confidence level, i.e.,
\begin{equation}
H_0: \bar{I}_1 = \bar{I}_2, \quad \bar{D}_1 = \bar{D}_2.
\end{equation}
The hypothesis test for this equality is exactly the same test we used for 2-D directional data.  Again, we use 
an \emph{F}-test to determine if the resultant from the pooled data is significantly different from the linear 
sum of the two individual resultants, i.e.,
\index{\emph{F}-test}
\index{Test!\emph{F}}
\begin{equation}
F= \frac{(n - 2) (R_1 + R_2 - R_p)}{n - R_1 - R_2},
\end{equation}
where again $n = n_1 + n_2$.  We need to find the resultant for the combined data set of 12 points.  It is found via (\ref{eq:Rlength}) and
gives $R_p = 10.50$.
We may then normalize by $n$ to find $\bar{R}_p = 0.875$.  Our observed \emph{F}-statistic thus becomes
\begin{equation}
F = \frac{10(5.86 + 5.88 - 10.50)}{12 - 5.86 - 5.88} = 47.4.
\end{equation}	 
The observed \emph{F} value by far exceeds the critical $F_{0.05,1,10} = 4.96$ (i.e., Table~\ref{tbl:Critical_F95})
and we must reject the null hypothesis that the two directions are the same.
\end{example}
\index{Spherical data distributions|)}
\index{Data!spherical|)}

\clearpage
\section{Problems for Chapter \thechapter}

\begin{problem}
	Analyze the current directions in surface waters off Hog Neck, MA for a two-day period, as listed in \emph{HogNeck.txt}.
	How do the current directions vary during a 24-hour period?  Do the current speeds show a similar trend?
\end{problem}
	
\begin{problem}
	Paleocurrent measurements from ripples in sandstones contain information on past current directions.
	\begin{enumerate}[label=\alph*)]
	\item Use the data in \emph{ripple\_A.txt} from sandstone formation A to test if there is a preferred direction in the data.  A
	local sedimentologist bravely suggests that the ripples were formed by long-shore currents along a paleo-coastline
	trending 200\DS.  Do the data support this hypothesis?
	\item A second sandstone formation (B) was also examined for paleocurrents (\emph{ripple\_B.txt}).
	Do these data suggest there was any change in current direction between the formations
	of these two layers?
	\end{enumerate}
\end{problem}

\begin{problem}
A geologist surveys a large section of an exposed batholith and measures the strike 
direction of numerous vertical joints.  The file \emph{joints.txt} contains the azimuths (measured from north toward east) 
of these orientations.

\begin{enumerate}[label=\alph*)]
\item At the 95\% level of confidence, is there a preferred orientation in the data?  If so, what is the 
preferred orientation?

\item Regional tectonic considerations seem to favor a general extensional stress regime in the west-northwest
-- east-southeast orientation.  Is this explanation for the joints consistent with the data 
at the 95\% level of confidence?

\item 50 km further north another exposure of the batholith reveals additional joints (\emph{morejoints.txt}).  Are 
they randomly oriented?

\item Do these new joints deviate significantly from the preferred orientation you determined for the first 
site?
\end{enumerate}
\end{problem}

\begin{problem}
A structural geologist has measured the strikes and dips of faults in a Jurassic sedimentary 
section.  The observations (in degrees from north toward east) are recorded below.
Find the mean direction and the mean resultant length.
Is this direction significant at the 95\% level of confidence?

$$
\begin{array}{|c||c|c|c|c|c|c|c|c|c|c|c|c|} \hline
\mbox{\bf{Strike}} & 142^{\circ} & 157^{\circ} & 145^{\circ} & 150^{\circ} & 149^{\circ} & 156^{\circ} & -13^{\circ} & 139^{\circ} & 155^{\circ} & -9^{\circ} & 148^{\circ} & 142^{\circ} \\ \hline
\mbox{\bf{Dip}} & 45^{\circ} & 50^{\circ} & 52^{\circ} & 38^{\circ} & 56^{\circ} & 44^{\circ} & 8^{\circ} & 49^{\circ} & 46^{\circ} & 11^{\circ} & 59^{\circ} & 43^{\circ} \\ \hline
\end{array}
$$

\end{problem}

\begin{problem}
Principal fracturing directions for three oil fields near Odessa, TX are given in \emph{odessa\_north.txt},
\emph{odessa\_northwest.txt} and \emph{odessa\_west.txt}.  Compare the mean directions of fractures from
the three fields.  Are they significantly different at the 95\% level of confidence?
\end{problem}

\begin{problem}
	Measurements of magnetic remanence (given as declination and inclination) from a sandstone in Western
	Australia are listed in \emph{Tumblagooda.txt}.
	\begin{enumerate}[label=\alph*)]
	\item Compute the mean direction and the mean resultant length.
	\item Estimate the angular radius of the 95\% confidence cone.
	\item Is this direction significant at the 95\% level of confidence?
	\end{enumerate}
\end{problem}

\begin{problem}
Scientific drilling in the Caribbean Sea on-board the \emph{Glomar Challenger} resulted in a set of paleomagnetic data (reproduced in \emph{glomar.txt}).
\begin{enumerate}[label=\alph*)]
\item Determine if these data (declination clockwise from north, inclination downward from horizontal) have a
preferred direction at the 95\% level of confidence.
\item Compare this mean paleomagnetic vector to the present field
at this site (declination = 354.3\DS and inclination = 46.4\DS).
Do these two directions differ significantly at the 95\% level of confidence?
What does your analysis suggest for the plate tectonic motions for this region?
\end{enumerate}
\end{problem}


\appendix

\chapter{STATISTICAL TABLES}
\label{ch:tables}
\epigraph{``The main purpose of a significance test is to inhibit the natural enthusiasm of the investigator.''}{\textit{Frederick Mosteller, Statistician}}
\index{Critical values|(}
This appendix contains a set of standard statistical tables of critical values
for a variety of tests or distributions at selected levels of confidence.  When appropriate,
both one-sided and two-sided critical values are provided.  A two-sided test is used when
the null hypothesis is asserting an \emph{equality}, while a one-sided test is appropriate when the
null hypothesis is stating an \emph{inequality}.  Reject the null hypothesis
if the calculated statistic \emph{exceeds} the critical value (except for the $U$-test where rejection of $H_0$ should result when the calculated $U$ is \emph{less} than the critical value.)
\vspace{0.25in}
\begin{table}[H]
\centering
\begin{tabular}{|c|c|} \hline
\bf{Test or Distribution} &  \bf{Table} \\ \hline
\rule{0pt}{2ex}Normal distribution &  \ref{tbl:Critical_z} \\[2pt] \hline
\rule{0pt}{2ex}Student's $t$ &  \ref{tbl:Critical_t} \\[2pt] \hline
\rule{0pt}{2ex}$\chi^2_{\alpha,\nu}$ &  \ref{tbl:Critical_chi2} \\[2pt] \hline
\rule{0pt}{2ex}$F_{0.9,\nu_1,\nu_2}$ &  \ref{tbl:Critical_F90} \\[2pt] \hline
\rule{0pt}{2ex}$F_{0.95,\nu_1,\nu_2}$ &  \ref{tbl:Critical_F95} \\[2pt] \hline
\rule{0pt}{2ex}$F_{0.975,\nu_1,\nu_2}$ &  \ref{tbl:Critical_F975} \\[2pt] \hline
\rule{0pt}{2ex}$F_{0.99,\nu_1,\nu_2}$ &  \ref{tbl:Critical_F99} \\[2pt] \hline
\rule{0pt}{2ex}$U_{0.05,n_1,n_2}$ (one-sided) &  \ref{tbl:Critical_U1} \\[2pt] \hline
\rule{0pt}{2ex}$U_{0.01,n_1,n_2}$ (one-sided) &  \ref{tbl:Critical_U2} \\[2pt] \hline
\rule{0pt}{2ex}$U_{0.05,n_1,n_2}$ (two-sided) &  \ref{tbl:Critical_U3} \\[2pt] \hline
\rule{0pt}{2ex}$U_{0.01,n_1,n_2}$ (two-sided) &  \ref{tbl:Critical_U4} \\[2pt] \hline
\rule{0pt}{2ex}Kolmogorov-Smirnov (one sample, given $\mu, \sigma$) &  \ref{tbl:Critical_KS1} \\[2pt] \hline
\rule{0pt}{2ex}Kolmogorov-Smirnov (one sample, estimate $\mu, \sigma$) &  \ref{tbl:Critical_KS2} \\
(Normally called Lilliefors test for normality) &  \\[2pt] \hline
\rule{0pt}{2ex}Kolmogorov-Smirnov (two sample) &  \ref{tbl:Critical_KS3} \\[2pt] \hline
\rule{0pt}{2ex}Spearman's rank correlation &  \ref{tbl:Critical_Spearman} \\[2pt] \hline
\rule{0pt}{2ex}Determine $\kappa$ from $\bar{R}$ (two-dimensional) &  \ref{tbl:Critical_kappa2} \\[2pt] \hline
\rule{0pt}{2ex}Lord Rayleigh test for $\bar{R}$ (two-dimensional) &  \ref{tbl:Critical_R2} \\[2pt] \hline
\rule{0pt}{2ex}Determine $\kappa$ from $\bar{R}$ (three-dimensional) &  \ref{tbl:Critical_kappa3} \\[2pt] \hline
\rule{0pt}{2ex}Fisher test for $\bar{R}$ (three-dimensional) &  \ref{tbl:Critical_R3} \\[2pt] \hline
\end{tabular}
\end{table}

\clearpage
\section{Cumulative Probabilities for the Normal Distribution}
\index{Normal distribution}
\PSfig[H]{Fig1_App_Normal}{Given a chosen $z_P$-value, the probability $P$ (gray area under the curve from $-\infty$ to $z_P$)
can be read from this table. Here, $z_P$ is given in the format \emph{-a.bc}, where \emph{-a.b} and \emph{0.0c}
correspond to a unique row and column combination.}
\begin{table}[H]
\centering
\small
\begin{tabular}{|c|cccccccccc|} \hline
$z_P$ & \bf{0.09} & \bf{0.08} & \bf{0.07} & \bf{0.06} & \bf{0.05} & \bf{0.04} & \bf{0.03} & \bf{0.02} & \bf{0.01}  & \bf{0.00} \\ \hline
\input{CriticalTables/DA1_Table_Normal}
\end{tabular}
\normalsize
\caption{Normal cumulative distribution function.  For $z > 0$ use $P(z) = 1 - P(-z)$.}
\label{tbl:Critical_z}
\end{table}
For critical $z_\alpha$ or $z_{\alpha/2}$ values, see the last entry ($\nu = \infty$) in the table of critical values
for the Student-$t$ (Table \ref{tbl:Critical_t}).
\clearpage

\section{Critical Values for the Student's $t$ Distribution}
\index{Critical values!Student's $t$}
\index{Critical values!Normal distribution}
\PSfig[H]{Fig1_App_Student_t}{Given a chosen confidence level, $\alpha$ (area of the tail(s)), and sample size, $n$, then the degrees of
freedom, $\nu$, is $n -1$. Find the corresponding row and column entries and read the critical $t_{\alpha,\nu}$ value (one-sided test, e.g., for $H_0: t < 0$)
or $t_{\alpha/2,\nu}$ value (two-sided test for $H_0: t = 0$).  The sign of $t$ depends on which
tail you are considering.}
\begin{table}[H]
\centering
\begin{tabular}{|c|ccccc|} \hline
\bf{One tail, } $\mathbf{\alpha}$:  &  \bf{0.10}  & \bf{0.05} & \bf{0.025} & \bf{0.01}  & \bf{0.005} \\ \hline
\bf{Two tails, } $\mathbf{\alpha}$:  &  \bf{0.20}  & \bf{0.1} & \bf{0.05} & \bf{0.02}  & \bf{0.01} \\ \hline
$\nu$ &    &  &  &   &  \\ \hline
\input{CriticalTables/DA1_Table_Student_t}
\end{tabular}
\caption{Critical values for the Student's $t$ distribution.}
\label{tbl:Critical_t}
\end{table}

\clearpage
\section{Critical Values for the $\chi^2$ Distribution}
\index{Critical values!$\chi^2$}
\PSfig[H]{Fig1_App_Chisquare}{Given $\alpha$ and degrees of freedom, $\nu = n -1$, read critical $\chi^2_{1-\alpha,\nu}$ or $\chi^2_{\alpha,\nu}$ values.}
\begin{table}[H]
\centering
\begin{tabular}{|c|ccccc||ccccc|} \hline
\bf{Degrees of} &  \multicolumn{5}{|c||}{\bf{Left tail} ($1-\alpha$)}  & \multicolumn{5}{c|}{\bf{Right tail} ($\alpha$)} \\  \cline{2-11}
\bf{freedom, }$\nu$ &  \bf{0.995}  & \bf{0.99} & \bf{0.975} & \bf{0.95}  & \bf{0.90} &  \bf{0.10}  & \bf{0.05} & \bf{0.025} & \bf{0.01}  & \bf{0.005} \\ \hline
\input{CriticalTables/DA1_Table_Chisquare}
\end{tabular}
\caption{Critical values for the $\chi^2$ distribution.}
\label{tbl:Critical_chi2}
\end{table}

\clearpage
\section{Critical Values for the $F$ Distribution}
\label{sec:Ftables}
\index{Critical values!$F$|(}
\PSfig[H]{Fig1_App_F}{Given chosen confidence level $\alpha$ (area of the tail) and the degrees of
freedom ($\nu_1 = n_1 -1$, $\nu_2 = n_2 -1$), find the corresponding row and column entries and read the critical
$F_{\alpha,\nu_1, \nu_2}$ value (two-sided test, e.g., $H_0: F = 1$).}

\subsection{$F$ table for 90\% confidence level}.

\begin{table}[h]
\centering
\footnotesize
\begin{tabular}{|c|cccccccccccccc|} \hline
$\downarrow \nu_2 | \nu_1 \rightarrow$ & \bf{1} & \bf{2} & \bf{3} & \bf{4} & \bf{5} & \bf{6} & \bf{7} & \bf{8} & \bf{9} & \bf{10} & \bf{15} & \bf{20} & \bf{25} & \bf{50} \\ \hline
\input{CriticalTables/DA1_Table_F_0.1}
\end{tabular}
\normalsize
\caption{Critical values for the $F$ distribution, $\alpha = 0.1$.  Note: $\nu_1, \nu_2$ are the degrees of freedom for
the numerator and denominator, respectively.}
\label{tbl:Critical_F90}
\end{table}

\clearpage
\subsection{$F$ table for 95\% confidence level}.

\begin{table}[h]
\centering
\footnotesize
\begin{tabular}{|c|cccccccccccccc|} \hline
$\downarrow \nu_2 | \nu_1 \rightarrow$ & \bf{1} & \bf{2} & \bf{3} & \bf{4} & \bf{5} & \bf{6} & \bf{7} & \bf{8} & \bf{9} & \bf{10} & \bf{15} & \bf{20} & \bf{25} & \bf{50} \\ \hline
\input{CriticalTables/DA1_Table_F_0.05}
\end{tabular}
\normalsize
\caption{Critical values for the $F$ distribution, $\alpha = 0.05$.  Note: $\nu_1, \nu_2$ are the degrees of freedom for
the numerator and denominator, respectively.}
\label{tbl:Critical_F95}
\end{table}

\clearpage
\subsection{$F$ table for 97.5\% confidence level}.

\begin{table}[h]
\centering
\footnotesize
\begin{tabular}{|c|cccccccccccccc|} \hline
$\downarrow \nu_2 | \nu_1 \rightarrow$ & \bf{1} & \bf{2} & \bf{3} & \bf{4} & \bf{5} & \bf{6} & \bf{7} & \bf{8} & \bf{9} & \bf{10} & \bf{15} & \bf{20} & \bf{25} & \bf{50} \\ \hline
\input{CriticalTables/DA1_Table_F_0.025}
\end{tabular}
\normalsize
\caption{Critical values for the $F$ distribution, $\alpha = 0.025$.  Note: $\nu_1, \nu_2$ are the degrees of freedom for
the numerator and denominator, respectively.}
\label{tbl:Critical_F975}
\end{table}

\clearpage
\subsection{$F$ table for 99\% confidence level}.

\begin{table}[h]
\centering
\footnotesize
\begin{tabular}{|c|cccccccccccccc|} \hline
$\downarrow \nu_2 | \nu_1 \rightarrow$ & \bf{1} & \bf{2} & \bf{3} & \bf{4} & \bf{5} & \bf{6} & \bf{7} & \bf{8} & \bf{9} & \bf{10} & \bf{15} & \bf{20} & \bf{25} & \bf{50} \\ \hline
\input{CriticalTables/DA1_Table_F_0.01}
\end{tabular}
\normalsize
\caption{Critical values for the $F$ distribution, $\alpha = 0.01$.  Note: $\nu_1, \nu_2$ are the degrees of freedom for
the numerator and denominator, respectively.}
\label{tbl:Critical_F99}
\end{table}
\index{Critical values!$F$|)}

\clearpage
\section{Critical Values for the Mann-Whitney ($U$-Test)}
\index{Critical values!$U$|(}
These critical values are used when comparing two samples based on their
\emph{ranks} instead of their values.  These tables are symmetrical so it does not matter
which sample you label ``1'' and which one is sample ``2'', i.e., $U_{\alpha,n_1,n_2} \equiv U_{\alpha,n_2,n_1}$.
For sample sizes larger than 20 you may use the normal distribution approximation (\ref{eq:U_approx}).
\subsection{One-sided tests}
This section concerns itself with \emph{one-sided} comparisons,
i.e., $H_0: \mu_1 > \mu_2$.  Depending on your level of confidence, use either
Table~\ref{tbl:Critical_U1} (for $\alpha = 0.05$) or 
Table~\ref{tbl:Critical_U2} (for $\alpha = 0.01$).
\begin{table}[h]
\centering
\footnotesize
\begin{tabular}{|c|ccccccccccccccccccc|} \hline
\input{CriticalTables/DA1_Table_U1}
\end{tabular}
\normalsize
\caption{Critical values for the one-sided $U$-test for $\alpha = 0.05$.  Note the use of $n$ rather than $\nu$.}
\label{tbl:Critical_U1}
\end{table}

\begin{table}[h]
\centering
\footnotesize
\begin{tabular}{|c|ccccccccccccccccccc|} \hline
\input{CriticalTables/DA1_Table_U2}
\end{tabular}
\normalsize
\caption{Critical values for the one-sided $U$-test for $\alpha = 0.01$.  Note the use of $n$ rather than $\nu$.}
\label{tbl:Critical_U2}
\end{table}

\clearpage
\subsection{Two-sided tests}
This section concerns itself with \emph{two-sided} comparisons,
i.e., $H_0: \mu_1 = \mu_2$.  Depending on your level of confidence, use either
Table~\ref{tbl:Critical_U3} (for $\alpha = 0.05$) or 
Table~\ref{tbl:Critical_U4} (for $\alpha = 0.01$).
\begin{table}[h]
\centering
\footnotesize
\begin{tabular}{|c|ccccccccccccccccccc|} \hline
\input{CriticalTables/DA1_Table_U3}
\end{tabular}
\normalsize
\caption{Critical values for the two-sided $U$-test for $\alpha = 0.05$.  Note the use of $n$ rather than $\nu$.}
\label{tbl:Critical_U3}
\end{table}

\begin{table}[h]
\centering
\footnotesize
\begin{tabular}{|c|ccccccccccccccccccc|} \hline
\input{CriticalTables/DA1_Table_U4}
\end{tabular}
\normalsize
\caption{Critical values for the two-sided $U$-test for $\alpha = 0.01$.  Note the use of $n$ rather than $\nu$.}
\label{tbl:Critical_U4}
\end{table}
\index{Critical values!$U$|)}

\clearpage
\section{Critical Values for the Kolmogorov-Smirnov Distribution}
\index{Critical values!Kolmogorov-Smirnov|(}

\subsection{One-sample comparison with a known distribution}
Use table \ref{tbl:Critical_KS1} when comparing a single sample's cumulative distribution to
that of a \emph{known} distribution, i.e., its parameters ($\mu, \sigma$) are prescribed
by the null hypothesis.
\begin{table}[h]
\centering
\small
\begin{tabular}{|c|ccccc|} \hline
\bf{One tail, } $\mathbf{\alpha}$:  &  \bf{0.10}  & \bf{0.05} & \bf{0.025} & \bf{0.01}  & \bf{0.005} \\ \hline
\bf{Two tails, } $\mathbf{\alpha}$:  &  \bf{0.20}  & \bf{0.1} & \bf{0.05} & \bf{0.02}  & \bf{0.01} \\ \hline
\input{CriticalTables/DA1_Table_KS1}
$n > 40$	&	$\frac{1.07}{\sqrt{n}}$	&	$\frac{1.22}{\sqrt{n}}$	&	$\frac{1.36}{\sqrt{n}}$	&	$\frac{1.52}{\sqrt{n}}$	&	$\frac{1.63}{\sqrt{n}}$	\\[4pt] \hline
\end{tabular}
\normalsize
\caption{Critical values for the one-sample Kolmogorov-Smirnov test.  Note the use of $n$ rather than $\nu$.}
\label{tbl:Critical_KS1}
\end{table}

\clearpage
\subsection{One-sample comparison with an unknown distribution (Lilliefors)}
Use table \ref{tbl:Critical_KS2} when comparing a single sample's cumulative distribution to
that of an \emph{unknown} normal distribution. Equate the unknown distribution's
parameters ($\mu, \sigma$) to those from the sample ($\bar{x}, s$) in order to compute its cumulative distribution.
The parameter estimation reduces the degrees of freedom by two and the standard
Kolmogorov-Smirnov critical values are not appropriate.  Lilliefors developed a numerical solution that
gives the correct critical values. This is a two-sided test.

\begin{table}[h]
\centering
\begin{tabular}{|c|cccc|} \hline
$\alpha$: & \bf{0.10} & \bf{0.05} & \bf{0.01} & \bf{0.001} \\ \hline
\input{CriticalTables/DA1_Table_KS2}
$n > 100$	&	$\frac{0.816}{\sqrt{n}}$	&	$\frac{0.888}{\sqrt{n}}$	&	$\frac{1.038}{\sqrt{n}}$	&	$\frac{1.212}{\sqrt{n}}$	\\[4pt] \hline
\end{tabular}
\caption{Critical values for the Lilliefors test.  Note the use of $n$ rather than $\nu$.}
\label{tbl:Critical_KS2}
\end{table}

\clearpage
\subsection{Two-sample comparison}
Table \ref{tbl:Critical_KS3} gives critical $D$-values for $\alpha = 0.05$ (upper row) and $\alpha = 0.01$ (lower row) for various sample sizes, $n_1$ and $n_2$.
The asterisk (*) means you cannot reject $H_0$ regardless of observed $D$.  For blank entries
simply reverse $n_1$ and $n_2$.
\begin{table}[h]
\centering
\begin{tabular}{|c||c|c|c|c|c|c|c|c|c|c|} \hline
$n_2 | n_1$:  &  \bf{3}  & \bf{4} & \bf{5} & \bf{6}  & \bf{7} &  \bf{8}  & \bf{9} & \bf{10} & \bf{11}  & \bf{12} \\ \hline
\input{CriticalTables/DA1_Table_KS3}
\end{tabular}
\caption{Critical values for the two-sample Kolmogorov-Smirnov test.  Note the use of $n$ rather than $\nu$.}
\label{tbl:Critical_KS3}
\end{table}

For larger sample sizes, the approximate critical value $D_\alpha$ is given by the equation
\begin{equation}
D_\alpha = c(\alpha) \sqrt{\frac{n_1 + n_2}{n_1\cdot n_2}},
\end{equation}
where the coefficient $c(\alpha)$ is given via the table below:
\begin{table}[H]
\centering
\begin{tabular}{|c||c|c|c|c|c|c|} \hline
$\alpha$	& \bf{0.10} & \bf{0.05} & \bf{0.025} & \bf{0.01} & \bf{0.005} & \bf{0.001} \\ \hline
$c(\alpha)$	& 1.22 & 1.36 & 1.48  & 1.63 & 1.73  & 1.95 \\ \hline
\end{tabular}
\label{tbl:Critical_KS3b}
\end{table}
Examples:
\begin{enumerate}
	\item For $\alpha = 0.05$ and sample sizes 5 and 8, $D_\alpha = 30/40 = 0.75$.
	\item For $\alpha = 0.01$ and sample sizes 15 and 28, $D_\alpha = 1.63 \sqrt{\frac{15+28}{15 \cdot 28}} = 0.522$.
\end{enumerate}
\index{Critical values!Kolmogorov-Smirnov|)}

\clearpage
\section{Critical Values for Spearman's Rank Correlation}
\index{Critical values!Spearman's rank}
This is either a one-sided (e.g., $H_0: \rho > 0$) or two-sided ($H_0: \rho = 0$) test
for the nonparametric rank correlation.  Given the number of pairs, $n$, in the sample and
the chosen confidence level, $\alpha$, Table~\ref{tbl:Critical_Spearman} shows the critical correlation that must be
exceeded for $H_0$ to be rejected.  The asterisk identify situations when $H_0$ cannot be rejected. 
\begin{table}[h]
\centering
\footnotesize
\begin{tabular}{|c|cccccccc|} \hline
\bf{One tail, } $\alpha$:  &  \bf{0.10}  & \bf{0.05} & \bf{0.025} & \bf{0.01}  & \bf{0.005}  & \bf{0.0025}  & \bf{0.001}  & \bf{0.0005} \\ \hline
\bf{Two tails, } $\alpha$:  &  \bf{0.20}  & \bf{0.1} & \bf{0.05} & \bf{0.02}  & \bf{0.01}   & \bf{0.005}   & \bf{0.002}   & \bf{0.001} \\ \hline
\input{CriticalTables/DA1_Table_Spearman}
\end{tabular}
\normalsize
\caption{Critical values for Spearman's rank correlation.  Note the use of $n$ rather than $\nu$.}
\label{tbl:Critical_Spearman}
\end{table}

\clearpage
\section{Relationship Between $\kappa$ and $\bar{R}$ for 2-D Directional Data}
\index{Mean resultant!kappa (2-D)}
Given a mean resultant ($\bar{R}$) we use Table~\ref{tbl:Critical_kappa2} to obtain the corresponding concentration parameter ($\kappa$)
for directional data in the plane.
Alternatively, one can solve the implicit equation for $\kappa$ given by
\begin{equation}
	\bar{R} = \frac{I_1(\kappa)}{I_0(\kappa)}.
	\label{eq:kappa2d}
\end{equation}
\begin{table}[h]
\centering
\begin{tabular}{|cc|cc|cc|} \hline
$\bar{R}$  &  $\kappa$  & $\bar{R}$  &  $\kappa$ & $\bar{R}$  &  $\kappa$ \\ \hline
\input{CriticalTables/DA1_Table_kappa2}
\end{tabular}
\caption{Relationship between $\kappa$ and $\bar{R}$ in 2-D.}
\label{tbl:Critical_kappa2}
\end{table}
\clearpage
\section{Critical Values of $\bar{R}$ for 2-D Directional Data}
\index{Critical values!$\bar{R}$ 2-D}
Given the level of significance we determine the critical value for the mean resultant
length under the null hypothesis of no preferred direction in the plane ($H_0: \bar{R} = 0$), with the alternative hypothesis
being that the data can be described via the von Mises distribution (\ref{eg:von_Mises}) with a preferred trend ($H_1: \bar{R} \neq 0$).
\begin{table}[h]
\centering
\begin{tabular}{|c|cccc|} \hline
$\alpha$:  &  \bf{0.10}  & \bf{0.05} & \bf{0.025} & \bf{0.01} \\ \hline
\input{CriticalTables/DA1_Table_R2mean}
\end{tabular}
\caption{Critical values for $\bar{R}$ in the plane.  Note the use of $n$ rather than $\nu$.}
\label{tbl:Critical_R2}
\end{table}

\clearpage
\section{Relationship Between $\kappa$ and $\bar{R}$ for 3-D Directional Data}
\index{Mean resultant!kappa (3-D)}
Given a mean resultant ($\bar{R}$) we use Table~\ref{tbl:Critical_kappa3} to obtain the corresponding concentration parameter ($\kappa$)
for directional data in space.
Alternatively, one can solve the implicit equation for $\kappa$ given by
\begin{equation}
	\coth{\kappa} - 1/\kappa = \bar{R}.
	\label{eq:kappa3d}
\end{equation}
\begin{table}[h]
\centering
\begin{tabular}{|cc|cc|cc|} \hline
$\bar{R}$  &  $\kappa$  & $\bar{R}$  &  $\kappa$ & $\bar{R}$  &  $\kappa$ \\ \hline
\input{CriticalTables/DA1_Table_kappa3}
\end{tabular}
\caption{Relationship between $\kappa$ and $\bar{R}$ in 3-D.}
\label{tbl:Critical_kappa3}
\end{table}
\clearpage
\section{Critical Values of $\bar{R}$ for 3-D Directional Data}
\index{Critical values!$\bar{R}$ 3-D}
Given the level of significance we determine the critical value for the mean resultant
length under the null hypothesis of no preferred direction in space ($H_0: \bar{R} = 0$), with the alternative hypothesis
being that the data can be described via the Fisher distribution (\ref{eq:fisher}) with a preferred trend ($H_1: \bar{R} \neq 0$).
\begin{table}[h]
\centering
\begin{tabular}{|c|cccc|} \hline
$\alpha$:  &  \bf{0.10}  & \bf{0.05} & \bf{0.025} & \bf{0.01} \\ \hline
\input{CriticalTables/DA1_Table_R3mean}
\end{tabular}
\caption{Critical values for $\bar{R}$ in 3-D space.  Note the use of $n$ rather than $\nu$.}
\label{tbl:Critical_R3}
\end{table}

\index{Critical values|)}
	

\printindex

\end{document}
