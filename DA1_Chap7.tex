%  $Id: DA1_Chap7.tex 670 2018-12-20 20:37:20Z pwessel $
%
\chapter{SEQUENCES AND SERIES ANALYSIS}
\label{ch:sequences}
\epigraph{``If you can't explain it simply, you don't understand it well enough.''}{\textit{Albert Einstein, Physicist}}
\index{Sequence}
\index{Data!sequence}
	The geosciences are replete with observational data that can be viewed as ordered sequences.  
Their single most important property is that they form a \emph{sequence}, and the \emph{positions} where 
data points occur within the sequence are paramount.  Contrast that arrangement to a set of repeated 
measurements of some quantity, say ten determinations of sandstone densities.  Our goal of 
determining the average density is not affected by how we order our data --- the order in the 
sequence is not important.  Sequential data therefore often consist of series with \emph{pairs} of 
variables:  The first indicates the position in the sequence, the other gives the observation.  A 
special family of such sequences consists of those in which an observation is given as a function of \emph{time}.  
The analysis of such data is traditionally called \emph{time series analysis}.  In this book we will 
extend these techniques to include ``space series'' as well, i.e., time and distance will be 
considered interchangeable, and we will discuss such data and the methods used to analyze them at
length in Chapter~\ref{ch:spectralanallysis}.
 
      We can think of many examples of natural data that are time --- or space ---
sequences, e.g., a series of temperatures as a function of time, tide gauge readings taken over  a 
long period, topography measured along a transect, and much more.  While such data sets will be our 
main concern, we should not forget that much sequential data do not have the format 
of a ``time''-series.  We might be considering a stratigraphic sequence consisting of the lithologic 
states encountered in a sedimentary succession.  The stratigraphy might show a cyclothem of 
shale--sandstone--shale--sandstone--shale--coal, etc., going from top to bottom.  We would like to 
investigate the significance of the succession, but cannot put a meaningful scale on the sequence.  
It is clear that the succession of lithologies represents changes over time, but we have no way of 
estimating the time scale.  Could we use thickness as a proxy for time?  Thickness is certainly related 
to time through sedimentation rates, but these are known to vary.  Additional complications include 
hard-to-estimate effects like compaction and erosion.  Furthermore, the thicknesses are likely to change 
significantly from location to location.  Thus, if we use thickness or any other measure of down-hole
position it may obscure the examination of the \emph{successions}, which is the primary objective of interest. 
For example, consider an observation that sandstone is the second state and coal the sixth state
in a sequence.  Clearly, this relationship has no meaning that can be 
expressed numerically, e.g., ``sixth'' is not $3 \times$ ``second''.  Obviously, we have here a problem of a 
different nature than the usual time-series analysis mentioned above.  

	Yet another type of sequence is a series of \emph{events}.  Such data may be historical records of 
earthquakes in California, volcanic eruptions of Mauna Loa, or reversals of the Earth's magnetic field.
In these cases, the data simply consist of the time interval \emph{between} events or a cumulative length of time over
which the events occurred, and special analysis techniques are required.

Thus, the nature of the sequential data and the type of sequence determine what questions we 
may hope to answer by subjecting our data to analysis.  The purpose of any of these methods is to 
facilitate answers to questions such as these:
\begin{enumerate}
\item	Are the data random, or do they exhibit a trend or pattern?
	\item	If there is a trend, what form does it have?
	\item	Are there any periodicities in the data?
	\item	What can be estimated or predicted from the data?
	\item	Are there other questions specific to the situation?
\end{enumerate}

	We shall see that while we often are concerned with the analysis of a single sequence of data, there 
are many instances in which we want to compare two or more sequences.  One obvious example 
to geologists is \emph{stratigraphic correlation}, based on either lithologic sections or well log data.  Sequence 
correlation may speed up routine correlations and detect subtle correlations which may be hard to 
detect by eye.

The methods for comparing two or more sequences can be grouped into two 
broad classes.
In the first, the exact position in a sequence matters, and a correlation is only 
significant if it takes place at the correct location.  One example is the comparison of an X-ray 
diffraction chart with standard charts in attempts to recognize minerals.  The comparison can 
only take place at certain angles.  If the shape of a spectral peak centered on 20$^\circ$ in the data looks 
exactly like the bump at 30$^\circ$ in the standard chart it is of no significance: both peaks would have 
to occur at the same angle.

     In the other class of methods the absolute position is not important, only relative position 
matters.  These processes, like \emph{cross-correlation}, are very similar to the mental process of 
geologic correlation.  However, these methods are limited because they cannot take stretching 
and compression of the scale into account.  Nevertheless, in many problems there is no distortion and we may 
use such techniques with some success.

\section{Markov Chains}
\index{Markov chains|(}

As mentioned above, many geological experiments result in data sequences consisting of 
ordered successions of \emph{mutually exclusive} states.  We already mentioned the lithology variations 
in stratigraphic sections.  Other examples include:
\begin{itemize}
\item	The changes in minerals across a line in a thin-section.
\item	Drill holes through zoned ore bodies.
\end{itemize}
The observations may be obtained at evenly spaced intervals, or we may simply register the 
position when a change of state occurs.  In the first case we would expect repeated states; the 
latter will obviously not contain such runs since we only record changes of state. 

     Such data may be subjected to \emph{cross-association} and/or \emph{auto-association} techniques, but right 
now we are primarily concerned with the nature of transitions rather than the relative positions of 
states in the sequence.  Therefore we will, for the moment, pay less attention to the positions of observations 
within the succession and instead concentrate on acquiring information about the \emph{tendency} 
of one state to follow another.
\begin{example}
\PSfig[H]{Fig1_lithology}{Example of a stratigraphic section with four separate lithologies.
Assessing the lithology at every 1-foot interval down the section resulted in 62 states and thus we recorded 61 transitions of state.}

     Let us look at an example of a stratigraphic section.  Here, we have determined the lithology 
at one foot intervals down a stratigraphic section.  This exercise resulted in a sequence of states.
	We find four mutually exclusive states: A) sandstone, B) limestone, C) shale, and 
D) coal.  There are 62 observed states, hence 61 transitions.  We tabulate the transitions in a 
4 x 4 matrix, since we have four possible transitions for each of the four states.  E.g., from sandstone 
(A) we may change to A, B, C, or D, since repeated states may occur.  The \emph{transition frequency 
matrix} $\mathbf{A}$ given in Table~\ref{tbl:markov1} expresses all the observed possibilities.
\index{Transition frequency matrix}
\index{Matrix!transition frequency}
\begin{table}[h]
\center
\begin{tabular}{|c|r|c|r|c|c|}
\hline
 & \bf{A} & \bf{B} & \bf{C} & \bf{D} & \bf{Row Total}\\ \hline
\bf{A} & 17 & 0 & 5 & 0 & 22 \\ \hline
\bf{B} & 0 & 5 & 2 & 0 & \, 7 \\ \hline
\bf{C} & 5 & 2 & 17 & 3 & 27 \\ \hline
\bf{D} & 0 & 0 & 3 & 2 & \, 5 \\ \hline
\bf{Column total} & 22 & 7 & 27 & 5 & 61 \\ \hline
\end{tabular} 
\caption{Transition frequency matrix for the transitions seen in Figure~\ref{fig:Fig1_lithology}.}
\label{tbl:markov1}
\end{table}
We can now see that element $a_{ij}$ reads ``number of transitions from state $i$ to state $j$''.
For our section, the matrix is symmetric.  However, in general this will not be the case, so $a_{ij} \neq   a_{ji}$.

	The tendency for one state to succeed another can be made clearer by converting the 
frequencies to percentages or fractions.  We do this by dividing each row by its row total.  These 
percentages may be considered conditional probabilities in that they measure the probability 
that state $j$ will follow \emph{given} that the present state is $i$, which we write as
$P(j|i)$ or $P(i \rightarrow  j)$.  The resulting \emph{transition probability matrix} $\mathbf{P}$
is given by Table~\ref{tbl:markov2} and graphically
illustrated in Figure~\ref{fig:Fig1_Markov}.
\index{Transition probability matrix}
\index{Matrix!transition probability}
\begin{table}[h]
\center
\begin{tabular}{|c|l|l|l|l|}
\hline
\bf{From/To} & \ \ \bf{A}  & \  \bf{B}  & \ \ \bf{C}  & \  \bf{D} \\ \hline
\bf{A} & 0.77 & 0.00 & 0.23 & 0.00 \\ \hline
\bf{B} & 0.00 & 0.71 & 0.29 & 0.00 \\ \hline
\bf{C} & 0.19 & 0.07 & 0.63 & 0.11 \\ \hline
\bf{D} & 0.00 & 0.00 & 0.60 & 0.40 \\ \hline
\end{tabular}
\caption{The transition probability matrix for the transitions in Table~\ref{tbl:markov1}.}
\label{tbl:markov2}
\end{table}

\PSfig[h]{Fig1_Markov}{A cyclic diagram may be used to represent the frequencies of transition between
the various lithologies.}

This operation will usually result in an asymmetrical matrix.  If we divide the row totals (the 
counts) by the grand total number of counts we find the relative \emph{proportions} of the four lithologies.  
This is called the marginal or \emph{fixed probability vector}:
\index{Marginal probability vector}
\index{Fixed probability vector}
\index{Vector!fixed probability}
\index{Vector!marginal probability}
\begin{equation}
\mathbf{f} = 
\left [ \begin{array}{cccc}
0.36 & 0.12 & 0.44 & 0.08
\end{array}
\right ].
\end{equation}
You may remember (if not, brush up on the probability theory in Chapter~\ref{ch:basics}) that the joint probability of two events 
$A$ and $B$ is
\begin{equation}
P(A \cap B) = P(B|A)\cdot P(A),
\end{equation}
which we can rearrange to give
\begin{equation}
P(B|A) = \frac{P(A \cap B)}{P(A)}.
\end{equation}
The probability that state $B$ will follow $A$ is the probability that both states $A$ and $B$ will occur,
divided by the probability that $A$ occurs.  Now, if $A$ and $B$ are independent states, then (e.g., \ref{eq:jointindependent})
\begin{equation}
P(A \cap B) = P(A) \cdot P(B).
\end{equation}
Therefore, if there are no dependencies then the probability that $B$ will follow $A$ is simply the 
probability that $B$ occurs.  This must hold for all independent states, so
\begin{equation}
P(B|A) = P(B|B)  = P(B|C) = P(B|D) = P(B).
\end{equation}
This result provides us with an opportunity to predict what the transition probability matrix should look 
like if the occurrence of a lithologic state at one point were completely independent of the 
lithology at the underlying point.  Naturally, that matrix will have rows matching the fixed 
probability vector.  So, for our stratigraphic example, we find the expected matrix to be
$$
\begin{array}
{|c|c|c|c|c|} \hline
 & \bf{A} & \bf{B} & \bf{C} & \bf{D} \\ \hline
\bf{A} & 0.36 & 0.12 & 0.44 & 0.08  \\ \hline
\bf{B} & 0.36 & 0.12 & 0.44 & 0.08 \\ \hline
\bf{C} & 0.36 & 0.12 & 0.44 &  0.08 \\ \hline
\bf{D} &  0.36 & 0.12 & 0.44 &  0.08\\ \hline
\end{array}
$$
Finally, we are now in the position to compare the observed transition frequencies to the predicted frequencies 
and test the null hypothesis that all lithologic states are independent of the state immediately below it.  
To do so we use a $\chi^2$-test after first converting the expected percentages back to counts or frequencies.  
We find
\begin{equation}
\left [\begin{array}{cccc}
22 &  &  & \\
 & 7 &  &  \\
 &  & 27 &  \\ 
 &  &  & 5 \end{array}
\right ] \cdot
\left [\begin{array}{cccc}
0.36 & 0.12 & 0.44 & 0.08\\
0.36 & 0.12 & 0.44 & 0.08 \\
0.36 & 0.12 & 0.44 & 0.08 \\ 
0.36 & 0.12 & 0.44 & 0.08 \end{array}
\right ] 
= 
\left [ \begin{array}{rcrc} 
7.9 & 2.6 & 9.7 & 1.8 \\
2.5 & 0.8 & 3.1 & 0.6 \\
9.7 & 3.2 & 11.9 & 2.2 \\
1.8 & 0.6 & 2.2 & 0.4 \end{array}
\right ].
\label{eq:markov_E}
\end{equation}
The test statistic is, as usual, given by 
\index{Test!$\chi^2$ (``chi-squared'')}
\index{Test!chi-squared ($\chi^2$)}
\index{$\chi^2$ test (``chi-squared'')}
\index{Chi-squared test ($\chi^2$)}
\begin{equation}
\chi^2 = \sum^n_{i=1} \frac{(O_i - E_i)^2}{E_i},
\end{equation}
where $O_i$ is the observed number of transitions and $E_i$ is the expectation for each transition, as given by (\ref{eq:markov_E}).  The 
degrees of freedom, $\nu$, is $(k-1) \cdot (k-1)$, with $k = 4$.  One degree of freedom is lost from each row and column 
because all rows of $\mathbf{P}$ must sum to 1 and we computed $\mathbf{f}$ from the row sums.  

     For the $\chi^2$-test to be valid, each category should have an expected value of at least 5.  Several 
of our categories do not fulfill that criteria.  Because we only are testing whether the transition 
frequencies are independent (random) or not, we may combine some categories to raise the 
expected values above 5.  Hence, we use the four largest categories $A\rightarrow A, A\rightarrow C, C\rightarrow A, C\rightarrow C$ and
the combinations $B\rightarrow$ any, $D\rightarrow$ any, and $[ A\rightarrow B, A\rightarrow D, C\rightarrow B, C\rightarrow D ]$.  We find $\chi^2$ to be 
\begin{equation}
\chi^2 = \begin{array}{c}
\displaystyle \frac{(17 - 7.9)^2}{7.9} +  \frac{(5 - 9.7)^2}{9.7} + \frac{(5-9.7)^2}{9.7} +   \frac{(17 - 11.9)^2}{11.9} + \\*[2ex]
\displaystyle \frac{(7-7.0)^2}{7.0} +  \frac{(5-5.0)^2}{5.0} +    \frac{(5-9.8)^2}{9.8} = 19.57.
\end{array}
\end{equation}
From Table~\ref{tbl:Critical_chi2} we find the critical value of $\chi^2$ with $\nu = 9$ and a 5\% level of significance to be 16.92.  
Since our computed value exceeds the critical value, we must reject the hypothesis that 
successive states are independent.  It appears there is a significant tendency for certain states to 
be followed by certain other states.  
\end{example}

     Sequences in which the state at one point is \emph{partially} dependent, in a probabilistic sense, on 
the previous state is called a \emph{Markov chain}.  It is intermediate between deterministic (fully 
predictable) and completely random sequences.  The section we examined has first-order 
Markov properties.  This means there is statistical dependency between points and their 
immediate predecessor. 

Higher-order Markov chains can exist as well. We can use the transition probability matrix to predict what the lithology might be \emph{two} feet 
above a point.  For example, we might want to fill in the missing part of a section in the 
statistically most reasonable way.  Let us say we start in state B (limestone).  The probabilities of 
reaching the next state is then given as 
\begin{equation}
\begin{array}{clr}
B \rightarrow A & \mbox{(sandstone)}	& 	0\%  \\
B \rightarrow B & \mbox{(limestone)}	&	71\% \\
B \rightarrow C & \mbox{(shale)}	&	29\% \\
B \rightarrow D & \mbox{(coal)}		& 	0\%
\end{array}
\end{equation}
Let us pretend that the next state is shale (C). Then, reaching the following state would be associated with the probabilities
\begin{equation}
\begin{array}{llr}
C \rightarrow A & \mbox{(sandstone)}	& 	19\% \\
C \rightarrow B & \mbox{(limestone)}	& 	7\%  \\
C \rightarrow C & \mbox{(shale)}	& 	63\% \\
C \rightarrow D & \mbox{(coal)}		& 	11\%
\end{array}
\end{equation}
Therefore, the probability that the sequence will be limestone $\rightarrow$ shale $\rightarrow$ limestone is 
\begin{equation}
P(B \rightarrow C) \cdot P (C \rightarrow B) = 29\% \cdot 7\% = 2\%.
\end{equation}
However, we can also reach limestone in two steps by way of the limestone $\rightarrow$ limestone $\rightarrow$
limestone path.  Now
\begin{equation}
P (B \rightarrow B) \cdot P (B \rightarrow B) = 71\% \cdot 71\% = 50\%.
\end{equation}
Since $B \rightarrow
A$  and $B\rightarrow D$ have zero probability, we can state that the probability of finding 
limestone two steps up above limestone, regardless of intervening lithology, is the sum 
\begin{equation}
P(B \rightarrow ? \rightarrow B) = P(B \rightarrow B \rightarrow B) + P(B \rightarrow C \rightarrow B) = 50\% + 2\% = 52\%.
\end{equation}
One can use the same approach to calculate the probability of any lithology two steps up, but 
fortunately there is a more efficient way: These multiplications and additions are exactly those that 
define a matrix multiplication.  Multiplying the transition matrix by itself (i.e., squaring it) yields
$\mathbf{P}^2$, which describes the \emph{second-order} Markov properties of the stratigraphic section:
\begin{equation}
\left [ \begin{array}{rrrr}
0.77 & 0 & 0.23 & 0 \\
0 & 0.71 & 0.29 & 0 \\
0.19 & 0.07 & 0.63 & 0.11 \\
0 & 0 & 0.60 & 0.40 \end{array} \right ]^2 = \left [ \begin{array}{rrrr}
0.64 & 0.02 & 0.32 & 0.02 \\
0.06 & 0.52 & 0.39 & 0.03\\
0.27 & 0.09 & 0.53 & 0.11 \\
0.11 & 0.04 & 0.62 & 0.23
\end{array} \right ]
\end{equation}
Note that again the rows have unit sums.  If we wanted to know whether the second-order Markov 
properties are significant we convert the frequency percentages to counts by multiplying $\mathbf{P}^2$
by the observed row totals again.  However, this time the product will approximate the second-order
transition we would have \emph{likely} observed had we measured them directly from the data.  We find
\begin{equation}
\left[ \begin{array}{rrrr}
14.1 & 0.4 & 7.0 & 0.6\\
0.4 & 3.7 & 2.7 & 0.2 \\
7.1 & 2.5 & 14.2 & 3.1 \\
0.6 & 0.2 & 3.1 & 1.1
\end{array} \right ]
\end{equation}
and carry out another 
$\chi^2$-test.  This time we obtain $\chi^2 = 7.73$ (critical value is still 16.92 since the expectations
are independent of the step length).  Hence, we must conclude that there are no 
significant second-order Markov properties present and that the lithology two steps away appears to 
be a random selection given the volume distribution of the rock types.
\index{Markov chains|)}

\section{Embedded Markov Chains}
\index{Embedded Markov chains|(}
\index{Markov chains!embedded|(}
\label{sec:embmarkov}

	The choice of a sampling interval introduces an arbitrary element into our sequence analysis.  
This can be avoided if we only record the transitions of state whenever they occur.  It follows 
that the transition frequency matrix will have zeros along the diagonal since no state can follow 
itself.  Sequences which cannot contain repeated states are called \emph{embedded Markov chains}.
\begin{example}
	Let us look at a particular example from a borehole through a sedimentary delta plain in
Scotland.  We have five lithologies: A (mudstone), B (shale), C (siltstone), D 
(sandstone), and E (coal).  The analysis yields
\begin{equation}
% MathType!MTEF!2!1!+-
% faaagCart1ev2aaaKnaaaaWenf2ys9wBH5garuavP1wzZbqedmvETj
% 2BSbqefm0B1jxALjharqqtubsr4rNCHbGeaGqiVu0Je9sqqrpepC0x
% bbL8FesqqrFfpeea0xe9Lq-Jc9vqaqpepm0xbba9pwe9Q8fs0-yqaq
% pepae9pg0FirpepeKkFr0xfr-xfr-xb9Gqpi0dc9adbaqaaeGaciGa
% aiaabeqaamaabaabaaGcbaqbamqabmGaaaqaaaqaauaadeqabuaaaa
% qaaiaadgeaaeaacaWGcbaabaGaam4qaaqaaiaadseaaeaacaWGfbaa
% aaqaauaadeqafeaaaaqaaiaadgeaaeaacaWGcbaabaGaam4qaaqaai
% aadseaaeaacaWGfbaaaaqaamaadmaabaqbamqabuqbaaaaaeaacaaI
% WaaabaGaaGymaiaaigdaaeaacaaIZaGaaGOnaaqaaiaaikdacaaIXa
% aabaGaaGynaiaaikdaaeaacaaIYaGaaGioaaqaaiaaicdaaeaacaaI
% 0aaabaGaaGinaaqaaiaaicdaaeaacaaIZaGaaGinaaqaaiaaikdaae
% aacaaIWaaabaGaaGinaiaaiwdaaeaacaaIXaGaaG4maaqaaiaaikda
% caaI5aaabaGaaGymaaqaaiaaisdacaaI1aaabaGaaGimaaqaaiaaio
% daaeaacaaIYaGaaGioaaqaaiaaikdacaaIZaaabaGaaGyoaaqaaiaa
% iIdaaeaacaaIWaaaaaGaay5waiaaw2faaaqaaaqaauaadeqabuaaaa
% qaaaqaaaqaaaqaaaqaaaaaaaqbamqabmqaaaqaaaqaauaadeqafeaa
% aaqaaiabg2da9aqaaiabg2da9aqaaiabg2da9aqaaiabg2da9aqaai
% abg2da9aaaaeaacqGH9aqpaaqbamqabmqaaaqaaiabfo6atbqaauaa
% deqafeaaaaqaaiaaigdacaaIYaGaaGimaaqaaiaaiodacaaI2aaaba
% GaaGyoaiaaisdaaeaacaaI3aGaaGioaaqaaiaaiAdacaaI4aaaaaqa
% amaanaaabaGaaG4maiaaiMdacaaI2aaaaaaaaaa!662C!
\begin{array}{*{20}{c}}
{}&{\begin{array}{*{20}{c}}
A&B&C&D&E
\end{array}}\\
{\begin{array}{*{20}{c}}
A\\
B\\
C\\
D\\
E
\end{array}}&{\left[ {\begin{array}{*{20}{c}}
0&{11}&{36}&{21}&{52}\\
{28}&0&4&4&0\\
{34}&2&0&{45}&{13}\\
{29}&1&{45}&0&3\\
{28}&{23}&9&8&0
\end{array}} \right]}\\
{}&{\begin{array}{*{20}{c}}
{}&{}&{}&{}&{}
\end{array}}
\end{array}\begin{array}{*{20}{c}}
{}\\
{\begin{array}{*{20}{c}}
 = \\
 = \\
 = \\
 = \\
 = 
\end{array}}\\
 = 
\end{array}\begin{array}{*{20}{c}}
\Sigma \\
{\begin{array}{*{20}{c}}
{120}\\
{36}\\
{94}\\
{78}\\
{68}
\end{array}}\\
{\overline {396} }
\end{array}
\end{equation}
The fixed probability vector is found by dividing the row totals by the grand total:
\begin{equation}
\mathbf{\mathbf{f}} = [0.30 \quad    0.09 \quad    0.24 \quad   0.20 \quad   0.17].
\end{equation}
To test whether the observed sequence has Markovian properties or independent states we 
may use the $\chi^2$-test in a similar way to what we did above.  The problem is that we cannot use 
the fixed vector to estimate the independent transition frequency matrix since that results in 
nonzero diagonal terms, which is forbidden.  We must therefore use a different method to 
estimate the necessary matrix.

	Imagine our sequence is a \emph{censored} sample from a sequence where repeats \emph{may} occur.  Its 
transition matrix would be identical to the one we observed except it would have nonzero 
diagonal terms.  If we converted this matrix to probabilities and raised it to a high power, we 
would find the transition probability matrix for a sequence with independent states.  We could 
then discard the diagonal terms, adjust the off-diagonal terms (to ensure they sum to 1), and end up with $P$ 
for an embedded sequence of independent states.  This result is achieved by trial-and-error 
since we do not know the number of repeated states in the envisioned sequence.  We want to find 
diagonal entries which do not change when the matrix is raised to higher powers.  An iterative scheme is used:
\begin{enumerate}
\item	Place arbitrary large estimates (1--2 magnitudes larger that your observations) into the diagonal positions in the observed 
matrix.
\item	Divide row totals by the grand total to get diagonal probabilities.
\item	Calculate new diagonal estimates by multiplying the diagonal probabilities from step 2 by the 
latest row sums.
\item	Repeat process steps (2) and (3) until the diagonal terms remain unchanged, typically after 
10--20 iterations.
\end{enumerate}
We will try this procedure on the Scottish data.  For our matrix, we try inserting 1000 first:
\begin{equation}
% MathType!MTEF!2!1!+-
% faaagCart1ev2aaaKnaaaaWenf2ys9wBH5garuavP1wzZbqedmvETj
% 2BSbqefm0B1jxALjharqqtubsr4rNCHbGeaGqiVu0Je9sqqrpepC0x
% bbL8FesqqrFfpeea0xe9Lq-Jc9vqaqpepm0xbba9pwe9Q8fs0-yqaq
% pepae9pg0FirpepeKkFr0xfr-xfr-xb9Gqpi0dc9adbaqaaeGaciGa
% aiaabeqaamaabaabaaGcbaqbamqabiqaaaqaamaadmaabaqbamqabu
% qbaaaaaeaacaaIXaGaaGimaiaaicdacaaIWaaabaGaaGymaiaaigda
% aeaacaaIZaGaaGOnaaqaaiaaikdacaaIXaaabaGaaGynaiaaikdaae
% aacaaIYaGaaGioaaqaaiaaigdacaaIWaGaaGimaiaaicdaaeaacaaI
% 0aaabaGaaGinaaqaaiaaicdaaeaacaaIZaGaaGinaaqaaiaaikdaae
% aacaaIXaGaaGimaiaaicdacaaIWaaabaGaaGinaiaaiwdaaeaacaaI
% XaGaaG4maaqaaiaaikdacaaI5aaabaGaaGymaaqaaiaaisdacaaI1a
% aabaGaaGymaiaaicdacaaIWaGaaGimaaqaaiaaiodaaeaacaaIYaGa
% aGioaaqaaiaaikdacaaIZaaabaGaaGyoaaqaaiaaiIdaaeaacaaIXa
% GaaGimaiaaicdacaaIWaaaaaGaay5waiaaw2faaaqaauaadeqabuaa
% aaqaaaqaaaqaaaqaaaqaaaaaaaqbamqabiqaaaqaauaadeqafeaaaa
% qaaiabg2da9aqaaiabg2da9aqaaiabg2da9aqaaiabg2da9aqaaiab
% g2da9aaaaeaafaWabeqabaaabaGaeyypa0daaaaafaWabeGabaaaba
% qbamqabuqaaaaabaGaaGymaiaaigdacaaIYaGaaGimaaqaaiaaigda
% caaIWaGaaG4maiaaiAdaaeaacaaIXaGaaGimaiaaiMdacaaI0aaaba
% GaaGymaiaaicdacaaI3aGaaGioaaqaaiaaigdacaaIWaGaaGOnaiaa
% iIdaaaaabaWaa0aaaeaafaWabeqabaaabaGaaGynaiaaiodacaaI5a
% GaaGOnaaaaaaaaaaaa!6EF7!
\begin{array}{*{20}{c}}
{\left[ {\begin{array}{*{20}{c}}
{1000}&{11}&{36}&{21}&{52}\\
{28}&{1000}&4&4&0\\
{34}&2&{1000}&{45}&{13}\\
{29}&1&{45}&{1000}&3\\
{28}&{23}&9&8&{1000}
\end{array}} \right]}\\
{\begin{array}{*{20}{c}}
{}&{}&{}&{}&{}
\end{array}}
\end{array}\begin{array}{*{20}{c}}
{\begin{array}{*{20}{c}}
 = \\
 = \\
 = \\
 = \\
 = 
\end{array}}\\
{\begin{array}{*{20}{c}}
 = 
\end{array}}
\end{array}\begin{array}{*{20}{c}}
{\begin{array}{*{20}{c}}
{1120}\\
{1036}\\
{1094}\\
{1078}\\
{1068}
\end{array}}\\
{\overline {\begin{array}{*{20}{c}}
{5396}
\end{array}} }
\end{array}
\end{equation}
We next obtain the new diagonal probabilities to be
\begin{equation}
\left [ \begin{array}{ccccc}
0.208 & & & & \\
& 0.192 & & &\\
& & 0.203 & & \\
& & & 0.200 & \\
& & & & 0.198
\end{array}
\right ] .
\end{equation}
Step (3) is to update the diagonal elements via the new row sums, hence we find
\begin{equation}
% MathType!MTEF!2!1!+-
% faaagCart1ev2aaaKnaaaaWenf2ys9wBH5garuavP1wzZbqedmvETj
% 2BSbqefm0B1jxALjharqqtubsr4rNCHbGeaGqiVu0Je9sqqrpepC0x
% bbL8FesqqrFfpeea0xe9Lq-Jc9vqaqpepm0xbba9pwe9Q8fs0-yqaq
% pepae9pg0FirpepeKkFr0xfr-xfr-xb9Gqpi0dc9adbaqaaeGaciGa
% aiaabeqaamaabaabaaGcbaqbamqabiqaaaqaamaadmaabaqbamqabu
% qbaaaaaeaacaaIYaGaaG4maiaaiodaaeaacaaIXaGaaGymaaqaaiaa
% iodacaaI2aaabaGaaGOmaiaaigdaaeaacaaI1aGaaGOmaaqaaiaaik
% dacaaI4aaabaGaaGymaiaaiMdacaaI5aaabaGaaGinaaqaaiaaisda
% aeaacaaIWaaabaGaaG4maiaaisdaaeaacaaIYaaabaGaaGOmaiaaik
% dacaaIYaaabaGaaGinaiaaiwdaaeaacaaIXaGaaG4maaqaaiaaikda
% caaI5aaabaGaaGymaaqaaiaaisdacaaI1aaabaGaaGOmaiaaigdaca
% aI1aaabaGaaG4maaqaaiaaikdacaaI4aaabaGaaGOmaiaaiodaaeaa
% caaI5aaabaGaaGioaaqaaiaaikdacaaIXaGaaGOmaaaaaiaawUfaca
% GLDbaaaeaafaWabeqafaaaaeaaaeaaaeaaaeaaaeaaaaaaauaadeqa
% ceaaaeaafaWabeqbbaaaaeaacqGH9aqpaeaacqGH9aqpaeaacqGH9a
% qpaeaacqGH9aqpaeaacqGH9aqpaaaabaqbamqabeqaaaqaaiabg2da
% 9aaaaaqbamqabiqaaaqaauaadeqafeaaaaqaaiaaiodacaaI1aGaaG
% 4maaqaaiaaikdacaaIZaGaaGynaaqaaiaaiodacaaIXaGaaGOnaaqa
% aiaaikdacaaI5aGaaGinaaqaaiaaikdacaaI4aGaaGimaaaaaeaada
% qdaaqaauaadeqabeaaaeaacaaIXaGaaGinaiaaiEdacaaI4aaaaaaa
% aaaaaa!67D6!
\begin{array}{*{20}{c}}
{\left[ {\begin{array}{*{20}{c}}
{233}&{11}&{36}&{21}&{52}\\
{28}&{199}&4&4&0\\
{34}&2&{222}&{45}&{13}\\
{29}&1&{45}&{215}&3\\
{28}&{23}&9&8&{212}
\end{array}} \right]}\\
{\begin{array}{*{20}{c}}
{}&{}&{}&{}&{}
\end{array}}
\end{array}\begin{array}{*{20}{c}}
{\begin{array}{*{20}{c}}
 = \\
 = \\
 = \\
 = \\
 = 
\end{array}}\\
{\begin{array}{*{20}{c}}
 = 
\end{array}}
\end{array}\begin{array}{*{20}{c}}
{\begin{array}{*{20}{c}}
{353}\\
{235}\\
{316}\\
{294}\\
{280}
\end{array}}\\
{\overline {\begin{array}{*{20}{c}}
{1478}
\end{array}} }
\end{array}
\end{equation}
Repeating step (2) with the new matrix gives
\begin{equation}
\left [ \begin{array}{ccccc}
0.239 & & & & \\
& 0.159 & & &\\
& & 0.214 & & \\
& & & 0.199 & \\
& & & & 0.189
\end{array}
\right ] ,
\end{equation}
and we keep repeating this process until it stabilizes.  In the end we find the marginal probability 
vector:
\begin{equation}
\left [ \begin{array}{ccccc}
0.335 \\
& 0.074 \\
& & 0.235 \\
& & & 0.181\\
& & & & 0.155
\end{array}
\right ]
\end{equation}
with a corresponding grand total of 524.

	Now, because all states are independent in the random case we will test for, the probability 
that state $j$ will follow state $i$ is simply $P(i \rightarrow j) = P(i) \cdot P(j)$.  This allows us to construct the 
\emph{expected} transition probability matrix
\begin{equation}
\mathbf{P}_e = \left [ \begin{array}{ccccc}
0.125 & 0.026 & 0.083 & 0.064 & 0.055 \\
0.026 & 0.006 & 0.017 & 0.013 & 0.012\\
0.083 & 0.017 & 0.055 & 0.043 & 0.03 \\
0.064 & 0.013 & 0.043 & 0.033 & 0.028 \\
0.055 & 0.012 & 0.036 & 0.028 & 0.024
\end{array} \right ].
\end{equation}
Scaling these probabilities by the grand total gives the expected frequencies
\begin{equation}
\mathbf{E} = \left [ \begin{array}{ccccc}
65.6  & 13.6 & 43.5 & 33.5  & 28.8  \\
13.6  & 3.1   & 8.9   & 6.8   & 6.3  \\
43.5  & 8.9   & 28.8 & 22.5  & 18.9 \\
33.5 & 6.8   & 22.5  &17.3   & 14.7  \\
28.8  & 6.3  & 18.9  & 14.7 &  12.6 
\end{array} \right ].
\end{equation}
\index{Test!$\chi^2$ (``chi-squared'')}
\index{Test!chi-squared ($\chi^2$)}
\index{$\chi^2$ test (``chi-squared'')}
\index{Chi-squared test ($\chi^2$)}
We strip off the diagonal elements and use the off-diagonal counts to evaluate the $\chi^2$ statistic.  
In this particular case, we find $\chi^2 = 172$ which greatly exceeds the critical value of 19.68 for $v = (k-1)^2 - 
k = 11$ degrees of freedom; the test indicates a strong first order Markov sequence.
\end{example}
\index{Embedded Markov chains|)}
\index{Markov chains!embedded|)}

\section{Series of Events}
\index{Series of events|(}
\label{sec:seriestest}
	One of many types of time series that occur in the natural sciences is the \emph{series of events}.  
Examples of such sequences include the historical records of earthquake occurrences,  volcanic 
eruptions, floods, storms and hurricanes, geomagnetic reversals (Figure~\ref{fig:Fig1_series}), landslides, and tsunamis.  These series
share some common characteristics:
\begin{itemize}
\item	Events are distinguished based on \emph{when} they occur in time.
\item	Events are essentially \emph{instantaneous} in the context of your range.
\item	Events are so \emph{infrequent} that they do not overlap in time.
\end{itemize}
In some cases, spatial data sequences may be considered a series of events.  Consider a traverse 
across a thin-section.  We may be interested in the occurrence of some rare mineral.  Another 
possibility may be the occurrence of bentonite (volcanic ash layers) in a sedimentary sequence.  
However, when a spatial scale acts as a proxy for the actual time scale, we know that the analysis 
will be susceptible to errors caused by varying sedimentation rates, compaction, and erosion.
\PSfig[H]{Fig1_series}{Times of reversals in the Earth's magnetic field during a 40 million year
interval.  While each reversal may take a few thousand years to complete, these reversals are
essentially instantaneous when seen as part of the much longer geological record.}

	With most studies of series of events we hope to find out what the basic features of the series 
are and how we can relate the distribution of length intervals to a physical mechanism.  We must 
first consider the possibility of a trend in the data.   Thus, we will use a test designed to detect 
trends in the rate of occurrence.  It works by simply comparing the mean (or centroid) of a series 
to its midpoint and test their separation for significance.
The centroid is given by
\begin{equation}
\bar{t} = \frac{1}{n} \sum^n_{i=1} t_i.
\end{equation}
We find
\index{Test!series of events}
\begin{equation}
z = \frac{\bar{t}-t_{1/2}}{T/ \sqrt{12 n}},
\end{equation}
where $t_{1/2}$ is the half-point and $T$ is the length of series.  The denominator is obtained by taking
the standard deviation of a uniform distribution over the range $\{0,T\}$ (which equals $T/\sqrt{12}$) and
using the central limits theorem to give the expected standard deviation of the centroid.
The test is very sensitive to changes in the rate 
of occurrence since the centroid is a $L_2$ estimate of location.   If no trend is detected, then we may 
conclude that the series is \emph{stationary}.
\index{Stationary}
	Many geological processes produce events that should be uniformly distributed in time.  For 
instance, the steady motion of the tectonic plates produce a steady increase in stress on a fault, 
which will slip to relieve the stress.  To test for uniformity we note that the cumulative 
distribution of a uniform series is a straight line from 0 to 1 over the time interval.  We can then 
compare this line to the stair-step cumulative function of our observed series of events and apply 
the Kolmogorov-Smirnov test on the largest discrepancy, as discussed in Chapter~\ref{ch:testing}.
\index{Series of events|)}

\section{Run Test}
\index{Run test|(}

	The simplest sequence imaginable is a succession of observations that can take on only two 
mutually exclusive states or values.  Consider a rock collector looking for fossils:  Each time he 
or she opens a concretion there may or may not be a fossil present.  We find true or false, 
yes or no, or 1 or 0.  Similar sequences are generated by coin tosses.  Twenty tosses may give the 
series
\begin{equation}
HTHHTHTTTHTHTHHTTHHH
\label{eq:heads}
\end{equation}
with 11 heads and 9 tails, close to the expected 10/10.  The probability of finding a given number 
of heads ($x$) in a series of $n$ tries is given by the binomial distribution (\ref{eq:binomial_dist}).
However, there is nothing in that expression that takes the \emph{order} in which the heads appear into account.  We would find 
a sequence of 10H followed by 10T to be highly unlikely; the same goes for alternate HTH...  Yet, the 
probability of the \emph{number} of heads is unchanged.  We test such binary sequences for randomness 
of occurrence by examining the number of \emph{runs}, defined as \emph{uninterrupted sequences of the same 
state}.
\index{Runs}
\begin{example}
In our sequence above (\ref{eq:heads}) we have the following 13 runs:
\begin{equation}
\begin{array}{rrcrrcrrrcccc}
                 H & T & HH &  T &  H &  TTT & H & T &  H &   T &   HH &   TT &   HHH\\
                  1 & 2 &  3 &     4 &   5 &     6 &    7 &   8 &   9 &    10 &  11 &    12 &     13\end{array} 
\end{equation}
This is a job for the $U$-test.
We test the significance of the runs by finding all possible ways of arranging $n_1$ items of state 1 
and $n_2$ items of state 2.  The total number of runs is called $U$, and we can consult tables for critical 
values of $U$ given $n_1, n_2$, and $\alpha$ (our confidence level).  However, for large $n_1, n_2 > 10$  the 
distribution is approximated by a normal distribution with a mean of 
\index{Test!run}
\index{Test!U}
\index{U test}
\begin{equation}
\bar{U} = \frac{2n_1 n_2}{n_1 + n_2} + 1
\end{equation}
and variance
\begin{equation}
s^2_U = \frac{2n_1 n_2 (2n_1 n_2- n_1 - n_2)}{(n_1 + n_2)^2 (n_1 + n_2 -1)}.
\end{equation}
We may then simply use the $z$-statistic
\begin{equation}
z = (U - \bar{U})/ s_U
\end{equation}
and see if our calculated $z$ value exceeds the $\pm z_{\alpha/2}$ interval.  For our H/T series
$n_1 = 11$ and $n_2 = 9$, so we find
\begin{equation}
\bar{U} = \frac{2 \cdot 11 \cdot 9}{11 + 9} + 1 = 10.9 \mbox{ and}
\end{equation}
\begin{equation}
s_U = \sqrt{\frac{(2 \cdot 11 \cdot 9)(2 \cdot 11 \cdot 9 - 11 - 9)}{(11 + 9)^2(11+9 -1)}} = \sqrt{4.6} = 2.1,
\end{equation}
which gives
\begin{equation}
z = \frac{13 - 10.9}{2.1} = 1.0.
\end{equation}
With $z_{\alpha/2} = z_{0.025} = \pm 1.96$ (i.e., Table~\ref{tbl:Critical_t}) we cannot reject the null hypothesis that the sequence appears random.
\end{example}

	The geological application of run tests may seem somewhat obscure, since most data 
consist of more than two mutually exclusive states.  A related procedure is a statistical method 
for examining runs up and down.  Here we are again considering two distinct ``states'', i.e., 
whether an observation is larger or smaller than the preceding observation.  Let us examine the 
data set illustrated in Figure~\ref{fig:Fig1_Run}.

\PSfig[H]{Fig1_Run}{Example of how a run test can be used on continuous data. Any multipoint segment with
the same sign of slope is defined as a ``run''.}

The segment ABC is a ``run up'' since the slopes AB and BC are both positive.  Likewise, GHI is a 
``run down''.  CDEF is also down because all slopes are negative except DE, which is zero.  IJ can 
be part of GHIJ or IJK.  For most floating point data there will be few points that exactly equal 
their neighbors.  If we only consider the sign of the slope we get the sequence
$$
+\ +\ -\ 0\ -\ +\ -\ -\ 0\ +
$$
Regarding the first 0 as a $`-$' and the second 0 as a $`+$', we find five runs: three of $`+$' and two of $`-$'.  Then, the $U$-test is directly 
applicable.  Again, we need more than 10 occurrences of each type to use the normal distribution 
approximation introduced earlier.  It is clear that one can apply runs test by converting data to a binary series
by almost any method, provided the hypothesis tested reflects the \emph{dichotomizing} method.  A 
common technique is to dichotomize a series by removing the median or mean value and look 
for randomness of runs about the central location. 
\begin{example}
Consider density measurements of ore samples across a magnetite body.  We 
want to know if the densities vary randomly about the median or if a trend is present.  The data 
are given in Table~\ref{tbl:runtest}.
\begin{table}[h]
\center
\begin{tabular}{|c|c|c|c|c|c|c|c|c|c|} \hline
3.57 & 3.63 & 2.86 & 2.94 & 3.42 & 2.85 & 3.67 & 3.78 & 3.86 & 4.02 \\ \hline
 -   &  -   &  -   &  -   &  -   &  -   &  -   &  -   &  -   &  +   \\ \hline
4.56 & 4.62 & 4.31 & 4.58 & 5.02 & 4.68 & 4.37 & 4.88 & 4.52 & 4.80 \\ \hline
 +   &  +   &  +   &  +   &  +   &  +   &  +   &  +   &  +   &  +   \\ \hline
4.55 & 4.61 & 4.93 & 4.60 & 4.51 & 3.98 & 4.22 & 3.52 & 2.91 & 3.87 \\ \hline
 +   &  +   &  +   &  +   &  +   &  +   &  +   &  -   &  -   &  -   \\ \hline
3.52 & 3.77 & 3.84 & 3.92 & 4.09 & 3.86 & 4.13 & 3.92 & 3.54 &      \\ \hline
 -   &  -   &  -   &  -   &  +   &  -   &  +   &  -   &  -   &      \\ \hline
\end{tabular}
\caption{Density measurements and their signs indicating if larger or smaller than the median density.}
\label{tbl:runtest}
\end{table}
The median density is found to be 3.98.  We subtract this value and store the signs of the 
deviations below the corresponding density.  We observe $U = 7$, with $n_1 = 19$, $n_2 = 20$.  For these numbers, 
\index{Test!U}
\index{U test}
\begin{equation}
\bar{U} = 20.5, \ \ s_U = 3.1,
\end{equation}
and we obtain $z = -4.4$.  Because our observed $U$ is far outside both the 95\% and 99\% confidence intervals we conclude 
that the variations about the median are not random but systematic.
\end{example}
There are of course many 
more variants of the run test shown here.  In general, such tests are \emph{nonparametric} in that they 
do not require the underlying distribution to be known to us.
\index{Run test|)}

\section{Autocorrelation}
\index{Autocorrelation|(}

	The main purpose of time series analysis is to take the order of the observations into account 
and try to learn the properties of the data set, such as discovering any periodicities, trends, or repeating patterns, and 
then use such characteristics to infer something about the process being observed.  Repetitions and 
other patterns in a sequence can be found by computing a measure of the ``self-similarity'' of the 
sequence, that is, to what extent a piece of the sequence looks like another piece of the same sequence.  One such 
measure is known as the \emph{autocorrelation}.
     
In Section~\ref{sc:cc} we discussed the correlation between two variables $x_i$ and $y_i$ and found it 
to be given by 
\begin{equation}
r = \displaystyle \frac{s_{xy}}{s_x s_y},
\end{equation}
where $s_x, s_y$ are the sample standard deviations and $s_{xy}$ the sample covariance, given by
\begin{equation}
s_{xy} = \frac{\displaystyle \sum^n_{i=1} (x_i - \bar{x})(y_i - \bar{y})}{n-1} = \frac{\displaystyle  \sum^n_{i=1} x_i y_i - n \bar{x} \bar{y}}{n-1}.
\end{equation}
The concept of the autocorrelation is to let both $x_i$ and $y_i$ be the same signal $y_i$ and then compare 
these two, identical, time-series.  Of course, with 
$x_i = y_i$ we find
\begin{equation}
s_{yy} = \frac{\displaystyle \sum^n_{i=1} y^2_i -  n\bar{y}^2}{n-1} = s^2 _y
\end{equation}
and
\begin{equation}
r =\frac{ \displaystyle  s^2_y}{s_y s_y} = 1,
\end{equation}
meaning the signal is perfectly correlated with itself.

\PSfig[H]{Fig1_Autocorrelation}{Autocorrelation is the correlation between a time-series and its
identical clone at different lags, $\tau$.}

To get more useful information from the autocorrelation we will shift all values in the second 
copy one step to the left (Figure~\ref{fig:Fig1_Autocorrelation}).  So, instead of having the covariance
be composed of the terms
\begin{equation}
s_{yy}(0) \propto  y_1 \cdot y_1 + y_2 \cdot y_2 + \ldots + y_n \cdot y_n
\end{equation}
we will instead have
\begin{equation}
s_{yy}(1) \propto y_2 \cdot y_1 + y_3 \cdot y_2 + \ldots + y_n \cdot y_{n-1}.
\end{equation}
We then get
\begin{equation}
s_{yy}(1) = \frac{\displaystyle  \sum^n_{i=2} y_i y_{i-1} - \frac{1}{n -1}\displaystyle  \sum^n_{i=2} y_i \sum ^n _{i=2} y_{i- 1}}{n-2}.
\end{equation}
Shifting one step further will give a different result.  In general, if we shift the second sequence 
$\tau$ steps relative to the first sequence we find the series' \emph{autocovariance}:
\begin{equation}
\index{Lag in autocorrelation}
\index{Autocorrelation!lag}
\index{Autocovariance}
\begin{array}{c}
s_{yy}(\tau) = \frac{\displaystyle \sum^n_{i=1 + \tau} y_i y_{i-\tau} -
\frac{1}{n-\tau}
\displaystyle \sum ^n_{i=1 + \tau} y_i \displaystyle \sum^n_{i = 1 + \tau} y_{i-\tau}}{n - \tau - 1} = \\[9pt]
\frac{(n-\tau) \displaystyle \sum ^n_{i=1+\tau} y_i y_{i - \tau} -
\displaystyle \sum^n_{i=1 + \tau} y_i \sum^n_{i = 1 + \tau} y_{i-\tau}}{(n - \tau)(n-\tau - 1)},
\end{array}
\end{equation}
where we call the number of shifts, $\tau$, the \emph{lag}.  It is assumed throughout our time-series 
discussion that the sequences are evenly spaced with spacing $\Delta t$ and contain $n$ points, so that the 
length of a sequence is $T = \Delta t(n-1)$.  Figure~\ref{fig:Fig1_Lag} illustrates the situation for a certain lag.

Computing the autocovariance for all lags from $\tau = 0$ to about $\tau = n/4$ results in
the autocovariance function $s_{yy}(\tau)$.  This function will tell us if the
sequence exhibits self-similarity 
and how much we must shift it (i.e., what is the lag) to reach a maximum in the autocovariance.
 
\PSfig[H]{Fig1_Lag}{Autocorrelations can be high for large lags $\tau$ if the data have ``repeating'' features.}

Plotting the autocovariance function for our sequence gives an \emph{autocovariogram},
illustrated in Figure~\ref{fig:Fig1_AC}.
\index{Autocovariogram}

\PSfig[H]{Fig1_AC}{The autocovariogram for a typical time-series.  At zero lag we obtain the variance
of the time-series.}

As was the case with the covariance for a set of paired values, the autocovariance depends on the 
units of the data, which makes it less useful for comparison purposes.  Again, the solution is to 
normalize it by the variance of the sequence. We find the variance to be given by
\begin{equation}
s^2_{y} = \frac{\displaystyle \sum^n_{i=1+\tau} (y_i - \bar{y})^2}{n - \tau - 1}.
\end{equation}
If we assume the mean and variance remain unchanged by the lag $\tau$, we obtain
\begin{equation}
r_\tau = \frac{(n-\tau ) \displaystyle \sum^n_{i=1+\tau} y_i y_{i-\tau} - \sum^n_{i=1+\tau} y_i \cdot \displaystyle \sum^n_{i=1+\tau} 
y_{i -\tau} }
{(n-\tau )(n-\tau - 1) \left[ \frac{1}{(n - \tau - 1)} \left ( \displaystyle \sum^n_{i=1+\tau} y^2_i - (n - \tau ) \bar{y}^2 \right) \right ]}
\end{equation}
which reduces to
\begin{equation}
r_\tau = \frac{ \displaystyle \sum^n_{i=1+\tau}y_i y_{i-\tau} - (n-\tau ) \bar{y}^2}
{ \displaystyle\sum^n_{i=1+\tau } y^2 _i - (n - \tau )\bar{y}^2}.
\end{equation}
The effect of normalizing by the variance is to obtain the \emph{autocorrelation} which only takes on 
values in the $[-1,+1]$ range.  This \emph{autocorrelogram} for our sequence remains unchanged in shape 
but now has a maximum of $+1$ for zero lag, i.e., the series is in perfect correlation with itself.

     Since it is arbitrary if we consider one copy of the sequence shifted by $-\tau$ or the other by $+\tau$, 
the autocorrelation function is symmetric about zero lag, i.e.
\begin{equation}
r_{-\tau} = r_\tau.
\end{equation}
The autocorrelogram can be used to reveal characteristics of a time series.  Commonly, one would 
like to compare the observed correlogram to predicted autocorrelograms for simple models or 
processes.  The simplest of all models is the one in which successive observations are (1) 
independent and (2) normally distributed.  Since each observation $y_i$ is independent of any other 
observation $y_{i-\tau}$ we expect 
\begin{equation}
r_\tau = \left \{ \begin{array}{cc}
1, & \tau = 0\\
0, & \mbox{elsewhere}
\end{array} \right . 
\end{equation}
\index{Autocorrelation!white noise}
\index{White noise}
\noindent
The expected autocorrelation for a totally random process (also called 
\emph{white noise}) is zero, with a variance of $\sigma^2 = 1/n$, when $n > 30$ (Figure~\ref{fig:Fig1_whitenoise}).
\PSfig[h]{Fig1_whitenoise}{White noise and its autocorellogram.  White noise is completely uncorrelated,
thus the expected value of correlation is zero for nonzero lags.  The finite length of a time-series
leads to departures from this theoretical prediction, and the gray band indicates expected variation
from zero correlation for a 95\% confidence level.}

	Stationary time-series with short term correlations will typically have an autocorrelogram where the first few 
coefficients are significantly nonzero whereas the remainder are close to zero.  If the time series 
has a tendency to alternate direction at every step, then the autocorrelogram will alternate too, with $r_1 < 0$.  If our time-series
is nonstationary (i.e., it includes a trend), then $r_\tau$ will not approach zero except for very 
large values of the lag.  Other correlations tend to be completely masked by the tendency for an 
observation to be systematically larger (or smaller) than its predecessor.  It is therefore always 
necessary to remove linear trends from time-series prior to analysis of $r_\tau$.  Strict periodicities in 
the data will be mimicked in the correlogram: A signal $y_t = A \cos \omega t$ will have an autocorrelation 
of $r_\tau \sim \cos \omega \tau$ (for large $n$).
	
Outliers can wreak havoc on the estimation of autocorrelation coefficients.  This is not surprising since estimating the
correlation is an L$_2$ process.  It is therefore important to suppress any outliers, using robust 
statistical methods, to insure good and stable coefficients.
\index{Autocorrelation|)}

\section{Cross-Correlation}
\index{Cross-correlation|(}

	Rather than using two identical series, an obvious extension of the autocorrelation method is to
compare two \emph{different} time-series at various lags.  From such an undertaking we would expect to learn 
two things: 1) The strength of the relationship between the two series, and 2) the lag that 
maximizes the correlation.  This process is called \emph{cross-correlation}, and it differs from 
autocorrelation in several ways:
\begin{enumerate}
\item It may not be possible to specify the zero lag position, unless the two series share a common 
origin and scale.
\item The cross-correlation will in general be asymmetric.
\item The two series may be of different length.
\end{enumerate}
The correlation coefficient for the match position $\tau$ (relative to an arbitrary origin, unless the series have a 
common origin) is simply
\begin{equation}
r_{\tau} = \frac{(n-1) \displaystyle \sum ^n_{i=1} x_i y_i - \sum^n_{i=1} x_i \sum ^n _{i-1} y_i}
{\sqrt{ \left ( (n-1) \displaystyle \sum ^n_{i=1} x_i ^2 - \left (\displaystyle \sum ^n_{i=1} x_i \right )^2 \right ) \left ( (n-1)
\displaystyle \sum ^n_{i=1} y_i^2 - \left ( \displaystyle \sum ^n_{i=1} y_i \right )^2 \right )  }},
\end{equation}
where $x_i$ and $y_i$ are the two series and the sum over $n$ represents the point pairs that overlap
for this particular lag position $\tau$ (Figure~\ref{fig:Fig1_CCLag}). One difference with 
the expression for the autocorrelation is that the denominator will depend (and thus varies) with $n$, while for 
the autocorrelation we used the variance for the entire chain.  
For this reason the cross-correlation is somewhat less stable.  A simple $t$-test for the correlation $r_{\tau}$
to determine significance can be obtained by calculating
\begin{equation}
	t = r_{\tau} \sqrt{\frac{n-2}{1-r_{\tau}^2}}
\end{equation}
and determine if the observed $t$ exceeds critical $t_{\alpha/2,n-2}$ as in a standard, two-sided $t$-test.

\PSfig[h]{Fig1_CCLag}{Cross-correlation between two separate time-series for an arbitrary lag.}

Cross-correlation as defined here is most useful when the two signals have a common origin and 
time scale so that zero lag can be identified.  While $\tau = 0$ always gives $r_\tau = 1$ for autocorrelation, 
$r_0$  may be zero for cross-correlation if one signal is delayed with respect to the other.   After 
compiling the cross-correlation for all lags (positive and negative) we may find that the 
correlation is maximized for a particular lag $\tau _m$. If the correlation is significantly  nonzero we may 
draw the conclusion that there is a direct correlation between the two series, but this effect is 
\emph{delayed} by some time $\Delta t \cdot \tau_m$.   Examples of data pairs that may exhibit 
such cross-correlation include
\begin{enumerate}
\item	Time series of the amount of water injected into a well and the intensity of seismicity.  Such 
a cross-correlation demonstrates the importance of pore-pressure on the friction on faults.  
The increased pore-pressure reduces the effective normal stress on the fault and allows 
for more earthquakes to occur after accounting for the delaying effect of groundwater flow.

\item Glacial loading histories and land emergence since the end of the ice age.  The viscous properties
of the mantle retard the isostatic response and produce a delay between ice melt and land emergence.

\item	Any process in which the input signal $x(t)$ is delayed and gives output $y(t)$.  The optimal 
lag $\tau$ in the cross-correlation between $x$ and $y$ will tell us something about  the process that 
caused the delay.  This could be flow through a permeable medium, viscous response of 
the mantle, the inelastic response of the solid Earth to tidal forces, etc.
\end{enumerate}
\index{Cross-correlation|)}

\section{Geologic Correlation}
\index{Geologic correlation|(}
\index{Correlation!geologic|(}

	The automated correlation of geological quantities quickly runs into trouble because
cross-correlation techniques require a constant and common time scale for the two time series, which is 
often not the case.  Depending on the particular nature of the data sets, systematic 
distortions such as variable sedimentation rates for sedimentary sequences in drill holes and variable 
seafloor spreading rates for magnetic anomalies will render cross-correlation problematic.  Also, at other times the 
two series may have nominal values such as lithologic states and therefore cannot be assigned a numerical 
value.  The extensions of autocorrelation and cross-correlation techniques to deal with nominal 
data are named \emph{autoassociation} and \emph{cross-association}, respectively.
Correlation of such data can be enumerated by \index{Auto-association}\index{Cross-association}sliding
the two data sets by one another and counting the number of matching states, e.g., 
sandstone at the same position as sandstone, and divide the result by the number of comparisons.  
Plotting this ratio $r_\tau$ as a function of match position $\tau$ may reveal a preferred location where the 
match is optimal.  
\begin{table}[h]
\center
\begin{tabular}{cccccccccccccccccc}
S & S & C & C   & L &  L  & C & L &  S  &  S  & S & C & S &  C &  C &  L  & L & \\
  &   &   & $|$ &   & $|$ &   &   & $|$ & $|$ &   &   &   & $|$ &   & $|$ &   &  \\   
  &   &   & C   & C &  L  & L & S &  S  &  S  & C & S & C &  C &  L &  L  & S & S  \\
\end{tabular}
\label{tbl:cross_assoc}
\caption{Two lithologic sequences of sandstone (S), clay (C), and limestone (L) are compared using the cross-association 
technique.  For the match position shown, there are six exact matches out of 14 possible, so $r_\tau = 6/14 = 0.43$.}
\end{table}

Simple binomial probability theory may then be used to test if this ratio is significant or if it is what one can 
expect from two random sequences of the same composition.
\index{Geologic correlation|)}
\index{Correlation!geologic|)}

\clearpage
\section{Problems for Chapter \thechapter}

\begin{problem}
The stratigraphic column below represents the lithologic successions in a sequence taken 
from a delta plain where we find the lithologies sandstone (light blue), siltstone (light gray), clay (light green), 
and coal (black).  Examine and record the transitions between lithologies every meter.
\PSfig[H]{Fig1_MarkovProblemSet1}{Observed stratigraphic section for Problem~\thechapter.\theproblem.  Distances are in meters.}

\begin{enumerate}[label=\alph*)]
\item What is the transition frequency matrix, $\mathbf{A}$, for this sequence?

\item Determine the fixed probability vector, $\mathbf{f}$.

\item Evaluate the transition probability matrix, $\mathbf{P}$.

\item At the 95\% level of confidence, are the transitions random?

\item $\mathbf{S = P \cdot P}$ gives the second order Markov transition matrix.  At the 95\% level, are there significant 
second-order properties in the sequence?
\end{enumerate}
\end{problem}

\begin{problem}
	Same questions as above, but this time using the transitions between just three lithologies:
	mudstone (light blue), siltstone (beige), and coal (black).
	\PSfig[H]{Fig1_MarkovProblemSet2}{Observed stratigraphic section for Problem~\thechapter.\theproblem.  Distances are in meters.}
\end{problem}

\begin{problem}
	The file \emph{embedded.txt} contains the embedded Markov chain transitions between four lithologies:
	A = shale with fossils, B = siltstone, C = sandstone, and D = coal.  Determine if there is evidence
	for a first-order cyclicity at the 95\% level of confidence using the test for embedded Markov chains.
\end{problem}

\begin{problem}
The file \emph{aso.txt} lists the years the Japanese volcano Aso has erupted during the period 1229--1962.  
Use a Kolmogorov-Smirnov test to determine (at the 95\% level of confidence) whether the 
events are uniformly distributed over the time period. Plot the two cumulative distributions involved in the test.
\end{problem}

\begin{problem}
The file \emph{GK2007.txt} lists all magnetic reversals and the duration of each chron from the last 155 million year.  
Use a series of events test (at the 95\% level of confidence) to determine whether the 
reversals are uniformly distributed over the time period.
\end{problem}

\begin{problem}
The data set \emph{limestone.txt} contains the thickness of successive limestone beds in the Lower Jurassic
from a formation in Wales.   Using the run test, is there a pattern in this sequence of thicknesses?
\end{problem}

\begin{problem}
\newcounter{Vostok}
\setcounter{Vostok}{\theproblem}
The 3-km long Vostok ice core from Antarctica resolves temperature variations relative to the present via
oxygen isotopes. These data are given in table \emph{vostok.txt}, which contains equidistant depths (in meter), the
corresponding times (in year), and the relative change in temperature (in $^{\circ}$C).
\begin{enumerate}[label=\alph*)]
\item Since the autocorrelation calculation requires an equidistant interval we must compute the autocorrelation
of the temperature changes as a function of depth.
At what lag $> 0$ is the autocorrelation maximized?  What does this lag represent and how is it related to
time?
\item Because of compaction, depth is not a good proxy for time, especially for the deeper (older) sections.
To analyze the temporal periodicities we thus need an equidistant time-series. Use MATLAB's \texttt{spline}
or other software to interpolate the data onto an
equidistant interval in time ($\Delta t = 25$) and compute the autocorrelation
of the resampled time-series.  How do the two autocorrelations differ? What period would you now select for the
dominant periodicity?
\end{enumerate}
\end{problem}

\begin{problem}
We will be revisiting Problem~\theConradchap.\theConrad\, so make sure you solve that problem first.
\begin{enumerate}[label=\alph*)]
\item Determine the residual bathymetry $r(x)$ and calculate their first differences, $n(x)$, using
MATLAB's \texttt{diff} operator, and plot these values versus distance.
\item Compute the autocorrelation of $n(x)$ for lags $\tau = 0$ through 5 (using MATLAB's \texttt{xcorr} function).
The 99\% confidence interval for white noise is known to be $\pm 3/\sqrt{n}$.  For these lags, are the data
compatible with a null hypothesis that states $n(x)$ may be considered white noise?
\end{enumerate}
\end{problem}

\begin{problem}
	An engineer generates a two-pulse, noisy signal that he sends through a ``black box'' filtering operator.
The experimental setup records the time (in seconds) and both the input and output magnitudes, reproduced in file
\emph{blackbox.txt}.  The black box filter seems to both smooth the output and delay it in time.
Use the cross-correlation technique to determine the lag induced by the filter.
\end{problem}

\begin{problem}
Two stratigraphic sections (A and B) separated by a few hundred meters have been obtained (Figure~\ref{fig:Fig1_crossassoc}).
Use the cross-association technique to determine for what shift the second section B best fits the first section A.
Plot your calculated match ratios $r_\tau$ versus the vertical offset of B relative to A and find the best fit.
Draw your new tie-lines between the various layers.
\PSfig[H]{Fig1_crossassoc}{Observed stratigraphic sections for Problem~\thechapter.\theproblem.}
\end{problem}
