%------------------------------------------
%	HW.tex
%
%	Dummy template for homework assignment via Latex/PDF
%  You simply copy paste the actual problem sets and/or
%  add additional questions to this file and run
%  pdflatex HW.txt
%  which generates the HW.pdf product.  Rename as you
%  see fit.
%
%	Copyright 1992-2023.
%	Paul Wessel
%------------------------------------------
%
% This almost empty document includes all the actual
% files with material.  It is the master
% file that sets up the general layout of the book

% EDITING STARTS AROUND LINE 100

\ifdefined\mypdfbook
   \documentclass[10pt,letter]{report}
   \usepackage[margin=1in]{geometry}
\else
   \documentclass[10pt,oneside,openany,letter]{book}
   \usepackage[margin=0.25in]{geometry}
\fi

\usepackage{ifpdf}
\usepackage{amsmath}
\usepackage{epsfig}
\usepackage{makeidx}
\usepackage{float}
\usepackage{times}
\usepackage{mathptm}
\usepackage{afterpage}
\usepackage{xcolor}
\usepackage{enumitem}
\usepackage{epigraph}
\usepackage{url}
\usepackage{bm}
\usepackage{wrapfig}
\usepackage[labelfont=bf]{caption}
% Special commands & environments for this book
%
%	PSfig will insert an eps file, add a label, and set a caption
%
\ifpdf
   % Here for creating PDF output with pdfLaTeX
   \usepackage{hyperref}
   \DeclareGraphicsExtensions{.pdf}
   \newcommand{\PSfig}[3][tbp]{\begin{figure}[#1] \centering \epsfig{figure=pdf/#2} \caption{#3} \label{fig:#2} \end{figure}}
   \newcommand{\PSfigplace}[1]{\epsfig{figure=pdf/#1.pdf}}
   \newcommand{\PSfignarrow}[3]{\begin{wrapfigure}{#1}{0.5\textwidth} \setlength\intextsep{0pt} \centering \epsfig{figure=pdf/#2} \caption{#3} \label{fig:#2} \vspace{-12pt} \end{wrapfigure}}
   \pdfcompresslevel=9
   \hypersetup{%
      pdfauthor={Paul Wessel},
      pdftitle={INTRODUCTION TO STATISTICS \& DATA ANALYSIS},
      pdfsubject={Introduction to Statistics \& Data Analysis},
      pdfkeywords={DATA ANALYSIS, statistics, hypothesis testing, modeling},
      pdfcreator={pdfLaTeX},
      bookmarksopen=true,
      bookmarksnumbered=true,
      hypertexnames=true,
      breaklinks=true,
      %pdfstartview={FitH},
      %linkbordercolor={1 1 0},
      %urlbordercolor={1 0 0},
   }%
\else
   % Here for creating PS or HTML output with LaTeX
   \usepackage[margin=0.5in]{geometry}
   \DeclareGraphicsExtensions{.svg}
   \newcommand{\PSfig}[3][tbp]{\begin{figure}[#1] \centering \epsfig{figure=svg/#2.svg} \caption{#3} \label{fig:#2} \end{figure}}
   \newcommand{\PSfigplace}[1]{\epsfig{figure=svg/#1.svg}}
   %\DeclareGraphicsExtensions{.png}
   %\newcommand{\PSfig}[3][tbp]{\begin{figure}[#1] \centering \epsfig{figure=png/#2.png} \caption{#3} \label{fig:#2} \end{figure}}
   %\newcommand{\PSfigplace}[1]{\epsfig{figure=png/#1.png}}
\fi

%%------------ THESE COMMANDS TO BE USED FOR POSTSCRIPT --------
\newcommand{\DS}{$^{\circ}$}
\newcommand{\DSm}{^{\circ}}
\newcounter{example}[chapter]
\newcounter{problem}[chapter]
\renewcommand{\bf}{\textbf}
%\newcommand{\important}[1]{\emph{#1}\index{#1}}

\newenvironment{example}{\stepcounter{example}\\\noindent\rule{\textwidth}{.01in}\\\textbf{Example~\thechapter--\theexample}.}{\\\noindent\rule{\textwidth}{.01in}\linebreak}
\newenvironment{example2}{\stepcounter{example}\noindent\rule{\textwidth}{.01in}\\\textbf{Example~\thechapter--\theexample}.}{\\\noindent\rule{\textwidth}{.01in}\linebreak}

\newenvironment{problem}{\stepcounter{problem}\noindent\\\textbf{Problem~\thechapter.\theproblem}.}{}
\DeclareMathOperator\erf{erf}
\DeclareMathOperator\sinc{sinc}

%%------------ THESE COMMANDS TOBE USED FOR HTML ---------
%\newcommand{\DS}{$^{o}$}%HTMLGMT

%--------------------------------------------------------------------------

% REPLACE THE PROBLEMS 9.3 and 9.7 with the equivalent source code for the problems you want
% ALSO UPDATE the HW xx string to the homework number

\begin{document}
	\begin{center} \bf{HW xx} \end{center}
		
	\bf{Problem 9.3)} \\

	A geologist surveys a large section of an exposed batholith and measures the strike 
	direction of numerous vertical joints.  The file \emph{joints.txt} contains the azimuths (measured from north toward east) 
	of these orientations.

	\begin{enumerate}[label=\alph*)]
	\item At the 95\% level of confidence, is there a preferred orientation in the data?  If so, what is the 
	preferred orientation?

	\item Regional tectonic considerations seem to favor a general extensional stress regime in the west-northwest
	-- east-southeast orientation.  Is this explanation for the joints consistent with the data 
	at the 95\% level of confidence?

	\item 50 km further north another exposure of the batholith reveals additional joints (\emph{morejoints.txt}).  Are 
	they randomly oriented?

	\item Do these new joints deviate significantly from the preferred orientation you determined for the first site?
	\end{enumerate}

	\noindent
	\bf{Answer)}
	\begin{enumerate}[label=\alph*)]
	\item  The mean direction of the data is $112.0\DSm$,
	with a mean resultant length $R = 0.9486$.  Given $n = 19$ and $\alpha = 0.05$ we use Table~A.17
	to find $R_{\alpha,n} = 0.394$.  Thus,
	this direction is significant at the 95\% level of confidence.

	\item The $R$ value corresponds to $\kappa = 10.031$ (via Table~A.16 by linear interpolation),
	which yields a standard error $s_e = 4.26\DSm$
	that we must divide by 2 to get the single-angle error.  For
	$\alpha = 0.05$ we find the confidence interval is $112.0\DSm\pm4.2\DSm$. The WNW-ESE direction of $112.5\DSm$
	means we would expect extensional failures to form at $90\DSm$ to that trend, i.e., $22.5\DSm$, which is outside
	the confidence interval on the mean direction of the joints.  Thus, we reject WNW-ESE
	extension as a possible origin for our joints.

	\item The similar analysis of this data set ($n = 18$) reveals a mean direction of $126.8\DSm$ with a mean resultant
	length $R = 0.9453$; again this is highly significant at the 95\% confidence level ($R_{\alpha,n} = 0.405$ per Table~A.17).

	\item The pooled estimate yields $R_p = 0.9154$ and $\kappa = 6.23$ (via Table~A.16 by linear interpolation);
	applying the $F$-test
	indicates that the observed $F = 22.1$ which exceeds critical $F_{1,35,0.05} = 4.1$ (Table~A.5) and
	we must reject the hypothesis that the two directions are the same.

	\end{enumerate}

	\bf{Problem 9.7)} \\

	Scientific drilling in the Caribbean Sea on-board the \emph{Glomar Challenger} resulted in a set of paleomagnetic data (reproduced in \emph{glomar.txt}).
	\begin{enumerate}[label=\alph*)]
	\item Determine if these data (declination clockwise from north, inclination downward from horizontal) have a
	preferred direction at the 95\% level of confidence.
	\item Compare this mean paleomagnetic vector to the present field
	at this site (declination = 354.3\DS and inclination = 46.4\DS).
	Do these two directions differ significantly at the 95\% level of confidence?
	What does your analysis suggest for the plate tectonic motions for this region?
	\end{enumerate}

	\bf{Answer)} \\
	\begin{enumerate}[label=\alph*)]
	\item We use the data to find the mean paleomagnetic vector
	$\mathbf{m} = (-0.2092, 0.1029, 0.6400)$, corresponding to a mean declination and inclination of $153.8^{\circ}$ and $39.8^{\circ}$,
	respectively.  We find the mean resultant length $\bar{R} = 0.681$.  Given $n = 12$ we find (via Table~A.19) that
	the critical value $R_{0.05,12} = 0.46$, hence there appears to be a significant direction in this data set.

	\item Based on $n$ and $\kappa \sim 3.09$ (from Table~A.18) we obtain $\delta_{95\%} \sim 23^{\circ}$.
	Converting the present magnetic vector to Cartesian coordinates gives $\mathbf{v} = (0.6862, -0.0685, 0.7242)$, whose
	angle with $\mathbf{m}$ is $62.6^{\circ}$. Since this greatly exceeds $\delta_{95\%}$ we conclude these directions are indeed
	different and suggest the seafloor was formed at a different location in the past.
	\end{enumerate}

\end{document}
