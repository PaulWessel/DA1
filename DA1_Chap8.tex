%  DA1_Chap5.tex
%
\chapter{SPECTRAL ANALYSIS}
\label{ch:spectralanallysis}
\epigraph{``Science is spectral analysis. Art is light synthesis.''}{\textit{Karl Kraus, Writer}}

	In Chapter~\ref{ch:sequences} we were preoccupied with the topic of time-series analysis in the time-domain 
and learned a few things about the autocorrelation and cross-correlation techniques.  
In this chapter, we will take the different perspective of studying the \emph{periodicities} present in a time 
series.
\index{Frequency content}
	At the heart of spectral analysis lies the notion of a signal's \emph{frequency content}.  This concept is 
utilized to decompose an observed signal into simpler components of known shape.  Because many real 
observations in fact contain periodic components that fluctuate in a predictable way (e.g., yearly, monthly, 
daily), it is desirable to use periodic functions as the basic building blocks of the time-series.  
The most obvious choices are the trigonometric functions \emph{sine} and \emph{cosine}.
\index{Fourier!series}
	The use of sines and cosines to approximate data and functions goes back to the early 1700s 
but was given mathematical rigor and extensive treatment by Joseph Fourier\index{Fourier, J.} late in the 18th century.  
Fourier proved that any continuous, single-valued function could be represented by a series of 
sinusoids --- today we know such series by the name \emph{Fourier series}.  Thus, spectral analysis 
involves finding the components of the Fourier series and interpreting the frequency content 
represented by the series.  Spectral analysis also goes by other names, such as frequency analysis and 
harmonic analysis.  Before we get into the details we must review some terminology and basic 
trigonometry.

\section{Basic Terminology}
\PSfig[h]{Fig1_sincos}{The periodic functions cosine and sine are defined as the $x$ and $y$-components of
the counter-clockwise spinning unit vector $r$ as a function of the rotation angle $\phi$.  a)
Spinning vector at a specific time $t_0$, yielding an angle $\phi_0$ and the corresponding $x$ and $y$
values indicated by the dashed lines. b) Over time, these components trace the sine and cosine functions.}

Consider a unit vector that rotates counterclockwise (Figure~\ref{fig:Fig1_sincos}).  The time it takes to
complete one cycle is 
called the \emph{period}, $T$.  The $y$ and $x$ coordinates are then periodic functions of $t$ and are given by the 
sine and the cosine, respectively.  The \emph{radial frequency}, $f$, of the signal is the number of complete 
revolutions per second.  Hence,
\index{Period}
\index{Sinusoid!period}
\index{Sinusoid!frequency}
\index{Sinusoid!wavelength}
\index{Wavelength}
\index{Frequency!radial}
\begin{equation}
x(t) = \cos (2 \pi ft) = \cos (\omega t) \quad y(t) = \sin (2\pi ft) = \sin (\omega t)
\end{equation}
The period $T = 1/f$ has units of seconds per cycle.  Instead of radial frequency we may use the \emph{angular 
frequency}, $\omega = 2 \pi f$, which has units of radians/sec.  For spatial data, the period $T$ corresponds to 
the \emph{wavelength}, $\lambda$, and the angular frequency is referred to as the \emph{wavenumber}, $k = 2\pi/\lambda$.
The \emph{amplitude}, \index{Wavenumber}\index{Frequency!angular}\index{Sinusoid!amplitude}$A$,
of the signal is the length of the radial vector, $r$.

Instead of requiring that the sine curve go 
through zero at an even number of $\pi$, we can shift it horizontally by subtracting a constant $\phi$  
from the argument.
\PSfig[H]{Fig1_phase}{A phase-shifted sine curve (dotted line) is shifted along the $t$-axis.}
The constant $\phi$ is called the \emph{phase} of the signal (Figure~\ref{fig:Fig1_phase}).
It is clear that the cosine and sine are out of phase 
by 90$^\circ$.  Let us assume that a particular time-series has one single periodic component with angular 
frequency $\omega$.  An example of such a series is shown in Figure~\ref{fig:Fig1_onecos},
where we would need to find both $A$ and $\phi$.  Unfortunately, while this model is linear in $A$ it is \emph{nonlinear} 
in $\phi$.  However, using the trigonometric identity for the cosine of a difference between two angles we find
\index{Phase}
\index{Sinusoid!phase}
\PSfig[H]{Fig1_onecos}{Sinusoid with arbitrary phase can be considered a sum of a sine and a cosine, both with zero phase.}
\noindent
\begin{equation}
d(t) = A\cos (\omega t -\phi ) = A [ \cos \phi \cos \omega t + \sin \phi \sin \omega t ] = a \cos \omega t + b \sin \omega t,
\label{eq:sinusoid}
\end{equation}
which \emph{is} linear in both $a$ and $b$.  Thus, instead of finding one amplitude and a phase we instead find the two 
amplitudes $a$ and $b$ for a cosine and sine pair, respectively, each with \emph{no} phase shift.  We may then easily recover the original
parameters $A$ and  $\phi$ using
\begin{equation}
	\index{Sinusoid!amplitude}
	\index{Amplitude!sinusoid}
	\index{Sinusoid!phase}
	\index{Phase!sinusoid}
A = \sqrt{a^2 + b^2 }, \quad \phi = \tan^{-1} b/a.
\end{equation}
	More often than not, the observed signal will contain many different sinusoids of different 
periods, phases, and amplitudes.  We can use the subscript $j$ to indicate the $j$'th component of the 
series.  Perhaps a complete Fourier series for a signal $y(t)$ could therefore be written
\begin{equation}
d(t) =  \sum^\infty_{j=0} a_{j} \cos \omega_{j}t + b_{j} \sin \omega_{j}t,
\end{equation}
where $\omega_j$ represent the various angular frequency components?  The $a_j$ and $b_j$ coefficients could then be found 
by standard least squares techniques.  However, if we try to solve this system for many 
components we quickly run into computational problems.  To avoid this problem we must look 
at \emph{harmonics}.
\subsection{Harmonics}
\index{Harmonics}
\index{Fundamental frequency}
\index{Frequency!fundamental}

\PSfig[h]{Fig1_fundamental}{The fundamental frequency (solid line) has period $T$ and typically represents the length of our data.  Also shown
is the second harmonic (dashed line).}
The \emph{fundamental} frequency of a signal has period $T$ (scaled to $2\pi$; see Figure~\ref{fig:Fig1_fundamental})
and reproduces a full cycle corresponding to the length of the data signal.  Consequently, 
\index{First harmonic}
\index{Second harmonic}
\begin{equation}
f_{F} = 1/T, \quad \omega_{F} = 2 \pi f_{F} = 2 \pi/T.
\end{equation}
The first harmonic is the sinusoid which makes two complete oscillations over the period $T$.  However, 
because $f_1 = 2f_F,$ this first harmonic sinusoid is usually called the \emph{second harmonic} (though 
some people still refer to it as the first harmonic).  Using that notation,
\begin{equation}
\begin{array}{ll}
f_1 = f_F = 1/T & \omega_{1} = 2\pi f_F = 2\pi/T\\
f_2 = 2f_F = 2/T & \omega_{2} = 4\pi f_F = 4\pi/T\\
\vdots & \vdots \\
f_n = nf_F = n/T & \omega_{n} = 2n\pi f_F= 2n\pi/T\\
\end{array}
\end{equation} 
Superposition of harmonics will always produce a new periodic function with period $T$.

\subsection{Beats}
Nearby frequency components can interact in interesting ways.  Consider 
\begin{equation}
d(t) = \cos \omega_1t + \cos \omega_2 t,
\end{equation}	 
where
\begin{equation}
\omega_1 = \omega + \delta \omega \quad \omega_2 = \omega - \delta \omega,
\end{equation}
and $\delta \omega$ is small.  Using trigonometric identities,
\begin{equation}
d(t) = 2 \cos (\delta \omega t)\cos(\omega t).
\end{equation}
\index{Amplitude!modulation (AM)}
\index{Beat (amplitude modulation)}
\index{Modulation!amplitude (AM)}
Thus, the component $\cos (\omega t)$ (with $T = 2 \pi/\omega$) has a slowly varying amplitude according to $\cos 
( \delta \omega t)$, which is the \emph{modulation} function.  This phenomenon is referred to as a \emph{beat} (Figure~\ref{fig:Fig1_AM}).
\PSfig[h]{Fig1_AM}{Graphical representation of a ``beat'' or amplitude modulation.  Modulating the amplitude
of a constant frequency carrier wave is the basis for AM radio, while FM broadcasts use a constant amplitude
carrier wave and modulate the frequency instead.}
As $\delta \omega$ gets smaller, the period of the beat curve gets longer and the phenomenon becomes more 
noticeable. In acoustics, the two frequencies $\omega _1$ and $\omega_ 2$ are often too high to hear but the beat is 
within an audible range.  \emph{Modulation} is the general phenomenon of a sinusoid with a varying 
amplitude (of any functional form).
\index{Modulation!frequency (FM)}
\index{Frequency!modulation (FM)}
\index{AM radio signals}
\index{FM radio signals}

\section{Fitting the Fourier Series}
\label{sec:FFS}
Consider fitting a set of data, $d_i$, using a set of sine and cosines as the basis functions.
At each observation time, $t_i$, our model predictions would be
\begin{equation}
\hat{d_i} = m_0 + m_1 \sin 2 \pi t_i/T + m_2 \cos 2 \pi t_i/T + m_3 \sin 4 \pi t_i/T + m_4 \cos 4 \pi t_i/T + \dots,
\end{equation}
where $T = n \Delta t$ if   $t_i$ are evenly spaced.  We want to interpolate the values exactly, hence at our observations,
\begin{equation}
\begin{array}{c}
m_0 + m_1 \sin 2 \pi t_1/T + m_2 \cos 2 \pi t_1/T + m_3 \sin 4 \pi t_1/T + \dots = d_1\\
m_0 + m_1 \sin 2 \pi t_2/T + m_2 \cos 2 \pi t_2/T + m_3 \sin 4 \pi t_2/T + \dots = d_2\\
\vdots\\
m_0 + m_1 \sin 2 \pi t_n/T + m_2 \cos 2 \pi t_n/T + m_3 \sin 4 \pi t_n/T + \dots = d_n\\
\end{array},
\end{equation}
and this gives us $n$ equations in $n$ unknowns.  Written as a matrix equation,
\begin{equation}
\left [ \begin{array}{ccccc}
1 & \sin 2 \pi t_1/T &  \cos 2 \pi t_1/T & \sin 4 \pi t_1/T & \dots \\
\vdots & \vdots & \vdots & \vdots & \dots \\        
1 & \sin 2 \pi t_n/T & \cos 2 \pi t_n/T & \sin 4 \pi t_n/T   & \dots \\
\end{array} \right ]
\times 
\left[ \begin{array}{c}
m_0 \\
\vdots \\
m_{n-1} \\
\end{array} \right]
=
\left[ \begin{array}{c}
d_1\\
\vdots\\
d_n \\
\end{array} \right] .
\end{equation}
This standard matrix equation, $\mathbf{G\cdot m = d}$, can now be solved for the $n$ coefficients of the sine and cosine 
series (i.e., the Fourier series) in the usual (matrix) way (i.e., $\mathbf{m} = (\mathbf{G}^T\mathbf{G})^{-1}\mathbf{G}^T\mathbf{d}$).
However, because of orthogonality 
relationships between harmonics, the coefficients can be found analytically (this was fist shown 
by Lagrange in the 1800's using just the sine components over a range of $x$ from 0 to $\pi$).
	
First, we rewrite the Fourier series as
\begin{equation}
\hat{d_i} = a_0 + \sum^{\leq n/2}_{j=1} \left[ a_j \cos \frac{2 \pi jt_i}{n \Delta t}
 + b_j \sin \frac{2 \pi jt_i}{n \Delta t} \right],\quad i = 1,n,
\label{eq:thesummation}
\end{equation}
where $\leq n/2$ means the largest whole integer resulting from the division $n/2$.
Thus, the unknown vector $\mathbf{m}$ would now contain the renamed components
\begin{equation}
\mathbf{m}^T = \left[ a_0 \quad a_1 \quad \dots a_ {\leq n/2} \quad b_1 \quad b_2 \quad \dots \quad b_{\leq n/2} \right] .
\end{equation}
The number of data points $n$ may be any integer value, but we will see it makes a minor difference if $n$ is or is not divisible by two.
If $n$ is an even number, then for $j = n/2$ we find the two last terms in (\ref{eq:thesummation}) to be
\begin{equation}
a_{n/2} \cos \frac{2 \pi n t_i}{2n \Delta t} + b_{n/2} \sin \frac{2 \pi n t_i}{2n \Delta t} = a_{n/2} \cos \frac { \pi t_i}{ \Delta t} + b_{n/2} \sin \frac{ \pi t_i}{ \Delta t}.
\end{equation}
Since $t_i = (i-1) \Delta t$, where $i = 1, \cdots , n$, we have
\begin{equation}
a_{n/2} \cos (i-1) \pi + b_{n/2} \sin (i-1) \pi = \left[-1 \right]^{(i-1)} a_{n/2}.
\end{equation}
This is true, since $\sin (i-1) \pi$ = 0 for all $i$, hence the $b_{n/2}$ term drops out and we find
\begin{equation}
\mathbf{m}^T = \left[ a_0 \quad a_1 \quad \dots \quad a_{n/2} \quad b_1 \quad b_2 \quad \dots \quad  b_{(n/2)-1} \right],
\end{equation}
which gives us $1 + (n/2) + (n/2) - 1 = n$ coefficients for $n$ unknowns, and thus a solvable system results.

On the other hand, if $n$ is \emph{odd} then for $j < (n/2) = (n-1)/2$, both the sine and cosine terms remain and we have
\begin{equation}
\mathbf{m}^T = \left[ a_0 \quad a_1 \quad \dots \quad  a_{(n-1)/2} \quad b_1 \quad b_2 \quad  \dots \quad b_{(n-1)/2} \right],
\end{equation}
which again yields $1 + (n-1)/2 + (n-1)/2 = n$ coefficients for $n$ unknowns.  Since, for $j = 0$,
\begin{equation}
a_0 \cos 0 + b_0 \sin 0 = a_0,
\end{equation}
and thus there is never a $b_0$ term.  Using the convention $a_0 = \frac{a_0}{2}$ and $a_{n/2} = \frac{a_{n/2}}{2}$,
the Fourier series may be written as
\begin{equation}
\hat{d_i} = \sum^{ \leq n/2}_{j=0} \left[a_j \cos \frac{2 \pi jt_i}{n \Delta t} + b_j \sin \frac {2 \pi j t_i}{n \Delta t} \right],\quad i = 1,n,\mbox{ with } a_0 = \frac{a_0}{2},\quad a_{n/2} = \frac{a_{n/2}}{2}
\end{equation}	 	
(the $a_{0}/2$ and $a_{n/2}/2$ convention is only a convenience that will become obvious later).  The frequency
\index{Fourier frequency}
\index{Frequency!Fourier}
\index{Fourier!frequency}
\begin{equation}
\omega_j = \frac{2 \pi j}{n \Delta t} = \frac{2 \pi j}{T} 
\end{equation}
is called the $j$'th \emph{Fourier frequency}.  Notice if $n$ is odd, then $j < n/2$, so $\omega_j < \pi/\Delta t$ and hence $\pi/\Delta t$ is \emph{not}
a Fourier frequency, otherwise it \emph{is} a 
Fourier frequency and the sine term is zero.  Also, the highest frequency $f_{n/2} = (n/2)/ (n \Delta t) = 1/(2 \Delta t)$ is called the \emph{Nyquist} 
frequency, which we will return to later.
\index{Nyquist frequency}
\index{Frequency!Nyquist}
\index{Fourier!series orthogonality|(}
\index{Orthogonality!Fourier series|(}
	
Lagrange took advantage  (in a brute force and laborious manner) of the following five
relationships of harmonic components (which are easily shown using the corresponding integral 
relationships):
\begin{equation}
\sum^{n}_{i=1} \cos \omega_{j}t_{i} = \left\{ \begin{array}{cc} 0, & j \neq 0 \\ n, & j = 0\\
\end{array} \right.,
\label{eq:cosonly}
\end{equation}
\begin{equation}
\sum^{n}_{i=1}\sin \omega_{j}t_{i} = 0,
\end{equation}
\begin{equation}
\sum^{n}_{i=1} \cos \omega_{j}t_{i} \cos \omega_{k}t_{i} = 
\left\{ \begin{array}{ll} n/2, & j=k\neq 0,n/2\\n, & j=k=0, n/2\\ 0& j \neq k\\ \end{array} \right.,
\label{eq:coscos}
\end{equation}
\begin{equation}
\sum^{n}_{i=1} \sin \omega_{j}t_{i} \cos \omega_{k}t_{i} = 0,
\label{eq:sincos}
\end{equation}
and
\begin{equation}
\sum^{n}_{i=1} \sin \omega_{j}t_{i} \sin \omega_{k}t_{i} = 
\left\{ \begin{array}{ll} n/2, & j=k\\ 0& j \neq k\\ \end{array} \right. .
\label{eq:sinsin}
\end{equation}
For a proof, consider the integral corresponding to (\ref{eq:cosonly}):
\begin{equation}
\int^{T}_0 \cos \omega_{j} tdt = \frac{1}{\omega_{j}} \int^{T\omega_{j}}_{0} \cos udu = \frac{1}{\omega_{j}} \sin u \left |^{T \omega {j}}_0 = \frac{T}{2 \pi j} \right. \left( \sin 2 \pi j - \sin 0\right) = 0 \left( j \neq 0 \right).
\end{equation}	 
For $j = 0$,
\begin{equation}
\int^{T}_{0} \cos 0tdt = \int^{T}_{0}dt = T.	 
\end{equation}
It is also easy to visualize these relationships.  For instance, see Figure~\ref{fig:Fig1_ortho} for the a graphical proof of (\ref{eq:sincos}).
\PSfig[h]{Fig1_ortho}{Orthogonality of two cosine harmonics drawn as solid ($\omega_1$) and dashed ($\omega_3$) lines,
with green circles and squares as the hypothetical discrete samples.  The products in (\ref{eq:coscos}) are represented by the
product of the red (negative) and blue (positive) signed lengths.  For this pair, you can see that for each red--blue
line there is another of opposite orientation, thus canceling each other and yielding a final sum of zero.}
\index{Discrete Fourier transform|(}
\index{Fourier!discrete transform|(}

Returning to the Fourier series, we have
\begin{equation}
\hat{d_i} = \sum^{ \leq n/2}_{j=0}\left[ a_{j} \cos \omega_{j}t_{i} + b_{j} \sin \omega_{j}t_{i}\right],\quad i = 1,n
\label{eq:fourierseries}
\end{equation}
and once again we want to minimize the misfit between data and model:
 \begin{equation}
E=\sum^{n}_{i=1}e^2_i = \sum^{n}_{i=1}\left[ d_i - \hat{d_i}\right]^2 = \sum^{n}_{i=1}\left[ d_{i}-\sum^{\leq \frac{n}{2}}_{j=0}\left( a_{j} \cos \omega_{j}t_{i} + b_{j} \sin \omega_{j}t_{i}\right)\right]^2.
\end{equation}
Our unknowns are the $n$ coefficients $a_j$ and $b_j$.  Taking the partial derivatives of $E$ with respect to those parameters gives
\begin{equation}
\frac{\partial E}{\partial a_{k}}=2 \sum^{n}_{i=1}\left[d_{i}-\sum^{ \leq n/2}_{j=0} \left(a_{j} \cos \omega_{j} t_{i} + b_{j} \sin \omega_{j} t_{i}\right) \right] \cos \omega_{k} t_{i} = 0, \quad k =0, \dots, \leq \frac{n}{2}
\end{equation}
\begin{equation}
\frac{\partial E}{\partial b_{k}} = 2 \sum^{n}_{i=1}\left[ d_{i}-\sum^{< n/2}_{j=1} \left( a_{j} \cos \omega_{j} t_{i} + b_{j} \sin \omega_{j} t_{i}\right) \right] \sin \omega_{k} t_{i} =0, \quad k = 1, \dots, < \frac{n}{2}.
\end{equation}
Eliminating the factor of 2 and rearranging, we obtain
\begin{equation}
\sum^{n}_{i=1}d_{i} \cos \omega_{k} t_{i} = \sum^{\leq n/2}_{j=0}\left[ a_{j}\sum^{n}_{i=1} \cos \omega_{j} t_{i} \cos \omega_{k} t_i + b_{j} \sum^{n}_{i=1} \sin \omega_{j} t_{i}\cos \omega_{k} t_i \right ],
\label{eq.thecosterm}
\end{equation}
\begin{equation}
\sum^{n}_{i=1}d_{i} \sin \omega_{k}t_{i} = \sum^{< n/2}_{j=1}\left[ a_{j}\sum^{n}_{i=1} \cos \omega_{j}t_{i} \sin \omega_{k}t_{i} + b_{j}\sum^{n}_{i=1} \sin \omega_{j}t_{i} \sin \omega_{k}t_{i} \right] .
\label{eq:dft_n2}
\end{equation}
The  five orthogonality relationships can now be employed.  For $k = 0$, (\ref{eq.thecosterm}) becomes
\begin{equation}
\sum^{n}_{i=1}d_{i}\cos \omega_{0}t_{i} = \sum^{\leq n/2}_{j=0}\left[a_{j} \sum^{n}_{i=1} \cos \omega_{j}t_{i} \cos \omega_{0} t_{i} +b_{j} \sum^{n}_{i=1} \sin \omega_{j}t_{i} \cos \omega_{0}t_{i} \right],
\end{equation}
and since $\omega_{0} = 0$, $\cos \omega _0 t_i$ is unity. Furthermore, we already determined that $b_0 = 0$, thus
\begin{equation}
\begin{array}{c}
\sum^{n}_{i=1}d_{i} = a_{0} \sum^{n}_{i=1} \cos \omega_{0}t_{i} + a_{1}\sum^{n}_{i=1} \cos \omega_{1} t_{i} + a_2 \sum^{n}_{i=1} \cos \omega _2 t_i \\[4pt]
+ \cdots + b_1 \sum^n_{i=1} \sin \omega _1 t_i + b_2 \sum ^n_{i=1} \sin \omega_2 t_i + \cdots .
\end{array}
\end{equation}
Using the relationships (\ref{eq:coscos}) and (\ref{eq:sincos}), we find
\begin{equation}
\sum^n_{i=1} d_i = a_0n + a_1 0+ a_2 0 +  \cdots b_1 0 + b_2 0 + \cdots,
\end{equation}
and with our convention $a_0 = a_0/2$ we obtain
\begin{equation}
\sum^n_{i=1} d_i = \frac{n}{2}a_0 \quad \Rightarrow \quad a_0 = \frac{2}{n} 
\sum^n_{i=1} d_i = 2 \bar{d},
\end{equation}
i.e., the first coefficient $a_0$ is simply twice the mean of the data (a consequence of our convention for $a_0$ 
that includes the factor of 1/2).	
Using the same approach for $k = 1$, we first obtain
\begin{equation}
\begin{array}{c}
\displaystyle \sum^n_{i=1} d_i \cos \omega_1 t_i = a_0 \sum^n_{i=1} \cos \omega _0 t_i \cos \omega _1 t_i +  a_1 \sum^n_{i=1} \cos \omega_1 t_i \cos \omega _1 t_i + a_2 \sum^n_{i=1} \cos \omega_2 t_i \cos 
\omega _1 t_i + \cdots \\*[2ex]
+ b_1 \displaystyle \sum^n_{i=1} \sin \omega_1 t_i \cos \omega_1 t_i + b_2 \displaystyle \sum^n_{i=1} \sin \omega_2 t_i \cos \omega_1 t_i + \cdots.
\end{array}
\end{equation}
Using the orthogonality relationships, we find
\begin{equation}
\displaystyle \sum^n_{i=1}  d_i \cos \omega_1 t_i = a_0 0 + a_1 \frac{n}{2} + a_2 0 + \cdots + b_1 0 + b_2 0 + \cdots .
\end{equation}
Therefore,
\begin{equation}
a_1 = \frac{2}{n} \displaystyle \sum^n_{i=1} d_i \cos \omega_1 t_i.
\label{eq:a1coeff}
\end{equation}
Since the orthogonality relationships are the same for all $k = 1, 2, \cdots < n/2$ as they were for $k = 1$  in 
(\ref{eq:a1coeff}), we find
\begin{equation}
a_k = \frac{2}{n} \displaystyle \sum^n_{i=1} d_i \cos \omega_{k} t_i, \quad k=0, 1, \cdots, < \frac{n}{2}.
\end{equation}
\PSfig[h]{Fig1_FourierFit}{Example of a data set (thin line with connected dots) and two fitted Fourier components.  Here,
we show the least-squares solution for the two harmonics $\omega_3$ (solid line) and $\omega_9$ (dashed line).  When
these and all other harmonics are evaluated they will sum to equal the original data set.}
Finally, we need to look at the last case, $k = n/2$ (for even $n$).  For $k = n/2$,
\begin{equation}
\begin{array}{c}
\displaystyle \sum^n_{i=1} d_i \cos \omega_{n/2} t_i = a_0 \sum^n_{i=1} \cos \omega_{0} t_i
\cos \omega_{n/2} t_i + a_1 \sum^n_{i=1}
\cos \omega_{1} t_i      \cos \omega_{n/2} t_i + a_2 
\sum^n_{i=1} \cos \omega_{2} t_i
 \cos \omega_{n/2} t_i + \cdots \\*[2ex]
+ b_1 \displaystyle \sum^n_{i=1} \sin \omega_1 t_i \cos \omega _{n/2}t_i + b_2 
\displaystyle \sum^n_{i=1} \sin \omega_2 t_i \cos \omega_{n/2} t_i + \cdots.
\end{array}
\end{equation}
Again, using the orthogonality relationships,
\begin{equation}
\displaystyle \sum^n_{i=1} d_i \cos \omega _{n/2} t_i = a_0 0 + a_1 0 + \cdots + na_{n/2} + \cdots + b_1 0 + b_2 0 + \cdots .
\end{equation}
Since we required $a_{n/2} = a_{n/2}/2$, we find
\begin{equation}
a_{n/2} = \frac{2}{n} \displaystyle \sum^n_{i=1} d_i \cos \omega_{n/2} t_i.
\end{equation}
Therefore, with the convention that $a_0 = a_0/2$ and $a_{n/2} = a_{n/2}/2$, and interchanging the dummy subscripts
$k$ and $j$, the $a_j$ can be defined as
\begin{equation}
	\index{Discrete cosine transform}
	\label{eq:DCT}
\boxed{a_j = \frac{2}{n} \displaystyle \sum^n_{i=1} d_i \cos \omega_j t_i, \quad 0 \leq j \leq \frac{n}{2}.}
\end{equation}
We now turn our attention to the other half of the normal equations (\ref{eq:dft_n2}) involving the $\sin \omega _j t_i$  
terms.  Previously, we have shown that $b_0 = b_{n/2} = 0$ so we will only need to look at one case $(k = 
1)$ and generalize the result for all $k$.  We find
\begin{equation}
\begin{array}{c}
\displaystyle \sum^n_{i=1} d_i \sin \omega_{1} t_i = a_0 \sum^n_{i=1} \cos \omega_{0} t_i
\sin \omega_{1} t_i + a_1 \sum^n_{i=1}
\cos \omega_{1} t_i \sin \omega_{1} t_i + a_2 \sum^n_{i=1} \cos \omega_{2} t_i
  \sin \omega_{1} t_i  + \cdots  \\*[2ex]
+ b_1 \displaystyle \sum^n_{i=1} \sin \omega_1 t_i \sin \omega _{1} t_1 +  b_2 
\displaystyle \sum^n_{i=1} \sin \omega_2 t_i \sin \omega_{1} t_i + \cdots.
\end{array}
\end{equation}
Thus,
\begin{equation}
\displaystyle \sum^n_{i=1} d_i \sin \omega_1 t_i = a_0 0 + a_1 0  + \cdots + b_1 \frac{n}{2} + b_2 0 + \cdots,
\end{equation}
and
\begin{equation}
b_1 = \frac{2}{n} \sum^n_{i=1} d_i \sin \omega_1 t_i.
\end{equation}
Since the orthogonality relationships hold for all $k$, we again interchange $k$ and $j$ and find
\begin{equation}
	\index{Discrete sine transform}
	\label{eq:DST}
\boxed{b_j = \frac{2}{n} \sum^n_{i=1} d_i \sin \omega_j t_i, \quad 0 < j < \frac{n}{2}.}
\end{equation}

The formulae for $a_j$ and $b_j$ are called the \emph{Discrete Cosine and Sine Transforms} and combined they
define the \emph{Discrete Fourier Transform}. Figure~\ref{fig:Fig1_FourierFit} shows an
example of a data set and the determination of two Fourier components.
\index{Discrete sine transform}
\index{Discrete cosine transform}
\index{Discrete Fourier transform|)}
\index{Fourier!discrete transform|)}
\begin{example}
Consider the time-series $d = [ 1 \ 0 \ \ \mbox{-2} \  \ \ \mbox{-1} \ \ 1 \ 2 \ 1 \ 0.5 ]^T$   with 
$\Delta t = 1, n = 8, T = 8$.  The Fourier frequencies are therefore
\begin{equation}
\begin{array}{lll}
\omega_j = 2 \pi j /T, & \omega_0 = 0, & \omega_1 = \pi/4, \\
\omega_2 = \pi/2,  & \omega_3 = 3 \pi/4, & \omega_4 = \pi.
\end{array}
\end{equation}
Solving for the coefficients, we find
\begin{equation} \begin{array}{lllll}
a_0 = 0.3125, & a_1 = -0.0884, & a_2 = 0.7508, & a_3 = 0.0804, & a_4 = -0.0625, \\
b_1 = -1.3687, & b_2 = 0.625, & b_3 = 0.1313.
\end{array}
\end{equation}
Thus, our Fourier series is
\begin{equation}
\begin{array}{lll}
d(t)&  = &  \displaystyle 0.3125 - 0.0884 \cos \frac{\pi}{4} t -1.3687 \sin \frac{\pi}{4} t + 0.7508 \cos  \frac{\pi}{2} t\\\\[2pt]
&  & + \displaystyle  0.625 \sin \frac{\pi}{2}t  + 0.0804 \cos \frac{3 \pi}{4}t + 0.1313 \sin \frac{3 \pi}{4} t - 0.0625 \cos \pi t \end{array}.
\end{equation}
\end{example}
Note that all the terms have periods that are multiples of the fundamental frequency; hence the 
Fourier representation must itself be periodic with the same fundamental period $T$.  We will later 
see that the assumption of periodic data may have grave consequences for our coefficients.

\subsection{The power of orthogonality}
Let us pause and lament the demise of our dear friend from Chapter~\ref{ch:matrix}, the design matrix $\mathbf{G}$.
What just happen to it in our analysis?
Well, recall equations (\ref{eq:gdotg}) and (\ref{eq:gdotd}), both overflowing with dot-products of the basis vectors $\mathbf{g}_j$.
Yet, with our choice of harmonics we found that these basis vectors were in fact orthogonal and thus their dot-products yielded
zero except along the matrix diagonal (where we got the simple constant $n/2$).  With $\mathbf{G}$ being diagonal, the formidable $[\mathbf{G}^T\mathbf{G}]^{-1}$
collapsed to the identity matrix $\mathbf{I}$ scaled by $2/n$, as we just witnessed. Now \emph{that} is the power of
orthogonal functions (of which sine and cosine are just two possibilities) and why they are so widely used in data analysis as well as for modeling in the physical sciences.


\index{Fourier!series orthogonality|)}
\index{Orthogonality!Fourier series|)}

\section{The Periodogram}
\label{sec:periodogram}
\index{Periodogram|(}
\index{Spectrum|(}

	We determined that the Fourier series expansion of our observed time-series $d_i$ could be 
written
\begin{equation}
\hat{d_i} = \sum^{\leq n/2}_{j=0} [ a_j \cos \omega_j t_i + b_j \sin \omega_j t_i ].
\end{equation}
Remember that (\ref{eq:sinusoid}) started out by trying to fit a cosine of arbitrary amplitude $A_j$ and phase $\phi_j$, 
but that we could rewrite this single term as a sum of a cosine and sine components with different amplitudes 
and zero phases.  We found
\begin{equation}
a_j = A_j \cos \phi_j, \quad b_j = A_j \sin \ \phi_j.
\end{equation}
From these expressions we readily find a component's full amplitude and phase.  Dividing the 
$b_j$ by $a_j$ gives 
\begin{equation}
\tan \phi_j = b_j / a_j \quad \Rightarrow \quad \phi _j = \tan ^{-1} (b_j / a_j).
\end{equation}
Squaring $a_j$ and $b_j$ and adding them gives
\index{Power spectrum}
\begin{equation}
A^2_j = a^2_j + b^2_j.
\end{equation}
The \emph{periodogram} is constructed by plotting $A_j^2$ versus $j$, $f_j$, $\omega_j$, or $P_j$.  While often called the 
\emph{power spectrum}, it is strictly speaking a raw, discrete periodogram.  The true spectrum is a 
smoothed periodogram showing frequency components of statistical regularity.  However, the 
periodogram is the most common form of output of a Fourier transform.  Figure~\ref{fig:Fig1_periodogram}
shows the periodogram for the function
\begin{equation}
d(t) = \frac{1}{2} \cos \omega_1 t + \frac{3}{4} \cos \omega_2 t + \frac{1}{2} \sin \omega_3 t
	+ \frac{1}{4} \cos \omega_3 t + \frac{1}{3} \cos \omega_4 t + \frac{1}{5} \sin \omega_4 t + \frac{1}{3} \sin \omega_6 t - \frac{3}{5}.
\label{eq:periodogram}
\end{equation}
\PSfig[h]{Fig1_periodogram}{Raw periodogram of the function given in (\ref{eq:periodogram}).  The peak corresponds to the
$A_0^2 = a_0^2$ term defined to be twice the mean (-0.6) squared.}
Let us look, for a moment, at the variance of the time series expansion.  Recall, the variance is 
given by
\begin{equation}
s^2 = \frac{1}{n-1} \sum^n _{i=1} (\hat{d_i} - \bar{d}) ^2.
\end{equation}
We shall write the Fourier series as
\begin{equation}
\hat{d_i} = \bar{d} + \sum_{j=1} ^{\leq \frac{n}{2}} \left (a_j \cos \omega _j t_i + b_j \sin \omega_j t_i \right ),
\end{equation}
by pulling the constant (mean) term out separately.  Since the two means cancel, we find
\begin{equation}
s^2 = \frac{1}{n-1}  \sum^n_{i=1} \left \{ \left [ \sum_{j=1} ^{\leq \frac{n}{2}} \left (a_j \cos \omega_{j} t_i + b_j \sin \omega_{j} t_i \right ) \right ] \left [ \sum_{q=1} ^{\leq \frac{n}{2}} \left (a_q \cos \omega_{q} t_i + b_q \sin \omega_{q} t_i \right ) \right ] \right \}.
\end{equation}
Also recall that, because of orthogonality, all the cross terms $(q \neq j)$ resulting from the full expansion 
of the two squared expressions will be zero when summed over $i$, while the remaining terms will sum to $n/2$ (since $j,q > 0$).
Hence, we are left with
\begin{equation}
s^2 = \frac{n}{2(n-1)} \sum_{j=1} ^{\leq \frac{n}{2}} (a^2_j + b^2_j) \sim \frac{1}{2}\sum_{j=1} ^{\leq \frac{n}{2}} A^2_j.
\end{equation}
Therefore, the power spectrum (periodogram) of $(a_j^2 + b_j^2)$ versus $\omega_j$ is a plot showing the 
contribution of individual frequency components to the total variance of the signal.  For this reason, 
the power spectrum is often called the variance spectrum.  However, most of the time it is simply called ``the 
spectrum.''  Hence, the Fourier transform converts a signal from the time domain to the frequency 
domain (or wavenumber domain), where the signal can be viewed in terms of the contribution of 
the different frequency components of which it is made.  The phase spectrum ($\phi_j$ versus $\omega_j$)
shows the relative phase of each frequency component.  In general, phase spectra are more difficult
to interpret than amplitude (or power) spectra.

\subsection{Aliasing of higher frequencies}
\index{Aliasing}
\index{Nyquist frequency}
\index{Frequency!Nyquist}
	We mentioned before that the highest frequency (or shortest period, or wavelength) that can be 
estimated from the data is called the Nyquist frequency (or period, or wavelength), given by
\begin{equation}
f_N = f_{n/2} = \frac{1}{2\Delta t}, \quad \omega_N = 2\pi f_N = \frac{\pi}{\Delta t}\quad P_{n/2} = 2 \Delta t.
\end{equation}
Higher frequencies, whose wavelengths are less than twice the spacing between sample points 
\emph{cannot be detected}.  However, when we sample a signal every $\Delta t$ and the original signal has 
higher frequencies than $f_{n/2}$, we introduce \emph{aliasing}.  Aliasing means that some frequencies will 
leak power into other frequencies.  This concept is readily seen by sampling a high-frequency 
signal at a spacing larger than the Nyquist interval.

\PSfig[h]{Fig1_aliasing}{Aliasing: A short-wavelength signal that is not sampled at the Nyquist frequency
or higher will instead appear as a longer-wavelength component that does not exist in the actual data.}
Sampling of the high-frequency signal actually results in a longer-period signal (Figure~\ref{fig:Fig1_aliasing}).
When Clint Eastwood's wagon wheels seem to spin backwards in an old Western movie --- that's aliasing:  The 24 
pictures/sec rate is simply too slow to capture the faster rotation of the wheels.
	
\subsection{Significance of a spectral peak}
\index{Test!spectral peak}
In some applications we may be interested in testing whether a particular 
component is dominant or if its larger amplitude is due to chance.  The statistician R. A. Fisher\index{Fisher, R. A.} devised a test that 
calculates the probability that a spectral peak $s_j^2$ will exceed the value $\sigma_j^2$ of a hypothetical time series 
composed of independent random points.  We must evaluate the ratio of the variance contributed by the
maximum peak to the entire data variance:
\begin{equation}
g = \frac{s^2 _j}{2s^2},
\label{eq:computed_g}
\end{equation}
where $s^2_j$ is the largest peak in the periodogram (we divide by two to get its variance contribution)
and $s^2$ is the variance of the entire series.  For 
a prescribed confidence level, $\alpha$, the critical value that we wish to compare to our observed $g$ is
\begin{equation}
g_{\alpha,k} \approx 1 - \exp \left( \frac{\ln \alpha - \ln k}{k-1} \right ),
\label{eq:critical_g}
\end{equation}
with $k = n/2$ (for even $n$) or $k = (n-1)/2$ (for odd $n$).  Should our observed $g$ (obtained via \ref{eq:computed_g}) exceed this 
critical value we decide that the dominant component is real and reflects a true 
characteristic of the phenomenon we are observing.  Otherwise, $s^2_j$ may be large simply by 
chance.

\subsection{Estimating the continuous spectrum}

	The power spectrum or periodogram obtained from the Fourier coefficients is discrete, yet 
we do not expect the power at frequency $\omega_j$ to equal the underlying continuous $P(\omega)$ at exactly 
$\omega_j$, since the discrete spectrum must necessarily represent some average value of power at all frequencies between $\omega_{j-1}$ and 
$\omega_{j+1}$.  In other words, the computed power at $\omega_j$ also represents the power from nearby frequencies 
not among the chosen harmonic frequencies $\omega_j$.  Furthermore, the uncertainty in any individual 
estimate $p^2_j$ is very large; in fact, it is equal to $\pm p^2_j$ itself.

 	Can we improve (i.e., reduce) the uncertainties in $p^2_j$ by using more data points or sample the 
data more frequently?  The unpleasant answer is that the periodogram estimates do not become 
more accurate at all!  The reason for this is that adding more points simply produces power 
estimates at a greater number of frequencies $\omega_j$.  The only way to reduce the uncertainty in the 
power estimates is to smooth the periodogram over nearby discrete frequencies.  This can be 
achieved in one of two ways:

\begin{enumerate}
\item	Use a time-series that is $M$ times longer (so $f_1' = f_1/M$) and \emph{sum} the $M$ power estimates $p^2_k$
straddling each original $\omega_j$ frequency to obtain a smooth estimate $p^2_j = \sum p^2_k$.
\item	Split the original data into $M$ smaller series, find the $p^2_j$ for each series, and take the \emph{mean} of 
the $M$ estimates for the same $j$ (i.e., the same frequency).
\end{enumerate}
\index{Windowing}
In both cases the variance of the power spectrum estimates drop by a factor of $M$, i.e., $s^2_j = p^2_j/M$.
The exact way the smoothing is achieved may vary among analysts.  Several different 
types of weights or spectral \emph{windows} have been proposed, but they are all relatively similar.  These windows 
arose because, historically, the power spectrum was estimated by taking the Fourier transform of 
the \emph{autocorrelation} of the data; hence many windows operated in the lag-domain.  The 
introduction of the Fast Fourier Transform made the FFT the fastest way to obtain the spectrum, 
which then is simply smoothed over nearby frequencies.  The FFT is a very rapid algorithm for 
doing a discrete Fourier transform, especially if $n$ is a power of 2.  It can be shown that one can 
always split the discrete transform into the sum of two discrete, scaled transforms of subsets of the data.  
Applying this result recursively, we eventually end up with a sum of transforms of data sets 
with one entry, whose transform equals itself.  While mathematically equivalent, there is a huge 
difference computationally:  While the discrete Fourier transform's execution time is proportional to $n^2$, 
the FFT only takes $n\cdot \log(n)$.  For a data set of $10^6$ points, the speed-up is a factor of $> 75,000$.

	By doing a Fourier Analysis, we have transformed our data from one domain (time or space) 
to another (frequency or wavenumber).  A physical analogy is the transformation of light sent 
through a triangular prism.  White light is composed of many frequencies, and the prism acts as a 
frequency analyzer that separates the various frequency components, here represented by colors.  Each color band 
is separated from its neighbor by an amount proportional to their difference in wavelength, and 
the intensity of each band reflects the amplitude of that component in the white light.  We know 
that by examining the spectrum we can learn much about the composition and temperature of the 
source and the material the light passed through.  Similarly, examining the power spectra of 
other processes may tell us something about them that may not be apparent in the time domain.
Consequently, spectral analysis remains one of the most powerful techniques we have for examining
temporal or spatial sequences.

\subsection{First-Order Spectrum Interpretation}
\label{sec:firstorderspectrum}
\PSfig[h]{Fig1_spectratypes}{Simplified representations of typical spectra that are called ``white'' (left; equal power at all frequencies),
``red'' (middle; power falling off with increasing frequency), and ``blue'' (right; power increasing with frequency).}
Per Section~\ref{sec:periodogram}, the raw power spectrum, or \emph{periodogram}, is obtained by plotting the squared amplitude $A_j^2$ versus
frequency.  Often, a spectrum will fall into one of three categories (see Figure~\ref{fig:Fig1_spectratypes}):
\begin{description}
	\item [white:] This is a spectrum that shows little or no amplitude variation with frequency.  Random values
	such as independent samples drawn from a normal distribution will have a white spectrum.\index{White spectrum}\index{Spectrum!white}
	\item [red:] This spectrum is dominated by long-wavelength (low-frequency) signals, with the spectrum tapering
	off for higher frequencies.  This is very common behavior in observed data, such as topography and potential fields (gravity, magnetics).
	It may also be indicative of data that represent an integrated phenomenon.\index{Red spectrum}\index{Spectrum!red}
	\item [blue:] This spectrum is dominated by short-wavelength (high-frequency) signal, with the spectrum tapering
	off for lower frequencies.  Data that depend on derivatives, such as slopes and curvatures, might behave this way, being higher-order
	derivatives of a red-spectrum topography signal.\index{Blue spectrum}\index{Spectrum!blue}
\end{description}
One reason for the prevalence of red or blue spectra for natural phenomena can be understood if we consider what effect a temporal derivative (e.g., $d/dt$) has in the frequency domain.  Given that the Fourier series representation of data can be written
\begin{equation}
\hat{d}(t) = \sum_{j = 0}^{\leq n/2} A_j \cos \left (\omega_j t - \phi_j \right ),
\end{equation}
taking the derivative yields
\begin{equation}
\frac{d}{dt}\hat{d}(t) = \sum_{j = 0}^{\leq n/2} -\omega_j \cdot A_j \sin \left (\omega_j t - \phi_j \right ),
\end{equation}
In effect, we multiply each Fourier amplitude by its corresponding frequency, hence amplitudes at higher frequencies are preferentially enhanced while those at lower frequencies are attenuated.  This scaling
will make the spectrum more ``blue''.  By analogy, integration in the temporal domain has the effect of \emph{dividing} the spectrum by the frequency, conversely
``reddening'' the spectrum.  This frequency effect is what we allude to when we say that taking a derivative typically make data noisier (it amplifies
the short-wavelength or high-frequency components in the data) while integration tends to make data smoother by attenuating the same
components.  Of course, these statements assume that the uncertainties in the data are mostly at high frequencies, but some data have more
uncertainty at low frequencies, in which case the situation is reversed.
These simple considerations may be useful when interpreting your observed spectra.  Finally, note that the derivative
also introduces a phase change of $\pi/2$ (90\DS) since the cosine and the negative sine are shifted by 90\DS.  A second multiplication (i.e., for a second-derivative result)
leads to a 180\DS\ change in phase since we are essentially multiplying by $-1$ (this does not affect power, which is proportional to amplitude squared).

\index{Periodogram|)}
\index{Spectrum|)}

\section{Convolution}
\index{Convolution|(}

	\emph{Convolution} represents one of the most fundamental operations of time series analysis and 
is one of the most physically meaningful.  Consider the passage of a signal through a linear filter, 
where the filter (a ``black box'') will modify a signal passing through it
(Figure~\ref{fig:Fig1_blackbox}).  For instance, it may
\begin{enumerate}
\item Amplify, attenuate or delay the signal.
\item Modify or eliminate specific frequency components.
\end{enumerate}
 
\PSfig[H]{Fig1_blackbox}{Example of convolution between an input signal and a filter.}

Consider the propagation of a seismic pulse through the upper layers of the Earth's crust, as illustrated
in Figure~\ref{fig:Fig1_earthfilter}.  The generated pulse may be sharp and thus have high
frequencies, yet the recorded signal that traveled through the crust may be much smoother and
include repeating signals that reflect internal boundaries.
 
\PSfig[H]{Fig1_earthfilter}{Convolving a seismic pulse with the Earth gives a seismic trace that may
reflect changing properties of the Earth with depth.}

Convolution is this process of linearly modifying one signal using another signal.  In Figure~\ref{fig:Fig1_earthfilter} we 
convolved the seismic pulse with the ``Earth filter'' to produce the observed returned seismogram.  
Symbolically, we write the convolution of a signal $d(t)$ by a filter $p(t)$ as the integral
\index{Deconvolution}
\index{Inverse filtering}
\index{Filtering!inverse}
\begin{equation}
h(t) = d(t) * p(t) = \int_{-\infty}^{+\infty} d(u) \cdot p(t-u) du,
\label{eq:convolution}
\end{equation}
where $*$ represents the convolution operator.
\emph{Deconvolution}, or \emph{inverse filtering}, is the process of unscrambling the convolved signal to 
determine the nature of the filter \emph{or} the nature of the input signal.  Consider these two cases:
\begin{enumerate}
\item If we knew the exact shape of our seismic pulse $d(t)$ and seismic signal received, $h(t)$, we could 
deconvolve the data with the pulse to determine the (filtering) properties of the upper layers of the Earth through 
which the pulse passed (i.e., $p(t) = d^{-1}(t) * h(t)$).
\item If we wanted to determine the exact shape of our pulse $d(t)$, we could pass it through a known 
filter $p(t)$ and deconvolve the output with the shape of the filter (i.e., $d(t) = p^{-1}(t) * h(t)$).
\end{enumerate}
The hard work here is to determine the inverse functions $d^{-1}(t)$ or $p^{-1}(t)$, which is akin to matrix inversion.
Other examples of convolution include:
\begin{enumerate}
\item Filtering data --- using running means, weighted means, removing specific frequency components, 
etc.
 \item Recording a phenomenon with an instrument that responds slower than the rate at which the 
phenomenon changes, or which produces a weighted mean over a narrow interval of time,
or which has lower resolving power than the phenomenon requires.
\item Conduction and convection of heat.
\item Deformation and the resulting gravity anomalies caused by the flexural response of the lithosphere
to a volcano.
\end{enumerate}

	Convolution is most easily understood by examining its effect on discrete functions.  
First, consider the discrete impulse $d(t)$ sent through the filter $p(t)$, as illustrated in Figure~\ref{fig:Fig1_conv1}:
\PSfig[H]{Fig1_conv1}{A filter's impulse response is obtained by sending an impulse $d(t)$ through the filter $p(t)$.}
\noindent
The output $h(t)$ from the filter is 
known as the \emph{impulse response function} since it represents the response of the filter to an 
impulse, $d(t)$.  It represents a fundamental property of the filter $p(t)$.
\index{Impulse response function}
Next, consider a more complicated input signal convolved with the filter, as shown in Figure~\ref{fig:Fig1_conv2}:
\PSfig[H]{Fig1_conv2}{Filtering seen as a convolution.}
\noindent
Since the filter is linear, we may think of the input as a series of individual impulses. The output 
is thus the sum of several impulse responses scaled by their amplitudes and shifted in 
time.  Calculating convolutions is a lot like calculating cross-correlations, except 
that the second time-series must be reversed.  Consider the two signals as finite sequences on separate strips of 
paper (Figure~\ref{fig:Fig1_conv3}).
\PSfig[H]{Fig1_conv3}{Graphical representation of a convolution.  We write the discrete values of $d(t)$ and $p(t)$ on two separate strips of paper.}
\noindent
We obtain the zero lag output by aligning the paper strips as shown in Figure~\ref{fig:Fig1_conv4},
after reversing the red strip.
\PSfig[H]{Fig1_conv4}{Convolution, zero lag.  Reverse one strip and arrange them to yield a single overlap.}
\noindent
The zero lag result $h_0$ is thus simply $d_0 \cdot p_0$.  Moving on, the first lag results from the alignment shown in  Figure~\ref{fig:Fig1_conv5}.
\PSfig[H]{Fig1_conv5}{Convolution, first lag.  We shift one strip by one to increase the overlap.}
\noindent
This simple process is repeated, and for each lag $k$ we evaluate $h_k$ as the sum of the products of the overlapping 
signal values.  This is a graphic (or mechanical) representation of the discrete convolution 
equation (compare this operation to the integral in \ref{eq:convolution}).
Consider the convolution of the two functions shown in Figure~\ref{fig:Fig1_conv6}.
 
\PSfig[H]{Fig1_conv6}{Moving averages is obtained by the convolution of data with a rectangular function of unit area.}
\noindent
If we look at this convolution with the moving strips of paper approach, we get the setup illustrated in Figure~\ref{fig:Fig1_conv7}:
 
\PSfig[h]{Fig1_conv7}{The mechanics of convolutions, this time without the paper strips.}
\noindent
Given the simple nature of $p(t)$, we can estimate the values of $h_k$ directly:
\begin{equation}
\begin{array}{rcl}
h_0 & = & d_{0}/5 \\[4pt]
h_1 & = &  \frac{1}{5} (d_0 + d_1)\\
 & & \vdots \\
h_4 & = &  \frac{1}{5} ( d_0 + d_1 + d_2 + d_3 + d_4)\\[4pt]
h_5 & = & \frac{1}{5} ( d_1 + d_2 + d_3 + d_4 + d_5)\\
& & \vdots \\
h_{18}&  = & d_{14}/5 \end{array}
\end{equation}	 

\PSfig[h]{Fig1_conv8}{The final result of the convolution is a smoothed data set since any short-wavelength signal
will be greatly attenuated.}
\noindent
This is simply a five-point running (or moving) average of $d(t)$, and the result is shown in Figure~\ref{fig:Fig1_conv8}.
An $n$-point average would be the 
result if $p(t)$ consisted of $n$ points, each with a value of $1/n$.

\subsection{Convolution theorem}
\index{Convolution theorem}

Although not shown here, it can be proven that a convolution of two functions $p(t)$ and $d(t)$ 
in the time-domain is equivalent to the product of $P(f)$ and $D(f)$ in the frequency domain 
(here, uppercase letters indicate the Fourier transforms of the lowercase, time-domain functions).
The converse is also true, thus
\begin{equation}
\begin{array}{rcl}
p(t) * d(t) & = & h(t) \quad \leftrightarrow \quad P(f) \cdot D (f) = H(f),\\[4pt]
p(t) \cdot d(t) & = & z(t) \quad \leftrightarrow \quad P(f) * D (f) = Z(f).\\
\end{array}
\end{equation}
Because convolution is a slow calculation it is often advantageous to transform our data from one
domain to the other, perform the simpler multiplication, and transform the data back to the original
domain.  The availability of \emph{fast Fourier transforms} (FFTs) makes this approach practical.
\index{Fast Fourier transform (FFT)}
\index{FFT (Fast Fourier transform)}
\section{Sampling Theory}
\index{Sampling!theory|(}
\index{Sampling!theorem}

\index{Sampling!theorem}
\index{Band-limited}
The \emph{sampling theorem} states that if a function is \emph{band-limited} (i.e., the transform is zero for all 
radial frequencies $f > f_N$), then the continuous function $d(t)$ can be uniquely determined from knowledge 
of its sampled values given a sampling interval $\Delta t \leq  1/(2 f_N)$.  From distribution theory, we have
\begin{equation}
d_t = \sum^{+\infty} _{j= - \infty} d(t) \delta (t-j \Delta t) = \sum^\infty _{j= - \infty} 
d(j \Delta t) \delta ( t - j \Delta t) = d(t) \cdot \Delta (t),
\end{equation}
where
\index{Sampling!function}\index{Comb function}
\begin{equation}
\Delta (t) = \sum^{+\infty }_{j= - \infty} \delta ( t - j \Delta t)
\end{equation}
is the sampling or ``comb'' function in the time domain (Figure~\ref{fig:Fig1_sampl1}).
Thus, $d_t$ is the continuous function $d(t)$ sampled at the discrete times $j\Delta t$.
Consequently, it is true that the original signal $d(t)$ can be reconstructed exactly from
its sampled values $d_t$ via the \emph{Whittaker-Shannon} interpolation formula\index{Whittaker-Shannon interpolation}\index{Interpolation!Whittaker-Shannon}
\begin{equation}
	d(t) = \sum_{j=-\infty}^{+\infty} d_j \sinc \left( \frac{t - j\Delta t}{\Delta t} \right),
	\label{eq:WhittakerShannon}
\end{equation}
where $d_j = d(j\Delta t)$ are the sampled data values and the $\sinc$ function\index{$\sinc$ (sinc function)} is defined as
\begin{equation}
	\sinc(x) =  \frac{\sin \pi x}{\pi x}.
	\label{eq:sincfunction}
\end{equation}

Recall that the multiplication of two functions in 
the time domain is equivalent to the convolution of the their Fourier transforms in the frequency domain, 
hence
\PSfig[h]{Fig1_sampl1}{The sampling or ``comb'' function, $\Delta (t)$, represents mathematically what we
do when we sample a continuous phenomenon $d(t)$ at discrete, equidistantly spaced times.}
\noindent
\begin{equation}
d(t) \cdot \Delta (t) \leftrightarrow D(f) * \Delta (f).
\end{equation}	 
The time-domain expression is visualized in Figure~\ref{fig:Fig1_sampl2}.
\PSfig[H]{Fig1_sampl2}{Sampling equals multiplication of a continuous signal $d(t)$ with a comb function $\Delta (t)$ in the time-domain.}
\noindent
The transformed function $\Delta(f)$ can be shown to be a series of impulses as well (Figure~\ref{fig:Fig1_sampl3}).
\PSfig[H]{Fig1_sampl3}{The Fourier transform of the comb function, $\Delta (t)$, is another comb function, $\Delta (f)$, with a spacing of $1/\Delta t$ between impulses.}
\noindent
In the frequency domain, $d(t)$ is represented as $D(f)$ and illustrated in Figure~\ref{fig:Fig1_sampl4}.
We note that while the time-domain comb function $\Delta(t)$ is a series of impulses spaced every $\Delta t$,
the frequency-domain comb function $\Delta(f)$ is also a series of impulses, but spaced every $1/\Delta t$.
The time and frequency domain spacings of the comb
functions are thus reciprocal: A finer sampling interval leads to a larger distance between the impulses in the
frequency domain.
\PSfig[H]{Fig1_sampl4}{The Fourier transform of our continuous phenomenon, $d(t)$.  We assume it is band-limited so that the transform
goes to zero beyond the highest frequency, $f_N$.}
\noindent
Given $D(f)$ and $\Delta(f), D(f) * \Delta(f)$ is schematically shown in Figure~\ref{fig:Fig1_sampl5}.
\PSfig[H]{Fig1_sampl5}{Replication of the transform, $D(f)$, due to its convolution with the comb function, $\Delta (f)$.}
\noindent
If the impulses in $\Delta(f)$ are spaced  closer than $1/\Delta t$ then there will be some overlap between the $D(f)$ replicas
that are centered at the location of each impulse (see Figure~\ref{fig:Fig1_sampl6}).
\PSfig[H]{Fig1_sampl6}{Aliasing in the frequency domain occurs when the sampling interval $\Delta t$ is too large.}
\noindent
This overlap introduces \emph{aliasing} (which we shall discuss more later).  To prevent aliasing, we must ensure $\Delta t \leq 1/(2 f_N)$,
where $f_N$ is the highest (radial) frequency component present in the time series.  As mentioned earlier, we call $f_N$ the
\emph{Nyquist frequency} and the Nyquist sampling interval is $\Delta t  =  1/(2 f_N)$, hence $f_N = 1/(2 \Delta t)$.
\index{Aliasing}
\index{Frequency!Nyquist}
\index{Nyquist frequency}
As long as we follow the sampling theorem and select  $\Delta t \leq 1/(2 f_N)$, with $f_N$ being the highest 
frequency component, there will be no spectral overlap in $D(f)* \Delta (f)$ and we will be able to recover 
$D(f)$ completely.  Therefore (and to prove the sampling theorem) we recover $D(f)$ by truncating the signal:
\begin{equation}
D(f) = [ D (f) * \Delta (f) ] \cdot H(f),
\end{equation}
\index{Gate function}
\noindent
which is illustrated in Figure~\ref{fig:Fig1_sampl7} as a multiplication of the replicating spectrum with a \emph{gate} function, $H(f)$.
\PSfig[h]{Fig1_sampl7}{Truncation of the Fourier spectrum via multiplication with a rectangular gate function, $H(f)$.}

\subsection{Aliasing, again}
\PSfig[h]{Fig1_aliasing2}{Aliasing as seen in the time domain.  Thin line shows a phenomenon with period $P$.
The circles and heavy dashed line show
the signal obtained using a sampling rate of $1.25P$, while the squares and dashed
line show a constant signal ($f = 0$) obtained with a sampling rate of $2P$.}
Aliasing can be viewed from several angles.  Conceptually, if $\Delta t > 1/(2 f_N)$ (where $f_N$ is 
the highest frequency component in phenomenon of interest), then a high frequency component will \emph{masquerade} in the 
sampled series as a lower, artificial frequency component, as shown in Figure~\ref{fig:Fig1_aliasing2}.
\noindent
If $\Delta t$ is a multiple of $P$ (e.g., see the squares in Figure~\ref{fig:Fig1_aliasing2}), then this frequency component is indistinguishable from a 
horizontal line (i.e., a constant, with frequency $f = 0$).  If  $\Delta t = 5 P/4$ (see circles in Figure~\ref{fig:Fig1_aliasing2}) then this frequency 
component is indistinguishable from a component with frequency $1/5 P$ (i.e., period of $5P$).  Therefore, the 
under-sampled frequency components manifest themselves as lower frequency components 
(hence the word alias).  In fact, every frequency \emph{not} in the range
\index{Principal alias}
\begin{equation}
0 \leq f \leq 1/(2\Delta t)
\end{equation}
has an alias in that range --- this is its \emph{principal alias}.  Furthermore, any frequency $f_H > f_N$ 
will be indistinguishable from its principal alias.  That is, the actual frequency $f_H = f_N + \Delta f$ will appear as the aliased frequency $f_L = f_N - \Delta f$.
\index{Folding frequency}
\index{Frequency!folding}
	Because of this relationship, the Nyquist frequency $(f_N)$ is often called the \emph{folding frequency} since the
aliased frequencies ($f > f_N$) will appear at their principal aliases folded back into the range $\leq f_N$ (Figure~\ref{fig:Fig1_nyquist}).
Therefore, when computing the transform of a data set,
any frequency components in the phenomenon with true frequencies $f > f_N$ have been folded back into 
the resolved frequency range during sampling.  Consequently, we must carefully choose $\Delta t$ so that the powers at frequencies $f' > f_N$ are either small 
or nonexistent, or we must ensure that $f_N$ is high enough so that the aliased part of the spectrum only affects 
frequencies higher than those of interest ($f \leq f_I$, see Figure~\ref{fig:Fig1_nyquist}).

\PSfig[h]{Fig1_nyquist}{Aliasing and folding frequency.  Power at higher frequencies than the Nyquist ($f_N$) will
reappear as power at lower frequencies, ``folded'' around $f_N$.  This extra power (orange) is then added to the
actual power and the result is a distorted, total power spectrum (red). Selecting the Nyquist frequency so that aliasing only affects frequencies higher
than the frequencies of interest $(f \leq f_I)$.  In this case, the extra power (orange) that is folded around $f_N$
does not reach into the lower frequencies of interest, and consequently the total spectrum is unaffected for frequencies
lower than $f_I$.}
\index{Convolution|)}
\index{Sampling!theory|)}

\section{Aliasing and Leakage}
\index{Aliasing|(}
\index{Leakage|(}

\PSfig[H]{Fig1_AL}{The continuous and band-limited phenomenon of interest, represented both in the time and frequency domains.
Left column represents the time domain and the right column represents the frequency domain, separated by a vertical dashed gray line.
The multiply, convolve, and equal signs indicate the operations that are being performed. a) Continuous phenomenon, b) Sampling function,
c) Infinite discrete observations, d) Gate function, e) Truncated discrete observations, f) Assumed periodicity $T$, g) Aliasing and leakage of signal.}

We were exploring the relationship between the continuous and discrete Fourier transform 
and  found that we could illustrate the process graphically.  First, we found that we had to sample 
the time-series $d(t)$ (Figure~\ref{fig:Fig1_AL}a).  The sampling of the phenomenon by the sampling function $\Delta(t)$
(Figure~\ref{fig:Fig1_AL}b) is a multiplication in the time-domain, which implies a convolution in the frequency domain.
This sampling yields discrete observations in the time domain, but the multiplication in the time
domain equals a convolution in the frequency domain, enforcing periodicity of the spectrum (Figure~\ref{fig:Fig1_AL}c).
Depending on the chosen sampling interval we may or may not have spectral overlap (aliasing). 
This discrete infinite series must then be truncated to contain a finite number of observations.  
The truncation is conceptually performed by multiplying our infinite time series with a finite gate function.
\index{Gate function}
\index{Data!truncation}
This truncation of the infinite and periodic signal amounts to a multiplication in the time domain with a gate function, $h(t)$,
whose transform is
\begin{equation}
	H(f) = \sinc (fT) = \frac{\sin \pi fT}{\pi fT},
\end{equation}
with both functions displayed in Figure~\ref{fig:Fig1_AL}d.  This process results in the finite discrete observations
shown in Figure~\ref{fig:Fig1_AL}e.
It is this truncation that is responsible for introducing \emph{leakage}.

Leakage arises because the truncation implicitly 
assumes that the time-series is periodic with period $T$ (Figure~\ref{fig:Fig1_AL}f).  Consequently, the
discretization of frequencies is equivalent to enforcing a periodic signal (Figure~\ref{fig:Fig1_AL}g).
Because both the time and frequency domain functions have been convolved with a series of 
impulses (by $\Delta(t)$ in time and $\Delta(f)$ in frequency), both functions are periodic in $n$ discrete values, so 
the final discrete spectrum  (for a real series as shown here) between $0$ and $f_N$ represents the 
discrete transform of the series on the left (which is periodic over $T$).

	If the procedure in Figure~\ref{fig:Fig1_AL} is followed mathematically, it is seen that the continuous Fourier 
transform is related to the discrete Fourier transform by the steps outlined graphically above.  
These show that a discrete Fourier transform will \emph{differ} from the continuous transform by two effects:
\begin{enumerate}
\item Aliasing --- from discrete time domain sampling.
\item Leakage --- from finite time domain truncation.
\end{enumerate}
Aliasing can be prevented by choosing $\Delta t \leq 1/(2 f_N)$ or reduced as discussed previously.  Leakage is 
always a problem for most observed (and hence truncated) time series.
As discussed, leakage arises from truncation in the time domain, which corresponds to a 
convolution with a $\sinc$ function in the frequency domain.  Conceptually, consider the effect of time domain 
truncation (Figure~\ref{fig:Fig1_trunc1}).
Fourier analysis is essentially fitting a series of sines and cosines (using the harmonics of the 
fundamental frequency $1/T$) to the series $d(t)$.  Since the Fourier series is necessarily periodic, it 
follows that
\begin{equation}
d(T/2 + \Delta t) = d( - T/2).
\end{equation}
In other words, the transform is equivalent to that of a time series in which $d(t)$ is repeated every $T$ (Figure~\ref{fig:Fig1_trunc2}).
 
 \PSfig[h]{Fig1_trunc1}{Truncation of a continuous signal, the equivalent of multiplying the signal
with a gate function $h(t)$, determines the fundamental frequency, $f = 1/T$.}
\PSfig[h]{Fig1_trunc2}{Artificial high frequencies are introduced due to the forced periodicity of a truncated time-series, which
produces a discontinuous signal (highlighted by the gray regions).}
\noindent
The leakage (conceptually) thus results from the frequency components that must be present to 
allow the discontinuity, occurring every $T$, to be fit by the Fourier series.  If the series $d(t)$ 
is perfectly periodic over $T$ then there is no leakage because $d(T + \Delta t) = d(\Delta t)$ and the transition will be 
continuous and smooth across $T$. 

	To minimize leakage we attempt to minimize the discontinuity (between $d(0)$ and $d(T)$) or 
minimize the lobes of the $\sinc (fT)$ function convolving the spectrum.  This is accomplished by 
truncating the time series with a more gently sloping gate function (called a taper, fader, window, 
etc.). In other words, we use a smoother function that has fewer high frequency components (Figure~\ref{fig:Fig1_trunc3}).

\noindent
\index{Gate function}
\index{Bartlett window}
\index{Windowing!Bartlett}
\index{Hanning window}
\index{Windowing!Hanning}
\index{Parzen window}
\index{Windowing!Parzen}
\index{Hamming window}
\index{Windowing!Hamming}
\index{Bartlett-Priestley window}
\index{Windowing!Bartlett-Priestley}
The triangular function is the \emph{Bartlett} window, which is the rectangle function convolved with 
itself (hence its transform is $\sinc^2 (fT)$).  The dashed line is the split cosine-bell window.  Other 
windows include:  \emph{Hanning} (a cosine taper), \emph{Parzen} (similar to Hanning but decays sooner and  
more steeply, \emph{Hamming} (like Hanning), and \emph{Bartlett-Priestley} (which is quadratic and has ``optimal'' properties, 
satisfying specific error considerations.)  All of these tapers have transforms that are less 
oscillatory than the $\sinc$ function but they are also wider. Therefore, multiplication of the time series with one 
of these gate functions results in a convolution whose transform in the frequency domain will smear spectral peaks 
more than the $\sinc$ function did. In return, it will not introduce ripples far away from these spectral peaks.

\PSfig[h]{Fig1_trunc3}{Alternative gate functions and their spectral representations.  The less abrupt
a gate function is in the time domain the less ringing it will introduce in the frequency domain.}
	Note that multiplying by, say, a Hanning window will make $d(T/2+\Delta t) \sim d(-T/2)$, so 
the bothersome discontinuity is eliminated --- however damping of all $d(t)$ away from $d(T)$ acts like a modulation, 
which accounts for the smearing of spectral peaks. Hence, leakage is still not completely eliminated.
\index{Aliasing|)}
\index{Leakage|)}

\section{Complex Fourier Series}

As mentioned, Fourier series combine \emph{even} (cosine) and \emph{odd} (sine) components.
It is common to simplify these expressions by using \emph{complex notation},
in which a \emph{complex number} $z$ is written
\begin{equation}
	z = x + iy.
\end{equation}
Here, $x$ is considered the \emph{real} part, $y$ is the \emph{imaginary} part,
and $i = \sqrt{-1}$ is the \emph{imaginary number}\index{Imaginary number}.  The concept of considering a complex number as a
point $(x, y)$ in the complex plane was first presented by Caspar Wessel\footnote{No relation!} in 1799\index{Wessel, C.}, but like many mathematical inventions
others had independently dabbled with this both before and after 1800.  In the end, our old friend K. F. Gauss made
many contributions to complex number theory.

Simple rules govern elementary
operations on complex numbers.  For instance, addition and subtraction follow
\begin{equation}
	\begin{array}{c}
	z = (a + ib) + (c + id) = (a + c) + i(b + d),\\
	z = (a + ib) - (c + id) = (a - c) + i(b - d),
	\end{array}
\end{equation}
while multiplication becomes
\begin{equation}
	z = (a + ib)(c + id) = ac + iad + ibd + i^2bd = (ac - bd) + i(bc + ad).
\end{equation}
Division is converted to a multiplication by a complex number and a division by a real number
simply by multiplying both numerator and denominator by the denominator's \emph{complex conjugate}, for
which the imaginary part changes sign:
\begin{equation}
	z = \frac{(a + ib)}{(c + id)} = \frac{(a + ib)(c - id)}{(c + id)(c - id)} = \frac{(ac + bd) + i(bc + ad)}{(c^2 + d^2)}.
\end{equation}
The use of the complex notation simplifies much of the algebra associated with Fourier
analysis and is therefore mathematically more convenient to use.  This is especially true
in two and higher dimensions. We will employ
\emph{Euler's formula}\index{Euler's formula} for complex numbers in our expressions of the sine and cosine transform.
Thus, we digress to discuss Euler's formula.

\subsection{Euler's formula}
\index{Euler's formula}
\PSfig[H]{Fig1_Euler_stamps}{Leonard Euler (1707--1783)\index{Euler, L.} was one of the most productive mathematicians of his era.
You know you did something right when your mugshot ends up on stamps, even in former countries like DDR and the Soviet Union.
His famous equation $e^{i\pi} + 1 = 0$, relating the five most important numbers in mathematics, follows from his complex
relation given in (\ref{eq:eulerrelation}) for $\omega t = \pi$.}
Using a Taylor series expansion\index{Taylor series}, we can write
\begin{equation}
\sin x = x - \frac{x^3}{3!} + \frac{x^5}{5!} - \frac{x^7}{7!} + \cdots,
%(16.1a)
\end{equation}

\begin{equation}
\cos x = 1 - \frac{x^2}{2!} + \frac{x^4}{4!} - \frac{x^6}{6!} + \cdots,
%(16.1b)
\end{equation}
and
\begin{equation}
e^x = 1+ x + \frac{x^2}{2!} + \frac{x^3}{3!} + \frac{x^4}{4!} + \cdots.
%(16.2)
\label{eq:exseries}
\end{equation}
Now, let us introduce
$$
x = i \theta = \sqrt{-1} \theta.
$$
Inserting this expression into (\ref{eq:exseries}) yields
%		(16.3)
\begin{equation}
\begin{array}{rcl}
e^{i\theta} & = & \displaystyle 1 + i\theta-\frac{\theta^2}{2!} - \frac{i \theta^3}{3!}
+ \frac{\theta^4}{4!} + \frac{i \theta^5}{5!} - \cdots \\[14pt]
& = & \displaystyle \left ( 1 - \frac{\theta^2}{2!} + \frac{\theta^4}{4!} - \cdots\right )
+ i \left ( \theta - \frac{\theta^3}{3!} + \frac{\theta^5}{5!} - \cdots \right ) = \\[14pt]
& & \cos \theta + i \sin \theta.
\end{array}
\end{equation}
For $x = -i\theta$ we instead get
$$
\begin{array}{rcl}
e^{-i\theta} & = & \displaystyle 1 - i\theta - \frac{\theta^2}{2!} + \frac{i \theta^3}{3!} - \frac{\theta^4}{4!} - \frac{i \theta^5}{5!} - \cdots \\[14pt]
& = & \displaystyle \left ( 1 -\frac{\theta^2}{2!} + \frac{\theta^4}{4!} - \cdots\right ) - i\left ( \theta - \frac{\theta^3}{3!} + \frac{\theta^5}{5!} - \cdots\right ) = \cos \theta - i \sin \theta.
\end{array}
$$
We will associate $\cos \theta$ with the positive $x$-axis and $i \sin \theta$ with the positive $y$-axis.
Thus, $e^{i\theta}$ is a \emph{unit vector} rotated an angle $\theta$ counter-clockwise from the $x$-axis
(examine Figure~\ref{fig:Fig1_sincos} one more time). For the
angular frequencies in the Fourier series (where $\theta = \theta(t) = \omega t)$, the Euler relation is
\begin{equation}
\boxed{e^{\pm i\omega t} = \cos \omega t \pm i \sin \omega t,}
\label{eq:eulerrelation}
%(16.4)
\end{equation}
while the \emph{inverse} relationships are
\begin{equation}
\begin{array}{c}
\cos \omega t = \frac{1}{2} (e^{i\omega t} + e^{-i \omega t} ), \\[14pt]
\sin \omega t = \frac{1}{2 i } (e^{i\omega t} - e^{-i \omega t} ).
\end{array}
%(16.5)
\end{equation}
We will refresh (or introduce) a few concepts that apply to complex numbers.
The \emph{complex conjugate}\index{Complex!conjugate} (denoted by $^*$) of a function $f(x)$ is written $f^*(x)$, with components
\begin{equation}
%(16.6)
\begin{array}{lcr}
f(x) & = & R (x) + i I(x), \\[14pt]
f^{\ast}(x) & = & R (x) - i I(x).
\end{array}
\end{equation}
Here, $R(x)$ is the real part and $I(x)$ is the imaginary part of $f(x)$, respectively. The \emph{magnitude}\index{Complex!magnitude} or
amplitude of a complex number is given by
\begin{equation}
A(x) = | f(x) | = \sqrt{f(x) \cdot f^*(x)} = \sqrt{R^2 (x) + I^2 (x)},
%(16.7)
\end{equation}
while the \emph{phase} is obtained via
\begin{equation}
\phi(x) = \tan^{-1} \frac{I(x)}{R(x)}.
%(16.7)
\end{equation}

\subsection{Using the complex notation}
\index{Orthogonality!Fourier series|(}
By using Euler's formula the \emph{five} orthogonality relations discussed in Section~\ref{sec:FFS} become just \emph{one}:
\begin{equation}
%(16.8)
\sum^n_{\ell=1} e^{i \omega_j t_{\ell}} \cdot e^{-i \omega_k t_{\ell}} =
\left \{ \begin{array}{cc}
n, & j=k \\[12pt]
0, & \mbox{otherwise}
\end{array} \right.
\label{eq:DA2_16.8}
\end{equation}
where we now use $\ell$ to indicate the point number (since $i$ here represents the imaginary number).
The equivalent integral relationship is obviously
\index{Orthogonality!Fourier series|)}
\begin{equation}
\int^T_{0} e^{i \omega_j t} \cdot e^{-i \omega_k t} dt =
\left \{ \begin{array}{cc}
T, & j=k \\[12pt]
0, & \mbox{otherwise}
\end{array} \right.
\label{eq:orthocomplex}
\end{equation}
Any real-valued function can be represented as a complex-valued function with a zero
imaginary part.
Using the inverse Euler relations, the Fourier series (\ref{eq:fourierseries}) can now be rewritten in complex form
as follows, again assuming that $n$ is an even number:
\begin{equation}
\begin{array}{rcl}
d_{\ell} & = & \displaystyle \sum^{n/2}_{j=0} \left [ a_j \cos \omega_j t_{\ell} + b_j \sin \omega_j t_{\ell} \right] =
\displaystyle \sum^{n/2}_{j=0}
\left [ \frac{a_j}{2} \left (e^{i\omega_j t_{\ell}} + e^{-i\omega_j t_{\ell}} \right ) +
\frac{b_j}{2i} \left (e^{i\omega_j t_{\ell}} - e^{-i\omega_j t_{\ell}} \right ) \right] \\[14pt]
& = & \frac{1}{2}\displaystyle \sum^{n/2}_{j=0} \left [\left (a_j + \frac{b_j}{i} \right) e^{i\omega t_j}
+ \left (a_j - \frac{b_j}{i} \right ) e^{- i \omega_j t_{\ell}} \right ] \\[14pt]
& = & \displaystyle \frac{1}{2} \sum^{n/2}_{j=0} \left [ \left (a_j - ib_j \right ) e^{i \omega_j t_{\ell}} + \left (a_j + ib_j \right)
e^{-i \omega_j t_{\ell}} \right ].
\end{array}
\label{eq:DA2_16.10}
\end{equation}
%(16.10)
Since the second term contains $e^{-\omega_j t_{\ell}}$, we shall consider the effect of introducing \emph{negative}
frequencies (i.e., let $j$ take on negative values) in simplifying this expression further\index{Frequency!negative}. In general, for negative $j$ (i.e., $-j$),
\begin{equation}
\left (a_{-j} - ib_{-j}\right )e^{i\omega_{-j} t_{\ell}} = \left (a_j +ib_j \right ) e^{-i \omega_j t_{\ell}},
%(16.11)
\end{equation}
which we see are related by the complex conjugate definition
\begin{equation}
J_{-j} = {J^*}_j,
%(16.12)
\end{equation}
because
$$
\begin{array} {rcl}
a_{-j} &=& \displaystyle \frac{2}{n} \sum^n_{\ell=1} d_{\ell} \cos \omega_{-j} t_{\ell} =\frac {2}{n} \sum^n_{\ell=1} d_{\ell} \cos \frac{-2 \pi j}{T}t_{\ell} = \frac{2}{n} \sum^n_{\ell=1} d_{\ell} \cos \frac{2\pi j}{T}t_{\ell} = a_j,\\[14pt]
b_{-j} &=& \displaystyle \frac{2}{n} \sum^{n}_{\ell=1} d_{\ell} \sin \omega_{-j} t_{\ell} = \frac {2}{n} \sum^n_{\ell=1} d_{\ell} \sin \frac {-2 \pi j}{T} t_{\ell} = \frac{-2}{n} \sum^n_{\ell=1} d_{\ell} \sin \frac{2 \pi j}{T} t_{\ell} = -b_j.
\end{array}
$$
Consequently, $a_j$ is an \emph{even} function, $b_j$ is an \emph{odd} function, and obviously
$$
e^{i \omega_{-j} t_{\ell}} = e^ {-i\omega_j t_{\ell}}.
$$
Therefore, the second term of (\ref{eq:DA2_16.10}) can be dropped if we merely extend the sum over
$-n/2 \leq j < n/2$:
\begin{equation}
\boxed{d_{\ell} = \sum^{< n/2}_{j = -n/2} \frac{1}{2} (a_j - ib_j) e^{i\omega_j t_{\ell}}= \sum^{< n/2}_{j = -n/2} J_j e^{i \omega_j t_{\ell}},\quad \ell=1,n.}
\label{eq:DA2_16.13}
%(16.13)
\end{equation}
Notice that the complex form has \emph{twice} as many coefficients as the real form, reflecting the fact
that each value of $d_{\ell}$ contains a real and an imaginary component.  Thus, there are actually \emph{twice} as many $d_{\ell}$
values as before, even though for most observations the imaginary components will all be zero.
Equation (\ref{eq:DA2_16.13}) represents the general complex form of the Fourier series, with complex coefficients
\begin{equation}
J_j = \frac{1}{2} (a_j-ib_j).
%(16.14)
\end{equation}
By substituting the expressions for $a_j$ (\ref{eq:DCT}) and $b_j$ (\ref{eq:DST}) into the expression above
(for $J_j$) \emph{or} by multiplying Equation (\ref{eq:DA2_16.13}) by $e^{-i\omega_k t_{\ell}}$, summing over all $t_{\ell}$, and using the
orthogonality relation (\ref{eq:DA2_16.8}), we can derive the general complex form for $J_j$ via
$$
\begin{array}{rcl}
\displaystyle \sum^{n}_{\ell=1} d_{\ell} e^{-i \omega_k t_{\ell}}&=& \displaystyle \sum^{< n/2}_{j = -n/2} J_j \sum^{n}_{\ell=1} e^{i \omega_j t_{\ell}} e^{-i \omega_k t_{\ell}},\\
%(16.15a)
\end{array}
$$
yielding
\begin{equation}
\displaystyle \sum^{n}_{\ell=1} d_{\ell} e^{-i \omega_k t_{\ell}} = n J_k.
%(16.15)
\end{equation}
Hence, and swapping the dummy indices $j$ and $k$, we find
\index{Complex!discrete Fourier transform}
\begin{equation}
\boxed{J_j = \displaystyle \frac{1}{n} \sum^{n}_{\ell=1} d_{\ell} e^{-i \omega_j t_{\ell}} , \quad -\frac{n}{2} \leq j < \frac{n}{2}.}
\label{eq:DA2_16.15}
%(16.15)
\end{equation}
Equation (\ref{eq:DA2_16.15}) is the \emph{complex discrete Fourier transform} of the series $d_{\ell}$ and equation (\ref{eq:DA2_16.13}) is
the \emph{complex discrete inverse Fourier transform}. Since $J_{n/2} = J_{-n/2}$ (and both are real since $b_{n/2} = 0$) we need not compute it for each index,
hence $-n/2 \leq j < n/2$. This arrangement also eliminates the awkward need to define $a_0$ and $a_{n/2}$ as one-half their values.

When $d_{\ell}$ is a \emph{real-valued} series, then $d_{\ell}=d^*_{\ell}$~because all the imaginary parts are zero, so
$$
J_j=\frac{1}{n} \sum^{n}_{\ell=1} d_{\ell} e^{-i \omega_j t_{\ell}} = \frac{1}{n} \sum^{n}_{\ell=1} d_{\ell} (\cos \omega_j t_{\ell} - i \sin \omega_j t_{\ell})
= \frac{1}{n} \sum^{n}_{\ell=1} d_{\ell} \cos \omega_j t_{\ell} - \frac{i}{n} \sum^{n}_{\ell=1} d_{\ell} \sin \omega_j t_{\ell}
$$
and hence
$$
J_{-j} = J^\ast_j.
$$
Therefore, the transform (i.e., $J_j$) is completely determined by the positive values of $j$, as
developed earlier, i.e., they are completely determined by the $n$ values of $a_j$ and $b_j$.
Specifically, if we need to recover $a_j$ and $b_j$ from the $J_j$ coefficients we simply note that
\begin{equation}
	J_{-j} + J_j = J^\ast_j + J_j = \frac{1}{2} (a_j+ib_j) + \frac{1}{2} (a_j-ib_j) = a_j,
\end{equation}
and likewise
\begin{equation}
	J_{-j} - J_j = J^\ast_j - J_j = \frac{1}{2} (a_j+ib_j) - \frac{1}{2} (a_j-ib_j) = i b_j.
\end{equation}
The negative frequencies confuse many practitioners of spectral analysis.  What are they? What do they represent?\index{Frequency!negative}
Are they components going backwards in time or space? 
It is important to remember that we simply introduced these as a \emph{mathematical convenience} in order to arrive at a
simple and compact transform, i.e., (\ref{eq:DA2_16.15}).  Physically, there are only positive frequencies.

\subsection{The continuous Fourier transform in 1-D}

For a continuous function $g(t)$,  equivalent expressions make up the \emph{forward} and \emph{inverse Fourier transform} pair,
typically written in complex (and symmetric) form as
\begin{equation}
%(16.16a)
G(f) = \int ^\infty _{-\infty} g(t) e^{-i 2 \pi t f} dt
\label{eq:FT1D}
\end{equation}
%(16.16b)
and
\begin{equation}
g(t) = \int ^\infty _{-\infty} G(f) e^{+i 2 \pi t f} df,
\label{eq:IFT1D}
\end{equation}
respectively, for radial frequency $f$.  As noted, the main differences between the continuous transforms and the discrete transforms
(which we necessarily must use to operate on observed data) are:
\begin{enumerate}
	\item Our data have a finite, nonzero sampling interval, whereas the theory considers continuous distributions.
	This difference opens up the possibility of \emph{aliasing}.
	\item Our data have a finite length (i.e., a fundamental period), whereas the theoretical distributions exist over all time.
	This limitation may lead to issues involving \emph{leakage}.
\end{enumerate}

\section{Computing Fourier Transforms}
Most practitioners of spectral analysis will find themselves performing the Fourier transform via a special
algorithm known as the \emph{Fast Fourier Transform} (FFT).  This algorithm as well as practical information on how
to use it and how to prepare our data for it is the focus of this section.

\subsection{The Fast Fourier Transform (FFT)}
\index{Fast Fourier transform (FFT)|(}
\index{FFT (Fast Fourier transform)|(}

The discrete Fourier transform involves $n$ coefficients that each requires $n$ multiplications to be determined.
Hence, the entire calculations is $O(n^2)$ in computer time.  Can we do better than that?
Since $\omega_j = 2 \pi j / T$, $T = n \Delta t$, and $t_\ell = \ell \Delta t$ for $\ell =0, ..., n-1$ it follows that the exponential term
in (\ref{eq:DA2_16.15}) is
\begin{equation}
e^{ -i \omega_j t_{\ell}} = e^{-\frac{i 2 \pi j}{n \Delta t}\ell \Delta t} = \left [ e^{-\frac{2 \pi i}{n}} \right ]^{j\ell} = W^{j\ell}.
\end{equation}
Note what happened to time here; it cancels, leaving just a ratio of integers.
We may now write
\begin{equation}
J_j = \displaystyle \frac{1}{n} \sum^{n-1}_{\ell=0} y_{\ell} W^{j\ell}.
\end{equation}
It is common in FFT implementations to perform the division by $n$ separately (at the end), so let us just consider
\begin{equation}
J_j = \displaystyle \sum^{n-1}_{\ell=0} y_{\ell} W^{j\ell}.
\end{equation}
You may note that if $j = 0$ we obtain $J_0 = \bar{y}$ (apart from the $1/n$ term, that is).  Now, consider this rearrangement:
\begin{equation}
J_j = \displaystyle \sum^{n-1}_{\ell=0} y_{\ell} e^{-\frac{2 \pi i j\ell}{n}} =
	\sum^{n/2-1}_{\ell=0} y_{2\ell} e^{-\frac{2 \pi i j (2\ell)}{n}} +
	\sum^{n/2-1}_{\ell=0} y_{2\ell+1} e^{-\frac{2 \pi i j (2\ell+1)}{n}}.	
\end{equation}
Here, we have simply rewritten the sum as two separate series containing the even ($2 \ell$) and odd ($2 \ell + 1$) terms of
the observations, respectively.  This expression can be manipulated further to yield
\begin{equation}
J_j = \displaystyle \sum^{n/2-1}_{\ell=0} y_{2\ell} e^{-\frac{2 \pi i j \ell}{(n/2)}} +
	e^{-2 \pi i j / n} \sum^{n/2-1}_{\ell=0} y_{2\ell+1} e^{-\frac{2 \pi i j \ell}{(n/2)}},
\label{eq:FFT1}
\end{equation}
which we can write as
\begin{equation}
J_j = J_j^e + W^j J_j^o.
\end{equation}
\PSfig[h]{Fig1_FFT_split}{Each dot-product making up a Fourier transform for a single frequency can be split in two by considering the odd (gray)
and even (white) observations separately, and by recursively applying this partitioning
until we reach a single element we can speed of the Fourier transform enormously.}
The first term is the Fourier transform of the even-numbered observations (i.e., considering only the 0th, 2nd, 4th, etc., values), while the
second term is the Fourier transform of just the odd-numbered observations (i.e., the 1st, 3rd, 5th, etc. values), scaled by the constant $W^j$. Since
each transform only deals with $n/2$ points we find the calculation time to be proportional to $O(2(n/2)^2) = O(n^2/2)$.
So while (\ref{eq:FFT1}) is mathematically identical to (\ref{eq:DA2_16.15}), the partitioning into two sums leads to a 50\% reduction in computation time.
But wait, there is more! The partitioning idea can be continued recursively on the two separate $J_j^e$ and $J_j^o$
sums as well.  For instance, the two terms in the transform for $J_j^e$ requiring $n/2$ data values each can themselves be written as
\begin{equation}
\begin{array}{ccc}
	J_j^e & = & J_j^{ee} + W^j J_j^{eo}, \\[14pt]
	J_j^o & = & J_j^{oe} + W^j J_j^{oo},
\end{array}
\end{equation}
with each term being a transform requiring just $n/4$ data points.  If $n$ is initially some power of 2, then carrying this splitting
all the way to a single number (whose transform is itself) yields a computational workload that is $O(n \log n)$.
These are huge savings compared to our initial estimate of $O(n^2)$!  For instance, with $n = 10^6$, the difference is
a factor of 50,000, while for $n = 10^7$ the operations are 400,000 times faster.  For each magnitude in $n$, this ratio
increases by another factor of $\sim 8$.  This clever result is essentially the \emph{Cooley-Tukey}\index{Cooley, J. W.}
\index{Tukey, J. W.}\index{Cooley-Tukey FFT algorithm} FFT algorithm from the 1960s.  However, its
origin goes further back to the 1940s with \emph{Danielson}\index{Danielson, G. C.} and \emph{Lanczos}\index{Lanczos, C.} and their work
on x-ray scattering as well as further back to \emph{Gauss}\index{Gauss, K. F.}.  Apparently, every neat mathematical idea can eventually
be traced back to Gauss or Euler...

\subsection{FFT implementations}

Both MATLAB and Octave implement the fast Fourier transform in a similar fashion, as do most mathematical function libraries.  There are three
functions we will need to become familiar with:
\begin{description}
	\item [fft:] The forward fast Fourier transform.\index{MATLAB!fft}\index{fft (MATLAB)}
	\item [ifft:] The inverse fast Fourier transform.\index{MATLAB!ifft}\index{ifft (MATLAB)}
	\item [fftshift:] Rearranging the order of spectral coefficients.\index{MATLAB!fftshift}\index{fftshift (MATLAB)}
\end{description}
To see how these work and what they expect as input we will look at the arguments of data and frequencies.
Let our data be represented by the data array
\begin{equation}
\mathbf{d} = [ \quad d_1, \quad d_2, \quad \ldots, \quad d_n \quad ],
\end{equation}
where $n$ is even and represents the number of data points.  Then, taking the transform [i.e., \texttt{J = fft(d);} in MATLAB or Octave] yields the array
\begin{equation}
\mathbf{J} = [ \quad J_0, \quad J_1, \quad J_2, \quad \ldots, \quad J_{\frac{n}{2}-1}, \quad J_{-\frac{n}{2}}, \quad J_{-\frac{n}{2}+1}, \quad \ldots, \quad J_{-1} \quad ].
\label{eq:FFT2}
\end{equation}
Clearly, the amplitudes are split between the positive (first half) and negative (second half) sets of the coefficients, i.e.,
\begin{equation}
| J_k | = | J_{-k} |.
\end{equation}
Plotting $J(f)$ is awkward since the order is discontinuous with respect to frequency, $f$.  To place $J_0$ symmetrically
in the middle requires shifting of the coefficients.  This is done by \emph{fftshift} [i.e., \texttt{J = fftshift(J);}], which results in the array
\begin{equation}
J = [ \quad J_{-\frac{n}{2}}, \quad J_{-\frac{n}{2}+1}, \quad \ldots, \quad J_{-1}, \quad J_0, \quad J_1, \quad J_2, \quad \ldots, \quad J_{\frac{n}{2}-1} \quad ],
\end{equation}
where the very first item ($J_{-\frac{n}{2}}$) is the real coefficient associated with the Nyquist frequency (since the
sine term is identically zero), and $J_0$ reflects the mean (real) value of $d$.  These are the only two coefficients
that \emph{do not} appear twice in (\ref{eq:FFT2}).
Note that your particular time coordinates $t_{\ell}$ play \emph{no part} in the transform.  It is implicitly assumed that
$t_{\ell} = {\ell}\Delta t$ and that $t_0 = 0$, so make sure you are careful in selecting your origin time.

\subsection{Detrending and windowing}

\PSfig[h]{Fig1_taper}{(top) Original topographic profile.  If this data set were subjected to Fourier analysis,
then the jump discontinuity due to the forced periodicity would introduce leakage. (middle) We reduce the leakage
by using a smooth rather than rectangular window function.  Here, a \emph{Tukey} window is shown, which is a cosine taper
between zero at the ends and unity in the middle. (bottom) Multiplying our data with the window function yields a data
set that no longer has a jump discontinuity.}
\index{Tukey window}
Prior to computing spectral estimates with an FFT we should do our utmost to reduce the influence of leakage.  As we discussed,
leakage arises due to truncation of a hypothetically infinite data series to a finite length data set (i.e., when making our observation), and the range of
the observed data becomes our fundamental (and probably artificial) period.  Since spectral analysis uses
periodic functions (sines and cosines) to represent the data, we are in effect forcing our data to be periodic as well, with the presumably
arbitrary data range as the fundamental period.  Unless the data happen to start and end on the same value we may have a potentially large
offset between the two values, which essentially constitutes a step function (e.g., upper panel of Figure~\ref{fig:Fig1_taper}).  Decomposing the data into sines and cosines
means we will approximate this step using  Fourier building blocks, thus leaking energy over a wide range of frequencies.
The standard approach to minimizing this problem is to use a gentler gate function than the rectangle we implicitly used when we captured our
data.  Figure~\ref{fig:Fig1_taper} (middle panel) shows what is called a Tukey window, which essentially is a half cosine wavelength that
connects the area outside the window (zero weight) to the central portion of the window (unit weight).  By multiplying our
observed data by this window we eliminate the step mismatch between the start and end of the data (e.g., lower panel of Figure~\ref{fig:Fig1_taper}), and this data modification dramatically
reduces the effect of leakage.  Note, however, that leakage is not completely eliminated since our tapering still changes the observed signal
in some way.  For one thing, it effectively shortens the fundamental period by an amount proportional to the ramp margin.

\subsection{Zero-padding}
\label{sec:zeropad}
These are two common and seemingly related problems in spectral analysis:
\begin{enumerate}
	\item For many data sets, the frequency resolution $\Delta f = 1/T$ may not be small enough to resolve spectral
	components that are closely spaced (remember, all the Fourier frequencies we use are integer multiples of $\Delta f$).
	\item The discrete points in the raw periodogram may be too widely spaced to resolve the actual frequencies
	of the peaks in question.    
\end{enumerate}
\PSfig[H]{Fig1_zeropad}{Zero-padding our data means to extend the range of the windowed data by adding zeros until the length
(i.e., the number of data points) reaches the next power of 2.}
Unfortunately, the first issue can only be addressed by collecting a longer time series so as to decrease $\Delta f$ to
a point where the two peaks are clearly separated in the periodogram.  Without this added resolution such peaks will
be seen as one wider (blurred) peak.  The second problem, however, can be addressed by
extending the finite time series with zeros prior to taking the FFT (Figure~\ref{fig:Fig1_zeropad}).  By adding zeroes, the length of the data
series ($n$) increases, effectively adding additional frequency components in between those that would be obtained for the
original, non-padded series.  In essence, we obtain an \emph{interpolation} of the spectral density estimates.  Note that
zero-padding helps fill in the shape of the spectrum but of course there is no improvement in the fundamental frequency resolution.
Nevertheless, zero-padding is widely used for several reasons:
\begin{enumerate}
	\item It smooths the shape of the periodigram via spectral interpolation.
	\item It may resolve potential ambiguities where the frequency difference between line spectra is greater
	than the fundamental frequency resolution.
	\item It may help define the exact frequency of the peaks of interest by reducing the quantization of the power.
	\item It may increase $n$ to an integer power of 2, thus speeding up the analysis when the FFT is used.
\end{enumerate}
While these benefits are all good, we note that adding zeros does not help us to distinguish closely spaced frequency components
that could not be resolved in the original time series.

\index{FFT (Fast Fourier transform)|)}
\index{Fast Fourier transform (FFT)|)}


\section{Filtering}
\index{Filtering|(}

Filtering data is a major data processing procedure that is used throughout the natural sciences.  Such procedures are applied to
reduce ``undesired'' features in the observations or to enhance the ``desired'' features.  What the desired and
undesired features are may change completely from application to application and may even be interchangeable.
Filters may be expressed as convolutions (and thus may take advantage
of the speed-up provided by the convolution theorem) or they must be executed relatively slowly in the time (or space) domains.
Both types of filters play important roles in data processing and interpretation.  Before we discuss a range of well-known
filters we will digress to revisit convolution and examine various properties of the Fourier transform.

\subsection{Convolution and the Fourier Transform}

Before revisiting convolution, we need to take a closer look at the analytical
Fourier transform as given by (\ref{eq:FT1D}).
Note that this general (continuous) form differs from the discrete form (\ref{eq:DA2_16.15}) in that it is no
longer dependent upon discrete Fourier frequencies nor is it restricted to a finite period, $T$.
Of course, for an actual transform of a discrete time series the discrete Fourier transform is always
used, thus in practice we will consider the Fourier frequencies only. The general integral form is used to transform
\emph{analytic} expressions, such as may arise in the solution to differential equations, and is thus useful in other ways as well.

	Using the notation of (\ref{eq:FT1D}), Fourier transform pairs are usually designated by
$$
g(t) \Leftrightarrow G(f),
$$
i.e., the transform of a function $g(t)$, which is a function of time (or distance) and traditionally denoted by lower-case letters,
is represented by the corresponding uppercase version of
the function, $G(f)$, and varies as a function of frequency $f$ (or wavenumber $k$ for spatial data).
\begin{example}
We will look at an important analytic Fourier transform pair.  Consider the instantaneous impulse or delta function
$$
%		(19.2)
g(t)= \delta (t).
$$
Per the definition of a \emph{Dirac delta function}\index{Dirac delta function}, its transform will be
\begin{equation}
G(f)= \int_{-\infty}^\infty \delta (t) e^{-i 2 \pi f t} dt= e^{0} = 1.
\label{eq:spike}
\end{equation}
\PSfig[H]{Fig1_delta}{(a) A delta-function (or spike) in one domain (here, the time domain) will be
transformed (b) to a constant in the other domain.  Note that the phase spectrum is zero since there
are no imaginary part to the transform.}
Thus, the narrowest possible function in the time domain has the broadest possible spectral expression, reflecting
the reciprocal nature of time and frequency (Figure~\ref{fig:Fig1_delta}).  Another way to look at this result is to say that an impulse is simply
a sum of cosines of unit amplitude for all frequencies.  This equivalence is used in geophysical exploration for oil and gas on
land. Instead of setting off explosions (i.e., an impulse), \emph{Vibroseis}\index{Vibroseis} trucks generate sweeping harmonic signals of increasing frequency.
In post-processing, the Earth's response to all these harmonics can be combined into the equivalent response to an actual impulse.  Because we cannot
use frequencies approaching infinity the resulting impulse will have a finite width, such as indicated in Figure~\ref{fig:Fig1_spike}.
\PSfig[h]{Fig1_spike}{Per (\ref{eq:spike}), if we sum up all the harmonics of uniform amplitude we should obtain a delta-function.
Here we add the first 40 harmonics, yielding an approximate impulse, which illustrates the principle behind Vibroseis.}
\end{example}
Other uses of analytic Fourier transforms include obtaining insight into the nature of spectra obtained from real processes
by studying those of very simple models.

\subsection{The scale theorem}
\index{Scale theorem}
\index{Theorem!scale}
\PSfig[h]{Fig1_scaletheorem}{(a) A wide (heavy line) and narrow (dashed line) function in the time domain will transform to (b) a
narrow and wide function in the frequency domain, respectively.}
What happens if we contract or expand the time axis?  We find the rule
\begin{equation}
\mbox{If} \ g(t) \leftrightarrow G (f)\ \mbox{then} \ g(at) \leftrightarrow \displaystyle \frac{1}{|a|} G \left (\frac{f}{a}\right ).
%(19.6)
\label{eq:scaletheorem}
\end{equation}
Proof, using $u = at$:
$$
\int^\infty_{-\infty} g(at) e^{-i 2 \pi f t} dt = \frac{1}{a} \int^\infty_{-\infty} g(u) e^{-i 2 \pi u/a} du = \frac {1}{|a|} G\left (\frac{u}{a}\right ).
$$
If $a$ is negative, then the limits of integration are reversed, so the frequency domain
amplitudes are still scaled positively, hence the $|a|$. Therefore, if a time scale is
contracted by $a$, the frequency is expanded $(1/|a|)$; this is expected since time is inversely
proportional to frequency. The amplitude of the frequency is decreased, which maintains
a constant area relative to the unscaled function.

\subsection{The shift theorem}
\index{Shift theorem}
\index{Theorem!shift}
If we apply a shift to the time axis, we find another rule:
\begin{equation}
\mbox{If} \ g(t) \leftrightarrow G(f) \ \mbox{then} \ g(t-t_0) \leftrightarrow G(f) e^{-i 2\pi f t_0}.
\label{eq:shifttheorem}
\end{equation}
%(19.7)
In polar form,
$$
G(f) e^{-i 2 \pi f t_0}=[A(f)e^{i \theta (f)}] e^{-i 2 \pi f t_0} = A(f)e^{i (\theta(f)- 2 \pi f t_0)}.
$$
Therefore, a shift in the time axis (a \emph{translation}) does not affect the amplitude spectrum, it
only affects the \emph{phase} spectrum.  It linearly shifts the phase by $\theta_0 = 2 \pi f t_0$, thus the
slope of the phase shift is directly proportional to the time shift $t_0$ (i.e., if phase were plotted
against $f$, then the slope would equal $t_0$).  This means the transform of a shifted impulse $\delta(t-t_0)$ is
$e^{i(\theta-\theta_0)}$, showing that the transform of random noise $n(t)$ has a constant amplitude spectrum $N(f) = 1$
but a random phase spectrum.

\subsection{The gate function}
\index{Gate function}
Finally, we will look at the transform of a very useful construction known as a
\emph{gate} function:
\begin{equation}
g(t)=\left \{\begin{array} {cl} 1, & |t| \leq T/2\\
0, & \mbox{elsewhere}
\end{array} \right..
\end{equation}
		%(19.9)
It transforms as follows:
$$
G(f)=\int^\infty_{-\infty} g(t) e^{-i 2\pi f t} dt = \int^{T/2}_{-T/2} e^{-i 2 \pi f t} dt = \int^{T/2}_{-T/2} (\cos 2\pi f t - i\sin 2\pi f t) dt = \int^{T/2}_{-T/2} \cos 2\pi f t dt.
$$
Here we have used the fact that the sine term, being odd, will integrate to zero. Using the substitution $u = 2 \pi f t$ we obtain
$$
G(f)=\frac{1}{2\pi f} \int^{\pi f T}_{-\pi f T} \cos u du = \frac{1}{2 \pi f}\sin u \mid ^{\pi f T}_{-\pi f T}= \frac {1}{2 \pi f}\left (\sin(\pi f T)-\sin(-\pi f T)\right).
$$
We rearrange this expression to give
\begin{equation}
G(f)= T \frac{\sin(\pi f T)}{\pi f T}= T \sinc (f T),
\end{equation}
%(19.10)
which is the definition of the $\sinc$-function.
\PSfig[h]{Fig1_sinc}{Truncating an infinite and periodic signal amounts to multiplying it in time with a gate function, $g(t)$,
whose transform $G(f) = T\sinc (fT)$.}

\subsection{The convolution theorem}
\index{Theorem!convolution}
\index{Convolution theorem}
\label{sec:convolutiontheorem}
	The \emph{convolution theorem} simply states that a convolution in the time domain is equivalent to
a multiplication in the frequency domain, and vice-versa, written as
\begin{equation}
s(t) * g(t) \leftrightarrow S (\omega) \cdot G (\omega).
%		(19.11)
\end{equation}
If $h(t) = s(t) * g(t)$, then
$$
H (\omega) = \int^\infty_{-\infty} h(t) e^{-i \omega t} dt = \int^\infty_{-\infty} \left [ \int^\infty_{-\infty} s(\tau) g (t- \tau) d\tau \right ]
e^{-i \omega t} dt.
$$
Notice the lag is defined as $t$ in the innermost integral. Interchanging the order of integration (this assumes
continuity of the integrand over $\pm\infty$) gives
$$
H (\omega) = \int^\infty_{-\infty} s(\tau) \left[ \int^\infty_{-\infty} g( t - \tau) e^{-i \omega t} dt \right ] d\tau.
$$
Let $u = t - \tau$, so $dt = du$. Then,
$$
\begin{array}{rcl}
H (\omega) & = & \displaystyle \int^\infty_{-\infty} s(\tau) \left[ \int^\infty_{-\infty} g(u)
e^{-i \omega(u + \tau)} du \right ] d\tau = \int^\infty_{-\infty} s(\tau) e^{-i \omega \tau}
\left[ \int^\infty_{-\infty} g(u) e^{-i \omega u} du \right ] d\tau \\[10pt]
& = & \displaystyle G (\omega) \int^\infty_{-\infty} s(\tau) e^{-i \omega \tau} d\tau = S(\omega) \cdot G (\omega).
\end{array}
$$
The result of this theorem provides the basic tool from which (among other things) the
continuous Fourier transform is related to the discrete Fourier transform. Also, because
of symmetry,
\begin{equation}
s(t) \cdot g (t) \leftrightarrow S (\omega) * G (\omega).
		%(19.12)
\end{equation}

\subsection{Parseval's theorem}
\index{Theorem!Parseval}
\index{Parseval's theorem}
\label{sec:Parseval}

\emph{Parseval's theorem} is a statement relating the variance of a data set to its power spectrum.
For a continuous function $g(t)$, the statement becomes
\begin{equation}
	\int_{-\infty}^{+\infty} \mid g(t) \mid^2 dt = \int_{-\infty}^{+\infty} \mid G(f) \mid^2 df.
	\label{eq:parceval}
\end{equation}
In Section~\ref{sec:periodogram} we derived a discrete version of Parseval's theorem by using a Fourier series expansion
for $g(t)$, removing its mean, and computing the variance of the finite time series.  Using
our complex notation we may now simply write
\begin{equation}
y_{\ell} = \sum^{< n/2}_{j \geq -n/2} J_j e^{i \omega_j t_{\ell}}, \quad \ell = 1,n
\end{equation}
and form the discrete version of (\ref{eq:parceval}) via (\ref{eq:DA2_16.15}) to yield
\begin{equation}
\sum^{n}_{\ell=1} y_{\ell}^2 = \sum^{n}_{\ell=1} \left \{ \left [ \sum^{< n/2}_{j \geq -n/2} J_j e^{i \omega_j t_{\ell}} \right ] \left [ \sum^{< n/2}_{k \geq -n/2} J_k e^{i \omega_k t_{\ell}} \right ] \right \}.
\end{equation}
Because of orthogonality (\ref{eq:DA2_16.8}), only the $j = k$ cross-terms will be nonzero and we find
\begin{equation}
\sum^{n}_{\ell=1} y_{\ell}^2 = n\sum^{< n/2}_{j \geq -n/2} J_j^2.
\end{equation}
By normalizing by $(n-1)$ we see the partitioning of data variance into its frequency contributions:
\begin{equation}
s^2 = \frac{1}{n-1}\sum^{n}_{\ell=1} y_{\ell}^2 = \frac{n}{n-1}\sum^{< n/2}_{j \geq -n/2} J_j^2  \approx \sum^{< n/2}_{j \geq -n/2} J_j^2.
\end{equation}

\subsection{Convolution filters}

	Filtering of data is typically performed in order to either smooth the signal or suppress power at 
particular frequencies or wavenumbers.  So far, we have learned that filtering can be considered an 
example of a convolution between the data $d(t)$ and the filter $p(t)$, i.e.,
\begin{equation}
h(t) = d(t) * p(t).
\end{equation}
Thus, we can immediately take advantage of the convolution theorem and write
\index{Convolution theorem}
\begin{equation}
H(f) = D(f) \cdot P(f).
\end{equation}
Hence, we can simply take the Fourier transform of $d(t)$ and multiply its spectral components $D(f)$ by 
$P(f)$.  For example, we learned previously that a simple MA filter consists of convolving the signal with a 
rectangle function of width $W$ (Figure~\ref{fig:Fig1_filter1}).\index{Filtering!moving average}
\PSfig[h]{Fig1_filter1}{Moving average filter convolution seen in the time domain.}
\noindent
Figure~\ref{fig:Fig1_filter2} shows how this operation might look like in the frequency domain.

\PSfig[h]{Fig1_filter2}{Moving average filtering seen in the frequency domain.}
\noindent
Thus, the MA filter output is obtained by multiplying the signal's Fourier coefficients by a $\sinc$ function.  This 
means that some of the coefficients will have their signs reversed, which we know equates to a 
phase change of $180^{\circ}$.  Also, while the filter coefficients do fall off with increasing $f$, they do so in a very 
oscillatory way and take a long time to approach zero.  This fact would suggest that the $\sinc$ function is a 
poor choice for a filter if you really wanted to cut out, say, high frequency information.  Perhaps we 
should instead design $P(f)$ directly so that it truncates all power at frequencies higher than $f_{\mbox{cut}}$?
Surely, by simply removing all power from frequencies of no interest we should retain the part of the signal
that we are most interested in.
This alternative operation, which is simply carried out in the frequency domain, is illustrated in Figure~\ref{fig:Fig1_filter3}.
The filtered output is finally obtained by transforming $H(f)$ back to the time domain (Figure~\ref{fig:Fig1_filter4}).
\PSfig[h]{Fig1_filter3}{Truncation of high frequencies is straightforward in the frequency domain.}

We discover that the output has a lot of ``ripples'' (here somewhat exaggerated) at approximately the period corresponding to $1/f_{\mbox {cut}}$.  
This ``ringing'' is caused by our convolving of $d(t)$ with $p(t)$, which in this case is a $\sinc$ function.  Because all power 
at higher frequencies are completely eliminated, the power at the remaining highest frequency $f_{\mbox{cut}}$
stands out.  This effect is referred to as ``Gibbs' phenomenon''.
\index{Ringing}
\index{Gibbs' phenomenon}

\PSfig[H]{Fig1_filter4}{``Ringing'' in the time domain due to truncation in the frequency domain.}
	It is clear from these two examples of the use of a rectangular-shaped function that the price we 
pay for having a sharp truncation of filter coefficients in one domain is excessive ``ringing'' in the 
other domain.  This, of course, is simply the convolution theorem at work.  A rectangle function 
is necessarily a poor choice for a filter because of the slowly decaying oscillatory nature of the 
$\sinc$ function.

\subsubsection{Gaussian filter}
\index{Filter!Gaussian}
\index{Gaussian filter}
\label{sec:gaussianfilter}
\PSfig[H]{Fig1_FilterWidth}{Gaussian filter is typically specified by its ``full width'', defined to be
$W = 6\sigma$.  While a Gaussian curve never reaches zero weight, its amplitude will have fallen to
$\sim 1$ percent of peak value at $t = \pm 3\sigma$.}
While we do desire to find a filter that rapidly tapers off power, say, beyond a certain
frequency, we know that in the limit, when the gradual taper approximates a step-function,
our transform will approximate a $\sinc$-function. How do we find a good filter?
It will depend on the application. Because of the inverse relationship between time and frequency (i.e., $f =
1/T)$, we saw that the \emph{scale theorem}\index{Scale theorem}\index{Theorem!scale} (\ref{eq:scaletheorem}) stated
$$
g(at)\leftrightarrow \frac{1}{|a|} G \left (\frac{f}{a} \right ).
$$
Thus, making a filter broader in one domain narrows it in the other. In the limit $(T \rightarrow \infty )$
we recover the transform pair
$$
1 \leftrightarrow \delta(t).
$$
Clearly, somewhere in the middle there must be a function that looks similar in both
domains, i.e.,
$$
g(t) \leftrightarrow G(f) = g(f).
$$
There is such a function; here we will simply state that the Gaussian normal distribution
behaves this way. Let
$$
g(t)=e^{-\pi t^2}.
$$
To find its transform we must follow (\ref{eq:FT1D}) and integrate
$$
G(f) = \int^{+ \infty}_{-\infty}
e ^{- \pi t^2} e ^{-2 \pi ift} dt =
\int ^{+ \infty}_{-\infty}
e^{-(\pi t ^2 + 2 \pi ift)}
dt.
$$
Notice that the exponent is almost of the form $(a+b)^2$. We complete the square by adding and
subtracting the missing term:
$$
G(f) = \int^{+ \infty}_{-\infty} e ^{-( \pi t ^2 + 2 \pi ift + i^2 \pi f ^2 -i ^2 \pi f^2)}
dt = \displaystyle \int ^{+ \infty}_{-\infty} e^{-( \sqrt{\pi} t + i \sqrt{\pi} f ) ^2 } e^{i^2 \pi f^2} dt.
$$
With $u = \sqrt{ \pi} t + i \sqrt{\pi} f $ and $du = \sqrt{\pi} dt$ we get
\begin{equation}
G(f) = \frac{1}{\sqrt{\pi}} e ^{- \pi f^2} \int ^{+ \infty}_{-\infty} e^{-u^2} du = e ^{- \pi f^2} = g (f),
\label{eq:gaussfilt1d}
\end{equation}
since the definite integral equals $\sqrt{\pi}$.  Filters based on the Gaussian function are some of the most used filters in data analysis and
data processing. Because of their smooth transform properties we know that they will
minimize ringing in the other domain.
\PSfig[h]{Fig1_gfilt_time}{(a) Raw topography profile shows considerable short-wavelength noise.
(b) Topography after application of a 80-km full-width Gaussian filter.  Note we lose 40 km of data
at either end, corresponding to the filter's half-width.}

\subsubsection{Butterworth filter}
\index{Butterworth filter}
\index{Filter!Butterworth}
\PSfig[h]{Fig1_BWfilter}{Various Butterworth filters for different orders.  The higher
the order (a), the more ringing will be apparent in the time-domain (b).}
	Another well-known set of frequency-domain filters has the functional form
\begin{equation}
%		(22.3)
G(f) = \sqrt{\frac{1}{1 + (f/f_0) ^{2n}} }.
\end{equation}
These are the so-called \emph{Butterworth} filters and are often used for low-pass filtering purposes.
The frequency $f_0$ defines the halfway point of the filter: The power of $G(f_0)$ is always 1/2,
regardless of the exponent $(2n)$. The value of $n$ determines how fast the filter falls off.
The higher the value of $n$, the faster the filter will drop off beyond $f_0$.
We see the sharper the drop off, the more ringing in the time domain, again reflecting Gibbs'
phenomenon caused by truncating the spectrum too rapidly. As always, there will have
to be a trade-off between how sharply you want to reduce the spectrum and how much
ringing you are willing to tolerate.\\

\subsubsection{Wiener filter}
\index{Wiener filter}
\index{Filter!Wiener}
\label{sec:Wiener}
Most observed time series can be thought of as a sum of two components:

\begin{enumerate}
\item The signal, $u(t)$, that we want to analyze,
\item The noise, $n(t)$.
\end{enumerate}
The measured signal is therefore corrupted by the noise. We will call this the \emph{observed signal},
$c(t)$. In addition, the measuring process may not be able to record all frequencies present in the phenomenon being observed.
Hence, the true signal $u(t)$ is ``blurred'' or smeared, which we can describe as a convolution
of $u(t)$ with the known instrument response, $r(t)$. The smeared signal is then
\begin{equation}
s(t)=r(t)*u(t)=\int^\infty_{-\infty} r(t-\tau)u(\tau)d \tau \quad \leftrightarrow \quad S(f)= R(f) \cdot U(f).
		%(16.1)
\end{equation}
Add in the noise component, $n(t)$, and we have
\begin{equation}
c(t)=s(t)+n(t)  \quad \leftrightarrow  \quad C(f)=S(f)+N(f).
\label{eq:DA2_16.2}
		%(16.2)
\end{equation}
In the absence of noise we can easily invert (or deconvolve) this equation by solving $U (f) =
S(f)/R(f)$. However, with noise a different method must be applied. We want to find
the optimal filter, $\phi(t)$ [or $\Phi(f)$] which, when applied to the measured signal $c(t)$ [or
$C(f)$], and then deconvolved by $r(t)$ [or $R(f)$], produces a signal $\hat{u}(t)$ [or $\hat{U}(f)$] that is as
close as possible to the uncorrupted signal $u(t)$ [or $U (f)$]. In other words, we will
estimate the true signal by
\begin{equation}
\hat{U}(f) = \frac{C(f) \cdot \Phi (f)}{R (f)}.
\label{eq:DA2_16.3}
%(16.3)
\end{equation}
What do we mean by being ``close'' to the true signal? We mean in a least-square sense, i.e.,
\begin{equation}
%		(16.4)
\min \int^\infty _{- \infty} \left [ \hat{u} (t) - u(t) \right ]^2 dt \leftrightarrow \min \int^\infty _{- \infty}
\left [ \hat{U} (f) - U (f) \right ]^2 d f,
\label{eq:percivalth}
\end{equation}
where we have used Parseval's theorem to express the misfit in the frequency domain.
With the transform of the noise given by $N(f)$, we substitute (\ref{eq:DA2_16.2}) and (\ref{eq:DA2_16.3})
into (\ref{eq:percivalth}) to find
$$
\min \int^\infty _{- \infty} \left \{
\frac{\left [ S(f) + N (f) \right ] \cdot \Phi (f)}{R (f) }
- \frac{S (f)}{R(f)} \right \} ^2 d f.
$$
Carrying out the square we obtain
$$
\min \int^\infty _{- \infty}
\frac{1}{ R^2 (f)}
\left \{ [ S^2 (f) + 2S (f) N (f) + N^2 (f) ] \Phi ^2 (f) - 2S (f) [ S(f) + N (f)] \Phi (f) + S^2 (f) \right \} d f,
$$
which simplifies to
\begin{equation}
%		(16.5)
\min \int^\infty _{- \infty}
\frac{1}{ R^2 (f)} \left \{ S^2 (f) [ 1 - \Phi (f)]^2 + N^2 (f) \Phi (f) \right \} d f.
\label{eq:DA2_16.5}
\end{equation}
This simplification is valid because we assume $S$ and $N$ are uncorrelated, hence their product integrated over all
frequencies must equal zero.

Obviously, (\ref{eq:DA2_16.5}) is only minimized if the integrand is minimized for all frequencies. Thus, we
can determine the best choice for $\Phi (f)$ by taking the derivative of the integrand with
respect to $\Phi$ and setting it to zero:
$$
-2S ^2 (f) \left [ 1 - \Phi (f) \right ] + 2 N^2 (f) \Phi (f) = 0.
$$
Since this identity must hold for all frequencies we solve for the optimal filter as
\index{Wiener, N.}
\begin{equation}
\Phi (f) = \frac{S^2 (f)} {S^2 (f) + N^2 (f)} = \frac{S^2(f) } {C^2(f)}.
%(16.6)
\label{eq:DA2_16.6}
\end{equation}
\PSfig[h]{Fig1_Wienerfilter}{An optimal Wiener filter depends on the practitioner's ability
to separate the spectrum ($C^2$) into the contributions from the signal ($S^2$) and the noise ($N^2$).}
This is the optimal filter known as the \emph{Wiener} filter, named after \emph{Norbert Wiener}. Note that it only involves the power of
$S$, the observed smeared signal, and the power of $N$, the noise. In particular, equation (\ref{eq:DA2_16.6}) does not contain $U$, the true
signal. This simplifies life: We can determine $\Phi(f)$ independently of $R(f)$. However, we
need to isolate $S^2$ and $N^2$ from $C^2$. There is no unique way to do that unless we have some
extra information. Luckily, a way out is often presented by looking at the spectrum $C^2$.
Often, it will be clear what shape the signal and noise spectra must have. Consider the
situation sketched in Figure~\ref{fig:Fig1_Wienerfilter}.
It appears that the noise spectrum is slightly tilted. We simply extrapolate this trend for all
frequencies and subtract it from $C^2$ to get $S^2$. We can now numerically form the filter $\Phi(f)$ from $S^2$ and $N^2$. As
you can see, the filter will be unity where the noise is minimal and drop smoothly to zero
where the noise dominates. Simple, but very powerful, and not very sensitive to errors in
isolating $S^2$ and $N^2$. In fact, a crude separation by eye based on a power spectrum is
usually adequate.


\subsection{Time-domain filters}
Many useful filters cannot be specified in the form of a convolution and thus cannot be made
extremely efficient via the magic of the convolution theorem and the availability of FFTs.  These time-domain (or space-domain)
filters must be executed on the data directly and may take considerable computational power.
\subsubsection{Median filter}
\index{Median filter|(}
\index{Filter!median}
\PSfig[h]{Fig1_MA_filterspike}{Convolution filters have a hard time when data sets contain outliers.}
	We have assumed for most of the time that all filtering can be described by a
convolution, and in general this is the case. The advantage of being able to describe
filtering as a convolution is very important: Thanks to the convolution theorem we can
simply transform both data and filter into the frequency domain, perform a multiplication
and take the inverse transform. This approach can be vastly faster than computing the convolution
in the time domain.  Why would we give up that advantage? Consider again a discrete
MA filter of a finite width, say
$$
g = \left \{ \frac{1}{5} \frac{1}{5} \frac{1}{5} \frac{1}{5} \frac{1}{5} \right \}.
$$
When convolved with data, it simply returns the mean value of the points inside the filter
width. However, we know the mean is a least squares estimate of ``average'' value. What
happens if we have the occasional bad data point? Because the filter uses an $L_2$ norm it returns bad values for
the entire filter width (see Figure~\ref{fig:Fig1_MA_filterspike}).
Clearly, this is not a desirable result. However, we can design a more \emph{robust} filter by
using a more robust estimate of ``average'' value. A good estimator that is insensitive
to the occasional outlier is the \emph{median}\index{Median}, defined as
$$
\tilde{x} = \left\{ \begin{array}{cl}
x_{\frac{n+1}{2}}, & n \ \mbox{is odd}\\
\frac{1}{2} \left ( x_{\frac{n}{2}} + x_{\frac{n+1}{2}} \right), & n \ \mbox{is even}
\end{array} \right.,
$$
where it is assumed the $x_i$ have been sorted.
\PSfig[h]{Fig1_median_filterspike}{Median filters excel in eliminating single or narrow sets of outliers.}
Unfortunately, the median is not an analytic function and has no transform (ponder what
its ``impulse response'' would be!). Hence, we are forced to calculate the result of median
filtering in the time domain only, as convolution is not applicable. An $n$-point median
filter will, for each lag, return the median value of the current $n$ points. Thus, the filtering
of the previous data example would result in the removal of the spike (Figure~\ref{fig:Fig1_median_filterspike}).
In addition to being robust, i.e., insensitive to outliers, a median filter also preserves step-functions,
provided the step length exceeds half the filter width.
\PSfig[h]{Fig1_filter_steps1d}{Convolution filters (top) will blur sudden steps while median filters (bottom)
will pass the steps unchanged.}
For data that consist of a noisy signal superimposed on a step-like background, median
filters are very useful  (see Figure~\ref{fig:Fig1_filter_steps1d}). Consider the task of finding the ``regional'' depth from
bathymetry data. The regional depth is ``contaminated'' by seamounts and faults and
may also abruptly increase or decrease across fracture zones (which are approximately step functions due to jumps in
crustal age).
	Finding the median of all points inside the filter width at each output point seems like
an expensive operation since the data must be sorted for each output location. Fortunately, a simple
iterative scheme may be used to speed up operations. Earlier, we found that the median satisfies the equality
\begin{equation}
\sum^n _{i=1} \frac{x_i - \tilde{x}}{|x_i - \tilde{x}|} = 0,
\label{eq:median_equation}
\end{equation}
since this ratio can only take on the values $-1, 0, +1$. With $\tilde{x}$ being defined as having as many values
above as below it (and any number of values may be equal to it) the sum must necessarily equal zero.
We may rewrite (\ref{eq:median_equation}) as
$$
\sum^n _{i=1} \frac{x_i}{|x_i - \tilde{x}|} -
\sum^n _{i=1} \frac{\tilde{x}} {|x_i - \tilde{x}|} = 0,
$$
which can be used to determine the median via successive iterations, i.e.,
\begin{equation}
\tilde{x}_k = \frac{\displaystyle \sum^n _{i=1} \frac{x_i}{|x_i - \tilde{x}_{k-1}|} } {\displaystyle \sum^n _{i=1} \frac{1}{|x_i - \tilde{x}_{k-1}|} },
\label{eq:median_iteration}
\end{equation}
where we obtain the $k$'th estimate of the median based on the previous value $\tilde{x}_{k-1}$.  Initializing $\tilde{x}_0$ to the mid-point between the minimum and maximum
data value, this scheme converges fairly quickly (e.g., Figure~\ref{fig:Fig1_medsearch}).  This is especially true for filtering since we would expect the median
output from the previous filter position to be similar to the median at the next output location.  In fact, the previous filter output is usually a good starting point
for the next output position.
\PSfig[h]{Fig1_medsearch}{We iteratively apply (\ref{eq:median_iteration}), and after seven iterations find the median ($\tilde{x} = -4956$) for a bathymetric data set with 11,313 points.}
\index{Median filter|)}

\subsubsection{Mode filter}
\index{Mode filter}
\index{Filter!mode}
\index{Least median of squares (LMS)}
\index{LMS (Least median of squares)}
	Finally, one can design a filter based on the \emph{mode}\index{Mode} rather than the median. Such filters
are called \emph{maximum likelihood} filters and return the most frequently occurring value
within the filter width. These filters are used to track representative levels through very noisy
data. We can implement such a filter using the \emph{Least Median of Squares} (LMS) approximation to the mode, as
discussed in great detail in Chapter~\ref{ch:regression}, or one can use a histogram (binning) approach and select the center
of the bin with most points.  Mode filtering has been used extensively in bathymetric studies to recover regional depths in the
presence of seamounts and plateaus.

\index{Least median of squares (LMS)}
\index{LMS (Least median of squares)}

\subsubsection{High-pass, low-pass, and band-pass filters}

As their names imply, these are filters that seek to retain a limited part of the frequency
content in the signal. Sometimes high-pass is called low-cut and low-pass is called high-cut,
for obvious reasons. In the frequency domain, low-pass and high-pass filters are illustrated
in Figure~\ref{fig:Fig1_low_high_filter}.
\PSfig[h]{Fig1_low_high_filter}{(left) A typical low-pass filter will
leave the low-frequency components unchanged and attenuate high-frequency components. (right) A high-pass filter
is complimentary and reverses what is being attenuated.}
\index{Filter!band-pass}
\index{Filter!low-pass}
\index{Filter!high-pass}
It is straightforward to design a high-pass filter that is complementary to the low-pass filter using
\begin{equation}
H (f) = 1 - L (f).
\label{eq:highpassf}
\end{equation}
In the time-domain, we find the corresponding filter as the inverse transform of $H(f)$:
\begin{equation}
H (f) = 1 - L (f) \leftrightarrow h(t) = \delta (t) - l (t).
\label{eq:highpasst}
\end{equation}
Thus, high-pass filtering the data $d(t)$ in the time domain is described by
$$
y(t) = d(t) * h (t) = d(t)* [\delta (t) - l(t)]
= d(t)* \delta (t)- d(t)* l (t) = d(t) - d(t) * l(t).
$$
This operation simply says that we high-pass-filter a data set by first using a (complementary) low-pass
filter and then subtract the output from the original signal.
Band-pass filters are simply a linear combination of high-pass and low-pass filters. You can
either design $B(f)$ directly or multiply data by $H(f)$ then by $L(f)$ (or two convolutions in
the time domain) or let $B(f) = L(f)\cdot H(f)$.
Note that (\ref{eq:highpasst}) is not restricted to convolution filters only.  Any filter, such as
our spatial median and mode filters, that can
be applied to data may be used to compute the complementary high-pass filter simply by subtracting
the low-passed output from the original observations.
\PSfig[h]{Fig1_band_filter}{A band-pass filter is simply a combination of a low- and high-pass filter
with different cutoff frequencies, which results in a certain band of frequencies being relatively unaffected.}

\index{Filtering!Gaussian}
\index{Filtering|)}

\clearpage
\section{Problems for Chapter \thechapter}

\begin{problem}
	The monthly flow of water in Cave Creek, Kentucky are given in the file \emph{cavecreek.txt},
	with the line number indicating months beginning with October, 1952.  Plot the data and compute the
	power spectrum.  What is the Nyquist frequency? Determine the most significant peak.  What period does it correspond to? Is the peak significant
	at the 95\% level of confidence?
\end{problem}

\begin{problem}
	The daily average temperatures in Billings, Montana (in \DS F) are given in file \emph{Billings.txt},
	with the first record corresponding to January 1, 1995.  Plot the data and compute the
	power spectrum.  What is the Nyquist frequency for this data set? Find the most significant peak in the spectrum.
	What period does it reflect, and is the peak significant at the 95\% level of confidence?
\end{problem}

\begin{problem}
	The numbers of sunspots recorded per month for the period 1749--2015 are given in \emph{sspots.txt}.
	Plot the data and compute the power spectrum.  What is the Nyquist frequency for this data set? Find the most significant peak in the spectrum.
	What period does it reflect, and is the peak significant at the 95\% level of confidence?
\end{problem}

\begin{problem}
	The daily discharge of the Columbia River (in m$^3$/day) are given in the file \emph{columbia.txt}, covering
	most of the interval 2001--2008. Plot the data and compute the
	power spectrum.  What is the Nyquist frequency? Determine the most significant peak.  What period does it correspond to? Is the peak significant
	at the 95\% level of confidence?
\end{problem}
	
\begin{problem}
	Revisit problem~\ref{ch:regression}.\theEDA\ and 
	remove the linear trends from the data.  Use \texttt{dft.m} (type \texttt{help dft}) to get the amplitudes $a_j$ and $b_j$ for 
	each Fourier series.  Make a plot of the raw power spectrum for each time-series using periods rather 
	than frequency on the $x$-axis (skip the zero frequency which represents the infinite period).  What is the 
	Nyquist period?
	What are the dominant periods in the data?  How well do the two series agree on this issue (a 
	qualitative answer is fine).
\end{problem}

\begin{problem}
	Scientists from Scripps Institution of Oceanography have been measuring the CO$_2$ concentration on top of Mauna Loa since 1958 (\url{http://scrippsco2.ucsd.edu}). The
	table \emph{CO2.txt} shows the record up to 2015 with decimal year versus CO$_2$ in ppm.  The second column
	contains the raw data, while the third column contains a few interpolated values when raw data were missing (this
	makes the data continuous and suited to Fourier analysis).
	\begin{enumerate}[label=\alph*)]
		\item Plot the data and determine the least-squares quadratic trend.
		\item Remove this trend from the data and subject the residuals to spectral analysis.
		Determine the two strongest spectral peaks in the signal.  What periods do these represent?
	\end{enumerate}
\end{problem}

\begin{problem}
	Revisit the noisy single-period data set \emph{noisy.txt} discussed in Problem~\thenoisyc.\thenoisyp.
	Remove the least-squares regression line to detrend the data
	and compute the power spectrum of the residuals.  Do you recover the single period you
	found earlier? Why might you get a different result in this analysis?
\end{problem}

\begin{problem}
	We will revisit problem~\ref{ch:sequences}.\theVostok\ where we discussed
	the 3-km long Vostok ice core from Antarctica that resolves temperature variations relative to the present via
	oxygen isotopes. These data are given in table \emph{vostok.txt}, which contains equidistant depths (in meter), the
	corresponding times (in year) and the relative change in temperature (in $^{\circ}$C).
	Because of compaction, depth is not a good proxy for time, especially for the deeper (older) sections.
	To analyze the temporal periodicities we thus need an equidistant time-series. Use MATLAB's \texttt{spline}
	or other software to interpolate the data onto an equidistant interval in time ($\Delta t = 25$).  
	What is the Nyquist frequency for this data set? Determine the most significant peak.  What period does it correspond to?
	Is the peak significant at the 95\% level of confidence?
\end{problem}
