% $Id: DA1_Chap2.tex 685 2019-01-12 07:54:38Z pwessel $
%
\chapter{REVIEW OF ERROR ANALYSIS}
\label{ch:error}
\epigraph{``An error does not become truth by reason of multiplied propagation, nor does the truth become error because nobody will see it.''}{\textit{Mahatma Gandhi, India Independence Leader}}

\index{Error analysis|(}
	Error analysis is the study and evaluation of uncertainty in measurements of continuous data 
(discrete data may have no errors).  We know that no measurement, however carefully made, 
can be completely free of uncertainties.  Since science depends on measurements, it is 
crucially important to be able to evaluate these uncertainties and to keep them to a minimum.  
Thus errors are not mistakes and you cannot avoid them by being very careful, but you should strive to make them 
as small as possible.  Understanding uncertainties is critical for determining the significance of models
that you may construct to explain your data.  In this chapter we will review the basic rules that govern
the reporting of measurements with uncertainties and how these uncertainties propagate when used to
obtain derived quantities (such as sums and products) as well as how they affect the uncertainties in
more complicated situations (such as being arguments to nonlinear functions).

\section{Reporting Uncertainties}
\index{Reporting uncertainties}
\index{Uncertainty!reporting}
\index{Data!uncertainties}

	When reporting the value of a measurement, care should be taken to give the best possible 
estimate of the uncertainty or error in the measurement.  Values read off scales or measured with 
mechanical instruments can usually be bracketed between lower and upper limits.  E.g., we may know
that a temperature measurement $T$ is not less  than 23\DS\ and not larger than 24\DS.  Hence, we
shall report the value of $T$ as
\begin{equation}
T = 23.5 \pm 0.5^{\circ},
\end{equation}
or, in general, $x \pm \delta x$.  For many types of measurements we can state with absolute certainty that 
$x$ must be within these bounds.  Unfortunately, very often we cannot make such a categorical 
statement.  To do so, we would have to specify unreasonably large values for $\delta x$ to be absolutely confident that 
the actual quantity lies within the stated interval.  In most cases, we will lower our confidence to, 
say, 90\% and use a smaller $\delta x$.  We need more detailed knowledge of the statistical laws that 
govern the process of measurement for finding a suitable $\delta x$, so we will return to this issue later in this book.

\subsection{Significant figures}
\index{Significant figures}
\index{Data!significant figures}

In general, the last significant figure of any stated answer should be of the same 
order of magnitude (i.e., same decimal position) as the uncertainty.
For intermediate calculations you should use one extra decimal (if calculating by 
hand) or all available decimals (if using calculators or computers).
\begin{example}
If a measured distance $d$ is 6051.78 m with an uncertainty of $\pm 30$ m, you should report the 
distance as $d = 6050 \pm 30$ m.  Similarly, if a time $t$ is measured to be 3 seconds and the
uncertainty is given as 1 part in 100, you should report the time as $t = 3.00 \pm 0.03$ s.
\end{example}
\section{Fractional Uncertainty}
\index{Uncertainty!fractional}
\index{Fractional uncertainty}
\index{Precision}
\index{Data!precision}

While the statement $x = x_0 \pm \delta x$ indicates the \emph{precision} of the measurement, it is clear that 
such a statement will have different meanings depending on the value of $x_0$ relative to $\delta x$.  
Clearly, with $\delta x = 1$ m, we imply a different precision for $x_0 = 3$ m than for $x_0 = 1000$ km.  Thus, we 
should consider the \emph{fractional uncertainty}, written as
\begin{equation}
\displaystyle \frac{ \delta x}{\left| x_0\right |}.
\end{equation}
Note that the fractional uncertainty is a \emph{nondimensional} quantity.
\begin{example}
We wish to report the uncertainty in the length of a 3m crocodile, which (due to a shortage of available arms)
we only were able to measure with a precision of 6 cm.  Using fractional uncertainty,
we report the uncertainty $\delta l = \pm 100 \frac{0.06}{3}$\% = $\pm2$\%.
\end{example}

\section{Uncertainty in Derived Quantities}
\index{Uncertainty!derived quantities}
It is common to obtain measurements (and estimate their uncertainties) and then use these values in expressions
that lead to derived quantities.  In such situations we must be careful to \emph{propagate} the uncertainties
of the initial measurements so that the derived quantities can be reported with their combined uncertainties.
The answers will differ depending on whether or not the individual measurements are
\emph{independent} or \emph{dependent} on each other.

\subsection{Uncertainty in sums and differences}

Consider the two values $x \pm \delta x$ and $y \pm \delta y$.  We would like to determine the correct
expressions for the uncertainties in the four derived quantities
\begin{equation}
	s = x + y,
\end{equation}
\begin{equation}
d = x - y,
\end{equation}
\begin{equation}
p = x  \cdot y,
\end{equation}
\begin{equation}
q = x / y.
\end{equation}
For the sum, common sense would suggest that the maximum value of $s$ must be $s = x + y + \delta x + \delta y$, while the minimum value is 
$s = x + y - \delta x- \delta y$.  Thus
\begin{equation}
\index{Uncertainty!sums}
s \approx (x + y) \pm (\delta x + \delta y ) = s_0 \pm \delta s
\end{equation}
and similarly for differences we reason that
\begin{equation}
\index{Uncertainty!differences}
d \approx (x - y) \pm (\delta x + \delta y) = d_0 \pm \delta d.
\end{equation}
Here, the subscript $0$ indicates the nominal or ``best'' value of the results.
We use the approximate sign $\approx$ since we anticipate that the stated uncertainties $\delta s$ and $\delta d$ 
probably overestimate the true uncertainties in the sum and difference, respectively.
Note that the uncertainties in both the sum and the difference of $x$ and $y$ are identical.
\subsection{Uncertainty in products and quotients}
For the product, we first use fractional uncertainties and rewrite $x$ and $y$ as
\begin{eqnarray*}
x & = & x_0 \left ( 1 \pm \frac{\delta x} {\left | x_0\right |} \right ),\\[10pt]
y & = & y_0 \left ( 1 \pm \frac{\delta y} {\left | y_0\right |} \right ).
\end{eqnarray*}
Then, the maximum value of $p$ is
\begin{equation}
\index{Uncertainty!products}
p_{high} = x_0 \left( 1 + \frac{\delta x} {\left | x_0 \right |} \right)
\cdot y_0  \left( 1 + \frac{\delta y} {\left | y_0\right |} \right),
\end{equation}
which becomes
\begin{equation}
\index{Uncertainty!products}
p_{high} = p_0  \left( 1 + \frac{\delta x} {\left | x_0\right |} +
\frac{\delta y} {\left | y_0\right |} + \ldots \right),
\end{equation}
where the higher order term proportional to $\delta x \cdot \delta y$ has been ignored.  We note that the best value is
\begin{equation}
p_0 = x_0 \cdot y_0.
\end{equation}
The minimum value is found by reversing the signs of $\delta x$ and $\delta y$.  Both expressions yield the uncertainty in 
$p$ as
\begin{equation}
\frac{\delta p} {\left | p_0\right |} \approx \frac{\delta x} {\left | x_0\right |} +
\frac{\delta y} {\left | y_0 \right |}.
\end{equation}	 
 For quotients, the maximum value will be
\begin{equation}
q_{high} = \frac{   x_0 \left ( 1 + \frac{\delta x} {\left | x_0\right |}\right) }
 { y_0 \left( 1 -  \frac{\delta y} {\left | y_0 \right |} \right)} =
q_0 \frac{1 + a}{1 -b}. 
\end{equation}
Using the binomial theorem, we expand $(1-b)^{-1}$ as 
$1+b+b^2+b^3\ldots$, hence
\begin{equation}
\frac{1+a}{1-b} \approx (1+a) (1+b)= 1+a + b + ab \approx 1 + a + b,
\end{equation}	 
where we again ignore higher-order terms in $b$ by assuming that $(\delta x/|x_0|) \ll 1$ and $(\delta y/|y_0|) \ll 1$.
Similarly, for the minimum value, find
\begin{equation}
\frac{1 - a}{1+b} \approx (1- a) (1- b)= 1 - a - b + ab \approx 1 - ( a + b).
\end{equation}	 
It therefore follows that
\begin{equation}
	\index{Uncertainty!quotients}
	q = q_0 \left[ 1 \pm \left( \frac{\delta x} {\left | x_0 \right |} +
	\frac{\delta y }{\left | y_0 \right |} \right) \right] =
	q_0 \left[ 1 \pm \frac{\delta q}{q_0} \right]
\end{equation}
and that the fractional uncertainty for both products and quotients are the same.

\subsection{Uncertainty for general expressions}
The stated uncertainties are the maximum values possible, but we suspect these are likely to be exaggerated.
Later, we will show that if we assume our errors to be \emph{normally} distributed (i.e., we have 
``Gaussian'' errors) and our measurements are \emph{independent}, then a better estimate of the uncertainty in 
sums and differences is 
\begin{equation}
\delta s = \delta d = \sqrt{ (\delta x)^2 + (\delta y) ^2},
\end{equation}
	and for products and quotients it becomes
\begin{equation}
\frac{\delta p }{p_0} = \frac{\delta q}{q_0} = 
\sqrt{ \left (  \frac{\delta x}{x_0} \right)^2 + \left( \frac{\delta y}{y_0}  \right)^2 }.
\end{equation}	 
Note that in the case $s = nx$, where $n$ is a constant, we must use $\delta s = n\delta x$ since all the $x$ are the 
same and obviously \emph{not independent} of each other.  Similarly, the product $p = x^n$ will have the 
fractional uncertainty
\begin{equation}
\frac{\delta p }{p_0} = n \frac{\delta x}{x_0},
\label{eq:poweruncertainty}
\end{equation}
since the (repeated) measurements are \emph{dependent}.  In conclusion, if $s = x + y + nz - u - v - mw$, then
\begin{equation}
\delta s = \sqrt{ (\delta x)^2 + (\delta y)^2 + (n \delta z)^2 +(\delta u)^2 + (\delta v)^2
+ (m \delta w)^2    }.
\end{equation}
Even if our assumption of independent measurements is incorrect, $\delta s$ cannot exceed 
the ordinary sum
\begin{equation}
\delta s = \delta x + \delta y + n \delta z + \delta u + \delta v + m \delta  w.
\label{eq:uncert_sum}
\end{equation}
Similarly, if
\begin{equation}
q = \frac{x \cdot y \cdot z^n}{u \cdot v \cdot w ^m},
\end{equation}
then we find the fractional uncertainty to be
\begin{equation}
\frac{\delta q }{q_0} = \sqrt{ \left ( \frac{\delta x}{x_0}   \right )^2 +
\left ( \frac{\delta y}{y_0}   \right )^2 + 
\left ( n \frac{\delta z}{z_0}   \right )^2 + \left ( \frac{\delta u}{u_0}   \right )^2  + \left ( \frac{\delta v}{v_0}   \right ) ^2 + 
\left ( m \frac{\delta w}{w_0}   \right )^2         }.
\label{eq:uncert_prod}
\end{equation}
While this is the likely error for independent data, we can confidently say that the fractional 
uncertainty will be less than the linear sum
\begin{equation}
\frac{\delta q}{q_0} = \frac{\delta x}{x_0} + \frac{\delta y }{y_0} +
n  \frac{\delta z}{z_0} +  \frac{\delta u}{u_0} +  \frac{\delta v}{v_0} +
m  \frac{\delta w}{w_0}.
\end{equation}
\begin{example}
As most students who have taken an introductory physics lab in mechanics will know, measuring
the period $T$ of a pendulum of length $\ell$ lets one estimate the 
acceleration of gravity as 
\begin{equation}
g = \frac{4\pi^2 \ell}{T^2}.
\end{equation}	 
We wish to determine the uncertainty in this estimate given measurements of $\ell$ and $T$
and their uncertainties.  Being independent measurements we use (\ref{eq:uncert_prod}) and find
\begin{equation}
\frac{\delta g}{\left| g_0    \right |} = 
\sqrt{ \left( \frac{\delta \ell}{\ell}   \right)^2 + \left( 2 \frac{\delta T}{T}   \right)^2      },
\end{equation}	 
since the constants $4$ and $\pi $ have no uncertainty.  Given the measurements
$\ell  = 92.95 \pm 0.10$ cm and $T = 1.936 \pm 0.004$ s, we obtain
\begin{equation}
g_0 = \frac{4 \pi^2 0.9295\mbox{ m}}{1.936^2\mbox{ s}^2} = 9.79035 \mbox{ m s}^{-2}.
\end{equation}	 
We can now evaluate the fractional uncertainty as     
\begin{equation}
\frac{\delta g}{\left| g_0    \right |} = 
\sqrt{ \left( \frac{0.1}{92.95}   \right)^2 + \left( 2 \frac{0.004}{1.936}   \right)^2 } \approx 0.4\%.
\end{equation}	 
The answer, therefore, is
\begin{equation}
g = 9.79 \pm 0.04 \mbox{ m s}^{-2},
\end{equation}
where we have only used two significant decimals.
\end{example}

\PSfig[h]{Fig1_Kelvin}{Lord Kelvin's model for the vertical temperature profile of the Earth
(the \emph{geotherm}\index{Geotherm}) at a time
$t_0$ since its initial formation at a constant temperature $T_0$ (dashed vertical geotherm).  The
tangent to the curve at the surface represents the vertical temperature gradient ($G$), which could
be estimated from temperature measurements in mines.}

Another case study revisits the debate that raged in the 19th century regarding the age of the Earth.
Observing the slow process of erosion, Charles Darwin\index{Darwin, C.} had implied that perhaps the Earth might be
as old as 300 million years.  Lord Kelvin, the preeminent physicist of his times, strongly objected
and set out to calculate the age using the conductive cooling of the Earth (this and many other fascinating
stories from the development of the geological sciences are portrayed in the classic book, \emph{Great Geological Controversies} by
A. Hallam [Oxford University Press]).
\index{Great geological controversies}
\begin{example}
\index{Lord Kelvin}
Lord Kelvin assumed the whole Earth
was once at a uniform temperature $T_{0}$ and had since cooled at the surface to $\sim 0^{\circ}$C (as indicated in Figure~\ref{fig:Fig1_Kelvin}).
Then, the physics of heat conduction in solids dictates that
\begin{equation}
	t_{0} = \frac{T_{0}^{2}}{\pi \kappa G^{2}},
\end{equation}
with initial temperature $T_{0}  \approx 2000 \pm 200\mbox{ }^{\circ}\mbox{K}$, thermal diffusivity
$\kappa = 1 \pm 0.25$ mm$^{2}$s$^{-1}$, and observed near-surface temperature gradient
$G = 25 \pm 5\mbox{ }^{\circ}\mbox{K}$ km$^{-1}$.  Given the procedures established earlier, we first determine
\begin{equation}
	\frac{\Delta t_{0}}{t_{0}} = \left[ \left(2\frac{\Delta T_{0}}{T_{0}}\right)^{2} + \left(\frac{\Delta \kappa}{\kappa}\right)^{2} + \left(2\frac{\Delta G}{G}\right)^{2} \right]^{\frac{1}{2}}.
\end{equation}
Inserting the estimated parameters, we find

\begin{equation}
	\frac{\Delta t_{0}}{t_{0}} = \left[ \left(2\frac{200}{2000}\right)^{2} + \left(\frac{0.25}{1}\right)^{2} + \left(2\frac{5}{25}\right)^{2} \right]^{\frac{1}{2}} \approx 51\%.
\end{equation}
We evaluate Kelvin's estimate of the age of the Earth to be
\begin{equation}
	t_{0} = \frac{(2000\mbox{ }^{\circ}\mbox{K})^{2}}{\pi \cdot 10^{-6} \mbox{m}^{2} \mbox{s}^{-1} (25\mbox{ }^{\circ}\mbox{K} \cdot 10^{-3} \mbox{m}^{-1})^{2}} \approx 2 \cdot 10^{15} \ \mbox{s} \ = 65 \ \mbox{Myr}.
\end{equation}
Given the fractional uncertainty, we obtain $t_{0} = 65 \pm 33$ Myr.  As the debate raged on,
positions hardened and Lord Kelvin continued to revise his
estimates downwards, finally settling on 25 Myr.  Modern science estimates that the Earth is closer to
4.6 \emph{billion} years old.  Where did Kelvin go wrong?
\end{example}

\PSfig[h]{Fig1_digline}{Example of a coastline segment whose length we attempt to estimate using both
a compass and via digitizing.}

	As a final case, let us imagine we are measuring the length of the coastline segment 
in Figure~\ref{fig:Fig1_digline} using two 
different methods: (1) Set a compass to a fixed aperture $\Delta x = 1 \pm 0.025$ cm and march along the 
line counting the steps, and (2) use a digitizing tablet and sample the line approximately every $\Delta x = 1 \pm 0.1$ cm.  
Let us assume that it took $N = 50$ clicks or steps so the measured line length in both cases is 50 cm.  What is the 
uncertainty in the length for the two methods?  First, let us state that there will be an uncertainty 
for both methods that has to do with the undersampling of short-wavelength coastline ``wiggles'' (in the continuation
of this volume we will learn about the \emph{fractal} nature of coastlines and that perhaps our simple approach
here is a bit naive).  That 
aside, we can see that the errors accumulate very differently.  For the compass length-sum the 
errors are all dependent (since the aperture is fixed) and we must use the summation rule to find the uncertainty $\delta l = 
N\cdot 0.025$ cm $= 1.25$ cm.  For the digitizing operations all the uncertainties associated with points 2 
through 49 largely cancel and we are left with the uncertainty of the endpoints.  Those are clearly
independent and hence the uncertainty is $\delta l = (0.1^2 + 0.1^2)^{1/2}$ cm $= 0.14$ cm.  The systematic errors 
using the compass accumulate while the errors in digitizing only affect the end-points.  This 
discussion of digitizing errors is a bit oversimplified, but it does illustrate the difference between 
the two types of errors and how they accumulate.

\subsection{Uncertainty in a function}
\index{Uncertainty!function}

\PSfig[h]{Fig1_func_uncertainty}{As $\delta x$ becomes very small, the derivative of any well-behaved function can be approximated by a
\emph{tangent} at the point $(x_0, y_0 = y(x_0))$; this is the second term in Taylor's expansion.}

Many solutions to scientific or engineering problems require the evaluation of functions
with our uncertain measurements as arguments.
If $x$ is measured with uncertainty $\delta x$ and is used to evaluate the function $y = f(x)$, then the 
uncertainty $\delta y$ is related to the \emph{derivative} of the function at $x$, i.e.,
\begin{equation}
\delta y = \left | \frac{df}{dx}  \right |_{x_0} \cdot \delta x,
\label{eq:funcdir}
\end{equation}
where the derivative is evaluated at $x_0$ (e.g., Figure~\ref{fig:Fig1_func_uncertainty}).
\begin{example}
Let  $y(x) = \cos x$ and $x = 20 \pm 3^{\circ}$.  Following (\ref{eq:funcdir}),
\begin{equation}
\delta y = \left| \frac{d\cos (x)}{dx} \right |_{x=20^{\circ}} \left( \frac{\pi}{180^{\circ}}\right) 3^{\circ} = \left|-\sin 20^{\circ} \right| 
\frac{3\pi}{180} = 0.342 \cdot 0.0524 \approx 0.02,
\end{equation}	 
where we have converted the angle from degrees to radians (why?).  The final answer then becomes
\begin{equation}
y = \cos(x) = 0.94 \pm 0.02.
\end{equation}
\end{example}
Finally, for a function of multiple variables, $f(x,...,z)$, we extend our analysis to find
\begin{equation}
\delta f = \sqrt{ \left ( \frac{\partial f }{\partial x} \delta x \right) ^2 + \cdots +
\left( \frac{\partial f }{\partial z} \delta z  \right)^2 }
\label{eq:uncert_func}
\end{equation}
when $x,..., z$ are all random and independent.  As before, $\delta f$ cannot exceed the ordinary sum
\begin{equation}
\delta f \leq \left | \frac{\partial f}{\partial x} \right | \delta x + \cdots + 
\left | \frac{\partial f}{\partial z}\right | \delta z,
\end{equation}
which is suitable for dependent measurements.
\begin{example}
Consider the spherical function
\begin{equation}
f (r,\theta, \phi ) = \frac{1}{2} r^2 \cos^2 \theta \sin \phi.	 \label{eq:spherical}	\end{equation}
We measured the parameters and found $r = 10 \pm 0.1, \theta = 60 \pm 1^{\circ}$, and 
$\phi = 10 \pm 1^{\circ}$.  Using (\ref{eq:uncert_func}), the 
uncertainty in the evaluated expression in (\ref{eq:spherical}) is found as
\begin{equation}
\delta f = \sqrt{ (r \cos ^2 \theta \sin \phi \delta r )^2 + 
(-r^2 \cos \theta \sin \theta\sin \phi \delta \theta )^2 + 
\left( \frac{1}{2} r^2 \cos^2 \theta \cos \phi \delta \phi\right )^2},
\end{equation}
which means our final estimate of $f(r, \theta, \phi)$ evaluates to
\begin{equation}
f(r, \theta, \phi) = 2.17 \pm 0.26.
\end{equation}
\end{example}
While the simple expressions for uncertainty in a derived quantity (\ref{eq:uncert_sum} or \ref{eq:uncert_prod}) generally apply, one must
be careful with expressions where one or more of the measurements appear in different \emph{subgroups} within the
expression.  In such cases one must treat the entire expression as a function of several variables and apply the
general expression given in (\ref{eq:uncert_func}), as our next example illustrates.
\begin{example}
Given the multivariate function
\begin{equation}
	f(a,b) = 2ab^2 + \pi b + 1,
\end{equation}
we wish to find its value and uncertainty for $a = 0.3 \pm 0.02$ and $b = 1 \pm 0.01$.
Because we have more than one term that depends on $b$ we cannot easily employ the rules
in (\ref{eq:uncert_sum}) and (\ref{eq:uncert_prod}) but must use (\ref{eq:uncert_func}) instead.
We first evaluate $f(0.3,1)$ to be 5.7416... and
next evaluate the uncertainty using (\ref{eq:uncert_func}):
\begin{equation}
	\delta f = \sqrt{\left (2b^2 \delta a \right)^2 + \left [ (4ab + \pi)\delta b \right ]^2} = 0.05903....
\end{equation}
Consequently, this yields a final estimate of
\begin{equation}
	f = 5.74 \pm 0.06,
\end{equation}
where we have rounded the answer to two decimals only.
\end{example}
\index{Error analysis|)}

\clearpage
\section{Problems for Chapter \thechapter}
%See course website for any data sets.

\begin{problem}
	The distance to a building is estimated from a map to be $2550\pm25$ m.  With a theodolite,
a student determines the angle between the horizontal plane and the building roof to be 1\DS 21' $\pm 1$'.
What is the height of the building and the uncertainty in this estimate?
\end{problem}

\begin{problem}
	The area of a cornfield is being estimated from aerial photographs.  Because of an unknown stretching factor
	the uncertainty in linear distances has been set to 1\% and hence uncertainties are \emph{dependent}.  What is the
	area and uncertainty of a field with measured dimensions 235.5 m by 115.6 m?
\end{problem}

\begin{problem}
Some neighborhood kid is driving his scooter way too fast while passing your house. Annoyed, you set out to apply
basic high-school physics in order to compute his speed: You measure a fixed length section
in front of your house and, while hiding in the bushes, use a stop-watch to time how long he takes to cover that distance.
Your old measuring tape reports $18.20 \pm 0.05$ m and you clock him covering that distance in
$0.82 \pm 0.10$ s.  In your passive-aggressive letter to the kid's parents, what is the speed and uncertainty
that you report?
\end{problem}

\begin{problem}
From compositional data you infer that two volcanic rock samples represent the initial and final
pulse of activity, respectively.  Your two samples have been radiometrically dated at
$25.53 \pm 0.1$ Ma and $29.66 \pm 0.2$ Ma, respectively.
What is the likely duration of volcanic activity (including the error)?
\end{problem}

\begin{problem}
You have carefully measured the densities of an exposed ore body and the surrounding bed rock and determined
$\rho_o = 3.15 \pm 0.05$ g cm$^{-3}$ and $\rho_r = 2.67 \pm 0.05$ g cm$^{-3}$, respectively.  A gravimetric survey over the region
will be sensitive to variations in lateral density only, in other words the density contrast between the
ore and the host rock.  Assuming your samples are independent, what is the density contrast and its uncertainty?
\end{problem}

\begin{problem}
Assigned to a crummy summer internship on Mars, you are tasked with the low-tech job of measuring the perimeters
of circular craters using a \$69.99 measuring wheel from Home Depot.  You obtain the circumferences of two craters
before you realize you are in fact the subject of a psychological experiment.  Although upset, you nevertheless decide
to complete the estimates for the two craters.  The measuring wheel reports 12,311 and 9,045
clicks, respectively, and the manufacturer says each click represents a distance increment of 25 cm (i.e., the circumference of the
measuring wheel).  Previous studies suggest that this method is accurate to 1\% of the circumference.
What is the area (and its uncertainty) of each crater in m$^2$?
\end{problem}

\begin{problem}
The Bouguer\index{Bouguer, P.} equation for gravity due to a constant thickness slab is given by $g = 2 \pi \rho \gamma h$, where
$\gamma = 6.6738\cdot10^{-11}$ m$^3$ kg$^{-1}$ s$^{-2}$ is the universal gravitational constant with fractional uncertainty
$1.2\cdot10^{-4}$, $\rho = 2850$ g cm$^{-3}$ is the slab density, and $h = 230\pm1$ m is the thickness of the slab.  
\begin{enumerate}[label=\alph*)]
	\item To achieve a precision of 1\% in the Bouguer calculation, how well do you need to know the density
	value (i.e., what is the maximum uncertainty you can tolerate)?
	\item Report the Bouguer value for the slab, including its uncertainty, in mGal ($= 10^{-5} \mbox{m s}^{-2}$).
\end{enumerate}
\end{problem}

\begin{problem}
The subsidence of young ($< 80$ Myr) oceanic crust due to lithospheric cooling has been
shown to follow approximately a linear $\sqrt {\mbox{age}}$ relationship, given by
$$
z = z_r + \frac{2 \rho_m \alpha_v T_m}{(\rho_m - \rho_w)} \sqrt{\frac{\kappa t}{\pi}}.
$$
Given estimates of thermal diffusivity $\kappa = 1.00 \pm 0.04 \mbox{ mm}^2s^{-1}$, water density
$\rho_w = 1.027 \pm 0.001$ g cm$^{-3}$, mantle density $\rho_m = 3.30 \pm 0.01$ g cm$^{-3}$,
volumetric thermal expansion coefficient $\alpha_v = (3.00 \pm 0.02) \cdot 10^{-5}$ $^{\circ}\mbox{K}^{-1}$,
average ridge depth $z_r = 2500 \pm 200$ m, and mantle temperature $T_m = 1300 \pm 25$ $^{\circ}$K,
determine the predicted depth and its uncertainty for a location where rocks of age
$t = 29.7 \pm 0.5$ Myr were recovered.  Which term dominates the final uncertainty?
\end{problem}

\begin{problem}
\index{Elastic plate thickness}
\index{Young's modulus}
\index{Poisson's ratio}
The flexural parameter $\alpha$ encountered in studies of elastic plate flexure is given by
$$
\alpha = \left [ \frac{4D}{(\rho_m - \rho_w)g} \right ]^\frac{1}{4},
$$
where $D$ is the flexural rigidity and is related to the \emph{elastic plate thickness}, $h$, via
$$
D = \frac{E h^3}{12 (1 - \nu^2)},
$$
where $E$ and $\nu$ are Young's modulus and Poisson's ratio, respectively.
It can be shown that the distance from a 2-D line load (approximating an island chain, perhaps) to
the maximum peripheral uplift (the ``bulge'') is given by $x_b = \pi \alpha$.  Given the measurements
$\rho_m = 3300 \pm 50$ kg m$^{-3}$, $E = 70 \pm 7$ GPa, $\nu = 0.25 \pm 0.01$, assuming no
uncertainty in $g = 9.81$ m s$^{-2}$ and $\rho_w = 1027$ kg m$^{-3}$, and observing $x_b = 252 \pm 10$ km,
what is the corresponding elastic plate thickness (and its uncertainty)?
\end{problem}

\begin{problem}
When collecting gravity measurements on-board a moving platform, such as a ship or aircraft, one must account
for the \emph{E{\"o}tv{\"o}s effect}, which will reduce or increase the apparent gravity depending on the moving platform's
heading ($\alpha$), latitude ($\theta$) and speed ($v$).  This effect (in mGal) is
$$
E = 14.585 v \cos \theta \sin \alpha + 0.015696 v^2,
$$
where speed $v$ is given in m/s.
\begin{enumerate}[label=\alph*)]
\item If the ship is at the equator and moving at $10.0 \pm 0.5$ knots with heading $\alpha = 35 \pm 2$\DS,
what is the E{\"o}tv{\"o}s effect and its uncertainty ? (You can assume $\alpha$ and $v$ have independent errors and
that the uncertainty in latitude is negligible).  Hint: Consider $E$ to be the function $E(v, \alpha)$ and use the
rule for uncertainty in a function (i.e., \ref{eq:uncert_func}).
\item Consider an aero-gravity survey using a plane capable of flying at 200 knots.  We wish to run
parallel lines in a certain direction $\alpha$ such that the \emph{changes} in $E$ with small changes in
$v$ are minimized.  Find the optimal survey line orientation.  You may again assume that the uncertainty
in latitude is negligible and that we are flying at a latitude of 45 degrees north.
\end{enumerate}
\end{problem}
